\documentclass{article}
\usepackage[utf8]{inputenc}
\usepackage{enumerate}

\author{kobriendublin }
\date{December 2018}

\begin{document}

%- Higher Certificate, Module 5, 2008. Question 1
\section{Introduction}
\begin{enumerate}[(i)]
\item HC5 2016 - solutions
2. (a) (i)
'( )
( ) log ( ) '( ) .
( )
X
X
X
M t
R t M t R t
M t
   1 (1st deriv correct)
Then using the quotient rule,
 
 
2
2
( ) ''( ) '( )
''( ) .
( )
X X
X
M t M t M t
R t
M t

 1 (2nd deriv correct)
TOTAL 2
(ii) We have that MX '(0)  E(X) and 2 ''(0) ( ). X M  E X 1 (using zero to find expectations)
Also by definition, 0 (0) ( ) (1) 1. X M  E e  E  1 (MX(0)=1)
So
( )
'(0) ( ).
1
E X
R   E X Also
2 2
2 2
2
1 ( ) ( ( ))
''(0) ( ) ( ( )) Var( ).
1
E X E X
R E X E X X
 
   
1 (1st deriv shown correct), 1 (2nd deriv shown correct)
TOTAL 4
(b) (i)
0 0
( )
( ) ( ) .
! !
t
y t y
tY ty e
Y
y y
e e
M t E e e e e e
y y

       
 
 
     = ( 1) .
t e e 
1 (definition as expectation), 1 (integration step),
1 (recognition of sum)
TOTAL 3
(ii) ( 1) 0 0 '( ) ( ) .
t t e
Y M t e e E Y e e          1 (1st deriv correct), 1 (correct substitution)
3 3 3 2 2 3 E (Y  E(Y))   E (Y )   E(Y ) 3E(Y )  3 E(Y)  .
1 (substitute for E(Y)), 1 (expansion)
Using 2 3 E(Y ),E(Y ) given in the question, we have
3 2 3 2 2 3 E (Y  E(Y))     3  3 (  )  3 . 
= 2 3 2 3 3 3   3  3 3  3   as required.
1 (substituted and simplified)
TOTAL 5
(iii) Mgf of sum of independent random variables is the product of their mgfs.
1 (mention independence),1 (state result)
Here they are identically distributed, so we raise to the nth power. 1 (nth power required)
  ( 1) ( ) ( )
n n et
S Y M t M t e     1 (power correctly applied)
This can be identified as the mgf of Poisson (n ). 1 (Poisson identified)
Since there is a unique one-to-one relationship between a distribution and its mgf, S follows a
Poisson (n ) distribution. 1 (state uniqueness)
TOTAL 6
\end{enumerate}
\end{document}
\end{enumerate}
\end{document}
\documentclass[a4paper,12pt]{article}
%%%%%%%%%%%%%%%%%%%%%%%%%%%%%%%%%%%%%%%%%%%%%%%%%%%%%%%%%%%%%%%%%%%%%%%%%%%%%%%%%%%%%%%%%%%%%%%%%%%%%%%%%%%%%%%%%%%%%%%%%%%%%%%%%%%%%%%%%%%%%%%%%%%%%%%%%%%%%%%%%%%%%%%%%%%%%%%%%%%%%%%%%%%%%%%%%%%%%%%%%%%%%%%%%%%%%%%%%%%%%%%%%%%%%%%%%%%%%%%%%%%%%%%%%%%%
\usepackage{eurosym}
\usepackage{vmargin}
\usepackage{amsmath}
\usepackage{graphics}
\usepackage{epsfig}
\usepackage{enumerate}
\usepackage{multicol}
\usepackage{subfigure}
\usepackage{fancyhdr}
\usepackage{listings}
\usepackage{framed}
\usepackage{graphicx}
\usepackage{amsmath}
\usepackage{chngpage}
%\usepackage{bigints}

\usepackage{vmargin}
% left top textwidth textheight headheight
% headsep footheight footskip
\setmargins{2.0cm}{2.5cm}{16 cm}{22cm}{0.5cm}{0cm}{1cm}{1cm}
\renewcommand{\baselinestretch}{1.3}

\setcounter{MaxMatrixCols}{10}
\begin{document}
%%%%%%%%%%%%%%%%%%%%%%%%%%%%%%%%%%%%%%%%%%%%%%%%%%%%%%%%%%%%%%%%%%%%%%%%%%%%%%%%%%%%%
\begin{table}[ht!]
     
\centering
     
\begin{tabular}{|p{15cm}|}
     
\hline 
7. A factory manager wishes to know whether or not the number of rejects produced by an
industrial process within a set period follows a Poisson distribution. The previous month’s
data, given below, are thought to be typical:
Number of rejects 0 1 2 3 4 5 6 or more
Frequency 38 49 43 17 11 2 0
(i) For what reasons might the rejects follow a Poisson distribution?
(ii) Test the hypothesis that the distribution of rejects is Poisson, and explain your result to
the factory manager.
\\ \hline
      
\end{tabular}
    
\end{table}
%%%%%%%%%%%%%%%%%%%%%%%%%%%%%%%%%%%%%%%%%%%%%%%%%%%%%%%%%%%%%%%%%%%%%%%%%%%%%%%%%%%%%

The value of the test-statistic is 
\[ {\displaystyle \chi ^{2}=\sum _{i=1}^{n}{\frac {(O_{i}-E_{i})^{2}}{E_{i}}} \]

\begin{enumerate}
\item If the process is producing individual rejects “at random”, i.e. singly and
at unpredictable instants of time, but at a constant rate over the period
of study, then the number of rejects during a fixed time of observation will
follow a Poisson distribution.
\item The mean must be estimated from the data:
¯x =
1
160
(0 + 49 + 86 + 51 + 44 + 10) =
240
160
= 1.5
Expected frequencies are 160e-1.5(1.5)r=r! for r = 0; 1; ¢ ¢ ¢.

\begin{center}
\begin{tabular}{cc|cc|cc|cc}
r : &0 & 1 & 2 & 3 & 4 & $\geq¸ 5$ & TOTAL\\
Obs:& 38 & 49 & 43 & 17 & 11 & 2&  160\\
Exp: &35.70& 53.55& 40.16 & 20.08 & 7.53&  2:98 160 \\
\end{tabular}
\end{center}
(The last two cells may be combined, but this is not really necessary.)
Â2 has 4 d.f., since 1 parameter had to be estimated and the totals of Obs
and Exp have to be the same.
15
Â2
(4) 

\begin{eqnarray*}
\chi^2 &=& \frac{(38-35.70)^2}{35.70} \;+\; \frac{(49-53.55)^2}{53.55} \;+\; \frac{(43-40.16)^2}{40.16} +\\ 
& & \frac{(17-20.08)^2}{20.08} \;+\; \frac{(11-7.53)^2}{7.53} \;+\; \frac{(2-2.98)^2}{2.98}\\\\
&=& \frac{(38-35.70)^2}{35.70} \;+\; \frac{(49-53.55)^2}{53.55} \;+\; \frac{(43-40.16)^2}{40.16} +  \\ 
& &  \frac{(17-20.08)^2}{20.08} \;+\; \frac{(11-7.53)^2}{7.53} \;+\; \frac{(2-2.98)^2}{2.98}\\\\
&=& 3:13;
\end{eqnarray*}
not significance.
\begin{itemize}
\item There is no reason to reject the hypothesis that the data follow a Poisson
distribution. 
\item Therefore the number of rejects per unit time is likely to remain
reasonably constant and they do not arise in any regular or predictable way.
\end{itemize}
\end{enumerate}
\end{document}

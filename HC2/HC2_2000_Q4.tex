4(i)A suitable N.H. is that the population distributions are the same with A.H.that
they are different. On the N.H. the number of instance where the new types will give a
higher figure is Binomial(10; p = y2).
there are 8 differences in favor of the original,and 2 for the new.
In B(10; 1
2 ), P(2 or less) = (1 + 10 + 45)(1
2 )10 = 0:0547 not significant. so the N.H. is
not rejected.
(ii)For a Wilcoxon test the N.H. is that population distributions are identical,and
the A.H is that they differ in location.
The actual difference(original-new) for each car are: +0.4,+0.3,-0.6,+0.8,+0.2,-0.1,+0.3+0.4,+1.1,
+0.2 and the ranks are 61
2 ; 41
2 ; 8; 9; 21
2 ; 1; 41
2 ; 61
2 ; 10; 21
2 : The sums of ranks are T¡ =
8 + 1 = 9; T+ = 46 the smaller of these is compared with the critical value for n=10,
which is T = 8; T¡ > 8 so the N.H. is not rejected.
(However, the result in (ii) is nearer to significance than that in(i)).
If we could assume that the differences for the 10 cars followed a normal distribution,
then a paired t-test could be applied. It would, if valid, be were powerful than either of
those above.

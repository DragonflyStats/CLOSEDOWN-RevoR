\documentclass[a4paper,12pt]{article}
%%%%%%%%%%%%%%%%%%%%%%%%%%%%%%%%%%%%%%%%%%%%%%%%%%%%%%%%%%%%%%%%%%%%%%%%%%%%%%%%%%%%%%%%%%%%%%%%%%%%%%%%%%%%%%%%%%%%%%%%%%%%%%%%%%%%%%%%%%%%%%%%%%%%%%%%%%%%%%%%%%%%%%%%%%%%%%%%%%%%%%%%%%%%%%%%%%%%%%%%%%%%%%%%%%%%%%%%%%%%%%%%%%%%%%%%%%%%%%%%%%%%%%%%%%%%
\usepackage{eurosym}
\usepackage{vmargin}
\usepackage{amsmath}
\usepackage{graphics}
\usepackage{epsfig}
\usepackage{enumerate}
\usepackage{multicol}
\usepackage{subfigure}
\usepackage{fancyhdr}
\usepackage{listings}
\usepackage{framed}
\usepackage{graphicx}
\usepackage{amsmath}
\usepackage{chngpage}
%\usepackage{bigints}

\usepackage{vmargin}
% left top textwidth textheight headheight
% headsep footheight footskip
\setmargins{2.0cm}{2.5cm}{16 cm}{22cm}{0.5cm}{0cm}{1cm}{1cm}
\renewcommand{\baselinestretch}{1.3}

\setcounter{MaxMatrixCols}{10}

\begin{document}

  \begin{table}[ht!]
     \centering
     \begin{tabular}{|p{15cm}|}
     \hline        
Describe and explain a linear model used for a one-way analysis of variance.  Explain clearly what each term in the model represents and state any assumptions required for the analysis to be valid. (6) 
 

 
 \\ \hline
      \end{tabular}
    \end{table}
    

        

\begin{enumerate}
\item (i) In the model $yij = u + ®i + "ij$ , the response $y_{ij}$ on the jM unit under ”treatment” i
contains:
U=a general mean response,
®i=an effect, or departure from general mean,due to treatment i,
"ij=random residual ”error”term, N » (0; ±2), with constant ±2, mutually independent
for all i,j.
The model is assumed additive. The standard assumptions thus are additivity, normality,
constant variance.



\newpage


  \begin{table}[ht!]
     \centering
     \begin{tabular}{|p{15cm}|}
     \hline  
 (ii) A farmer is considering adding diet supplements to the food currently given to his dairy cows in an attempt to increase milk yields.  In order to assist his decision, he sets up an experiment using a random sample of 24 of his cows who are assigned randomly to one of the following 3 diet regimes: standard diet, standard diet plus supplement A, standard diet plus supplement B.  At the end of the three month study period the average daily milk yield (in pints) is recorded for each cow with the following results. 
 
  
1:  Standard diet 
2:  Standard diet plus supplement A 
3:  Standard diet plus supplement B 16 24 25 18 20 21 19 29 23 21 25 19 24 27 24 21 26 22 25 23 26 17 21 20 
 
 
Carry out an analysis of these data, and write a report on your findings for the farmer.  
\\ \hline 
\end{tabular}
\end{table}
%%%%%%%%%%%%%%%%%%%%%%%%%%%%%%%%%%%%%%%%%%%%%%%%
\item H0 is ”$\mu_1 = \mu_2 = \mu_3$”, and the alternative H1 is that at least one is different from the
others.
Total for treatments: 1:161, 2:195, 3:180. Grand total=536
P
y2
ij ¡ 12222
9
Corrected total s.s. = 12222 ¡ 5362=24 = 251:33 s.s. for treatments= 1612+1952+1802
8 ¡
5362
24 = 12043:25 ¡ 11970:67 = 72:583

Analysis of Variance.
\begin{center}
 \begin{tabular}{|c|c|c|c|c|}
SOUERCE OF VARIATION & D:F: & SUM OF SQUARES & MEAN SQUARE & F \\
TREATMENTS & 2 & 72:583&  36:292&   4:26\\
RESIDUAI & 21&  178.750 & 8:512&  \\
TOTAL & 23 & 251. 333& & \\
\end{tabular}   
\end{center}




Mean: 1:20.125, 3:22.500, 2:24.375.

\begin{itemize}
    \item The significant difference between any two means
(since all have 8 replicates)is t21
q
2£8:512
8 = 2:08 £ 1:459 = 3:03 at the 5\% level.
\item The milk yield for treatment 2 (using supplement A)gave the highest average in this
trial. 
\item But it should be noted that when allowance is made for the variation between
individual animals the statistical evidence only indicates that this treatment is better
than ”no supplement”. 
\item The position for supplement B remains in doubt.
\end{itemize}

\end{enumerate}
\end{document}

\documentclass[a4paper,12pt]{article}
%%%%%%%%%%%%%%%%%%%%%%%%%%%%%%%%%%%%%%%%%%%%%%%%%%%%%%%%%%%%%%%%%%%%%%%%%%%%%%%%%%%%%%%%%%%%%%%%%%%%%%%%%%%%%%%%%%%%%%%%%%%%%%%%%%%%%%%%%%%%%%%%%%%%%%%%%%%%%%%%%%%%%%%%%%%%%%%%%%%%%%%%%%%%%%%%%%%%%%%%%%%%%%%%%%%%%%%%%%%%%%%%%%%%%%%%%%%%%%%%%%%%%%%%%%%%
\usepackage{eurosym}
\usepackage{vmargin}
\usepackage{amsmath}
\usepackage{graphics}
\usepackage{epsfig}
\usepackage{enumerate}
\usepackage{multicol}
\usepackage{subfigure}
\usepackage{fancyhdr}
\usepackage{listings}
\usepackage{framed}
\usepackage{graphicx}
\usepackage{amsmath}
\usepackage{chngpage}
%\usepackage{bigints}

\usepackage{vmargin}
% left top textwidth textheight headheight
% headsep footheight footskip
\setmargins{2.0cm}{2.5cm}{16 cm}{22cm}{0.5cm}{0cm}{1cm}{1cm}
\renewcommand{\baselinestretch}{1.3}

\setcounter{MaxMatrixCols}{10}
\begin{document}
%%%%%%%%%%%%%%%%%%%%%%%%%%%%%%%%%%%%%%%%%%%%%%%%%%%%%%%%%%%%%%%%%%%%%%%%%%%%%%%%%%%%%
\begin{table}[ht!]
     
\centering
     
\begin{tabular}{|p{15cm}|}
     
\hline    
4. The table below shows results from a clinical trial of antibiotics in controlling stomach
ulcers, giving the number who did or did not recover within four weeks under each of two
treatments. (Patients were allocated to Drug 1 or Drug 2 at random.) Patients who dropped
out of treatment because of side-effects were excluded in (A), but included as ‘not
recovered’ in (B).
(A). Without drop-outs (B). With drop-outs
Drug 1 Drug 2 Drug 1 Drug 2
Recovered 11 13 11 13
Not recovered 7 2 18 13
(Source: D.J Hand, Journal of the Royal Statistical Society, Series A, Part 3, 1994.)
(i) Use Fisher’s exact test to analyse the data in Table (A), and a chi-square test for
Table (B). Explain clearly in each case what you conclude from your test.
(ii) What is the effect of including or excluding the drop-outs from this analysis?
\\ \hline
      
\end{tabular}
    
\end{table}
%%%%%%%%%%%%%%%%%%%%%%%%%%%%%%%%%%%%%%%%%%%%%%%%%%%%%%%%%%%%%%%%%%%%%%%%%%%%%%%%%%%%%
4. (A)
1 2
R 11 13 : 24
NR 7 2 : 9
18 15 33
j
More extreme tables are
10 14 : 24
8 1 : 9
18 15 33
and 9 15 : 24
9 0 : 9
18 15 33
Together, these form the “tail” of the distribution when margins are fixed.
probability are
18! 15! 24! 9!
33! 11! 7! 13! 2!
;
18! 15! 24! 9!
33! 10! 8! 14! 1!
;
18! 15! 24! 9!
33! 9! 9! 15! 0! :
i.e. 18£17£14
55£31£29 = 0:08664; 9£17
31£290 = 0:01703; 17
31£29£15 = 0:00126.
\begin{itemize}
    \item For a 2-tail test of the Null Hypothesis of no difference, the probability is
2(0:08664+0:01702+0:00126) = 0:2098. 
\item There is no significant evidence for
any difference between the two drugs.

\end{itemize}

(B) The Â2
(1) test also tests the N. H. that the proportions recovering are the
same on each drug. ‘Expected’ frequencies are those given by this N. H. with
the same marginal totals as the ‘Observed’.
OBSERVED (EXPECTED) Drug1 Drug2
Recovered 11(12:65) 13(11:35) 24
Not recovered 18(16:35) 13(14:65) 31
29 26 55
Â2
(1) = (1:65)2( 1
12:65 + 1
11:35 + 1
16:35 + 1
14:65 ) = 2:7225 £ 0:296578 = 0:807 n.s.
Again there is no significant evidence against the N. H.


(ii) 
\begin{itemize}
\item The inference is the same in this, rather small, trial whether or not the
drop-outs are included. 
\item There were 11 drop-outs on each drug, which is a
considerable proportion of patients beginning the trial; 
\item however, no particular
reasons for drop-out are known.
\end{itemize}
\end{document}

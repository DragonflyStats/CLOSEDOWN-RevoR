\documentclass{article}
\usepackage[utf8]{inputenc}
\usepackage{framed}
\usepackage{enumerate}

\begin{document}

\maketitle

\section{Introduction}

%%%%%%%%%%%%%%%%%%%%%%%%%%%%%%%%%%%%%%%%%%%%%%%%%%%%%%%%%%%%%%%%%%%%%
6.(a)When a large sample of data is available from any population, not necessarily normal, and
including discrete data as well as continuous, the sample mean or total follows a distribution that
is approximately normal. Therefore with large samples of data significance tests of, and confidence
intervals for, a population mean may be found, at least approximately, without knowing what
distribution the population has. This extends, for example to proportions in a binomial.
In practice, when distributions are reasonably symmetrical, even when not normal, samples
may be as small as about 30, while if data are very skew then very large samples-several hundredmay
be required to give acceptable results. An examination of data, possibly by graphical methods,
is a useful guide when applying the approximation.
We may treat ¯x as N(¹; ¾2=n) when n is sample size and ¹; ¾2 are the (known or estimated)
mean and variance of the population distribution. The only theoretical restriction is that ¹ and
¾2 must be finite.
Many estimates of practically important items are the sum of several independent components,
e.g. crop yields of many plants forming a plot, and so their total tends to be normally distributed,
N(n¹; n¾2).
(b)(i)For difference in means, a 95% confidence interval is approximately
(x¯1 ¡ x¯2) § 1:96
p
s21
=n1 + s22
=n2.
For the given data, this is (3:75 ¡ 2:10) § 1:96
q
2:742
125 + 1:402
108 , i.e. 1:65 § 1:96 £ 0:2797
8
giving 1:65 § 0:55 or (1.10 to 2.20).
Since this interval does not contain zero, it is likely that there will be more calls each shift in
district 1 than in 2, the mean difference being between 1.1 and 2.2(with probability 0.95).
(ii)For difference in proportions,q pˆ1¡pˆ2
p1(1¡p1)
n1
+p2(1¡p2)
n2
» N(0; 1).
The NH is ”p1 = p2”.
Hence 26
125 ¡ 15
108 = 0:208 ¡ 0:139 = 0:069 is the estimated difference.
Its variance is 0:208£0:792
125 + 0:139£0:861
108 = 2:426 £ 10¡3; S.E.=0.049.
Hence 0:069
0:049 = 1:40 n.s. as N(0,1)and there is no significant evidence of a difference in proportions.
\end{enumerate}
\end{document}
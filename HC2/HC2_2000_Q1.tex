Paper II: Statistical Methods
1(i) Use stems of 5 units, rather than 10,to give clearer results.
Division1
1 j 4
¢ j 6 6 7 8 9 9
2 j 0 0 0 1 2 3 4 4
¢ j 5 6 7 7 8 9
3 j 4
Median = 1
2(21 + 22) = 21:5
Lower quartile = 19
Upper quartile = 26
Interquartile range = 7
Division2
9
1 j
¢ j 5 5 8
2 j 1 3 4 4
¢ j 5 6 8 8 9 9 9
3 j 1 2 3 4 4
¢ j 7 8 9
Median = 1
2(28 + 29) = 28:5
Lower quartile = 24
Upper quartile = 33
Interquartile range = 9
The means and variances are required later.
Division1 : ˜ x1 = 22:23; s21
= 24:7554; s1 = 498:
Division2 : ˜ x2 = 27:82; s21
= 46:8225; s1 = 684:
(ii)The average values are considerably higher in Division 2;So is the variability
especially because of the three high values 37,38,39.The distributions are otherwise reasonably
symmetrical.
(iii)assuming the normal distribution can be used as a model, ( ¯ x1¡ ¯ x2)§t(42)
q
2s2=22
is the confidence interval, where s2is the pooled variance 21s21
+20s22
42 , with 42 d.f. ¯ x1 ¡
¯ x2 = ¡5:59; t(42) at 5% is (approx.) 2.019;s2 = 35:7890 so the interval is ¡5:59 §
2:019
q
s2
11 i:e: ¡5:59§2:019£1:804.giving ¡5:59§3:642; i:e: (¡9:23 to ¡1:95):The
variance in the two populations are assumed to be the same.

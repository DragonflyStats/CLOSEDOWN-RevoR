\documentclass[a4paper,12pt]{article}
%%%%%%%%%%%%%%%%%%%%%%%%%%%%%%%%%%%%%%%%%%%%%%%%%%%%%%%%%%%%%%%%%%%%%%%%%%%%%%%%%%%%%%%%%%%%%%%%%%%%%%%%%%%%%%%%%%%%%%%%%%%%%%%%%%%%%%%%%%%%%%%%%%%%%%%%%%%%%%%%%%%%%%%%%%%%%%%%%%%%%%%%%%%%%%%%%%%%%%%%%%%%%%%%%%%%%%%%%%%%%%%%%%%%%%%%%%%%%%%%%%%%%%%%%%%%
\usepackage{eurosym}
\usepackage{vmargin}
\usepackage{amsmath}
\usepackage{graphics}
\usepackage{epsfig}
\usepackage{enumerate}
\usepackage{multicol}
\usepackage{subfigure}
\usepackage{fancyhdr}
\usepackage{listings}
\usepackage{framed}
\usepackage{graphicx}
\usepackage{amsmath}
\usepackage{chngpage}
%\usepackage{bigints}

\usepackage{vmargin}
% left top textwidth textheight headheight
% headsep footheight footskip
\setmargins{2.0cm}{2.5cm}{16 cm}{22cm}{0.5cm}{0cm}{1cm}{1cm}
\renewcommand{\baselinestretch}{1.3}

\setcounter{MaxMatrixCols}{10}
\begin{document}
%%%%%%%%%%%%%%%%%%%%%%%%%%%%%%%%%%%%%%%%%%%%%%%%%%%%%%%%%%%%%%%%%%%%%%%%%%%%%%%%%%%%%%%%%
\begin{table}[ht!]
     

\centering
     

\begin{tabular}{|p{15cm}|}
     

\hline 

 
3. To treat patients effectively the drugs prescribed by doctors must have an accurately defined potency.  Consequently, the distribution of potency values for shipments of a drug must have a mean value as indicated on the container, and must have a small variance.  If this does not happen pharmacists may be distributing drug prescriptions that could be harmfully potent or have low potency and be ineffective.  In the marketing leaflet for a particular drug the manufacturer claims that the drug has a potency of 10 milligrams per cubic centimetre (mg/cc) with a standard deviation of 0.04 mg/cc.  To investigate the manufacturer's claims a pharmacist selects a simple random sample of ten containers from the latest shipment of the drug and measures the potency of each with the following results: 
 
\[\{9.96   9.99   10.05   10.04   9.97   10.06   10.00   10.08   10.07   9.95 . \}\]
 
 Test whether the manufacturer’s claims concerning the potency of the drug are being met, clearly stating any assumptions on which your analysis depends. (14) 
 
 Determine whether your conclusions would have been the same if the sample had been three times as large with the same sample mean and standard deviation. (6) 

\\ \hline


\end{tabular}
    

\end{table}

%%%%%%%%%%%%%%%%%%%%%%%%%%%%%%%%%%%%%%%%%%%%%%%%%%%%%%%%%%%%%%%%%%%%%%%%%%%%%%%%%%%%%%%%%

\begin{enumerate}
\item N.H. “¹ = 10”, Also for s.d., N.H. is “¾ = 0:04”
for the sample,
n = 10
X
x = 100:17
X
x2 = 1003:4242; s2 = 2:35667 £ 10¡3; s = 0:04865
t(9) =
¯x ¡ ¹
s
p
10
=
10:017 ¡ 10:000
p
2:235667 £ 10¡4
=
0:017
0:0150
= 1:137 n:s:
\begin{itemize}
    \item The N.H. ”¹ = 10” is not rejected on this evidence.
X2
(9) =
(n ¡ 1)s2
¾2 =
9 £ 2:35667 £ 10¡3
(0:04)2 = 13:256
less than the value for 5% significance, so the N.H. is not rejected.
\item We have assumed the determinations of potency are independently normally distributed
with the same variance.
with n=30,t(9) has
p
10 replace by
p
30, and it is now t(29). the value of t(29) will be
p
p30
10
£
1:137 = 1:97, which is approaching the 5\% significance point, so now the N.H. is open
to doubt (although technically not rejected at 5%)
\item For the variance,x2
(29) = 29s2
¾2 , so in the previous calculation 9 is replaced by 29 and
x2
(29) = 29
9 £13:256 = 42:714:. 
\item This again is very near the 5% point, but this time is just
significant and we may reject the N.H. for ¾2
11
In both cases the large amount of data has given a more powerful test
\end{itemize}

\end{enumerate}
\end{document}

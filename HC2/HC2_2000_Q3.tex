3.N.H. “¹ = 10”, Also for s.d., N.H. is “¾ = 0:04”
for the sample,
n = 10
X
x = 100:17
X
x2 = 1003:4242; s2 = 2:35667 £ 10¡3; s = 0:04865
t(9) =
¯x ¡ ¹
s
p
10
=
10:017 ¡ 10:000
p
2:235667 £ 10¡4
=
0:017
0:0150
= 1:137 n:s:
The N.H. ”¹ = 10” is not rejected on this evidence.
X2
(9) =
(n ¡ 1)s2
¾2 =
9 £ 2:35667 £ 10¡3
(0:04)2 = 13:256
less than the value for 5% significance, so the N.H. is not rejected.
We have assumed the determinations of potency are independently normally distributed
with the same variance.
with n=30,t(9) has
p
10 replace by
p
30, and it is now t(29). the value of t(29) will be
p
p30
10
£
1:137 = 1:97, which is approaching the 5% significance point, so now the N.H. is open
to doubt (although technically not rejected at 5%)
For the variance,x2
(29) = 29s2
¾2 , so in the previous calculation 9 is replaced by 29 and
x2
(29) = 29
9 £13:256 = 42:714:. This again is very near the 5% point, but this time is just
significant and we may reject the N.H. for ¾2
11
In both cases the large amount of data has given a more powerful test

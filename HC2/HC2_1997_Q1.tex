9
II. STATISTICAL METHODS
1. (i) For 1988, ¯x1 = 53:4, s1 = 19:7; also n = 750;
for 1990, ¯x2 = 55:3, s2 = 19:5; also n = 633.
If ¹1; ¹2 are the corresponding population means, H0 is ¹1 = ¹2 (or, strictly,
¹1 ¸ ¹2) and H1, to be tested, is ¹2 > ¹1.
V (¯x2 ¡ ¯x1) = s21
n1
+ s22
n2
= 19:72
750 + 19:52
633 = 1:11816, SE = 1:057.
As these are large samples of date we use a normal (r) test:
r = 55:3¡53:4
1:057 = 1:9
1:057 = 1:798.
The form of H1 requires a one-tail test, with critical value 1.645 at 5%.
Hence we reject H0.
A 95% confidence interval for the increase is 1:9§1:96£1:057 = 1:9§2:07,
or (-0.17; 3.97).
If we are certain that there must have been an increase we may prefer to
quote this result as (0; 3:97).
(ii) For 1988, pM = 349
750 = 0:4653 and pF = 0:5347; n = 750.
For 1990, pM = 321
633 = 0:5071 and pF = 0:4929; n = 633.
The hypotheses H0: pM;1988 = pM;1990 and H1 : pM has changed can be
examined in a 2 £ 2 table of ‘observed’ frequencies and ‘those expected on
H0’.
OBSERVED(EXPECTED) 1988 1990 TOTAL
MALE 349(363:34) 321(306:66) 670
FEMALE 401(386:66) 312(326:34) 713
750 633 1383
Â2
(1) =
(349 ¡ 363:34)2
363:34
+ ¢ ¢ ¢ +
(312 ¡ 326:34)2
326:34
= (14:34)2f
1
363:34
+
1
306:66
+
1
386:66
+
1
326:34
g
= 205:6356 £ 0:011664 = 2:40n:s:
There is no evidence of change.
[An alternative method is to use normal approximations for pM: N(p; p(1¡p)
n )
in each year and consider the difference. This would be needed if confidence
intervals had been required. ]
2. (a) yij = ¹ + ¿i + ²ij , where yij is the observation measured as the jth of the
items receiving treatment i; ¹ is a grand (overall) mean term; ¿i is an effect
10
(deviation from mean) due to treatment i; ²ij are i.i.d. N(0; ¾2) residual
terms. There are i = 1 to v treatments, ri replicates of each, and
Xv
i=1
ri = N,
the total number of items in the experiment.
(b)
“Treatment” ri
P
yij
P
y2
ij ¯yi
1 6 128:0 2792:00 21:33
2 4 79:4 1582:06 19:85
3 4 90:7 2064:51 22:68
4 3 60:5 1243:25 20:17
17 358:6 7681:82
Although results 1 are rounded to whole numbers, analysis of variance will
have to assume that all observations on all treatments have the same variance
¾2. Also the material used in the trial should have been selected at
random from what was available, and the samples examined under identical
conditions in random order.
(i) Total corrected S:S: = 7681:82¡G2=N = 7681:82¡7564:35 = 117:47. “Treatments”
S:S: =
1282
6
+
79:42 + 90:72
4
+
60:52
3
¡
G2
N
= 7583:4625 ¡ 7564:35 = 19:11.
Analysis of Variance D:F: S:S: M:S:
Treatments(Storages) 3 19:11 6:371 F < 1
Residual 13 98:36 7:566 = ˆ¾2
TOTAL 16 117:47
We are not given any specific contrasts among storages to be tested, but
even if the whole Treatments S:S: were due to one contrast this would still
not be significant as F(1;13) ( 19:11
7:566 = 2:52, less than the 5% point 4.67). We
may say confidently, that there are no significant differences among these
“Treatments”.
(ii) Given the result in (i), there could be no change to the inference. [In a
borderline case, some intelligence in looking at individual differences may be
called for, as ¾2 may be slightly overestimated.]

\documentclass[a4paper,12pt]{article}
%%%%%%%%%%%%%%%%%%%%%%%%%%%%%%%%%%%%%%%%%%%%%%%%%%%%%%%%%%%%%%%%%%%%%%%%%%%%%%%%%%%%%%%%%%%%%%%%%%%%%%%%%%%%%%%%%%%%%%%%%%%%%%%%%%%%%%%%%%%%%%%%%%%%%%%%%%%%%%%%%%%%%%%%%%%%%%%%%%%%%%%%%%%%%%%%%%%%%%%%%%%%%%%%%%%%%%%%%%%%%%%%%%%%%%%%%%%%%%%%%%%%%%%%%%%%
\usepackage{eurosym}
\usepackage{vmargin}
\usepackage{amsmath}
\usepackage{graphics}
\usepackage{epsfig}
\usepackage{enumerate}
\usepackage{multicol}
\usepackage{subfigure}
\usepackage{fancyhdr}
\usepackage{listings}
\usepackage{framed}
\usepackage{graphicx}
\usepackage{amsmath}
\usepackage{chngpage}
%\usepackage{bigints}

\usepackage{vmargin}
% left top textwidth textheight headheight
% headsep footheight footskip
\setmargins{2.0cm}{2.5cm}{16 cm}{22cm}{0.5cm}{0cm}{1cm}{1cm}
\renewcommand{\baselinestretch}{1.3}

\setcounter{MaxMatrixCols}{10}
\begin{document}
%%%%%%%%%%%%%%%%%%%%%%%%%%%%%%%%%%%%%%%%%%%%%%%%%%%%%%%%%%%%%%%%%%%%%%%%%%%%%%%%%%%%%%%%%
\begin{table}[ht!]
     

\centering
     

\begin{tabular}{|p{15cm}|}
     

\hline 

 
6. A random sample of 100 shops in a particular city was taken and the profit margins (%) were calculated with the following results: 
 
\begin{center}
\begin{tabular}{|c|c|}
Profit margin (\%) & Number of shops \\
0 but less than 2   &  3 \\
2 but less than 4   &  8 \\
4 but less than 5   & 15 \\
5 but less than 6   & 16 \\
6 but less than 7   & 17 \\
7 but less than 8   & 18 \\
8 but less than 10   & 19 \\
10 but less than 12    &  3 \\
12 or over     & 1   \\
Total & 100 \\
\end{tabular}
\end{center}
 
 
Draw a histogram depicting the above data. 
(8) 
 
Estimate the mean and median profit margins (%), using the above data.  
Which is the modal class interval? (8) 
 
You are informed that the mean and median are really 6.42% and 6.41% respectively.  
Account for any differences of these figures from your answers. (4) 
\\ \hline


\end{tabular}
    

\end{table}

%%%%%%%%%%%%%%%%%%%%%%%%%%%%%%%%%%%%%%%%%%%%%%%%%%%%%%%%%%%%%%%%%%%%%%%%%%%%%%%%%%%%%%%%%
\begin{verbatim}
6:As shown in figure 1
The 6th question
The model class interval is “7 but less than 8” For mean, we use mid-points of interval
as x:
13
x f F fx
1 3 3 3
4:5 8 11 24
4:5 15 26 67:5
5:5 16 42 88
6:5 17 59 110:5
7:5 18 77 135
9 19 96 171
11 3 99 33
13 1 100 13
645
Estimation of mean ¯x = 645
100 = 6:45

\end{verbatim}
\begin{itemize}
    \item Median is midway between 50M and 51sr in rank orders. There are 42 up to 6, and 17
in the interval 6 to 7.
M = 6 + 50:5¡42
17 £ 1 = 6:5
\item By using mid-points to calculate ¯x, we assume the observations in the interval are uniformly
distributed. Because we have a slight overestimation,it seems that there was a
slight tendency for them to occur nearer the lower limits of intervals. 
\item we do not know
what to use for x in the final interval; 13 may be too high but as the frequency is only
1 the effect is small.
\item the median being overestimated indicates that in “‘6 to 7”most of
the observations actually fall in the lower part of the interval.
\end{itemize}

\end{document}

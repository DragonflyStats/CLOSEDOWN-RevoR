6:As shown in figure 1
The 6th question
The model class interval is “7 but less than 8” For mean, we use mid-points of interval
as x:
13
x f F fx
1 3 3 3
4:5 8 11 24
4:5 15 26 67:5
5:5 16 42 88
6:5 17 59 110:5
7:5 18 77 135
9 19 96 171
11 3 99 33
13 1 100 13
645
Estimation of mean ¯x = 645
100 = 6:45
Median is midway between 50M and 51sr in rank orders. There are 42 up to 6, and 17
in the interval 6 to 7.
M = 6 + 50:5¡42
17 £ 1 = 6:5
By using mid-points to calculate ¯x, we assume the observations in the interval are uniformly
distributed. Because we have a slight overestimation,it seems that there was a
slight tendency for them to occur nearer the lower limits of intervals. we do not know
what to use for x in the final interval; 13 may be too high but as the frequency is only
1 the effect is small. the median being overestimated indicates that in “‘6 to 7”most of
the observations actually fall in the lower part of the interval.

\documentclass[a4paper,12pt]{article}
%%%%%%%%%%%%%%%%%%%%%%%%%%%%%%%%%%%%%%%%%%%%%%%%%%%%%%%%%%%%%%%%%%%%%%%%%%%%%%%%%%%%%%%%%%%%%%%%%%%%%%%%%%%%%%%%%%%%%%%%%%%%%%%%%%%%%%%%%%%%%%%%%%%%%%%%%%%%%%%%%%%%%%%%%%%%%%%%%%%%%%%%%%%%%%%%%%%%%%%%%%%%%%%%%%%%%%%%%%%%%%%%%%%%%%%%%%%%%%%%%%%%%%%%%%%%
\usepackage{eurosym}
\usepackage{vmargin}
\usepackage{amsmath}
\usepackage{graphics}
\usepackage{epsfig}
\usepackage{enumerate}
\usepackage{multicol}
\usepackage{subfigure}
\usepackage{fancyhdr}
\usepackage{listings}
\usepackage{framed}
\usepackage{graphicx}
\usepackage{amsmath}
\usepackage{chngpage}
%\usepackage{bigints}

\usepackage{vmargin}
% left top textwidth textheight headheight
% headsep footheight footskip
\setmargins{2.0cm}{2.5cm}{16 cm}{22cm}{0.5cm}{0cm}{1cm}{1cm}
\renewcommand{\baselinestretch}{1.3}

\setcounter{MaxMatrixCols}{10}

\begin{document}


\section{Introduction}
\begin{enumerate}
    \item  Type I Error is to reject the Null Hypothesis H0 when it is true. Level of significance,
®, is the probability of making a Type I Error.
\item Type II Error is to ”accept”, not reject, to when it is false. Tower is the probability
of rejecting H0 when it is false. So if P(Type II Error) = ¯; power = (1 ¡ ¯)
\item  H0 is that the number of accidents is uniformly distributed. The sum of the number
of accidents is 65, and so the expected number per day is B. Thus we have the table:
Day M Tu W Th F
OBS 17 10 12 11 15
WXP 13 13 13 13 13
and Â2
(4) =
FP
M
(O¡E)2
E is a test of the NH against an Alt that the numbers vary from
8
day to day. Hence Â2
(4) = 1
13(42 + 32 + 12 + 22 + 22) = 34
13 = 2:615 n.s. There is no
statistical evidence against the Null Hypothesis.
\item  We have the same hypotheses and test, with expected frequencies 130, so Â2
(4) =
1
130(402+302+102+202+202), is 10 times what it was in (i), so Â2
(4) = 26:15 giving
very strong evidence to reject the NH. The increased a mount of data will increase
the power of the test to discriminate between the two hypotheses.
\end{enumerate}
\end{document}

\documentclass[a4paper,12pt]{article}
%%%%%%%%%%%%%%%%%%%%%%%%%%%%%%%%%%%%%%%%%%%%%%%%%%%%%%%%%%%%%%%%%%%%%%%%%%%%%%%%%%%%%%%%%%%%%%%%%%%%%%%%%%%%%%%%%%%%%%%%%%%%%%%%%%%%%%%%%%%%%%%%%%%%%%%%%%%%%%%%%%%%%%%%%%%%%%%%%%%%%%%%%%%%%%%%%%%%%%%%%%%%%%%%%%%%%%%%%%%%%%%%%%%%%%%%%%%%%%%%%%%%%%%%%%%%
\usepackage{eurosym}
\usepackage{vmargin}
\usepackage{amsmath}
\usepackage{graphics}
\usepackage{epsfig}
\usepackage{enumerate}
\usepackage{multicol}
\usepackage{subfigure}
\usepackage{fancyhdr}
\usepackage{listings}
\usepackage{framed}
\usepackage{graphicx}
\usepackage{amsmath}
\usepackage{chngpage}
%\usepackage{bigints}

\usepackage{vmargin}
% left top textwidth textheight headheight
% headsep footheight footskip
\setmargins{2.0cm}{2.5cm}{16 cm}{22cm}{0.5cm}{0cm}{1cm}{1cm}
\renewcommand{\baselinestretch}{1.3}

\setcounter{MaxMatrixCols}{10}

\begin{document}

  \begin{table}[ht!]
     \centering
     \begin{tabular}{|p{15cm}|}
     \hline        
3. (a) Explain the meaning of and the association between the following statistical terms used in hypothesis tests. 
 
  (i) Type I error and level of significance. 
(4) 
  (ii) Type II error and power. 
(4) 
 

\\ \hline
      \end{tabular}
    \end{table}
    
  \begin{table}[ht!]
     \centering
     \begin{tabular}{|p{15cm}|}
     \hline  
 (b) For each of the 65 minor accidents occurring in a particular factory over a one year period in which only full five day weeks were worked, the safety officer noted the day of the week on which it occurred.  The safety officer then compiled the following table. 
 
Day of week Number of accidents Monday 17 Tuesday 10 Wednesday 12 Thursday 11 Friday 15 
 
 
 (i) Investigate the hypothesis that an accident is equally likely to occur on any day of the working week using an appropriate test. (6) 
 
   
 \\ \hline 
      \end{tabular}
    \end{table}
  
% Please add the following required packages to your document preamble:

% \usepackage{graphicx}
\begin{table}[]
\centering
\resizebox{\textwidth}{!}{%
\begin{tabular}{ll}
Day of week  &  Number of accidents \\
Monday &  170 \\
Tuesday &  100 \\
Wednesday &  120 \\
Thursday &  110 \\
Friday &  150\\
\end{tabular}%
}
\end{table}        

\begin{enumerate}
    \item  Type I Error is to reject the Null Hypothesis H0 when it is true. Level of significance,
®, is the probability of making a Type I Error.
\item Type II Error is to ”accept”, not reject, to when it is false. Tower is the probability
of rejecting H0 when it is false. So if P(Type II Error) = ¯; power = (1 ¡ ¯)
\item  H0 is that the number of accidents is uniformly distributed. The sum of the number
of accidents is 65, and so the expected number per day is B. Thus we have the table:
Day M Tu W Th F
OBS 17 10 12 11 15
WXP 13 13 13 13 13
and Â2
(4) =
FP
M
(O¡E)2
E is a test of the NH against an Alt that the numbers vary from
8
day to day. Hence Â2
(4) = 1
13(42 + 32 + 12 + 22 + 22) = 34
13 = 2:615 n.s. There is no
statistical evidence against the Null Hypothesis.
%%%%%%%%%%%%%%%%%%%
\begin{table}[ht!]
     \centering
     \begin{tabular}{|p{15cm}|}
     \hline  
 (ii) In the following year, data were collected from all the factories covered by the manufacturer with the following results. 
 
 
Day of week Number of accidents Monday 170 Tuesday 100 Wednesday 120 Thursday 110 Friday 150 
 
 
Perform a similar test on this second set of data and compare and comment on the results of the two tests. (
\\ \hline
      \end{tabular}
    \end{table}
\item  We have the same hypotheses and test, with expected frequencies 130, so Â2
(4) =
1
130(402+302+102+202+202), is 10 times what it was in (i), so Â2
(4) = 26:15 giving
very strong evidence to reject the NH. The increased a mount of data will increase
the power of the test to discriminate between the two hypotheses.
\end{enumerate}
\end{document}

\documentclass[a4paper,12pt]{article}
%%%%%%%%%%%%%%%%%%%%%%%%%%%%%%%%%%%%%%%%%%%%%%%%%%%%%%%%%%%%%%%%%%%%%%%%%%%%%%%%%%%%%%%%%%%%%%%%%%%%%%%%%%%%%%%%%%%%%%%%%%%%%%%%%%%%%%%%%%%%%%%%%%%%%%%%%%%%%%%%%%%%%%%%%%%%%%%%%%%%%%%%%%%%%%%%%%%%%%%%%%%%%%%%%%%%%%%%%%%%%%%%%%%%%%%%%%%%%%%%%%%%%%%%%%%%
\usepackage{eurosym}
\usepackage{vmargin}
\usepackage{amsmath}
\usepackage{graphics}
\usepackage{epsfig}
\usepackage{enumerate}
\usepackage{multicol}
\usepackage{subfigure}
\usepackage{fancyhdr}
\usepackage{listings}
\usepackage{framed}
\usepackage{graphicx}
\usepackage{amsmath}
\usepackage{chngpage}
%\usepackage{bigints}

\usepackage{vmargin}
% left top textwidth textheight headheight
% headsep footheight footskip
\setmargins{2.0cm}{2.5cm}{16 cm}{22cm}{0.5cm}{0cm}{1cm}{1cm}
\renewcommand{\baselinestretch}{1.3}

\setcounter{MaxMatrixCols}{10}

\begin{document}
  \begin{table}[ht!]
     \centering
     \begin{tabular}{|p{15cm}|}
     \hline        
Question Text 1
\\ \hline
      \end{tabular}
    \end{table}
    
  \begin{table}[ht!]
     \centering
     \begin{tabular}{|p{15cm}|}
     \hline  
Question Text 2   
 \\ \hline 
      \end{tabular}
    \end{table}
  \begin{table}[ht!]
     \centering
     \begin{tabular}{|p{15cm}|}
     \hline  
Question Text 3 
\\ \hline
      \end{tabular}
    \end{table}
        

\section{Introduction}
\begin{enumerate}
    \item If two samples are drawn from Normal distributions N(ui; ¾2
i ) i = 1; 2 and ui; ¾2
i not
known, an estimate of each variance is s2
i =
mPi
j=1
(xij ¡ ¯xi)2=(mi ¡1) where mi are sample
sizes. A test of the N.H, ”¾2
1 = ¾2
2” is given as F(m1¡1;m2¡1) = s21
=s22
, where s21
> s22
to
make use of standard tables. (see the example below)
In the Analysis of Variance (see, e.g., question 5), mean squares computed on the
Null Hypothesis are all estimates of the same residual variation ¾2, and so the ratio
treatments mean square
residual mean sqare will follows an F-distribution with the appropriate degrees of freedom.
\item Supplier 1: m = 16;
P
xi = 30:1;
P
x2
i = 58:85; s2 = 0:14829.
Supplier 2: m = 16;
P
xi = 30:3;
P
x2
i = 57:97; s2 = 0:03929; F(15;15) = 0:14829
0:03929 = 3:77;
leads us to reject a N.H. of equal variances in favor of an A.H. ¾2
1 > ¾2
2. The company is
well advised to use supplier 1.
\end{enumerate}
\end{document}

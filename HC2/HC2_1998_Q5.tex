\documentclass[a4paper,12pt]{article}
%%%%%%%%%%%%%%%%%%%%%%%%%%%%%%%%%%%%%%%%%%%%%%%%%%%%%%%%%%%%%%%%%%%%%%%%%%%%%%%%%%%%%%%%%%%%%%%%%%%%%%%%%%%%%%%%%%%%%%%%%%%%%%%%%%%%%%%%%%%%%%%%%%%%%%%%%%%%%%%%%%%%%%%%%%%%%%%%%%%%%%%%%%%%%%%%%%%%%%%%%%%%%%%%%%%%%%%%%%%%%%%%%%%%%%%%%%%%%%%%%%%%%%%%%%%%
\usepackage{eurosym}
\usepackage{vmargin}
\usepackage{amsmath}
\usepackage{graphics}
\usepackage{epsfig}
\usepackage{enumerate}
\usepackage{multicol}
\usepackage{subfigure}
\usepackage{fancyhdr}
\usepackage{listings}
\usepackage{framed}
\usepackage{graphicx}
\usepackage{amsmath}
\usepackage{chngpage}
%\usepackage{bigints}

\usepackage{vmargin}
% left top textwidth textheight headheight
% headsep footheight footskip
\setmargins{2.0cm}{2.5cm}{16 cm}{22cm}{0.5cm}{0cm}{1cm}{1cm}
\renewcommand{\baselinestretch}{1.3}

\setcounter{MaxMatrixCols}{10}
\begin{document}
\begin{enumerate}
\item If p is the probability of success at any attempt, and the rat does not ’learn’ which routes are
failures, so that each result is independent of others, then the geometric distribution explains the
number of trials needed to gain one success.
\item The value of p must be estimated from the data.
ˆp = 1
¯x , ¯x = [(1 £ 56) + (2 £ 27) + (3 £ 13) + (4 £ 3) + (6 £ 1)]=100 = 1:67.
Hence ˆ ¯ = 0:5988. Calculate P(1) etc. on geometric distribution.
\[P(1)=0.5988 . P(2)=0.5988£0.4012=0.2402 .
P(3)=0:5988 £ (0:4012)2=0.0964 . P(4)=0.0387 etc.\]
x : 1 2 3 ¸ 4 TOTAL
OBS : 56 27 13 4 100
EXP: ON GEOMETRIC : 59:88 24:02 9:64 6:46 100
Combine ”¸ 4” into one class to avoid very small expected frequencies. One parameter was
estimated, so Â2 has 2 d.f. for testing fit to the geometric.
\[Â2
(2) =
(56 ¡ 59:88)2
59:88
+
(27 ¡ 24:02)2
24:02
+
(13 ¡ 9:64)2
9:64
+
(4 ¡ 6:46)2
6:46
= 2:73 n:s:\]
There is no evidence against the hypothesis of fit to a geometric distribution, nor therefore against
the conditions stated in (a).
\end{enumerate}
\end{document}

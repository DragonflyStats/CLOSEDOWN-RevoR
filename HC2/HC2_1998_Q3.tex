\documentclass[a4paper,12pt]{article}
%%%%%%%%%%%%%%%%%%%%%%%%%%%%%%%%%%%%%%%%%%%%%%%%%%%%%%%%%%%%%%%%%%%%%%%%%%%%%%%%%%%%%%%%%%%%%%%%%%%%%%%%%%%%%%%%%%%%%%%%%%%%%%%%%%%%%%%%%%%%%%%%%%%%%%%%%%%%%%%%%%%%%%%%%%%%%%%%%%%%%%%%%%%%%%%%%%%%%%%%%%%%%%%%%%%%%%%%%%%%%%%%%%%%%%%%%%%%%%%%%%%%%%%%%%%%
\usepackage{eurosym}
\usepackage{vmargin}
\usepackage{amsmath}
\usepackage{graphics}
\usepackage{epsfig}
\usepackage{enumerate}
\usepackage{multicol}
\usepackage{subfigure}
\usepackage{fancyhdr}
\usepackage{listings}
\usepackage{framed}
\usepackage{graphicx}
\usepackage{amsmath}
\usepackage{chngpage}
%\usepackage{bigints}

\usepackage{vmargin}
% left top textwidth textheight headheight
% headsep footheight footskip
\setmargins{2.0cm}{2.5cm}{16 cm}{22cm}{0.5cm}{0cm}{1cm}{1cm}
\renewcommand{\baselinestretch}{1.3}

\setcounter{MaxMatrixCols}{10}
\begin{document}
%%%%%%%%%%%%%%%%%%%%%%%%%%%%%%%%%%%%%%%%%%%%%%%%%%%%%%%%%%%%%%%%%%%%%%%%%%%%%%%%%%%%%%%%%%%%%%%%%%%%%%%%%%%%%%%%%%%%%
\begin{table}[ht!]
 
\centering
 
\begin{tabular}{|p{15cm}|}
 
\hline  

3. (a) A manufacturer of industrial light bulbs wishes to control the variability in the length of life of the bulbs so that its standard deviation σ is less than or equal to 150 hours.  A random sample of 10 bulbs was taken and tested in the laboratory giving the following results in hours:
2100,  2302,  1951,  2415,  2067,  1911,  2149,  2489,  2083,  2124
Test the hypothesis H0: σ ≤150 hours against a suitable alternative.  Explain your results and state any assumptions which you made.

\\ \hline
  
\end{tabular}

\end{table}
%%%%%%%%%%%%%%%%%%%%%%%%%%%%%%%%%%%%%%%%%%%%%%%%%%%%%%%%%%%%%%%%%%%%%%%%%%%%%%%%%%%%%%%%%%%%%%%%%%%%%%%%%%%%%%%%%%%%%
\begin{enumerate}
\item If we can assume that the lifetime distribution for the bulbs is normal with variance ¾2,
and all observations are independent of one another, then (n ¡ 1)s2=¾2 will be distributed Â2
n¡1.
Here n=10, and on H0 we take ¾2 = 1502. Then effectively we test H0 : ¾2 · 1502 against
H1 : ¾2 > 1502.
For the data,(n ¡ 1)s2 = 9 £ 35410:99. Hence Â2
(9) = 14:16, which is not significant at the 5%
level. Therefore we do not have enough evidence to reject H0 which says ¾ · 150.
\begin{table}[ht!]
 
\centering
 
\begin{tabular}{|p{15cm}|}
 
\hline  

(b) A chemical manufacturer using two production lines has made slight adjustments to the second in an attempt to reduce the variability in the levels of impurities in the chemical produced.  Twelve randomly selected batches of chemical from each process were analysed and the level of impurities found to be as follows:

\begin{center}
\begin{tabular}{c|c|}
Process 1 & 2.12 2.45 2.43 2.51 2.52 2.44 2.56 2.51 2.41 2.46 2.43 2.38\\
Process 2 & 2.46 2.45 2.46 2.44 2.55 2.56 2.55 2.36 2.50 2.52 2.48 2.42
\end{tabular}
\end{center}
Using an appropriate statistical test, investigate whether the manufacturer has been successful in reducing the process variability for process 2.  
Explain your conclusions  stating any assumptions which you made.

\\ \hline
  
\end{tabular}

\end{table}

%%%%%%%%%%%%%%%%
\item Since twelve randomly selected batches were used from each process we have independent estimates
of variances ¾2
1, ¾2
2. The Null Hypothesis will be ¾2
1 = ¾2
2, and AH ¾2
1 > ¾2
2.
\begin{itemize}
\item From the data, s21
= 0:012536 and s22
= 0:003590.
\item Assuming that the distributions of impurity levels are normal, s21
=s22
is distributed as F(11,11).
s21
s22
= 3:49¤, significant at the 5% level so that H0 is rejected in favor of H1: there is evidence of a
reduction in process variability.
\end{itemize}
\end{enumerate}
\end{document}

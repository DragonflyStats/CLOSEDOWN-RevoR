\documentclass[a4paper,12pt]{article}
%%%%%%%%%%%%%%%%%%%%%%%%%%%%%%%%%%%%%%%%%%%%%%%%%%%%%%%%%%%%%%%%%%%%%%%%%%%%%%%%%%%%%%%%%%%%%%%%%%%%%%%%%%%%%%%%%%%%%%%%%%%%%%%%%%%%%%%%%%%%%%%%%%%%%%%%%%%%%%%%%%%%%%%%%%%%%%%%%%%%%%%%%%%%%%%%%%%%%%%%%%%%%%%%%%%%%%%%%%%%%%%%%%%%%%%%%%%%%%%%%%%%%%%%%%%%
\usepackage{eurosym}
\usepackage{vmargin}
\usepackage{amsmath}
\usepackage{graphics}
\usepackage{epsfig}
\usepackage{enumerate}
\usepackage{multicol}
\usepackage{subfigure}
\usepackage{fancyhdr}
\usepackage{listings}
\usepackage{framed}
\usepackage{graphicx}
\usepackage{amsmath}
\usepackage{chngpage}
%\usepackage{bigints}

\usepackage{vmargin}
% left top textwidth textheight headheight
% headsep footheight footskip
\setmargins{2.0cm}{2.5cm}{16 cm}{22cm}{0.5cm}{0cm}{1cm}{1cm}
\renewcommand{\baselinestretch}{1.3}

\setcounter{MaxMatrixCols}{10}

\begin{document}
  \begin{table}[ht!]
     \centering
     \begin{tabular}{|p{15cm}|}
     \hline        
Question Text 1
\\ \hline
      \end{tabular}
    \end{table}
    
  \begin{table}[ht!]
     \centering
     \begin{tabular}{|p{15cm}|}
     \hline  
Question Text 2   
 \\ \hline 
      \end{tabular}
    \end{table}
  \begin{table}[ht!]
     \centering
     \begin{tabular}{|p{15cm}|}
     \hline  
Question Text 3 
\\ \hline
      \end{tabular}
    \end{table}
        
\section{Introduction}
\begin{enumerate}
    \item (a) Parametric methods require assumptions about the distribution from which a sample is
drawn, usually that it is normal (or approximately so). When samples are small, results
can be seriously wrong if this not true, although for large sample the Central Limit
Theorem makes it satisfactory to examine means or totals from most distributions.
Nonparametric methods require no distributional assumptions, and are generally based
on rankings, medians and combinatorial results which allows tables of critical values to be constructed.
\begin{itemize}
    \item  Skew data can be analyzed without the need for transformations to make
them more symmetrical (approximately normal). 
\item But in general a nonparametric test is
much less powerful than the corresponding parametric one-although an exception is the
Mann Whitely test which is almost as good as the corresponding.
\item In general, data that are (approximately) normal, either in the original units or after a
suitable transformation, are best analyzed by parametric methods. 
\item A dot-plot can be a
useful guide to the shape of a set of data.
\end{itemize}

\item For Stimulus 1, the data appear skew,with a possible upper outlier. For Stimulus 2, they
are also skew, but more spread out,and again with a possible upper outlier.
Because of the skewness, and the possible outliers, parametric joint ranking and stimulus:
0:5 0:5 0:7 0:8 1:0 1:1 1:3 1:3 1:6 1:8 1:8 2:3 2:4 2:6 3:0 3:5 4:0 4:4 5:3
1 1 1 2 1 2 1 2 2 1 1 2 1 2 2 1 2 2 1 2
Number of times a stimulus 1 time is below a stimulus 2 time U = 10+10+8+8+6+5+
5+3+1 = 66 The number of observations are m = n = 10 for the two stimuli. The rank
sum statistic T = U+1
2£10£11 = 66+55 = 121 A normal approximation to this for 10 or
more observations in each set is N(105; 175). Standardizing, Z = 12p1¡105
175
= 16
13:23 = 1:21
n.s., giving no evidence of difference in location of the two sets of times.

\end{enumerate}
\end{document}

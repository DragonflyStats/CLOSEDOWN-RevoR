\documentclass[a4paper,12pt]{article}
%%%%%%%%%%%%%%%%%%%%%%%%%%%%%%%%%%%%%%%%%%%%%%%%%%%%%%%%%%%%%%%%%%%%%%%%%%%%%%%%%%%%%%%%%%%%%%%%%%%%%%%%%%%%%%%%%%%%%%%%%%%%%%%%%%%%%%%%%%%%%%%%%%%%%%%%%%%%%%%%%%%%%%%%%%%%%%%%%%%%%%%%%%%%%%%%%%%%%%%%%%%%%%%%%%%%%%%%%%%%%%%%%%%%%%%%%%%%%%%%%%%%%%%%%%%%
\usepackage{eurosym}
\usepackage{vmargin}
\usepackage{amsmath}
\usepackage{graphics}
\usepackage{epsfig}
\usepackage{enumerate}
\usepackage{multicol}
\usepackage{subfigure}
\usepackage{fancyhdr}
\usepackage{listings}
\usepackage{framed}
\usepackage{graphicx}
\usepackage{amsmath}
\usepackage{chngpage}
%\usepackage{bigints}

\usepackage{vmargin}
% left top textwidth textheight headheight
% headsep footheight footskip
\setmargins{2.0cm}{2.5cm}{16 cm}{22cm}{0.5cm}{0cm}{1cm}{1cm}
\renewcommand{\baselinestretch}{1.3}

\setcounter{MaxMatrixCols}{10}

\begin{document}
  \begin{table}[ht!]
  \centering
  \begin{tabular}{|p{15cm}|}
  \hline  
6. (a) Using one or two illustrative examples in each case, discuss the uses of the following distributions in tests of hypotheses relating to measures of location: 
 
  (i) the Normal distribution, 
(4) 
  (ii) the t-distribution. 
(4) 
 
 \\ \hline
   \end{tabular}
 \end{table}
 
  \begin{table}[ht!]
  \centering
  \begin{tabular}{|p{15cm}|}
  \hline  
 (b) A company wishes to study whether the presence of a supervisor has an effect on the productivity of its work force.  A random sample of 12 workers was selected and their work rate recorded on two separate occasions; once when a supervisor was present and on another occasion when a supervisor was absent.  The order of the two occasions was determined randomly for each worker.  The results obtained were as follows. 
 
Worker Supervisor present 
Supervisor absent 1 23 28 2 35 38 3 29 29 4 32 35 5 43 42 6 32 30 7 30 24 8 29 32 9 33 33 10 34 37 11 43 42 12 30 33 
 
 
Carry out a suitable analysis of these data and write a short report of your findings for a manager. (12) \\ \hline 
   \end{tabular}
 \end{table}
  \begin{table}[ht!]
  \centering
  \begin{tabular}{|p{15cm}|}
  \hline  
Question Text 3 
\\ \hline
   \end{tabular}
 \end{table}
  
\begin{enumerate}
 \item (a) (i) When it is reasonable to assume that a set of data come from a population that can
be modelled normal distribution,a test for the location ”center” of the distribution
can be based on the standard normal distribution. If a sample fxig is small, we
require to know a value for the variance ¾2, Then a test of H0 : ”mean = u”is to
calculate z = (¯x¡u)
¾=
p
n , where n is sample size, as test z as N(0; 1). In large samples,
¾2 need not be know exactly but can be estimated by s2 = 1
n¡1
Pn
i=1(xi ¡ ¯x2).
This is satisfactory for n¿30 in faith symmetrical distributions (even if not exactly
normal) but sometimes a value of ¾2 from earlier similar work and be used, e.g,
in industry for smaller n, Differences between means from N(u1; ¾2
2) and N(u2; ¾2
2)
will be N(u1 ¡ u2; ¾2
1
n1
+ ¾2
2
n2
) and then a hypothesis that they have the same mean
(location ”center”) is tested using Z = (x¯p1¡x¯2)¡(u1¡u2)
±2
1=n1+±2
2=n2
Very large sample of data
from normal approximations such as Binomial (n; p) » N(np; np(1 ¡ p)).
\item For a distribution that is normal with unknown variance but n < 30; ¾2 is estimated
by s2 and t = ¯x¡u
s=
p
n follows the t-distribution with (n-1)degrees of freedom.
Also, provided two populations have the same variance, the different between their
locations can be tested by t = (¯x1¡¯x2)¡(u1¡u2)
s
p
1=n1+1=n2
, where sample sizes are n1; n2; the
N.H. is ”u1 = u2” and s2 = (n1¡1)s21
+(n2¡1)s22
n1+n2¡2 These test meet the same purposes as
those in (i).
\item  These data are paired, involving the same workers, and therefore a test must be based on
the single sample of 12 differences di =(rate with supervisor present)-(rate with supervisor
absent). We may examine the N.H ”mean of D=0”. Values fdig are ¡5; ¡3; 0; ¡3; 1; 2; 6;
¡3; 0; ¡3; 1; ¡3:
P
di = ¡10 ¯ d = ¡0:833:
P
d2
i = 112; s2 = 1
11(112¡(¡10)2
12 ) = 9:4242.
The test is t(11) = p¡0:833¡0
9:4242=12
= 0:94, n.s., so there is no evidence that, on average, the
supervisor’s presence made any difference. But the manager may be interested to note
10
that only one worker(7) showed a substantially higher rate with the supervisor present;
usually there was little difference or a small decrease.
\end{enumerate}
\end{document}

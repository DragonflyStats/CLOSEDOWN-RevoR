\documentclass{article}
\usepackage[utf8]{inputenc}
\usepackage{framed}
\usepackage{enumerate}

\begin{document}

\maketitle

\section{Introduction}


4.(a)Because the measurements are taken on the same volunteers, the paired t-test is appropriate.
Differences (B-A) are: -5, -2, -8, 1, -3, 0, -6, 2, -1, -5, 0, -4.
Assuming that these are normally distributed, the N.H. that the mean difference is 0
uses t(11) = ¯ d¡0
s=
p
12
.
The observed mean difference is ¯ d = ¡31
12 = ¡2:583. s2 = (3:088)2.
Hence t(11) = ¡ 2:583
3:088=3:464 = ¡2:898¤.
Reject the N.H. There is evidence of a change in blood pressure.
The estimated mean increase is 2.583 units. A 95% confidence interval for this is 2:583§2:201£
3:088=3:464 = 2:583 § 1:962 or (0.62 to 4.55) units.
(b)On the Null Hypothesis of no difference in improvement under the two treatments, expected
numbers are calculated:
7
OBS(EXP) Improved Not Improved TOTAL
A 45(54) 55(46) : 100
B 63(54) 37(46) : 100
108 92 200
Â2
(1) =
P (O¡E)2
E = 92( 2
54 + 2
46 ) = 162( 1
54 + 1
46 ) = 6:52¤.
(Â2
(1) = 5:80 if Yates’ correction is used: not essential).
We have evidence to reject the Null Hypothesis at the 5% significance level. This is an indication
of treatment difference.
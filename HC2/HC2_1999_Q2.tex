\documentclass[a4paper,12pt]{article}
%%%%%%%%%%%%%%%%%%%%%%%%%%%%%%%%%%%%%%%%%%%%%%%%%%%%%%%%%%%%%%%%%%%%%%%%%%%%%%%%%%%%%%%%%%%%%%%%%%%%%%%%%%%%%%%%%%%%%%%%%%%%%%%%%%%%%%%%%%%%%%%%%%%%%%%%%%%%%%%%%%%%%%%%%%%%%%%%%%%%%%%%%%%%%%%%%%%%%%%%%%%%%%%%%%%%%%%%%%%%%%%%%%%%%%%%%%%%%%%%%%%%%%%%%%%%
\usepackage{eurosym}
\usepackage{vmargin}
\usepackage{amsmath}
\usepackage{graphics}
\usepackage{epsfig}
\usepackage{enumerate}
\usepackage{multicol}
\usepackage{subfigure}
\usepackage{fancyhdr}
\usepackage{listings}
\usepackage{framed}
\usepackage{graphicx}
\usepackage{amsmath}
\usepackage{chngpage}
%\usepackage{bigints}

\usepackage{vmargin}
% left top textwidth textheight headheight
% headsep footheight footskip
\setmargins{2.0cm}{2.5cm}{16 cm}{22cm}{0.5cm}{0cm}{1cm}{1cm}
\renewcommand{\baselinestretch}{1.3}

\setcounter{MaxMatrixCols}{10}

\begin{document}
\begin{enumerate}
    \item For the combined data(both sexes), minium=2412, maximim=3473; quartiles are 2761.5
and 3201.0, median=2951.5.
\begin{itemize}
    \item The median is below the center of the box, so that the central half of the weights show
some tendency to have more below ”average” than above.
\item The lower whisker is somewhat
longer than the upper one, suggesting that relatively small babies can be distinctly small
while relatively large ones are not quite so far from the others.
\item The second smallest is 2596, 184 above the minimum, whereas the second largest is
3462(and the next 3386), new to the maximum. 
\item So the smallest of all may be an ”outlier”.
Some medical information would be interesting.
\end{itemize}

\item  It seems not unreasonable to assume that we have a sample from an approximately
normal distribution, and or this basis the 95% confidence interval for u is ¯x§t23
p
s2=24,
where ¯x =
P
xi=24 and s2 = 1
23 (
P
x2
i ¡ (
P
x1)2=24)
P
xi = 71227:
Therefore ¯x = 2967:79 and s2 = 83213:2156; so s = 288:467 The (approximate) interval
is 2967:79 § 2:069 £ 288:467=4:899 or 2846 to 3090.
\item Full-term babies:
Males 3279; 3462; 3386; 3199; 3153 3067: n = 6
P
xi = 19546; ¯x = 3257:67; s2 =
21885:5; s = 147:94;
Females 2967; 3203; 3294; 3473; 2962; 2952: n = 6;
P
xi = 18851; ¯x = 3141:83; s2 =
47102:2; s = 217:03.
F(5;5) = 2:15 for the ratio of variances, which is not significant so it is valid to pool them
and use s20
= 68987:6335=2 = 34493:82. so =185.725.
A two-sample t-test can be used: ¯xM¡¯xF
S0
p
1=t+1=t
that is 115.84/107.23=1.08 n.s. as t(10),
giving no evidence against the Null Hypothesis of equal mean weights. Each population
must be assumed normally distributed, with the same variance.
\end{enumerate}
\end{document}
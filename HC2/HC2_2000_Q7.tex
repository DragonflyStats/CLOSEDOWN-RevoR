\documentclass[a4paper,12pt]{article}
%%%%%%%%%%%%%%%%%%%%%%%%%%%%%%%%%%%%%%%%%%%%%%%%%%%%%%%%%%%%%%%%%%%%%%%%%%%%%%%%%%%%%%%%%%%%%%%%%%%%%%%%%%%%%%%%%%%%%%%%%%%%%%%%%%%%%%%%%%%%%%%%%%%%%%%%%%%%%%%%%%%%%%%%%%%%%%%%%%%%%%%%%%%%%%%%%%%%%%%%%%%%%%%%%%%%%%%%%%%%%%%%%%%%%%%%%%%%%%%%%%%%%%%%%%%%
\usepackage{eurosym}
\usepackage{vmargin}
\usepackage{amsmath}
\usepackage{graphics}
\usepackage{epsfig}
\usepackage{enumerate}
\usepackage{multicol}
\usepackage{subfigure}
\usepackage{fancyhdr}
\usepackage{listings}
\usepackage{framed}
\usepackage{graphicx}
\usepackage{amsmath}
\usepackage{chngpage}
%\usepackage{bigints}

\usepackage{vmargin}
% left top textwidth textheight headheight
% headsep footheight footskip
\setmargins{2.0cm}{2.5cm}{16 cm}{22cm}{0.5cm}{0cm}{1cm}{1cm}
\renewcommand{\baselinestretch}{1.3}

\setcounter{MaxMatrixCols}{10}
\begin{document}
\begin{enumerate}
    \item 
The mean of the data is 1
100(0 + 25 + 36 + 36 + 28 + 45 + 30) = 2.00. In a
poisson distribution with mean 2, P(0) = e¡2 = 0:1353,so the expected frequency of
zeros is 13.53.similarly P(1) = 2e¡2 = 0:2707; P(2) = 4e¡2
2 = 0:2707 P(3) = 8e¡2
3 =
0:1804 P(4) = 16e¡2
4 = 0.0902 P(5) = 0.0361; P(¸ 6) = 0.0166: The table of observed
and corresponding expected frequencies is :
count 0 1 2 3 4 ¸ 5 Total
Obs:freq 24 25 18 12 7 14 : 100
Exp:freq 13:53 27.07 27.07 18.04 9.02 5:27 : 100
Comparing these in a Â2 test, there will be 4 degrees o freedom since we are using an
estimate of the mean.
\begin{eqnarray*}
\chi^2{2,4}
&=&
XO ¡ E
E
\\ &=& 
10:472
13:53
+
2.072 + 9.072
27.07
+
6.042
18.04
+
2.022
9.02
+
8:732
5:27
\\ &=& 28:236
\end{eqnarray*}
There is strong evidence against the N.H. of a poisson distribution, which is therefore
rejected.
\item kolmogorov-smirnor uses cumulative probabilities:
0 1 2 3 4 5 6 (1)
OBS 0:24 0:49 0:67 0:79 0:86 0:95 1.00 (¡)
EXP 0:1353 0:4060 0:6767 0:8571 0:9473 0:9843 0:9945 (1.0000)
jO ¡ Ej 0:1047 0.0840 0.0067 0.0671 0.0873 0.0334 0.0046 (¡)
\begin{itemize}
    \item The maximum modulus of difference in cumulative probabilities is 0.1047; for n=100
observations, the upper 5\% tail starts at 1p:36
n i.e. 0.1360, so the observed difference is
not significant and there is no evidence for rejecting the poisson N.H. Chi-squared tests
the pattern of frequencies, which had too heavy a tail that was balanced by too many
zeros for a mean=2; Kolmogrorov-smirnov by using cumulative probabilities is not so
affected by the upper-tail.
\item The gross value added by all manufacturing fell in the period 1991/2/3, so any study
of individual component must allow for this. 
\item Also the data are based on 1995 price, so
change in costs of raw materials, labor and manufacture over the nine years will reflect
this: not all price can be automatically increase to compensate for increase in expenditure.
\item The weightings are 1995,so if there were substantial change over the period.
\end{itemize}
 This
would influence index movements.
One approach is to compare trends, and illustrate these on a graph with time on the
horizontal axis. The extent to which individuals reflected the general trend (bottom
row) should be commented on. with 14 individual rows, there is a lot of information,
and perhaps the main components as given by 1995 weight are enough to illustrate.
An alternative is to recalculate indices relative to the overall index for each year, e.g.
Food 1989:95.6/97.9=97.7(see table below).
1989 1990 1991 1992 1993 1994 1995 1996 1997
Food:etc 97:7 99:5 104:5 106:4 105:2 103.0 100.0 100:6 102.0
Pulp; paper; etc 96:2 98:7 99:1 100:3 102:1 100.0 100.0 97:6 96:9
chemical 85:4 85:5 92:5 95:4 96:1 96:5 100.0 100:3 100:3
Basic Metals; etc 114:5 113:8 108:8 103:4 101.0 98:8 100.0 99:3 99:9
Electrical 81:7 82:7 83:6 85.0 88:4 94:7 100.0 103:6 103:8
Transport 113:7 111:4 109:7 107:5 104:3 102:2 100.0 105:3 108:9
Thus relative to the picture for total manufacturing, both chemicals and electrical have
steadily increased their value added over time,metals have steadily reduced, food rose
until 1993 and has since dropped back, pulp and paper rose and fell again while transport
fell and then rose again.
Transport was never below the level for total.
although most other categories formed small parts of the total (by 1995 weight things),
some points stand out; coke etc. is nearly always well below. Total except for 1995;
textiles show a steady fall,rubber and plastic a steady rise.
Higher value added will often reflect greater efficiency, but will also be affected by the
15
factors mentioned above.
\end{enumerate}
\end{document}

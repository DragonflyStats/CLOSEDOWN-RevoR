\documentclass[a4paper,12pt]{article}
%%%%%%%%%%%%%%%%%%%%%%%%%%%%%%%%%%%%%%%%%%%%%%%%%%%%%%%%%%%%%%%%%%%%%%%%%%%%%%%%%%%%%%%%%%%%%%%%%%%%%%%%%%%%%%%%%%%%%%%%%%%%%%%%%%%%%%%%%%%%%%%%%%%%%%%%%%%%%%%%%%%%%%%%%%%%%%%%%%%%%%%%%%%%%%%%%%%%%%%%%%%%%%%%%%%%%%%%%%%%%%%%%%%%%%%%%%%%%%%%%%%%%%%%%%%%
\usepackage{eurosym}
\usepackage{vmargin}
\usepackage{amsmath}
\usepackage{graphics}
\usepackage{epsfig}
\usepackage{enumerate}
\usepackage{multicol}
\usepackage{subfigure}
\usepackage{fancyhdr}
\usepackage{listings}
\usepackage{framed}
\usepackage{graphicx}
\usepackage{amsmath}
\usepackage{chngpage}
%\usepackage{bigints}

\usepackage{vmargin}
% left top textwidth textheight headheight
% headsep footheight footskip
\setmargins{2.0cm}{2.5cm}{16 cm}{22cm}{0.5cm}{0cm}{1cm}{1cm}
\renewcommand{\baselinestretch}{1.3}

\setcounter{MaxMatrixCols}{10}
\begin{document}
\begin{enumerate}[(a)]
\item N=48 mice, Table of results is:
survival : No Y es
control : 14(10:5) 10(13:5) 24
Drug 7(10:5) 17(13:5) 24
21 27 48
values in brackets are the frequencies expected on the null hypothesis that survival rates
are the same in both groups of mice.
\begin{eqnarray*}x2
(1) &=&
X(O ¡ E)2
E
\\ &=& (3:5)2(
1
10:5
+
1
10:5
+
1
13:5
+
1
13:5
) \\ &=&  4:148
\end{eqnarray*}
with Yates correction, each $(O-E)$ is reduced by 0.5 before squaring, which gives 32( 2
10:5+
2
13:5 ) = 3:408 n.s. as X2
(1):
\begin{itemize}
    \item Yates’ correction can sometimes over-correct for continuity (discreteness). so the ”exact”
probability of obtain the results on the N.H.-which could be found by Fisher’s
Exact Test-is found about the 5\% value(3.84). 
\item this leaves the significance or otherwise
of the result in doubt, but indicates that further data would be needed before a clear
decision could be made. 
\item 48 is a very small number of units upon which to compare two
proportions.
\end{itemize}

For the second trial, N=144, and the results are:
survival : No Y es
control : 42(31:5) 30(40:5) 72
Drug 21(31:5) 51(40:5) 72
63 81 144
x(1)
2 = (10:5)2(
2
31:5
+
2
40:5
) = 12:44
and with Yates’ correction,the value of X2
(1): is 102( 2
31:5 + 2
40:5 ) = 11:29
Both values are significant at the 0.1% lever, leaving little doubt that there is a difference
between Control and Drug.
since we are only asked to test ”the effect” of the drug a 1-tail test may not be valid;
but the data from the second experiment show a firm indication in favor of the drug.
\end{enumerate}
\end{document}

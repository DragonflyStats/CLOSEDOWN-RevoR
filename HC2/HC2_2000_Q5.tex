5(i) The 1-way analysis compares the N.H.“all ¹i are the same ”, where f¹ig denote
the means of the sores using the 4 points i=1,2,3,4; The A.H. is that there are differences
among the means
Paint total are 315,335,350,370,grand total=1370, N=16 observations. Sum of squares
of observations=118290.Total corrected sum of squares = 118290 ¡ 13702
16 = 983:75. s.s.
for paint = 1
4(3152 + 3352 + 3502 + 3702) ¡ 13702
16 = 406:25
Analysis Of Variance
Source of V ariation Degrees of freedom Sum of squre Mean square
Paints 3 406:25 135:42
Residual 12 577:50 48:125
Total 15 983:75
F(3; 12) = 2:81 n:s:
Assuming the scores can be modelled by a normal distribution, the linear model underlining
this analysis is y = ¹i +"ij ; f"ijg all N(0; ¾2). we do not reject the N.H. that all
¹iare the same.
If the columns repeat geographical differences, we should remove these in a 2-way analysis.
The area total are 343,321,336,370 and give a sum of square 1
4(3432 +3212 +3362 +
3702) ¡ 13702=16 = 315:25.
12
ANALYSIS OF VARIANCE
Source DF SS MS
Paint 3 406:25 136:42 F(3; 9) = 4:65
Areas 3 315:25 105:08 F(3; 9) = 3:61n:s:
Residual 9 262:25 29:139 = ¾ˆ2
Total 15 983:75
Making allowance for area differences reduce the residual variation to that which is
purely random natural variation, so make the paints comparison more precise. Now the
N.H.should be rejected. there is evidence of a different among the paints.

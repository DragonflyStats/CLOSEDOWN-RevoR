\documentclass[a4paper,12pt]{article}
%%%%%%%%%%%%%%%%%%%%%%%%%%%%%%%%%%%%%%%%%%%%%%%%%%%%%%%%%%%%%%%%%%%%%%%%%%%%%%%%%%%%%%%%%%%%%%%%%%%%%%%%%%%%%%%%%%%%%%%%%%%%%%%%%%%%%%%%%%%%%%%%%%%%%%%%%%%%%%%%%%%%%%%%%%%%%%%%%%%%%%%%%%%%%%%%%%%%%%%%%%%%%%%%%%%%%%%%%%%%%%%%%%%%%%%%%%%%%%%%%%%%%%%%%%%%
\usepackage{eurosym}
\usepackage{vmargin}
\usepackage{amsmath}
\usepackage{graphics}
\usepackage{epsfig}
\usepackage{enumerate}
\usepackage{multicol}
\usepackage{subfigure}
\usepackage{fancyhdr}
\usepackage{listings}
\usepackage{framed}
\usepackage{graphicx}
\usepackage{amsmath}
\usepackage{chngpage}
%\usepackage{bigints}

\usepackage{vmargin}
% left top textwidth textheight headheight
% headsep footheight footskip
\setmargins{2.0cm}{2.5cm}{16 cm}{22cm}{0.5cm}{0cm}{1cm}{1cm}
\renewcommand{\baselinestretch}{1.3}

\setcounter{MaxMatrixCols}{10}
\begin{document}

%%%%%%%%%%%%%%%%%%%%%%%%%%%%%%%%%%%%%%%%%%%%%%%%%%%%%%%%%%%%%%%%%%%%%%%%%%%%%%%%%%%%%%%%%%%%%%%%%%%%%%%%%%%%%%%%%%%%%
\begin{table}[ht!]
 
\centering
 
\begin{tabular}{|p{15cm}|}
 
\hline  

8. (a) Discuss the advantages and disadvantages of using non-parametric rather than parametric methods in statistical analyses.
(b) The weights of fish in two populations were compared by analysing the differences in the weights of a random sample of fish from each population matched by length.  The weight differences in grams for 10 pairs of fish was as follows:
12,   −13,   −125,   −120,   −73,   2,   3,   −147,   −12,   −4.
Explain why a parametric test would be unsuitable for comparing the weights of the fish from these two populations.
Analyse these data using a test which:
(i) ignores the magnitudes of the differences;
(ii) uses both the signs and magnitudes of the differences;
and comment on the comparison of your results.

\\ \hline
  
\end{tabular}

\end{table}

%%%%%%%%%%%%%%%%%%%%%%%%%%%%%%%%%%%%%%%%%%%%%%%%%%%%%%%%%%%%%%%%%%%%%%%%%%%%%%%%%%%%%%%%%%%%%%%%%%%%%%%%%%%%%%%%%%%%%



\begin{enumerate}
    \item 
8.(a)Parametric methods require a distribution (often the normal) to be specified as a model for
the observations. 
\begin{itemize}
    \item If this is not correct, inferences can be seriously affected.\item Non-parametric methods
rely on such things as rank ordering of data, and require no distributional assumptions. 
\item They
allow more general types of analysis, not based on means and variances(parameters) but are less
powerful than parametric methods when there are available for the corresponding problem.
\end{itemize}
(b)These data are very skew, even after differences have been taken. Even if a logarithmic transformation
is taken, the resulting data cannot be taken as anywhere near symmetrical.
\item A sign test uses only +/-, and there are 3 +’s, 7 -’s in 10 pairs. If the populations are the
same, H0 says that the proportion of + signs is 1
2 , so the number of +’s is Binomial(10,1/2).
P(· 3 +0 s) = P(0) + P(1) + P(2) + P(3) = 1
210 (1 + 10 + 45 + 120) = 0:172.
This is >0.5, so does not provide evidence of any difference.
\item Wilcoxon’s signed ranks test uses magnitudes also.
H0 is “opulations the same”, H1“different”. Ranked absolute values
are 2 3 4 12 12 13 73 120 125 147
rank 1 2 3 41
2 41
2 6 7 8 9 10
sign + + ¡ + ¡ ¡ ¡ ¡ ¡ ¡

\begin{itemize}
    \item The sum of the + ve ranks is 712 and is below the(2-sided) critical table value 8 for n=10,
and so the NH is rejected.
\item  Using this test, there is evidence of different location of population
values.
\item By using the values, more information is available, since the + signs are attached to values
that are numerically quite small. 
\item The Wilcoxon test thus gains greater power to discriminate between
the two sets of data that produced the differences given.
\end{itemize}

\end{enumerate}
\end{enumerate}
\end{document}

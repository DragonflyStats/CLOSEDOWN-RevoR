\documentclass{article}
\usepackage[utf8]{inputenc}
\usepackage{framed}
\usepackage{enumerate}

\begin{document}

\maketitle

\section{Introduction}
%%%%%%%%%%%%%%%%%%%%%%%%%%%%%%%%%%%%%%%%%%%%%%%%%%%%%%%%%%%%%%%%%%%%%%%%%%%%%%%%%%%%%%%%%%%%%%%%%%%%%%%%%%%%%%%%%%%%5
8.(a)Parametric methods require a distribution (often the normal) to be specified as a model for
the observations. If this is not correct, inferences can be seriously affected. Non-parametric methods
rely on such things as rank ordering of data, and require no distributional assumptions. They
allow more general types of analysis, not based on means and variances(parameters) but are less
powerful than parametric methods when there are available for the corresponding problem.
(b)These data are very skew, even after differences have been taken. Even if a logarithmic transformation
is taken, the resulting data cannot be taken as anywhere near symmetrical.
(i)A sign test uses only +/-, and there are 3 +’s, 7 -’s in 10 pairs. If the populations are the
same, H0 says that the proportion of + signs is 1
2 , so the number of +’s is Binomial(10,1/2).
P(· 3 +0 s) = P(0) + P(1) + P(2) + P(3) = 1
210 (1 + 10 + 45 + 120) = 0:172.
This is >0.5, so does not provide evidence of any difference.
(ii)Wilcoxon’s signed ranks test uses magnitudes also.
H0 is “opulations the same”, H1“different”. Ranked absolute values
are 2 3 4 12 12 13 73 120 125 147
rank 1 2 3 41
2 41
2 6 7 8 9 10
sign + + ¡ + ¡ ¡ ¡ ¡ ¡ ¡
The sum of the + ve ranks is 71
2 and is below the(2-sided) critical table value 8 for n=10,
and so the NH is rejected. Using this test, there is evidence of different location of population
values.
By using the values, more information is available, since the + signs are attached to values
that are numerically quite small. The Wilcoxon test thus gains greater power to discriminate between
the two sets of data that produced the differences given.

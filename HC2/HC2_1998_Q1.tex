\documentclass{article}
\usepackage[utf8]{inputenc}

\title{RSS_Jan_2019_HC2}
\author{kobriendublin }
\date{December 2018}

\begin{document}

\maketitle

\section{Introduction}
PAPER II : Statistical Methods
5
1.(a) Yij
"
observation
= ¹
"
general mean
+ ®i
"
effect
due to
treatments
+ ¯ + j
"
effect of
being in
block j
+ ²ij
®i and ¯j are deviations from the general mean, due to which treatment has been given and
which block the unit(plot) is in; these are independent of one another.
f²ijg are mutually independent random residual terms, representing natural variation between
experimental units, each distributed normally with mean C and (constant) variance ¾2.
(b)Location totals: (1)31; (2)47; (3)42; (4)34. G=154. N=12.
Treatment totals: A,47; B,59; C,48.
P
y2=2060.
Total ss=2060-1542/12=83.667 .
Location ss=1
3 (312 + 472 + 422 + 342) ¡ 1542=12 = 53:667.
Treatment ss=1
4 (472 + 592 + 482) ¡ 1542=12 = 22:167.
Analysis of Variance:
SOURCE D:F: SUM OF SQUARES M:S:
Treatments 2 22:167 11:084 F(2; 6) = 8:49¤
Locations 3 53:667 17:889 F(3; 6) = 13:70¤¤
Residuals 6 7:883 1:306
TOTAL 11 83:667
The locations differed significantly, and therefore it was useful to use the randomized block scheme
with locations as blocks. We assume there is no blocks £ treatments interaction.
For treatments, means are: A 11:75
C 12:00
B 14:75
. We are not told which comparisons(contrasts) among
treatments are important, but it is clear that the significance of F(2,6) must be due to the difference
between B and the other two.
2.(i)Histogram of time taken to complete a standard task.
6
Mid-point of
time interval Frequency Cumulative
(mins:)(t) (f) ft ft2 frequency
10:5 3 31:5 330:75 3
11:5 7 80:5 925:75 10
13:5 33 445:5 6014:25 43
16:5 18 297:0 4900:50 61
19:0 9 171:0 3249:00 70
70 1025:5 15420:25
Mean=1025:5
70 = 14:65.
Median=11:95 + 25
33 £ 3 = 14:22.
(assuming records to nearest 0.05min).
Variance=(15420:25 ¡ 1025:52=70) ¥ 69 = 5:7489. SD=2.40 .
With few intervals, the histogram alone is not very informative, but the mean and median are
roughly the same, and near to the middle of the range of the data. Therefore we may treat the
data as approximately normal, and certainly as sufficiently symmetrical to use large-sample tests.
(ii)An approximate 95% confidence interval is 14:65 § 1:96
q
5:7489
70 i.e. 14:65 § 0:56, which is
(14.09 to 15.21).
3.(a)If we can assume that the lifetime distribution for the bulbs is normal with variance ¾2,
and all observations are independent of one another, then (n ¡ 1)s2=¾2 will be distributed Â2
n¡1.
Here n=10, and on H0 we take ¾2 = 1502. Then effectively we test H0 : ¾2 · 1502 against
H1 : ¾2 > 1502.
For the data,(n ¡ 1)s2 = 9 £ 35410:99. Hence Â2
(9) = 14:16, which is not significant at the 5%
level. Therefore we do not have enough evidence to reject H0 which says ¾ · 150.
(b)Since twelve randomly selected batches were used from each process we have independent estimates
of variances ¾2
1, ¾2
2. The Null Hypothesis will be ¾2
1 = ¾2
2, and AH ¾2
1 > ¾2
2.
From the data, s21
= 0:012536 and s22
= 0:003590.
Assuming that the distributions of impurity levels are normal, s21
=s22
is distributed as F(11,11).
s21
s22
= 3:49¤, significant at the 5% level so that H0 is rejected in favor of H1: there is evidence of a
reduction in process variability.
4.(a)Because the measurements are taken on the same volunteers, the paired t-test is appropriate.
Differences (B-A) are: -5, -2, -8, 1, -3, 0, -6, 2, -1, -5, 0, -4.
Assuming that these are normally distributed, the N.H. that the mean difference is 0
uses t(11) = ¯ d¡0
s=
p
12
.
The observed mean difference is ¯ d = ¡31
12 = ¡2:583. s2 = (3:088)2.
Hence t(11) = ¡ 2:583
3:088=3:464 = ¡2:898¤.
Reject the N.H. There is evidence of a change in blood pressure.
The estimated mean increase is 2.583 units. A 95% confidence interval for this is 2:583§2:201£
3:088=3:464 = 2:583 § 1:962 or (0.62 to 4.55) units.
(b)On the Null Hypothesis of no difference in improvement under the two treatments, expected
numbers are calculated:
7
OBS(EXP) Improved Not Improved TOTAL
A 45(54) 55(46) : 100
B 63(54) 37(46) : 100
108 92 200
Â2
(1) =
P (O¡E)2
E = 92( 2
54 + 2
46 ) = 162( 1
54 + 1
46 ) = 6:52¤.
(Â2
(1) = 5:80 if Yates’ correction is used: not essential).
We have evidence to reject the Null Hypothesis at the 5% significance level. This is an indication
of treatment difference.
5.(a)If p is the probability of success at any attempt, and the rat does not ’learn’ which routes are
failures, so that each result is independent of others, then the geometric distribution explains the
number of trials needed to gain one success.
(b)The value of p must be estimated from the data.
ˆp = 1
¯x , ¯x = [(1 £ 56) + (2 £ 27) + (3 £ 13) + (4 £ 3) + (6 £ 1)]=100 = 1:67.
Hence ˆ ¯ = 0:5988. Calculate P(1) etc. on geometric distribution.
P(1)=0.5988 . P(2)=0.5988£0.4012=0.2402 .
P(3)=0:5988 £ (0:4012)2=0.0964 . P(4)=0.0387 etc.
x : 1 2 3 ¸ 4 TOTAL
OBS : 56 27 13 4 100
EXP: ON GEOMETRIC : 59:88 24:02 9:64 6:46 100
Combine ”¸ 4” into one class to avoid very small expected frequencies. One parameter was
estimated, so Â2 has 2 d.f. for testing fit to the geometric.
Â2
(2) =
(56 ¡ 59:88)2
59:88
+
(27 ¡ 24:02)2
24:02
+
(13 ¡ 9:64)2
9:64
+
(4 ¡ 6:46)2
6:46
= 2:73 n:s:
There is no evidence against the hypothesis of fit to a geometric distribution, nor therefore against
the conditions stated in (a).
6.(a)When a large sample of data is available from any population, not necessarily normal, and
including discrete data as well as continuous, the sample mean or total follows a distribution that
is approximately normal. Therefore with large samples of data significance tests of, and confidence
intervals for, a population mean may be found, at least approximately, without knowing what
distribution the population has. This extends, for example to proportions in a binomial.
In practice, when distributions are reasonably symmetrical, even when not normal, samples
may be as small as about 30, while if data are very skew then very large samples-several hundredmay
be required to give acceptable results. An examination of data, possibly by graphical methods,
is a useful guide when applying the approximation.
We may treat ¯x as N(¹; ¾2=n) when n is sample size and ¹; ¾2 are the (known or estimated)
mean and variance of the population distribution. The only theoretical restriction is that ¹ and
¾2 must be finite.
Many estimates of practically important items are the sum of several independent components,
e.g. crop yields of many plants forming a plot, and so their total tends to be normally distributed,
N(n¹; n¾2).
(b)(i)For difference in means, a 95% confidence interval is approximately
(x¯1 ¡ x¯2) § 1:96
p
s21
=n1 + s22
=n2.
For the given data, this is (3:75 ¡ 2:10) § 1:96
q
2:742
125 + 1:402
108 , i.e. 1:65 § 1:96 £ 0:2797
8
giving 1:65 § 0:55 or (1.10 to 2.20).
Since this interval does not contain zero, it is likely that there will be more calls each shift in
district 1 than in 2, the mean difference being between 1.1 and 2.2(with probability 0.95).
(ii)For difference in proportions,q pˆ1¡pˆ2
p1(1¡p1)
n1
+p2(1¡p2)
n2
» N(0; 1).
The NH is ”p1 = p2”.
Hence 26
125 ¡ 15
108 = 0:208 ¡ 0:139 = 0:069 is the estimated difference.
Its variance is 0:208£0:792
125 + 0:139£0:861
108 = 2:426 £ 10¡3; S.E.=0.049.
Hence 0:069
0:049 = 1:40 n.s. as N(0,1)and there is no significant evidence of a difference in proportions.
7.Given ¹ = 1:81; ¾2 = (0:025)2. n=10.
(i)For A, ¯x=1.80 and s2=0.001977.
For B, ¯x=1.85 and s2=0.000689.
(n¡1)s2
A
¾2 = 9£0:001977
0:0252 = 28:47¤¤¤ » Â2
(9), giving very strong evidence to reject an NH that A’s
variability is the same as the laboratory standard, and to accept an AH that it is greater.
(n¡1)s2
B
¾2 = 9£0:000689
0:0252 = 9:92; n:s: as Â2
(9), so there is no statistical evidence that B’s variability
is unacceptable.
(ii)For A,p x¯¡1:81
0:001977=10
= ¡0:01
0:014 = ¡0:71 n:s: as t(9).
No evidence that A’s results are biased.
For B,p x¯¡1:81
0:000689=10
= 0:04
0:0083 = 4:82¤¤¤ as t(9).
B’s results do seem to be biased, because this value of t(9) leads us to reject the N.H. “mean=1.81”
Hence worker A produces results which are unbiased but very variable, while B is biased but precise.

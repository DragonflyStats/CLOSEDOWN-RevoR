Estimating Decision Forest
Models
The rxDForest function in RevoScaleR fits a decision forest, which is an ensemble of decision
trees. Each tree is fitted to a bootstrap sample of the original data, which leaves about 1/3 of
the data unused in the fitting of each tree. Each data point in the original data is fed through
each of the trees for which it was unused; the decision forest prediction for that data point is
the statistical mode of the individual tree predictions, that is, the majority prediction (for
classification; for regression problems, the prediction is the mean of the individual predictions).
Unlike individual decision trees, decision forests are not prone to overfitting, and they are
consistently shown to be among the best machine learning algorithms. RevoScaleR implements
decision forests in the rxDForest function, which uses the same basic tree-fitting algorithm as
rxDTree (see Section 10.1, The rxDTree Algorithm). To create the forest, you specify the
number of trees using the nTree argument and the number of variables to consider for splitting
in each tree using the mTry argument. In most cases, you will also want to specify the maximum
depth to grow the individual trees: greater depth typically results in greater accuracy, but as
with rxDTree, also results in significantly longer fitting times. 

Chapter 9.
Fitting Generalized Linear
Models
Generalized linear models (GLM) are a framework for a wide range of analyses. They relax the
assumptions for a standard linear model in two ways. First, a functional form can be specified
for the conditional mean of the predictor. This is referred to as the “link” function. Second,
you can specify a distribution for the response variable. The rxGlm function in RevoScaleR
provides the ability to estimate generalized linear models on large data sets.
The following family/link combinations are implemented in C++ for performance
enhancements: binomial/logit, gamma/log, poisson/log, and Tweedie. Other family/link
combinations use a combination of C++ and R code. Any valid R family object that can be used
with glm() can be used with \texttt{rxGlm()}, including those that are user-defined. The following table
shows all of the supported family/link combinations (in addition to user-define
%=================================================================%

9.1 A Simple Example Using the Poisson Family
The Poisson family is used to estimate models of count data. Examples from the literature
include the following types of response variables:

\begin{itemize}
\item Number of drinks on a Saturday night
\item Number of bacterial colonies in a Petri dish
\item Number of children born to married women
\item Number of credit cards a person has
\end{itemize}
%==================================================================================================================%
We’ll start with a simple example from Kabacoff’s R in Action book (2011), using data provided
with the robust R package. The data are from a placebo-controlled clinical trial of 59 epileptics.
Patients with partial seizures were enrolled in a randomized clinical trial of the anti-epileptic
drug, progabide. Counts of epileptic seizures were recorded during the trial. The data set also
includes a baseline 8-week seizure count and the age of the patient.
To access this data, first make sure the robust package is installed, then use the data command
to load the the data frame:
data(breslow.dat, package = "robust")
First, let’s get some basic information on the data set, then draw a histogram of the sumY
variable, containing the total count of seizures during the trial.

9.3 An Example Using the Tweedie Family
The Tweedie family of distributions provide flexible models for estimation. The power
parameter var.power determines the shape of the distribution, with familiar models as special
cases: if var.power is set to 0, Tweedie is a normal distribution; when set to 1, it is Poisson;
when 2, it is Gamma; when 3, it is inverse Gaussian. If var.power is between 1 and 2, it is a
compound Poisson distribution and is appropriate for positive data that also contains exact
zeros, for example, insurance claims data, rainfall data, or fish-catch data. If var.power is
greater than 2, it is appropriate for positive data.
In this example, we’ll use a subsample from the 5% sample of the U.S. 2000 census. We will
consider the annual cost of property insurance for heads of household ages 21 through 89, and 
Fitting Generalized Linear Models 115
its relationship to age, sex, and region. A variable “perwt” in the data set represents the
probability weight for that observation. First, to create the subsample (specify the correct data
path for your downloaded data):

%==================================================================================================================%
\begin{framed}
\begin{verbatim}
bigDataDir = "C:/Revolution/Data"
bigCensusData <- file.path(bigDataDir, "Census5PCT2000.xdf")
propinFile <- "CensusPropertyIns.xdf"
propinDS <- rxDataStep(inData = bigCensusData, outFile = propinFile,
 rowSelection = (related == 'Head/Householder') & (age > 20) & (age < 90),
 varsToKeep = c("propinsr", "age", "sex", "region", "perwt"),
 blocksPerRead = 10, overwrite = TRUE)
rxGetInfo(propinDS)
File name: C:\YourWorkingDir\CensusPropertyIns.xdf
Number of observations: 5175270
Number of variables: 5
Number of blocks: 10
Compression type: zlib
\end{verbatim}
\end{framed}
%---------------------------------------------%
An \texttt{Xdf} data source representing the new data file is returned. The new data file has over 5
million observations.
Let’s do one more step in data cleaning. The variable region has some very long factor level
character strings, and it also has a number of levels for which there are no observations. We
can see this using rxSummary:
\begin{framed}
\begin{verbatim}
rxSummary(~region, data = propinDS)
Call:
rxSummary(formula = ~region, data = propinDS)
Summary Statistics Results for: ~region
File name: C:\YourWorkingDir\CensusPropertyIns.xdf
Number of valid observations: 5175270
Category Counts for region
Number of categories: 17
Number of valid observations: 5175270
Number of missing observations: 0
\end{verbatim}
\end{framed}
region Counts
New England Division 265372
Middle Atlantic Division 734585
Mixed Northeast Divisions (1970 Metro) 0
East North Central Div. 847367
West North Central Div. 366417
Mixed Midwest Divisions (1970 Metro) 0
South Atlantic Division 981614
East South Central Div. 324003
West South Central Div. 553425
Mixed Southern Divisions (1970 Metro) 0
Mountain Division 328940
Pacific Division 773547
Mixed Western Divisions (1970 Metro) 0
Military/Military reservations 0
PUMA boundaries cross state lines-1% sample 0



%==================================================================================================================%

Chapter 13.
Clustering
Clustering is the general name for any of a large number of classification techniques that
involve assigning observations to membership in one of two or more clusters on the basis of
some distance metric.
%-------------------------------------------------------------------------------%
\subsection*{13.1 K-means Clustering}
K-means clustering is a classification technique that groups observations of numeric data using
one of several iterative relocation algorithms—that is, starting from some initial classification,
which may be random, points are moved from cluster to another so as to minimize sums of
squares. In RevoScaleR, the algorithm used is that of Lloyd (1982).
To perform k-means clustering with RevoScaleR, use the rxKmeans function.
13.1.1 Clustering the Airline Data
As a first example of k-means clustering, we will cluster the arrival delay and scheduled
departure time in the airline data 7% subsample. To start, we extract variables of interest into a
new working data set to which we’ll be writing additional information :
\begin{framed}
\begin{verbatim}
bigDataDir <- "C:/Revolution/Data"
sampleAirData <- file.path(bigDataDir, "AirOnTime7Pct.xdf")
rxDataStep(inData = sampleAirData, outFile = "AirlineDataClusterVars.xdf",
varsToKeep=c("DayOfWeek", "ArrDelay", "CRSDepTime", "DepDelay"))

\end{verbatim}
\end{framed}


%=============================================================%
Using the Cluster Membership Information
A common follow-up to clustering is to use the cluster membership information to see whether
a given model varies appreciably from cluster to cluster. Since we can use the rowSelection
argument to extract a single cluster on the fly, there is no need to sort the data first. As an
example, we fit our original linear model of ArrDelay by DayOfWeek for two of the clusters:
\begin{framed}
\begin{verbatim}
clust1Lm <- rxLinMod(ArrDelay ~ DayOfWeek, "AirlineDataClusterVars.xdf",
rowSelection = .rxCluste r == 1 )
clust5Lm <- rxLinMod(ArrDelay ~ DayOfWeek, "AirlineDataClusterVars.xdf",
rowSelection = .rxCluster == 5)
summary(clust1Lm)
summary(clust5Lm)
\end{verbatim}
\end{framed}

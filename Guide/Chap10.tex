Chapter 10.
Estimating Decision Tree
Models
The rxDTree function in RevoScaleR fits tree-based models using a binning-based recursive
partitioning algorithm. The resulting model is similar to that produced by the recommended R
package rpart. Both classification-type trees and regression-type trees are supported; as with
rpart, the difference is determined by the nature of the response variable: a factor response
generates a classification tree; a numeric response generates a regression tree.
10.1 The rxDTree Algorithm
Decision trees are effective algorithms widely used for classification and regression. Building a
decision tree generally requires that all continuous variables be sorted in order to decide where
to split the data. This sorting step becomes time and memory prohibitive when dealing with
large data. Various techniques have been proposed to overcome the sorting obstacle, which
can be roughly classified into two groups: performing data pre-sorting or using approximate
summary statistic of the data. While pre-sorting techniques follow standard decision tree
algorithms more closely, they cannot accommodate very large data sets. These big data
decision trees are normally parallelized in various ways to enable large scale learning: data
parallelism partitions the data either horizontally or vertically so that different processors see
different observations or variables and task parallelism builds different tree nodes on different
processors.

Chapter 8.
Fitting Logistic Regression
Models
Logistic regression is a standard tool for modeling data with a binary response variable. In R,
you fit a logistic regression using the glm function, specifying a binomial family and the logit link
function. In RevoScaleR, you can use rxGlm in the same way (see Chapter 9, Fitting Generalized
Linear Models) or you can fit a logistic regression using the optimized rxLogit function;
because this function is specific to logistic regression, you need not specify a family or link
function.
As an example, consider the kyphosis data set in the rpart package. This data set consists of
81 observations of four variables (Age, Number, Kyphosis, Start) in children following corrective
spinal surgery; it is used as the initial example of glm in the White Book(Chambers & Hastie,
1992). The variable Kyphosis reports the absence or presence of this deformity.
We can use rxLogit to model the probability that kyphosis is present as follows:
library(rpart)
rxLogit(Kyphosis ~ Age + Start + Number, data = kyphosis)
The following output is returned:
Logistic Regression Results for: Kyphosis ~ Age + Start + Number
Data: kyphosis
Dependent variable(s): Kyphosis
Total independent variables: 4 
Fitting Logistic Regression Models 105
Number of valid observ
Number of valid observations: 81
Number of missing observations: 0
Coefficients:
 Kyphosis
(Intercept) -2.03693354
Age 0.01093048
Start -0.20651005
Number 0.41060119
The same model can be fit with glm (or rxGlm) as follows:
glm(Kyphosis ~ Age + Start + Number, family = binomial, data = kyphosis)
Call: glm(formula = Kyphosis ~ Age + Start + Number, family = binomial,
data = kyphosis)
Coefficients:
(Intercept) Age Start Number
 -2.03693 0.01093 -0.20651 0.41060
Degrees of Freedom: 80 Total (i.e. Null); 77 Residual
Null Deviance: 83.23
Residual Deviance: 61.38 AIC: 69.38
8.1 Stepwise Logistic Regression
Stepwise logistic regression is an algorithm that helps you determine which variables are most
important to a logistic model. You provide a minimal, or lower, model formula and a maximal,
or upper, model formula, and using forward selection, backward elimination, or bidirectional
search, the algorithm determines the model formula that provides the best fit based on an AIC
or significance level selection criterion.
RevoScaleR provides an implementation of stepwise logistic regression that is not constrained
by the use of "in-memory" algorithms. Stepwise linear regression in RevoScaleR is implemented
by the functions rxLogit and rxStepControl.
Stepwise logistic regression begins with an initial model of some sort. We can look at the
kyphosis data again and start with a simpler model: Kyphosis ~ Age:
initModel <- rxLogit(Kyphosis ~ Age, data=kyphosis)
initModel
Logistic Regression Results for: Kyphosis ~ Age
Data: kyphosis
Dependent variable(s): Kyphosis
Total independent variables: 2
Number of valid observations: 81
Number of missing observations: 0
Coefficients:
 Kyphosis
(Intercept) -1.809351230
Age 0.005441758
106 Prediction
We can specify a stepwise model using rxLogit and rxStepControl as follows:
KyphStepModel <- rxLogit(Kyphosis ~ Age,
data = kyphosis,
variableSelection = rxStepControl(method="stepwise",
scope = ~ Age + Start + Number ))
KyphStepModel
Logistic Regression Results for: Kyphosis ~ Age + Start + Number
Data: kyphosis
Dependent variable(s): Kyphosis
Total independent variables: 4
Number of valid observations: 81
Number of missing observations: 0
Coefficients:
 Kyphosis
(Intercept) -2.03693354
Age 0.01093048
Start -0.20651005
Number 0.41060119
The methods for variable selection (forward, backward, and stepwise), the definition of model
scope, and the available selection criteria are all the same as for stepwise linear regression; see
section 7.8 and the rxStepControl help file for more details.
8.2 Prediction
As described above for linear models, the objects returned by the RevoScaleR model-fitting
functions do not include fitted values or residuals. We can obtain them, however, by calling
rxPredict on our fitted model object, supplying the original data used to fit the model as the
data to be used for prediction.
For example, consider the mortgage default example in Section 6 of the manual RevoScaleR:
Getting Started Guide. For that example, we used ten input data files to create the data set
used to fit the model. But suppose instead we use nine input data files to create the training
data set and use the remaining data set for prediction. We can do that as follows (again,
remember to modify the first line for your own system):
bigDataDir <- "C:/Revolution/Data"
mortCsvDataName <- file.path(bigDataDir, "mortDefault", "mortDefault")
trainingDataFileName <- "mortDefaultTraining"
mortCsv2009 <- paste(mortCsvDataName, "2009.csv", sep = "")
targetDataFileName <- "mortDefault2009.xdf"
ageLevels <- as.character(c(0:40))
yearLevels <- as.character(c(2000:2009))
colInfo <- list(list(name = "houseAge", type = "factor",
 levels = ageLevels), list(name = "year", type = "factor",
 levels = yearLevels))
append= FALSE
for (i in 2000:2008)
{
 importFile <- paste(mortCsvDataName, i, ".csv", sep = "")
Fitting Logistic Regression Models 107
 rxImport(inData = importFile, outFile = trainingDataFileName,
colInfo = colInfo, append = append)
 append = 
 
 8.3 Prediction Standard Errors and Confidence Intervals
You can use rxPredict to obtain prediction standard errors and confidence intervals for models
fit with rxLogit in the same way as for those fit with rxLinMod. The original model must be fit
with covCoef=TRUE:
logitObj2 <- rxLogit(default ~ year + creditScore + yearsEmploy + ccDebt,
data = trainingDataFileName, blocksPerRead = 2, verbose = 1,
reportProgress=2, covCoef=TRUE)
You then specify computeStdErr=TRUE to obtain prediction standard errors; if this is TRUE, you
can also specify interval="confidence" to obtain a confidence interval:
rxPredict(logitObj2, data = targetDataFileName,
outData = targetDataFileName, computeStdErr = TRUE,
interval = "confidence", overwrite=TRUE)
The first ten lines of the file with predictions can be viewed as follows:
rxGetInfo(targetDataFileName, numRows=10)
File name: C:\Users\yourname\Documents\Revolution\mortDefault2009.xdf
Number of observations: 1e+06
Number of variables: 10
Number of blocks: 2
Compression type: zlib
Data (10 rows starting with row 1):
 creditScore houseAge yearsEmploy ccDebt year default default_Pred
1 617 20 8 4410 2009 0 6.620773e-06
2 623 11 7 5609 2009 0 4.610861e-05
3 758 17 4 7250 2009 0 4.259884e-04
4 687 22 5 3761 2009 0 3.770789e-06
5 663 15 6 6746 2009 0 2.312827e-04
6 676 10 2 7106 2009 0 1.092593e-03
108 Prediction Standard Errors and Confidence Intervals
7 721 23 2 2280 2009 0 8.515912e-07
8 680 18 7 2831 2009 0 6.011109e-07
9 734 9 5 3867 2009 0 3.144299e-06
10 688 16 8 6238 2009 0 5.350031e-05
 default_StdErr default_Lower default_Upper
1 3.143695e-07 6.032422e-06 7.266507e-06
2 1.953612e-06 4.243427e-05 5.010109e-05
3 1.594783e-05 3.958500e-04 4.584203e-04
4 1.739047e-07 3.444893e-06 4.127516e-06
5 8.733193e-06 2.147838e-04 2.490486e-04
6 3.952975e-05 1.017797e-03 1.172880e-03
7 4.396314e-08 7.696409e-07 9.422675e-07
8 3.091885e-08 5.434655e-07 6.648706e-07
9 1.469334e-07 2.869109e-06 3.445883e-06
10 2.224102e-06 4.931401e-05 5.804197e-05

\end{document}

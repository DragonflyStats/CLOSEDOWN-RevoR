\documentclass[a4paper,12pt]{article}
%%%%%%%%%%%%%%%%%%%%%%%%%%%%%%%%%%%%%%%%%%%%%%%%%%%%%%%%%%%%%%%%%%%%%%%%%%%%%%%%%%%%%%%%%%%%%%%%%%%%%%%%%%%%%%%%%%%%%%%%%%%%%%%%%%%%%%%%%%%%%%%%%%%%%%%%%%%%%%%%%%%%%%%%%%%%%%%%%%%%%%%%%%%%%%%%%%%%%%%%%%%%%%%%%%%%%%%%%%%%%%%%%%%%%%%%%%%%%%%%%%%%%%%%%%%%
  \usepackage{eurosym}
\usepackage{vmargin}
\usepackage{amsmath}
\usepackage{graphics}
\usepackage{epsfig}
\usepackage{enumerate}
\usepackage{multicol}
\usepackage{subfigure}
\usepackage{fancyhdr}
\usepackage{listings}
\usepackage{framed}
\usepackage{graphicx}
\usepackage{amsmath}
\usepackage{chngpage}
%\usepackage{bigints}

\usepackage{vmargin}
% left top textwidth textheight headheight
% headsep footheight footskip
\setmargins{2.0cm}{2.5cm}{16 cm}{22cm}{0.5cm}{0cm}{1cm}{1cm}
\renewcommand{\baselinestretch}{1.3}

\setcounter{MaxMatrixCols}{10}

%%%%%%%%%%%%%%%%%%%%%%%%%%%%%%%%%%
\begin{framed}
5. The amount of oil, in suitable units, recoverable from a test well has a distribution
with probability density function given by
1
2
f x x ( ) for 2
x




 
and is zero otherwise, where  (> 0) is an unknown parameter. The amounts of oil
recoverable from a random sample of tests are
1 2 , , , X X Xn
. It is required to test
the null hypothesis  = 2 against the alternative hypothesis   2.

\end{framed}
%%%%%%%%%%%%%%%%%%%%%%%%%%%%%%%%%%%%%%%%%%%%%%%%%%%%%%%%%%%%%%%%%%%%%%%%%%%%%%%%%%%%%%%%%%%%%%%%
\begin{enumerate}[(a)]

\item Show that the maximum likelihood estimator
ˆ

of  satisfies the equation
log log2  
ˆ
i
n
X n

   .
%%%%%%%%%%%%%%%%%%%%%%%%%%%%%%%%%%%%%%%%%%%%%%%%%%%%%%%%%%%%%%%%%%%%%%%%%5
\item  Hence show that
 
ˆ
log 0.5 i
n
X
 

.
%%%%%%%%%%%%%%%%%%%%%%%%%%%%%%%%%%%%%%%%%%%%%%%%%%%%%%%%%%%%%%%%%%%%%%%%%5
\item Using the result in part (i), show that the generalised likelihood ratio test
statistic  satisfies
 
2 2 log log ?
n
 n n
 
   .
%%%%%%%%%%%%%%%%%%%%%%%%%%%%%%%%%%%%%%%%%%%%%%%%%%%%%%%%%%%%%%%%%%%%%%%%%%
\item Suppose now that n is large and that the size of the test is to be 0.05. Show that
the acceptance region is
2 1.92 ˆ
log 1 log 2 ˆ
n


    .
(5)
\end{enumerate}
\end{document}

\documentclass[a4paper,12pt]{article}
%%%%%%%%%%%%%%%%%%%%%%%%%%%%%%%%%%%%%%%%%%%%%%%%%%%%%%%%%%%%%%%%%%%%%%%%%%%%%%%%%%%%%%%%%%%%%%%%%%%%%%%%%%%%%%%%%%%%%%%%%%%%%%%%%%%%%%%%%%%%%%%%%%%%%%%%%%%%%%%%%%%%%%%%%%%%%%%%%%%%%%%%%%%%%%%%%%%%%%%%%%%%%%%%%%%%%%%%%%%%%%%%%%%%%%%%%%%%%%%%%%%%%%%%%%%%
  \usepackage{eurosym}
\usepackage{vmargin}
\usepackage{amsmath}
\usepackage{graphics}
\usepackage{epsfig}
\usepackage{enumerate}
\usepackage{multicol}
\usepackage{subfigure}
\usepackage{fancyhdr}
\usepackage{listings}
\usepackage{framed}
\usepackage{graphicx}
\usepackage{amsmath}
\usepackage{chngpage}
%\usepackage{bigints}

\usepackage{vmargin}
% left top textwidth textheight headheight
% headsep footheight footskip
\setmargins{2.0cm}{2.5cm}{16 cm}{22cm}{0.5cm}{0cm}{1cm}{1cm}
\renewcommand{\baselinestretch}{1.3}

\setcounter{MaxMatrixCols}{10}

\section{Monte Carlo Simulation}
\begin{enumerate}

%%%%%%%%%%%%%%%%%%%%%%%%%%%%%%%%%%%%%%%%%%%%%%%%%%%%%%%%%%%%%%%%%%%%%%%%%%%%%%%%%%%%%%%%%%%%%%%%%%%%%%%%%%
\item a) Describe how computer Monte Carlo simulation can be used to
(i) compare estimators,
(4)


\item draw inferences in Bayesian analysis.
(4)
(b) Each item on a production line is given a quick test which has two possible
results: s1
(appears satisfactory) and s2
(appears unsatisfactory). However, the
test is itself prone to error so that if the item is satisfactory, P(s1
) = 0.9 and
P(s
2
) = 0.1, while if the item is unsatisfactory, P(s
1
) = 0.4 and P(s
2
) = 0.6.
After each item is inspected, it is either sold or scrapped. If a satisfactory item
is sold, there is a net loss of –2 units (i.e. a profit of 2 units), while if an
unsatisfactory item is sold there is a penalty resulting in a net loss of 10 units.
Any item that is scrapped results in a net loss of 1 unit.
\item List the four decision rules for deciding whether each item should be
scrapped.
(2)
(ii) Calculate the risk table.
(8)
\item  State, with reasons, which is the minimax rule.
(2)
\end{enumerate}
\end{document}

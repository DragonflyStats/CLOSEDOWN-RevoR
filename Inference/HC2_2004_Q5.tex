\documentclass[a4paper,12pt]{article}

%%%%%%%%%%%%%%%%%%%%%%%%%%%%%%%%%%%%%%%%%%%%%%%%%%%%%%%%%%%%%%%%%%%%%%%%%%%%%%%%%%%%%%%%%%%%%%%%%%%%%%%%%%%%%%%%%%%%%%%%%%%%%%%%%%%%%%%%%%%%%%%%%%%

\usepackage{eurosym}
\usepackage{vmargin}
\usepackage{amsmath}
\usepackage{graphics}
\usepackage{epsfig}
\usepackage{enumerate}
\usepackage{multicol}
\usepackage{subfigure}
\usepackage{fancyhdr}
\usepackage{listings}
\usepackage{framed}
\usepackage{graphicx}
\usepackage{amsmath}
\usepackage{chngpage}

%\usepackage{bigints}



\usepackage{vmargin}

% left top textwidth textheight headheight

% headsep footheight footskip

\setmargins{2.0cm}{2.5cm}{16 cm}{22cm}{0.5cm}{0cm}{1cm}{1cm}

\renewcommand{\baselinestretch}{1.3}

\setcounter{MaxMatrixCols}{10}

\begin{document}

%% -- Higher Certificate, Paper II, 2004.  Question 5 

%%%%%%%%%%%%%%%%%%%%%%%%%%%%%%%%%%%%%%%%%%%%%%%%%%%%%%%%%%%%%%%%%%%%%%% 

\begin{table}[ht!]
 
\centering
 
\begin{tabular}{|p{15cm}|}
 
\hline  

\large
\noindent Explain the meaning of the following terms used in hypothesis tests. 
 
 (a) Type I error. 

 
 (b) Type II error. 

 
 (c) Level of significance. 

 
 (d) Power. 

 

\\ \hline
  
\end{tabular}

\end{table} 

\large

%%-- Part (i) 
\begin{itemize}
    \item[(a)] A type I error is to reject the null hypothesis, in favour of the alternative hypothesis, when in fact the null hypothesis is true. 
 
    \item[(b)] A type II error is to fail to reject the null hypothesis when in fact the alternative hypothesis is true. 
 
    \item[(c)] The level of significance of a test is the probability of rejecting the null hypothesis when in fact it is true, i.e. it is the probability of making a type I error.  It is conventionally denoted by $\alpha$. 
 
    \item[(d)] The power of a test is the probability of rejecting the null hypothesis, expressed as a function of the parameter (or equivalently, if it is not a test for a single parameter) being investigated.  So it is given by $1 \;-\; \beta$  , where $\beta$  is the probability of making a type II error similarly expressed as a function. 
 
\end{itemize} 

 
%%%%%%%%%%%%%%%%%%%%%%%%%%%%%%%%%%%%%%%%%%%%%%%%%%%%%%%%%%%%%%%%% 
\newpage
\begin{table}[ht!]
 
\centering
 
\begin{tabular}{|p{15cm}|}
 
\hline  

 \large
\noindent A manufacturer of coffee uses a machine to fill jars.  The machine is calibrated so that the amount of coffee dispensed into each jar is Normally distributed with mean ($\mu$) 200 grams and standard deviation ($\sigma$) 15 grams. 


Each hour, a random sample of 9 jars is taken from the previous hour's output and the sample mean amount ($\bar{x}$) is evaluated.  


If the sample mean lies in the interval $190 < \bar{x} < 210$, the previous hour's output is accepted, otherwise it is rejected and the machine is recalibrated before continuing. 
 
 \begin{itemize}
     \item[(a)] Calculate the probability of committing a type I error by rejecting the previous hour's output when $\mu = $ 200 grams and $\sigma  =$ 15 grams. 
 
\item[(b)] Calculate the probability that the previous hour's output will be accepted when $\mu = $ 216 grams and $\sigma  =$  15 grams. 
 
 \end{itemize}

\\ \hline

\end{tabular}

\end{table} 

\large 
\noindent  Let $X$ represent the amount of coffee in a jar.  We have $X \sim N( \mu , 15^2)$.  

\noindent  The sample size is $n = 9$, so $\bar{X} \sim N( \mu , 15^2/9)$. ( Let $Z \sim N(0, 1)$. )
 
\noindent \textbf{Part (a)}\\ 
\noindent We have $\mu  = 200$. 
 \[ P \left( \bar{X} < 190 \right) = P \left( Z <  \frac{190 \;-\; 200}{ 15/\sqrt{9} } \right) \;=\; P( Z <-2.00) \;=\; 0.02275 \]

\[ P \left( \bar{X} > 210 \right) = P \left( Z >  \frac{210 \;-\; 200}{ 15/\sqrt{9} } \right) \;=\; P( Z > 2.00) \;=\; 0.02275 \]
 So the probability of committing a type I error is 0.02275 + 0.02275 = 0.0455. 
 
\noindent \textbf{Part (b)}\\ 
\noindent  Here $\mu  = 216$. 


\[ P \left( \bar{X} < 190 \right) = P \left( Z <  \frac{190 \;-\; 216}{ 15/\sqrt{9} } \right) \;=\; P( Z < -5.2) \approx 0\;\;\;   \]

\[ P \left( \bar{X} < 210 \right) = P \left( Z <  \frac{210 \;-\; 216}{ 15/\sqrt{9} } \right) \;=\; P( Z < -1.2) \;=\; 0.1151 \]


\large 
\noindent   So the total probability of accepting the output is 0.1151. 

\medskip
 
\large 
\noindent  This is the probability of a Type II error for this procedure, i.e. the value of \textbf{beta} , for $\mu  = 216$.  Thus the power of the procedure when in fact $\mu  = 216$ is $1 \;-\; 0.1151 = 0.8849$.

%\end{enumerate}
\end{document}

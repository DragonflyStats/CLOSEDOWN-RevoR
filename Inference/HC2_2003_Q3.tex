\documentclass[a4paper,12pt]{article}
%%%%%%%%%%%%%%%%%%%%%%%%%%%%%%%%%%%%%%%%%%%%%%%%%%%%%%%%%%%%%%%%%%%%%%%%%%%%%%%%%%%%%%%%%%%%%%%%%%%%%%%%%%%%%%%%%%%%%%%%%%%%%%%%%%%%%%%%%%%%%%%%%%%%%%%%%%%%%%%%%%%%%%%%%%%%%%%%%%%%%%%%%%%%%%%%%%%%%%%%%%%%%%%%%%%%%%%%%%%%%%%%%%%%%%%%%%%%%%%%%%%%%%%%%%%%
\usepackage{eurosym}
\usepackage{vmargin}
\usepackage{amsmath}
\usepackage{graphics}
\usepackage{epsfig}
\usepackage{enumerate}
\usepackage{multicol}
\usepackage{subfigure}
\usepackage{fancyhdr}
\usepackage{listings}
\usepackage{framed}
\usepackage{graphicx}
\usepackage{amsmath}
\usepackage{chngpage}
%\usepackage{bigints}

\usepackage{vmargin}
% left top textwidth textheight headheight
% headsep footheight footskip
\setmargins{2.0cm}{2.5cm}{16 cm}{22cm}{0.5cm}{0cm}{1cm}{1cm}
\renewcommand{\baselinestretch}{1.3}

\setcounter{MaxMatrixCols}{10}
\begin{document}
	%%--- Higher Certificate, Paper II, 2003. Question 3
	
	%%%%%%%%%%%%%%%%%%%%%%%%%%%%%%%%%%%%%%%%%%%%%%%%%%%%%%%%%%%%%%%%%%%%%%%%%%%%%%%%%%%%%%%%%%%%%%%%%%%%%%%%%%%%%%%%%%%%%%%%%%%%%%%%%
	
	%% --- 3. 
	\large
	\noindent One of the tasks routinely undertaken by a particular laboratory is to establish the potassium content of blood serum.  It has been established that, when apparatus is working properly, in repeated tests of the same blood serum the standard deviation should not exceed 0.05g (\%). 
	
	The manager of the laboratory decides to perform a quality control study of the two sets of apparatus used by the laboratory to measure the potassium content of various compounds.  A test sample is prepared in which the potassium content is known to be 10.5g (\%) and each set of apparatus is used to make eight repeat analyses of the test sample.  The results in g (\%) are as follows. 
	
	\begin{center}
		
		\begin{tabular}{|c|c|}
			\hline
			Apparatus A & 10.55 10.62 10.40 10.52 10.46 10.31 10.50 10.49 \\ \hline
			Apparatus B & 10.30 10.25 10.35 10.30 10.28 10.35 10.24 10.43 \\ \hline
		\end{tabular}
	\end{center}
	
	\noindent For each set of apparatus is there significant evidence that 
	
	\begin{enumerate}[(a)]
		\item the readings are more variable than the general laboratory standard? 
		\item the readings are biased? 
		
	\end{enumerate} 
	
	
	Comment on how each set of apparatus should be altered to improve the accuracy of its measurements. 
	
	
	%%%%%%%%%%%%%%%%%%%%%%%%%%%%%%%%%%%%%%%%%%%%%%%%%%%%%%%%%%%%%%%%%%%%%%%%%%%%%%%%%%%%%%%%%%%%%%%%%%%%%%%%%%%%%%%%%%%%%%%%%%%%%%%%%
	
	\begin{framed}
		\noindent Null hypotheses to be tested are $\sigma = 0.05$ and $\mu = 10.5$.\\ \medskip
		
		\noindent Summary statistics for the two sets of apparatus are:
		\begin{itemize}
			\item[A:] $\bar{x} = 10.48$, $s^2 = 0.008898$ (s = 0.09433)
			\item[B:] $\bar{x} = 10.31$, $s^2 = 0.003876$ (s = 0.06228)
		\end{itemize}
		$n = 8$ in both cases.
	\end{framed}
	
	\large
	\noindent \textbf{Part (a)}
	\begin{itemize}
		\item Alternative hypothesis is $\sigma^2 > (0.05)^2 = 0.0025$. 
		\item 
		Test statistic is $  { \displaystyle \frac{(n - 1) s^2}{\sigma^2}  }$
		refer to  $\chi^2_{n-1}$, i.e. $\chi^2_{7}$ here: upper 5\% point is 14.07, upper 1\% point is 18.48.
		\item 
		For A, we get  $  { \displaystyle   \frac{7 \times 0.008898}{0.0025}  = 24.91 }$, highly significant. 
		\item 
		For B, we get $  { \displaystyle   \frac{7 \times 0.003876}{0.0025}  = 10.85 }$, not significant.
		\item The null hypothesis is rejected for A, but cannot be rejected for B. There is
		evidence that A is more variable than standard, but not that B is more variable.
	\end{itemize}
	
	
	\newpage
	
	\large
	\noindent \textbf{Part (b)}
	\begin{itemize}
		\item  Alternative hypothesis is $\mu \neq 10.5$.
		
		\[ \mbox{Test statistic} \;=\;  \frac{ \bar{x} \;-\;10.5 } {s / \sqrt{n}},  \]
		
		\item The test statistic is distributed as $t_{n- 1}$, i.e.
		$t_7$ here. The absolute values of the test statistics are compared to a critical value of 2.364.
		
		\begin{itemize}
			\item[$\bullet$] For A, we get $  { \displaystyle \frac{ -0.02 }{0.09433/\sqrt{8} }  \;=\; \frac{ -0.02 \sqrt{8} }{0.09433}  \;=\;   - 0.60}$ (not significant).
			\item[$\bullet$]  For B, we get $  { \displaystyle  \frac{ -0.19  }{0.06228/\sqrt{8}}  \;=\; \frac{ -0.19 \sqrt{8} }{0.06228}  \;=\;   - 8.63}$ (very significant).
		\end{itemize}
		
		There is very strong evidence that B is biased (downwards) but none that A is
		biased.
		\item  Probably B only needs a scale of measurement adjusted, if the complete
		process is automated; more seriously there may be a fault in the way the
		potassium content is measured. 
		\item For A, there is too much variation, though the
		mean value is acceptable, and this is likely to need adjustment to that part of
		the process which can give rise to variability. In each case a laboratory
		technician should be called in.
	\end{itemize}
\end{document}

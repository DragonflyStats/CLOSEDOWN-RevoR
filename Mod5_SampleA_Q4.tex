\documentclass{article}
\usepackage[utf8]{inputenc}
\usepackage{enumerate}

\author{kobriendublin }
\date{December 2018}

\begin{document}

%- Higher Certificate, Module 5, 2008. Question 1
\section{Introduction}
\begin{enumerate}[(i)]
\item  The sum of all 12 table entries is 30c. These probabilities must add up to 1, so $c = 1/30$.
%%%%%%%%%%%%%%%%%%%%%%%%%%%%%%%%%%%
\item The marginal distributions are given by the row and column totals.
Hence: 
\begin{itemize}
    \item P(X = 1) = 15c = 1/2; 
    \item P(X = 2) = 10c = 1/3; 
    \item P(X = 3) = 5c = 1/6.
\end{itemize}
Similarly:
\begin{itemize}
    \item P(Y = 1) = 12c = 2/5; 
    \item P(Y = 2) = 6c = 1/5; 
    \item P(Y = 3) = 6c = 1/5; 
    \item P(Y = 4) = 6c = 1/5.
    
\end{itemize}    
%%%%%%%%%%%%%%%%%%%%%%%%%%%%%%%%%%%
\item 111121()1232362323EX⎛⎞⎛⎞⎛⎞=×+×+×=++=⎜⎟⎜⎟⎜⎟⎝⎠⎝⎠⎝⎠ .
2111143()1492362323EX⎛⎞⎛⎞⎛⎞=×+×+×=++=⎜⎟⎜⎟⎜⎟⎝⎠⎝⎠⎝⎠ .
21055Var()33X⎛⎞ ∴=−=⎜⎟⎝⎠.
We also need E(Y) later: 223411()55555EY=+++=.
Distribution of XY:
  Values of xy
1
2
3
4
6
12
Probability
6c
7c
4c
6c
5c
2c
[c = 1/30, see above]
614465211011()1234612303030303030303EXY⎛⎞⎛⎞⎛⎞⎛⎞⎛⎞⎛⎞=×+×+×+×+×+×==⎜⎟⎜⎟⎜⎟⎜⎟⎜⎟⎜⎟⎝⎠⎝⎠⎝⎠⎝⎠⎝⎠⎝⎠
Also we have 51111()()353EXEY=×=.
Cov(,)()()()0XYEXYEXEY∴=− .
%%%%%%%%%%%%%%%%%%%%%%%%%%%%%%%%%%%
\item X and Y are not independent [even though Cov(X, Y) = 0 and even though some cells have P(X = x, Y = y) = P(X = x).P(Y = y)]. For example, we have P(X = 1, Y = 4) = 2/15, but P(X = 1).P(Y = 4) = 1/10.
%%%%%%%%%%%%%%%%%%%%%%%%%%%%%%%%%%%
\item (v) U = 1 if X = 1 or 3 U = 0 if X = 2
V = 1 if Y = 1 or 3 V = 0 if Y = 2 or 4
Table of joint distribution of U and V, with margins.
Values of V
0
1
0
2c = 1/15
8c = 4/15
10c = 1/3
Values of U
1
10c = 1/3
10c = 1/3
20c = 2/3
12c = 2/5
18c = 3/5
Consider for example the cell with (U, V) = (0, 0). The cell probability is 1/15 but the product of the marginal probabilities is 2/15. So U and V are not independent.
\end{enumerate}
\end{document}
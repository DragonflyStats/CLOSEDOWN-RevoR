\documentclass[a4paper,12pt]{article}

%%%%%%%%%%%%%%%%%%%%%%%%%%%%%%%%%%%%%%%%%%%%%%%%%%%%%%%%%%%%%%%%%%%%%%%%%%%%%%%%%%%%%%%%%%%%%%%%%%%%%%%%%%%%%%%%%%%%%%%%%%%%%%%%%%%%%%%%%%%%%%%%%%%

\usepackage{eurosym}
\usepackage{vmargin}
\usepackage{amsmath}
\usepackage{graphics}
\usepackage{epsfig}
\usepackage{enumerate}
\usepackage{multicol}
\usepackage{subfigure}
\usepackage{fancyhdr}
\usepackage{listings}
\usepackage{framed}
\usepackage{graphicx}
\usepackage{amsmath}
\usepackage{chngpage}

%\usepackage{bigints}



\usepackage{vmargin}

% left top textwidth textheight headheight

% headsep footheight footskip

\setmargins{2.0cm}{2.5cm}{16 cm}{22cm}{0.5cm}{0cm}{1cm}{1cm}

\renewcommand{\baselinestretch}{1.3}

\setcounter{MaxMatrixCols}{10}

\begin{document}
Higher Certificate, Paper II, 2004. Question 4
%%%%%%%%%%%%%%%%%%%%%%%%%%%%%%%%%%% 
\begin{framed} 
 
4. The yields on the ordinary shares of 95 large United Kingdom companies at close of trading on 6 March 2003, as given in the Financial Times, were as follows. 
 
Yield (%) Number of companies  ≥ 0 but < 1   5  ≥ 1 but < 2   8  ≥ 2 but < 3 13  ≥ 3 but < 4 18  ≥ 4 but < 5 19     ≥ 5 but < 7.5 21 ≥ 7.5 but < 10   8  ≥ 10 but < 15   2  ≥ 15 but < 20   1 Total 95 
           Source:  "Financial Times", 6 March 2003. 
 
You may treat the yields as constituting a random sample from a large population. 
 
(i) Draw a histogram depicting the above data. 
(7) 
 
(ii) State the modal class interval and estimate the mean, median and standard deviation of the observations. (6) 
 
(iii) Why is it only possible to estimate the mean, median and standard deviation rather than find them? (3) 
 
(iv) Let p be the proportion of companies in the underlying population for which the shares yield 5 per cent or more.  Construct a 95% confidence interval for p. (4) 
 

\end{framed}


\newpage
%%%%%%%%%%%%%%%%%%%%%%%%%%%%%%%%%%%
\begin{framed}
Yield (%) Interval
width
Frequency f Frequency
density
Midpoint x fx fx2 Cum freq
F
≥ 0 but < 1 1 5 5 0.5 2.5 1.25 5
≥ 1 but < 2 1 8 8 1.5 12.0 18.00 13
≥ 2 but < 3 1 13 13 2.5 32.5 81.25 26
≥ 3 but < 4 1 18 18 3.5 63.0 220.50 44
≥ 4 but < 5 1 19 19 4.5 85.5 384.75 63
≥ 5 but < 7.5 2.5 21 8.4 6.25 131.25 820.3125 84
≥ 7.5 but < 10 2.5 8 3.2 8.75 70.0 612.50 92
≥ 10 but < 15 5 2 0.4 12.5 25.0 312.50 94
≥ 15 but < 20 5 1 0.2 17.5 17.5 306.25 95
95 439.25 2757.3125
\end{framed}

(i)
(ii) The modal class interval is "≥ 4 but < 5" (based, of course, on frequency
density; it would be "≥ 5 but < 7.5" if based only on frequency).
x = 439.25/95 = 4.62 (%).
For the median, we require the 48th observation from the beginning, which is
estimated as being at 4 + (4/19 × 1) = 4.21 (%).
2
2 1 2757.3125 439.25
94 95
s
 
=  − 
 
= 7.7272. So s = 2.78 (%).
(iii) These are only estimates because we have the data grouped into intervals, not
the 95 individual values.
(iv) p is estimated by ˆp = 32/95 = 0.337. The estimated variance of ˆp is
(0.337)(0.663)/95 = 0.002352, so the estimated standard deviation is 0.0485. Thus a
95% confidence interval for p is given by, approximately, 0.337 ± (1.96)(0.0485), i.e.
it is (0.242, 0.432).\end{enumerate}
\end{document}

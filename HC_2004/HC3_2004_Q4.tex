\documentclass[a4paper,12pt]{article}

%%%%%%%%%%%%%%%%%%%%%%%%%%%%%%%%%%%%%%%%%%%%%%%%%%%%%%%%%%%%%%%%%%%%%%%%%%%%%%%%%%%%%%%%%%%%%%%%%%%%%%%%%%%%%%%%%%%%%%%%%%%%%%%%%%%%%%%%%%%%%%%%%%%

\usepackage{eurosym}
\usepackage{vmargin}
\usepackage{amsmath}
\usepackage{graphics}
\usepackage{epsfig}
\usepackage{enumerate}
\usepackage{multicol}
\usepackage{subfigure}
\usepackage{fancyhdr}
\usepackage{listings}
\usepackage{framed}
\usepackage{graphicx}
\usepackage{amsmath}
\usepackage{chngpage}

%\usepackage{bigints}



\usepackage{vmargin}

% left top textwidth textheight headheight

% headsep footheight footskip

\setmargins{2.0cm}{2.5cm}{16 cm}{22cm}{0.5cm}{0cm}{1cm}{1cm}

\renewcommand{\baselinestretch}{1.3}

\setcounter{MaxMatrixCols}{10}

\begin{document}
Higher Certificate, Paper III, 2004.  Question 4 
 
 
(i) 
Year Quarter Sales 4-quarter totals 8-quarter totals 
Moving average 
1997 1 31.54    1997 2 22.33   1997 3 20.29 209.73 26.216(25) 1997 4 30.30 
 104.46 105.27   1998 1 32.35    
 
 
(ii) There is a sharp seasonal variation, "Sales – MA" being always substantially negative in quarters 2 and 3, always substantially positive in quarters 1 and 4.  To estimate the pattern of seasonal variation, we need the average of the "Sales – MA" figures for each quarter. 
 
 Q1 Q2 Q3 Q4   1997     –5.93   3.73   1998   5.42   –2.84   –6.38   3.90   1999   6.24   –3.93   –5.88   3.95   2000   5.15   –4.20   –4.78   3.88   2001   5.77      Seasonal totals 22.58 –10.97 –22.97 15.46   Seasonal averages     5.645     –3.657     –5.743     3.865 Sum:  0.110 Correction:  –0.028 Corrected seasonal averages     5.617     –3.685     –5.771     3.837 (–0.002)  
 
 
(iii) Using this case as an example, the visual pattern can show detail which is lost in the table of figures, such as in the year 2000 where the fluctuation was not so great as in other years, although the pattern was the same.  We can also see that, while it rises overall, the MA trend shows a slight dip from late 1998 onwards before a sharper rise in 2000.  We can visualise the trend from the table when it is fairly smooth like this, but not always so easily.  Trend and seasonal variation are important properties to observe, and a clear method of doing so is invaluable. 
 
 
(iv) We use observed sales minus estimated seasonal variation.  For year 2000: 
 
Q1 Q2 Q3 Q4 32.64 – 5.62 = 27.02 23.64 + 3.69 = 27.33 23.37 + 5.77 = 29.14 32.20 – 3.84 = 28.36 
 
It would not be a good idea to use this method on the 2001 sales because the estimated seasonal variation might have changed if we had had enough data to make "Sales – MA" up to the end of the year. 
 

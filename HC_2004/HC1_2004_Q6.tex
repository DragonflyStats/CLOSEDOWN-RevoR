\documentclass[a4paper,12pt]{article}

%%%%%%%%%%%%%%%%%%%%%%%%%%%%%%%%%%%%%%%%%%%%%%%%%%%%%%%%%%%%%%%%%%%%%%%%%%%%%%%%%%%%%%%%%%%%%%%%%%%%%%%%%%%%%%%%%%%%%%%%%%%%%%%%%%%%%%%%%%%%%%%%%%%%%%%%%%%%%%%%%%%%%%%%%%%%%%%%%%%%%%%%%%%%%%%%%%%%%%%%%%%%%%%%%%%%%%%%%%%%%%%%%%%%%%%%%%%%%%%%%%%%%%%%%%%%

\usepackage{eurosym}
\usepackage{vmargin}
\usepackage{amsmath}
\usepackage{graphics}
\usepackage{epsfig}
\usepackage{enumerate}
\usepackage{multicol}
\usepackage{subfigure}
\usepackage{fancyhdr}
\usepackage{listings}
\usepackage{framed}
\usepackage{graphicx}
\usepackage{amsmath}
\usepackage{chngpage}

%\usepackage{bigints}
\usepackage{vmargin}

% left top textwidth textheight headheight

% headsep footheight footskip

\setmargins{2.0cm}{2.5cm}{16 cm}{22cm}{0.5cm}{0cm}{1cm}{1cm}

\renewcommand{\baselinestretch}{1.3}

\setcounter{MaxMatrixCols}{10}

\begin{document}
Higher Certificate, Paper I, 2004. Question 6
\begin{framed}
%%%%%%%%%%%%%%%%%%%%%%%%%%%%%%%%%
The random variable X has probability density function $f(x)$ given by
\[ f(x) =  \frac{k}{x^{k+1}}, \qquad \mbox{  where  } x \geq  1, k > 0.\]
Sketch the graph of $f(x)$ and find the cumulative distribution function $F(x)$.



%%%%%%%%%%%%%%%%%%%%%%%%%%%%%%%%%%%%%%%%%%%%%%%%%%%%%%
\end{framed}
( ) 1, 1; 0 k
f x k x k
x + = ≥ >

\begin{enumerate}

\item  Sketch of f (x).
[NOTE. The curve should of course appear as a smooth curve; it might not do so,
due to the limits of electronic reproduction.]

;
The CDF is ecomputed as follows
\begin{eqnarray*}
F_X(x) &=& P(X \leq x)\\
&=& \int^{x}_{1}  f(u) du\\
&=& \int^{x}_{1}  \frac{k}{u^{k+1}} du\\
&=& \left[ - \frac{1}{u^k} \right]^x_1  \\
&=& 1 - \frac{1}{x^k}  \qquad \mbox{ for } x \geq 1 \\
\end{eqnarray*}



%%%%%%%%%%%%%%%%%%%%%%%%%%%%%%%%%%%%%%%%%%%%%%%%%%%%%%%%%%
\newpage
\begin{framed}
(ii) Find the median and the lower and upper quartiles of X and deduce the semiinterquartile
range of X.
\end{framed}

\item  




\begin{description}

\item[Median]       
M has ${ \displaystyle \frac{1}{2} = F(m) = 1 – M^{–k} }$, so 
${ \displaystyle \frac{1}{2} = M^{–k} }$ and hence $M = 2^{1/k}$. 

\item[Lower quartile] $Q_1$ has ${ \displaystyle \frac{1}{4} = F(Q_1) = 1 – Q_1^{–k} }$, so
${ \displaystyle \frac{3}{4} = Q_1^{–k} }$, i.e. $Q_1 = (4/3)^{1/k}$

\item[Upper quartile] $Q_3$ has ${ \displaystyle \frac{3}{4} = F(Q_3) = 1 – Q_3^{–k} }$, so
${ \displaystyle \frac{1}{4} = Q_3^{–k} }$, i.e. $Q_3 = 4^{1/k}$

\end{description}

\begin{eqnarray*}
E(X) &=& \int^{\infty}_{1} x f(x) dx\\
% &=& \int^{\infty}_{1} x \frac{k}{x^2}f(x) dx\\
&=& \int^{\infty}_{1} x \frac{k}{x^{k+1}}f(x) dx\\
&=& \int^{\infty}_{1}  \frac{k}{x^{k}}f(x) dx\\
&=& \left[ \frac{-k}{(k-1)x^{k-1}} \right]^{\infty}_{1}\\
&=& \frac{k}{k-1}
\end{eqnarray*}

\begin{eqnarray*}
E(X^2) 
&=& \int^{\infty}_{1} x^2 f(x) dx\\
&=& \int^{\infty}_{1} x^2 \frac{k}{x^{k+1}}f(x) dx\\
&=& \int^{\infty}_{1}  \frac{k}{x^{k-1}}f(x) dx\\
&=& \left[ \frac{-k}{(k-2)x^{k-2} } \right]^{\infty}_{1}\\
&=& \frac{k}{k-2}
\end{eqnarray*}


This can be written down
directly from the c.d.f. found in part (i).
%%%%%%%%%%%%%%%%%%%%%%%%%%%%%%%%%%%%%%%%%%%%%%%%%%%%%%
\newpage
\begin{framed}

(iii) Assuming $k > 2$, find the expectation and variance of X. What is the probability that X exceeds its expectation?

(iv) In the country of Utopia, incomes in units of £10000 are distributed as is $X$ with k = 3. Find (a) the median income, (b) the mean income, (c) the
proportion of incomes greater than \$100000.
\end{framed}
\item  For the case k = 3,
\begin{itemize}
\item (a) M = 21/3 in the units given, or \$12599,
\item (b) mean = 3/2 in the units given, or \$15000,
\item (c) inserting X = 10, $P(X \leq 10) = 3$
1 1
10
− , so $P(X > 10)$ = 3
1
10
, i.e. 0.1\%.
\end{itemize}
\end{enumerate}
\end{document}

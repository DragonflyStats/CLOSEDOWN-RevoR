Higher Certificate, Paper I, 2004. Question 6
( ) 1, 1; 0 k
f x k x k
x + = ≥ >
(i) Sketch of f (x).
[NOTE. The curve should of course appear as a smooth curve; it might not do so,
due to the limits of electronic reproduction.]
C.d.f. is ( ) ( ) 1 1
1
1 1 1
x
x
k k k
F x P X x k du
u + u x
= ≤ = = −  = −   ∫ (for x ≥ 1).
(ii) Median M has ½ = F(m) = 1 – M –k, so ½ = M –k and hence M = 21/k.
Lower quartile Q1 has ¼ = F(Q1) = 1 – Q1
–k, so ¾ = Q1
–k, i.e. Q1 = (4/3)1/k.
Upper quartile Q3 has ¾ = F(Q3) = 1 – Q3
–k, so Q3 = (4)1/k.
Hence the semi-interquartile range is
1/
1/ 4
4
3
1
2
k
k      −   
   
.
Continued on next page
f (x)
1 x
(iii) ( ) 1 ( ) 1 ( ) 1
1 k 1 k 1
E X x f x dx k dx k k
x k x k
∞
∞ ∞
−
 −  = = =   =  −  −
∫ ∫ .
( ) ( ) ( )
2 2
1 1 1 2
1 k 2 k 2
E X x f x dx k dx k k
x k x k
∞
∞ ∞
− −
 −  = = =   =  −  −
∫ ∫ .
( ) ( ) { ( )} ( )
2
2 2
2 Var
2 1
X E X E X k k
k k
∴ = − = −
− −
( )( )
{( ) ( )} ( )( )
2
2 2 1 2
2 1 1 2
k k k k k
k k k k
= − − − =
− − − −
.
( ( )) /( 1) 1
/( 1)
1 1k
k k k k
k k
P X E X k dx k
x x k
∞
∞
− +
−
> = = −  =  −          ∫ , or this can be written down
directly from the c.d.f. found in part (i).
(iv) For the case k = 3,
(a) M = 21/3 in the units given, or £12599,
(b) mean = 3/2 in the units given, or £15000,
(c) inserting X = 10, P(X ≤ 10) = 3
1 1
10
− , so P(X > 10) = 3
1
10
, i.e. 0.1%.

\documentclass[a4paper,12pt]{article}

%%%%%%%%%%%%%%%%%%%%%%%%%%%%%%%%%%%%%%%%%%%%%%%%%%%%%%%%%%%%%%%%%%%%%%%%%%%%%%%%%%%%%%%%%%%%%%%%%%%%%%%%%%%%%%%%%%%%%%%%%%%%%%%%%%%%%%%%%%%%%%%%%%%%%%%%%%%%%%%%%%%%%%%%%%%%%%%%%%%%%%%%%%%%%%%%%%%%%%%%%%%%%%%%%%%%%%%%%%%%%%%%%%%%%%%%%%%%%%%%%%%%%%%%%%%%

\usepackage{eurosym}
\usepackage{vmargin}
\usepackage{amsmath}
\usepackage{graphics}
\usepackage{epsfig}
\usepackage{enumerate}
\usepackage{multicol}
\usepackage{subfigure}
\usepackage{fancyhdr}
\usepackage{listings}
\usepackage{framed}
\usepackage{graphicx}
\usepackage{amsmath}
\usepackage{chngpage}

%\usepackage{bigints}
\usepackage{vmargin}

% left top textwidth textheight headheight

% headsep footheight footskip

\setmargins{2.0cm}{2.5cm}{16 cm}{22cm}{0.5cm}{0cm}{1cm}{1cm}

\renewcommand{\baselinestretch}{1.3}

\setcounter{MaxMatrixCols}{10}

\begin{document}
Higher Certificate, Paper I, 2004. Question 6

( ) 1, 1; 0 k
f x k x k
x + = ≥ >

\begin{enumerate}

\item  Sketch of f (x).
[NOTE. The curve should of course appear as a smooth curve; it might not do so,
due to the limits of electronic reproduction.]
C.d.f. is ( ) ( ) 1 1
1
1 1 1
x
x
k k k
F x P X x k du
u + u x
= ≤ = = −  = −   ∫ (for x ≥ 1).
\item  Median M has ½ = F(m) = 1 – M –k, so ½ = M –k and hence M = 21/k.
\begin{itemize}
    \item Lower quartile Q1 has ¼ = F(Q1) = 1 – Q1
–k, so ¾ = Q1
–k, i.e. Q1 = (4/3)1/k.
\item Upper quartile Q3 has ¾ = F(Q3) = 1 – Q3
–k, so Q3 = (4)1/k.
\item Hence the semi-interquartile range is
1/
1/ 4
4
3
1
2
k
k      −   
   
.
\end{itemize}

\begin{eqnarray*}
E(X) &=& \int^{\infty}_{1} x f(x) dx\\
% &=& \int^{\infty}_{1} x \frac{k}{x^2}f(x) dx\\
&=& \int^{\infty}_{1} x \frac{k}{x^{k+1}}f(x) dx\\
&=& \int^{\infty}_{1}  \frac{k}{x^{k}}f(x) dx\\
&=& \left[ \frac{-k}{(k-1)x^{k-1}} \right]^{\infty}_{1}\\
&=& \frac{k}{k-1}
\end{eqnarray*}

\begin{eqnarray*}
E(X^2) 
&=& \int^{\infty}_{1} x^2 f(x) dx\\
&=& \int^{\infty}_{1} x^2 \frac{k}{x^{k+1}}f(x) dx\\
&=& \int^{\infty}_{1}  \frac{k}{x^{k-1}}f(x) dx\\
&=& \left[ \frac{-k}{(k-2)x^{k-2} } \right]^{\infty}_{1}\\
&=& \frac{k}{k-2}
\end{eqnarray*}
\[
2 1 1 2
k k k k k
k k k k
= − − − =
− − − −
.
( ( )) /( 1) 1
/( 1)
1 1k
k k k k
k k
P X E X k dx k
x x k
∞
∞
− +
−
> = = −  =  −     
     ∫ , or this can be written down
directly from the c.d.f. found in part (i).
\item  For the case k = 3,
\begin{itemize}
\item (a) M = 21/3 in the units given, or £12599,
\item (b) mean = 3/2 in the units given, or £15000,
\item (c) inserting X = 10, P(X ≤ 10) = 3
1 1
10
− , so P(X > 10) = 3
1
10
, i.e. 0.1%.
\end{itemize}
\end{enumerate}
\end{document}

\documentclass[a4paper,12pt]{article}

%%%%%%%%%%%%%%%%%%%%%%%%%%%%%%%%%%%%%%%%%%%%%%%%%%%%%%%%%%%%%%%%%%%%%%%%%%%%%%%%%%%%%%%%%%%%%%%%%%%%%%%%%%%%%%%%%%%%%%%%%%%%%%%%%%%%%%%%%%%%%%%%%%%

\usepackage{eurosym}
\usepackage{vmargin}
\usepackage{amsmath}
\usepackage{graphics}
\usepackage{epsfig}
\usepackage{enumerate}
\usepackage{multicol}
\usepackage{subfigure}
\usepackage{fancyhdr}
\usepackage{listings}
\usepackage{framed}
\usepackage{graphicx}
\usepackage{amsmath}
\usepackage{chngpage}

%\usepackage{bigints}



\usepackage{vmargin}

% left top textwidth textheight headheight

% headsep footheight footskip

\setmargins{2.0cm}{2.5cm}{16 cm}{22cm}{0.5cm}{0cm}{1cm}{1cm}

\renewcommand{\baselinestretch}{1.3}

\setcounter{MaxMatrixCols}{10}

\begin{document}
Higher Certificate, Paper II, 2004. Question 8
(a) The total number of nests is 240. On the null hypothesis that each of the 8
directional categories has the same probability of being used, the expected number
will be 30 for each. Thus the observed and expected frequencies (O and E) are as
follows.
Position N NE E SE S SW W NW
O 27 24 35 33 38 33 24 26
E 30 30 30 30 30 30 30 30
The test statistic is
( )2 2 2 2
2 3 6 ... 4 204 6.80
30 30 30 30
O E
X
E
−
=Σ = + + + = = ,
which is referred to 2
7 χ (note 7 degrees of freedom because the table has 8 cells and
there are no estimated parameters here). This is not significant (the 5% point is
14.07); we cannot reject the null hypothesis.
Splitting the data table into 8 positions, each with a fairly small expected frequency
for a chi-squared test, limits the power.
A Kolmogorov-Smirnov test uses an empirical cumulative distribution function,
taking a starting point which in this case would be arbitrary, say N (North), and
following in order round the positions. Thus it is not testing a relevant null hypothesis
for this problem.
(b) We have a 2×3 contingency table. The null hypothesis is that males and
females have the same ratio of preferences for designs A, B and C. The contingency
table is as follows, with the expected frequencies in brackets in each cell.
A B C Total
Males 129 (120) 24 (32) 47 (48) 200
Females 126 (135) 44 (36) 55 (54) 225
Total 255 68 102 425
The test statistic is
( )2 2 2 2 2 2 2
2 9 8 1 9 8 1 5.09
120 32 48 135 36 54
O E
X
E
−
=Σ = + + + + + = ,
which is referred to 22
χ . This is not significant (the 5% point is 5.99); we cannot
reject the null hypothesis.

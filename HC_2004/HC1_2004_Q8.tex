\documentclass[a4paper,12pt]{article}

%%%%%%%%%%%%%%%%%%%%%%%%%%%%%%%%%%%%%%%%%%%%%%%%%%%%%%%%%%%%%%%%%%%%%%%%%%%%%%%%%%%%%%%%%%%%%%%%%%%%%%%%%%%%%%%%%%%%%%%%%%%%%%%%%%%%%%%%%%%%%%%%%%%%%%%%%%%%%%%%%%%%%%%%%%%%%%%%%%%%%%%%%%%%%%%%%%%%%%%%%%%%%%%%%%%%%%%%%%%%%%%%%%%%%%%%%%%%%%%%%%%%%%%%%%%%

\usepackage{eurosym}
\usepackage{vmargin}
\usepackage{amsmath}
\usepackage{graphics}
\usepackage{epsfig}
\usepackage{enumerate}
\usepackage{multicol}
\usepackage{subfigure}
\usepackage{fancyhdr}
\usepackage{listings}
\usepackage{framed}
\usepackage{graphicx}
\usepackage{amsmath}
\usepackage{chngpage}

%\usepackage{bigints}
\usepackage{vmargin}

% left top textwidth textheight headheight

% headsep footheight footskip

\setmargins{2.0cm}{2.5cm}{16 cm}{22cm}{0.5cm}{0cm}{1cm}{1cm}

\renewcommand{\baselinestretch}{1.3}

\setcounter{MaxMatrixCols}{10}

\begin{document}Higher Certificate, Paper I, 2004. Question 8
\begin{framed}
%%%%%%%%%%%%%%%%%%%%%%%%%%%%%%%%%



8. In a study of office efficiency, a firm has established benchmark times, x1, x2, …, x10,
for the completion of 10 different routine office tasks. A newly recruited trainee is
timed for his performance on each of these tasks, giving times y1, y2, …, y10. The
times in minutes, (xi, yi), i = 1, 2, …, 10, are tabulated below.
Task A B C D E F G H I J
Benchmark time x 5 5 10 10 10 15 15 20 20 40
Trainee's time y 8 12 16 16 21 18 20 25 31 53
Note:Σ = 150, Σ 2 = 3200, Σ = 220, Σ 2 = 6280, Σ = 4440 i i i i i i x x y y x y .
(i) Plot a scatter diagram of these data and briefly comment on the suitability of
simple linear regression analysis in this case.
\end{framed}

\begin{framed}
(ii) Stating clearly your assumptions, use the method of least squares to fit a
simple linear regression model to the data, and calculate (a) the residual mean
square (the unbiased estimate of the variance σ 2 of the stochastic term in the
regression model) and (b) the coefficient of determination, R2.
\end{framed}


\begin{enumerate}[(a)]
\item Trainee's time (y)
0
10
20
30
40
50
60
0 10 20 30 40 50
Benchmark time (x)
Simple linear regression analysis seems quite suitable.
\item  The model is $y_i = \alpha_i + \beta xi + e_i$, where {ei} are uncorrelated with zero mean and
(constant) variance σ 2 (independent identically distributed $N(0, \sigma^2)$ for the purpose of
undertaking statistical tests, as in part (iii)). \begin{itemize}
\item Estimating by the method of least squares
gives
ˆ xy
xx
S
S
β = , αˆ = y −βˆ x ,
where (standard notation)
\begin{eqnarray*}
S_{XY} &=&
\sum x_iy_i - \frac{\sum x_i\sum y_i}{n}\\
S_{XX} &=&
\sum x_i^2 - \frac{(\sum x_i)^2}{n}\\
S_{YY} &=&
\sum y_i^2 - \frac{(\sum y_i)^2}{n}\\
\end{eqnarray*}
\begin{itemize}
\item \textbf{ Slope Estimate}
\begin{eqnarray*}
b_1 = \frac{S_{XY}}{S_{XX}}
\end{eqnarray*}
\item \textbf{ Intercept Estimate}
\begin{eqnarray*}
 b_0 = \bar{y} -b_1\bar{x}
\end{eqnarray*}
\end{itemize}
%%%%%%%%%%%%%%%%%%%%%%%%%%%%%%%%%%
\item We have
( )
( 2 )
ˆ 4440 150 220 /10 1140 1.20
3200 150 /10 950
xy
xx
S
S
β
− ×
= = = =
−
and αˆ = 22 − (1.20×15) = 4 ,
\item so the line is
y = 4 + 1.2x.
\item 
The total sum of squares is ( ) ( )2
2 2 1440
10
i
yy i i
y
S = y − y = y − = Σ Σ Σ .

\item The sum of squares for regression is ˆ
xy β S (or 2 / xy xx S S ) = 1368.
\item Therefore the residual sum of squares is 1440 – 1368 = 72.
\item This has 8 degrees of freedom, so the residual mean square ($\hat{\sigma}^2$ ) is 72/8 = 9.
\item The coefficient of determination $R^2 = 1368/1440 = 0.95$ (usually given as 95\%).
\end{itemize}
%%%%%%%%%%%%%%%%%%%%%%%%%%%%%%%%%%%%%%%%%%%%
\newpage
\begin{framed}

(iii) Given that the variance of the slope estimate is Σ( − )2
2
x x i
σ , where x denotes
the sample mean benchmark time, test at the 5% level of significance the null
hypothesis that the slope of this regression is 1.
(4)

\end{framed}
\item  The estimated variance of $\hat{\beta}$ is 9/950 = 0.009474. 
\begin{itemize}
    \item So the test statistic for
testing the null hypothesis $\beta = 1$ is 1.2 1
0.009474
− = 2.05, which we refer to t8.
\item This is not significant at the 5\% level, so the null hypothesis $\beta = 1$ cannot be rejected.
\end{itemize}
%%%%%%%%%%%%%%%%%%%%%%%%%%%%%%%%%%%
\newpage
\begin{framed}
(iv) The office manager is dissatisfied with the regression relationship which you
obtain. He believes that when x = 0 it should also be the case that y = 0. State
the appropriate form of the linear regression model embodying this restriction,
derive a formula for the estimate of its slope parameter, and estimate this
parameter for the above data.
\end{framed}
\item  The model here is yi = bxi + ei.
Estimating b by least squares, we minimise ( )2
1
n
i i
i
y bx
=
Ω =Σ − .
\begin{itemize}
    \item Differentiating with respect to b, we have 2 ( ) i i i
d y bx x
db
Ω = − Σ − .
\item Setting this equal to zero gives ˆ 2
i i i Σx y = bΣx , i.e. ˆ / 2 i i i b = Σx y Σx .
\end{itemize}

(Note that
2
2
2 2 0 i
d x
db
Ω = Σ > , so this is a minimum.)
Thus we have $\hat{b} = 4440/3200 = 1.3875$.
\end{enumerate}
\end{document}

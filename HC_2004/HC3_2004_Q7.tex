\documentclass[a4paper,12pt]{article}

%%%%%%%%%%%%%%%%%%%%%%%%%%%%%%%%%%%%%%%%%%%%%%%%%%%%%%%%%%%%%%%%%%%%%%%%%%%%%%%%%%%%%%%%%%%%%%%%%%%%%%%%%%%%%%%%%%%%%%%%%%%%%%%%%%%%%%%%%%%%%%%%%%%

\usepackage{eurosym}
\usepackage{vmargin}
\usepackage{amsmath}
\usepackage{graphics}
\usepackage{epsfig}
\usepackage{enumerate}
\usepackage{multicol}
\usepackage{subfigure}
\usepackage{fancyhdr}
\usepackage{listings}
\usepackage{framed}
\usepackage{graphicx}
\usepackage{amsmath}
\usepackage{chngpage}

%\usepackage{bigints}



\usepackage{vmargin}

% left top textwidth textheight headheight

% headsep footheight footskip

\setmargins{2.0cm}{2.5cm}{16 cm}{22cm}{0.5cm}{0cm}{1cm}{1cm}

\renewcommand{\baselinestretch}{1.3}

\setcounter{MaxMatrixCols}{10}

\begin{document}

Higher Certificate, Paper III, 2004.  Question 7 
 
 
(i) A trimmed mean is likely to have removed any major outliers, and in the case of a skew distribution it will be a better central measure than the mean of all the data.  Although hypothesis testing is still very approximate, descriptive statistics are improved. 
 
 
(ii) These particular sets of wage data appear to be skew, as would be anticipated, rather than having many obvious outliers.  In sector 0, the 44.50 and perhaps the 1.75 appear to be outliers, but there is doubt about regarding any others as such.  The other noticeable "gap" is between 16.42 and 19.00 in sector 1;  some people would argue for regarding the top three values in that sector as outliers. 
 
One convention is to regard as outliers any observations below Q1 – 1.5R or above Q3 + 1.5R, where R is the interquartile range (R = Q3 – Q1).  This would cover all above 22.4 in sector 0, i.e. the top nine;  only 20.4 in sector 1 (in spite of the obvious "gap" already mentioned);  and none in sector 2. 
 
The lowest value in sector 0 is suspect, but in a distribution of this shape no automatic calculation is likely to declare it to be an outlier. 
 
In the boxplots, the "whiskers" have been extended all the way to the minimum and maximum for each sector except for the lowest and highest values in sector 0. 
 
 
 
 
 
 
 
 
 
 
[Note.  The limits of electronic reproduction may mean that the boxplots will not appear in their correct locations with precise accuracy.] 
 
 
 
Continued on next page 
Sector 0
Sector 1
Sector 2
0 10 20 30 40 50 Wages 

 
 
(iii) Most of the available data are for sector 0.  The extreme values for this sector are under $2 and over $40, but there is only one very small and one very large value.  The median is below that for the other two sectors, indicating a general tendency towards lower payments.  The overall pattern is skew. 
 
Wages in sectors 1 and 2 show a rather similar pattern, but this is based on a much larger sample of data for sector 1 than for sector 2.  In sector 1, there are top values (three of them) around $19 or $20, and three of $4 or less.  The three top values could be checked to see if they are indeed from this sector, or perhaps from a distinctly different sub-sector compared with the rest. 
 
Wages in sector 2 do not exceed $15 in these (few) data, but there are none below $3.75.  In fact there are only three below $7.  This suggests that workers are on the whole better paid at the lower end of this sector but wages do not rise to the level of the other sectors. 
 
 

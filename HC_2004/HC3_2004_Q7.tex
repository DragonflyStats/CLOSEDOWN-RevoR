\documentclass[a4paper,12pt]{article}

%%%%%%%%%%%%%%%%%%%%%%%%%%%%%%%%%%%%%%%%%%%%%%%%%%%%%%%%%%%%%%%%%%%%%%%%%%%%%%%%%%%%%%%%%%%%%%%%%%%%%%%%%%%%%%%%%%%%%%%%%%%%%%%%%%%%%%%%%%%%%%%%%%%

\usepackage{eurosym}
\usepackage{vmargin}
\usepackage{amsmath}
\usepackage{graphics}
\usepackage{epsfig}
\usepackage{enumerate}
\usepackage{multicol}
\usepackage{subfigure}
\usepackage{fancyhdr}
\usepackage{listings}
\usepackage{framed}
\usepackage{graphicx}
\usepackage{amsmath}
\usepackage{chngpage}

%\usepackage{bigints}



\usepackage{vmargin}

% left top textwidth textheight headheight

% headsep footheight footskip

\setmargins{2.0cm}{2.5cm}{16 cm}{22cm}{0.5cm}{0cm}{1cm}{1cm}

\renewcommand{\baselinestretch}{1.3}

\setcounter{MaxMatrixCols}{10}

\begin{document}

Higher Certificate, Paper III, 2004.  Question 7 
 
%%%%%%%%%%%%%%%%%%%%%%%%%%%%%%%%%%%%%%%%%%%%%%%%%%%%%%%%%%%%%%%%%%%%%%%%
\begin{framed}
7.
A random sample of 250 people has been taken from the 1985 US Current Population
Survey. Some computer output relating to the variables WAGE (wage in dollars per
hour) and SECTOR (0 = Other, 1 = Manufacturing, 2 = Construction) is given below.
(i) "TrMean" in the output is the trimmed mean. This is the mean obtained when
the largest 5% and smallest 5% (rounded to the nearest integer) of values are
removed. What advantage might there be in using a trimmed mean rather than
a conventional mean?
(4)
(ii) Draw a boxplot of WAGE for each of the three sectors. Indicate on your
boxplots any values which might be regarded as outliers.
\end{framed}
\begin{framed}
(iii) Write a report comparing wages by sector, based on the computer output and
your answer to part (ii).
(10)
Descriptive Statistics: wages by sector
Variable
WAGE SECTOR
0
1
2
Variable
WAGE SECTOR
0
1
2
N
188
48
14
SE Mean
0.432
0.613
0.971
Mean
9.609
9.492
9.507 Median
8.500
9.245
10.375 TrMean
9.055
9.293
9.529 StDev
5.920
4.244
3.633
Minimum
1.750
3.000
3.750 Maximum
44.500
20.400
15.000 Q1
5.250
5.890
6.500 Q3
12.120
11.658
11.808
Data Display – Wages in sample from sector 0 (in ascending order)
\begin{verbatim}
1.75	4	5	5.87	7.5	8.75	9.86	12	15	22.5
3.35	4.13	5	6	7.53	8.75	10	12.16	15	22.5
3.35	4.17	5	6	7.67	8.8	10	12.5	15	24.98
3.4	4.25	5.13	6.25	7.69	8.85	10	12.5	15.56	24.98
3.45	4.25	5.2	6.25	7.75	8.89	10	12.65	15.79	24.98
3.5	4.28	5.21	6.25	7.8	8.9	10	12.67	15.95	24.98
3.5	4.35	5.25	6.25	7.88	8.93	10	13	16	26
3.5	4.35	5.25	6.25	7.96	9	10	13.12	16.14	44.5
3.5	4.5	5.3	6.25	8	9	10.2	13.16	16.65	
3.5	4.5	5.5	6.5	8	9	10.25	13.2	17.25	
3.51	4.5	5.5	6.67	8.2	9	10.5	13.33	18	
3.55	4.5	5.5	6.75	8.49	9	10.81	13.45	19.98	
3.56	4.55	5.5	6.85	8.5	9.15	11.11	13.45	19.98	
3.65	4.55	5.71	7	8.5	9.33	11.25	13.45	20	
3.65	4.55	5.71	7	8.5	9.37	11.25	13.65	20	
3.75	4.85	5.75	7	8.56	9.42	11.35	13.95	20.5	
3.75	5	5.75	7	8.63	9.5	11.79	14	20.55	
3.8	5	5.8	7.14	8.75	9.5	11.84	14.29	22.2	
3.84	5	5.8	7.5	8.75	9.6	12	14.53	22.2	
4	5	5.85	7.5	8.75	9.6	12	14.67	22.5	
\end{verbatim}
%---------------------------------------------------------------%
Data Display – Wages in sample from sector 1 (in ascending order)
3.00
5.40
7.78
10.50
11.71
19.00
3.35
5.65
8.40
10.53
12.00
19.98
4.00
5.77
8.50
10.58
12.00
20.40
4.50
6.25
8.89
10.62
12.00
4.62
6.50
9.00
11.00
12.50
4.80
6.67
9.24
11.00
13.89
4.85
6.75
9.25
11.25
15.00
4.95
6.80
10.00
11.32
15.38
5.10
7.00
10.00
11.50
16.42
Data Display – Wages in sample from sector 2 (in ascending order)
3.75
11.43
4.30
11.67
5.00
12.22
7.00
15.00
7.30
15.00
8.90
10.00
10.75
10.78
8

\end{framed}
%%%%%%%%%%%%%%%%%%%%%%%%%%%%%%%%%%%%%%%%%%%%%%%%%%%%%%%%%%%%%%%%%%%%%%%% 
\begin{enumerate}[(a)]
\item A trimmed mean is likely to have removed any major outliers, and in the case of a skew distribution it will be a better central measure than the mean of all the data.  Although hypothesis testing is still very approximate, descriptive statistics are improved. 
 
 
\item These particular sets of wage data appear to be skew, as would be anticipated, rather than having many obvious outliers.  In sector 0, the 44.50 and perhaps the 1.75 appear to be outliers, but there is doubt about regarding any others as such.  The other noticeable "gap" is between 16.42 and 19.00 in sector 1;  some people would argue for regarding the top three values in that sector as outliers. 
 
One convention is to regard as outliers any observations below Q1 – 1.5R or above Q3 + 1.5R, where R is the interquartile range (R = Q3 – Q1).  This would cover all above 22.4 in sector 0, i.e. the top nine;  only 20.4 in sector 1 (in spite of the obvious "gap" already mentioned);  and none in sector 2. 
 
The lowest value in sector 0 is suspect, but in a distribution of this shape no automatic calculation is likely to declare it to be an outlier. 
 
In the boxplots, the "whiskers" have been extended all the way to the minimum and maximum for each sector except for the lowest and highest values in sector 0. 
 
 
 
 
 
 
 
 
 
 
[Note.  The limits of electronic reproduction may mean that the boxplots will not appear in their correct locations with precise accuracy.] 
 
 
 
Continued on next page 
Sector 0
Sector 1
Sector 2
0 10 20 30 40 50 Wages 

 
 
\item Most of the available data are for sector 0.  The extreme values for this sector are under $2 and over $40, but there is only one very small and one very large value.  The median is below that for the other two sectors, indicating a general tendency towards lower payments.  The overall pattern is skew. 
 
Wages in sectors 1 and 2 show a rather similar pattern, but this is based on a much larger sample of data for sector 1 than for sector 2.  In sector 1, there are top values (three of them) around $19 or $20, and three of \$4 or less.  The three top values could be checked to see if they are indeed from this sector, or perhaps from a distinctly different sub-sector compared with the rest. 
 
Wages in sector 2 do not exceed $15 in these (few) data, but there are none below $3.75.  In fact there are only three below \$7.  This suggests that workers are on the whole better paid at the lower end of this sector but wages do not rise to the level of the other sectors. 

\end[enumerate}
\end{document} 

\documentclass[a4paper,12pt]{article}

%%%%%%%%%%%%%%%%%%%%%%%%%%%%%%%%%%%%%%%%%%%%%%%%%%%%%%%%%%%%%%%%%%%%%%%%%%%%%%%%%%%%%%%%%%%%%%%%%%%%%%%%%%%%%%%%%%%%%%%%%%%%%%%%%%%%%%%%%%%%%%%%%%%%%%%%%%%%%%%%%%%%%%%%%%%%%%%%%%%%%%%%%%%%%%%%%%%%%%%%%%%%%%%%%%%%%%%%%%%%%%%%%%%%%%%%%%%%%%%%%%%%%%%%%%%%

\usepackage{eurosym}

\usepackage{vmargin}

\usepackage{amsmath}
\usepackage{graphics}
\usepackage{epsfig}
\usepackage{enumerate}
\usepackage{multicol}
\usepackage{subfigure}
\usepackage{fancyhdr}
\usepackage{listings}
\usepackage{framed}
\usepackage{graphicx}
\usepackage{amsmath}

\usepackage{chngpage}

%\usepackage{bigints}



\usepackage{vmargin}

% left top textwidth textheight headheight

% headsep footheight footskip

\setmargins{2.0cm}{2.5cm}{16 cm}{22cm}{0.5cm}{0cm}{1cm}{1cm}

\renewcommand{\baselinestretch}{1.3}



\setcounter{MaxMatrixCols}{10}

\begin{document}

%-Higher Certificate, Paper I, 2004. Question 1
\begin{enumerate}

\item (a)
\[P(\mbox{all four favour the complex}) = (0.6)^4.\]
\[P(\mbox{all four oppose the complex}) = (0.3)^4.\]
\[P(\mbox{all four are indifferent}) = (0.1)4.\]
So 

\begin{eqnarray*}
P(\mbox{all four think alike}) 
&=& (0.6)^4 + (0.3)^4 + (0.1)^4 \\ &=& 0.1378.
\end{eqnarray*}


\item  P(an individual is not opposed) = 0.6 + 0.1 = 0.7.
So 

\begin{eqnarray*}
P(\mbox{none of the four is opposed}) &=& (0.7)^4\\ &=& 0.2401
\end{eqnarray*}

%%%%%%%%%%%%%%%%%%%%%%%%%%%%%%

\item  Possible favourable results are FFOI, FOOI, FOII, in any order.
\begin{itemize}
\item $P(FFOI) = (0.6)^2(0.3)(0.1) = 0.0108$
\item $P(FOOI) = (0.6)(0.3)^2(0.1) = 0.0054$
\item $P(FOII) = (0.6)(0.3)(0.1)^2 = 0.0018$
\end{itemize}

Each result can be arranged in 4!
2!1!1!
= 12 ways.

So overall probability is 12(0.0108 + 0.0054 + 0.0018) = 0.216.
\item \begin{itemize}
    \item From (a), P(all four in favour) = (0.6)^4 = 0.1296.
    \item From (b), P(none
opposed) = 0.2401. 
\item So the required conditional probability is
0.1296/0.2401 = 0.5398.
\end{itemize} 
\item The number in favour is binomially distributed with n = 4 and p = 0.6. 
\begin{itemize}
    \item So the
expectation (mean) is $4 \times 0.6 = 2.4$ 
\item the variance is $4 \times 0.6 \times 0.4 = 0.96$.
\end{itemize}
\item P(opposed) = P(opposedyoung)P(young) + P(opposedolder)P(older)
= (0.12 × 0.25) + (p × 0.75)
where p = P(opposedolder). But we are given that P(opposed) = 0.3. Hence
p = 0.36.
\item In samples of one "young" and three "olders",
\begin{eqnarray*}
P(exactly one opposes) &=& P("young" opposes, "olders" do not)
 + P("young" does not oppose, one "older" opposes)\\
&=& {(0.12)(0.64)3} + {3(0.88)(0.36)(0.64)2}\\ &=& 0.03146 + 0.38928\\
&=& 0.4207.
\end{eqnarray*}
\end{enumerate}
\end{document}

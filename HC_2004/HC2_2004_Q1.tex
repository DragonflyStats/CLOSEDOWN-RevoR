\documentclass[a4paper,12pt]{article}

%%%%%%%%%%%%%%%%%%%%%%%%%%%%%%%%%%%%%%%%%%%%%%%%%%%%%%%%%%%%%%%%%%%%%%%%%%%%%%%%%%%%%%%%%%%%%%%%%%%%%%%%%%%%%%%%%%%%%%%%%%%%%%%%%%%%%%%%%%%%%%%%%%%

\usepackage{eurosym}
\usepackage{vmargin}
\usepackage{amsmath}
\usepackage{graphics}
\usepackage{epsfig}
\usepackage{enumerate}
\usepackage{multicol}
\usepackage{subfigure}
\usepackage{fancyhdr}
\usepackage{listings}
\usepackage{framed}
\usepackage{graphicx}
\usepackage{amsmath}
\usepackage{chngpage}

%\usepackage{bigints}



\usepackage{vmargin}

% left top textwidth textheight headheight

% headsep footheight footskip

\setmargins{2.0cm}{2.5cm}{16 cm}{22cm}{0.5cm}{0cm}{1cm}{1cm}

\renewcommand{\baselinestretch}{1.3}

\setcounter{MaxMatrixCols}{10}

\begin{document}

Higher Certificate, Paper II, 2004. Question 2
(i) n = 16. Σxi = 49.4, Σxi
2 = 157.3; x = 3.0875, s2 = 0.3185.
We need to assume that diameters are Normally distributed.
A 95% confidence interval for the true mean of this grower's tomatoes is given by
x ± t s / 16 where t is the double-tailed 5% point of t15, i.e. 2.131. So the interval is
3.0875 ± 2.131 0.3185 /16 , i.e. 3.0875 ± 0.3007 , i.e. (2.787, 3.388).
As the specified mean of 3.0 is within this interval, it seems this grower could be
accepted.
It is also specified that the true variance σ 2 should not be greater than (0.5)2, which is
0.25. A 95% confidence interval for σ 2 is given by
( ) 2 ( ) 2
2
2 2
U L
1 1
χ χ
n s n s
σ
− −
< <
where 2
L χ and 2
U χ are the lower and upper 2½% points of 2
1 χn− , i.e. of 2
15 χ , which are
6.262 and 27.488. Thus the interval is 0.1738 < σ 2 < 0.7629, which is equivalent to
0.42 < σ < 0.87. This interval does contain the specified greatest value of 0.5 for σ,
but caution is suggested by the fact that the upper limit is well above 0.5; the grower
might well not be acceptable on this basis. (Note that the comparatively large value
of s2 has also affected the confidence interval for the mean calculated above – it is,
relatively speaking, rather a wide interval.)
The short report should say that although the mean diameter in the sample is near to
3.0, the material is so variable that the specified greatest value of 0.5 for the standard
deviation is quite likely to be exceeded, perhaps by a substantial amount. If the
directors are still interested, they should examine a larger sample.
The variability could well contain a large between-plant component, so a method
which mainly measures within-plant variation is not a good one – however quick and
easy it may be.\end{enumerate}
\end{document}

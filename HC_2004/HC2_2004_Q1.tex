\documentclass[a4paper,12pt]{article}

%%%%%%%%%%%%%%%%%%%%%%%%%%%%%%%%%%%%%%%%%%%%%%%%%%%%%%%%%%%%%%%%%%%%%%%%%%%%%%%%%%%%%%%%%%%%%%%%%%%%%%%%%%%%%%%%%%%%%%%%%%%%%%%%%%%%%%%%%%%%%%%%%%%

\usepackage{eurosym}
\usepackage{vmargin}
\usepackage{amsmath}
\usepackage{graphics}
\usepackage{epsfig}
\usepackage{enumerate}
\usepackage{multicol}
\usepackage{subfigure}
\usepackage{fancyhdr}
\usepackage{listings}
\usepackage{framed}
\usepackage{graphicx}
\usepackage{amsmath}
\usepackage{chngpage}

%\usepackage{bigints}



\usepackage{vmargin}

% left top textwidth textheight headheight

% headsep footheight footskip

\setmargins{2.0cm}{2.5cm}{16 cm}{22cm}{0.5cm}{0cm}{1cm}{1cm}

\renewcommand{\baselinestretch}{1.3}

\setcounter{MaxMatrixCols}{10}

\begin{document}

Higher Certificate, Paper II, 2004. Question 1


%%%%%%%%%%%%%%%%%%%%%%%%%%%%%%%%%%%%%%%%%%%%%%%%%%%%%%%%%%%%%%%%%%%%%%%  
\begin{table}[ht!]
 
\centering
 
\begin{tabular}{|p{15cm}|}
 
\hline  

 (i) State an analysis of variance model suitable for a randomised block design.  
Explain clearly what each term in the model represents and state any assumptions required for the analysis based on this model to be valid. 
 

\\ \hline
  
\end{tabular}

\end{table} 

\begin{table}[ht!]
 
\centering
 
\begin{tabular}{|p{15cm}|}
 
\hline  

 An experiment was conducted to investigate the toxic effects of four different chemical compounds A, B, C and D on the skin.  

Four adjacent regions, each a square of side 3 cm, were marked on the left forearm of each of six subjects, and each chemical was applied to each subject, 
choosing the sites of application of the chemicals at random for each subject.  

After three hours, the skin was examined and scored from 0 to 10 depending on the degree of irritation, with 0 representing no irritation and 10 severe irritation.  
The data are given in the following table. 

\begin{center}
\begin{tabular}{|c|c|c|c|c|c|} 


Subject 1 &  Subject 2 &  Subject 3&  Subject 4 &  Subject 5 & Subject 6 \\ \hline

D  5  &  A  7  &  B  2 &  B  4 & A  3 &  D  6  \\ \hline
B  3  & C  4 & A  1 & A  6&  B  1 & B  7 \\ \hline 
A  3  &   D  7 &  D  3 & C  6 &  D  2 &  C  3 \\ \hline
C  2  &  B  6 & C  1  & D  7 & C  2 & A  5  \\ \hline
\end{tabular}
\end{center}
 
Carry out a suitable analysis of these data.  Explain clearly your conclusions and comment on the differences, if any, between the toxic effects of the compounds.  Discuss briefly whether all the necessary assumptions can safely be made. (14) 
 

\\ \hline
  
\end{tabular}

\end{table} 
%%%%%%%%%%%%%%%%%%%%%%%%%%%%%%%%%%%%%%%%%%%%%%%%%%%%%%%%%%%%%%%%%%%%%%% 
\begin{enumerate}
\item , 1, 2, ..., , 1, 2, ..., , { } ~ ind N(0, 2 ) ij i j ij ij y = μ +τ +β +ε i = ν j = b ε σ .
There are ν treatments and b blocks. yij is the observation (response) on the unit (plot)
in block j which receives treatment i. 0, 0 i i j j Στ = Σ β = (i.e. fixed effects model).
μ is the overall population general mean, τ
i the population mean effect due to
treatment i, β
j the population mean effect due to block j. The Normally distributed
residual (error) terms εij all have variance σ 2 and are uncorrelated (independent). All
non-random variation is covered by the τ
i and β
j terms.
\item The "blocks" here are subjects 1 to 6. The "treatments" are compounds A to
D. In the notation of part (i), ν = 4 and b = 6.
Totals are: Block 1 Block 2 Block 3 Block 4 Block 5 Block 6
13 24 7 23 8 21
Treatment A Treatment B Treatment C Treatment D
25 23 18 30
The grand total is 96. ΣΣyij
2 = 486.
"Correction factor" is
962 384
24
= .
Therefore total SS = 486 – 384 = 102.
SS for blocks =
132 242 72 232 82 212 384 457 384 73
4 4 4 4 4 4
+ + + + + − = − = .
SS for treatments =
252 232 182 302 384 396.33 384 12.33
6 6 6 6
+ + + − = − = .
The residual SS is obtained by subtraction.
SOURCE DF SS MS F value
Blocks 5 73.00 14.600 13.14 Compare F5,15
Treatments 3 12.33 4.111 3.70 Compare F3,15
Residual 15 16.67 1.111 = σˆ 2
TOTAL 23 102.00

\begin{itemize}
    \item 
The upper 0.1\% point of F5,15 is 7.57; the blocks effect is very highly significant.
The upper 5\% point of F3,15 is 3.29; the treatments effect is significant.
Continued on next page
Clearly there are block (subject) differences. Even after removing these, the results
are quite variable.
To investigate treatment differences, first calculate the treatment means, which are (in
ascending order, for clarity)
C : 3.00 B : 3.8333 A : 4.1667 D : 5.00
The least significant difference between any pair of these means is
15 15
2 1.111 0.6086
6
t × = t where 15
2.131 at 5%
2.947 at 1%
4.073 at 0.1%
t

=


so the least significant differences are 1.30 for 5\%, 1.79 for 1% and 2.48 for 0.1%.
\item Thus the only apparent difference is between C and D, significant at the 1% level.
The results must be interpreted with caution.
The data are on a 10-point scale of integers, so obviously cannot have an underlying
Normal distribution. 
\item However, when the means of 6 replicates are being compared,
the (necessarily approximate) results should give a good guide to likely treatment (i.e.
compound) differences.
Some of the subjects are considerably more prone to irritation than others. 
\item Because of
this, the underlying variances might be different in some blocks from others. This
would be contrary to an assumption in the modelling, and would thus be a further
feature making the results only approximate.

\end{itemize}
\end{enumerate}
\end{document}

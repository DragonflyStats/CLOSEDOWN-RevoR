\documentclass[a4paper,12pt]{article}

%%%%%%%%%%%%%%%%%%%%%%%%%%%%%%%%%%%%%%%%%%%%%%%%%%%%%%%%%%%%%%%%%%%%%%%%%%%%%%%%%%%%%%%%%%%%%%%%%%%%%%%%%%%%%%%%%%%%%%%%%%%%%%%%%%%%%%%%%%%%%%%%%%%

\usepackage{eurosym}
\usepackage{vmargin}
\usepackage{amsmath}
\usepackage{graphics}
\usepackage{epsfig}
\usepackage{enumerate}
\usepackage{multicol}
\usepackage{subfigure}
\usepackage{fancyhdr}
\usepackage{listings}
\usepackage{framed}
\usepackage{graphicx}
\usepackage{amsmath}
\usepackage{chngpage}

%\usepackage{bigints}



\usepackage{vmargin}

% left top textwidth textheight headheight

% headsep footheight footskip

\setmargins{2.0cm}{2.5cm}{16 cm}{22cm}{0.5cm}{0cm}{1cm}{1cm}

\renewcommand{\baselinestretch}{1.3}

\setcounter{MaxMatrixCols}{10}

\begin{document}
Higher Certificate, Paper III, 2004.  Question 5 
 
 
(i) The total number of collisions rose steadily for the first three years and then fell in 1997.  Total casualties rose in 1995, then dropped in 1996 and then rose again in 1997.  Percentage changes from one year to the next were as follows. 
  1994 to 1995 1995 to 1996 1996 to 1997 Total collisions +6.9 +4.1 –2.8 Total casualties +6.0 –8.0 +4.2 
 
To consider seriousness of collisions, we might combine "fatal" and "major":- 
  1994 1995 1996 1997 Number 30734 32323 30557 30849 % of total collisions 6.6 6.5 5.9 6.1 
 The actual numbers were very similar except for the increase in 1995, but in the last two years they were a smaller proportion of the total. 
 
Similarly we might combine "killed" and "seriously injured", noting that the number of fatal casualties was least in 1997 whereas the number of seriously injured was least in 1994:- 
  1994 1995 1996 1997 Number 46529 50036 48321 48993 % of total casualties 33.6 34.1 35.8 34.8 
 The actual numbers increased in 1995 but then steadied off;  however, the percentages of the total increased for three years before falling slightly. 
 
Useful diagrams would be component bar charts showing the annual totals of collisions and of casualties, with the components in each category (fatal etc, or killed etc) shown in different shading or colouring.  Because annual changes are fairly small relative to total sizes, these diagrams will not show obvious or clear trends. 
 
 
(ii) Examples of useful background information are 
 • occupancy of vehicles involved – e.g. driver only, few passengers, many passengers as in coaches • type of vehicle – e.g. heavy lorry, car or other small vehicle (some collisions, e.g. between a lorry and a small car, seem more likely to cause fatalities) • traffic density at the time • numbers of registered vehicles in the year, indicating general level of road use / congestion • mileage travelled by drivers involved • ages of vehicles • ages of drivers and length of experience • roadworks or other local hazards. 
 

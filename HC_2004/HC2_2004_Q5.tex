\documentclass[a4paper,12pt]{article}

%%%%%%%%%%%%%%%%%%%%%%%%%%%%%%%%%%%%%%%%%%%%%%%%%%%%%%%%%%%%%%%%%%%%%%%%%%%%%%%%%%%%%%%%%%%%%%%%%%%%%%%%%%%%%%%%%%%%%%%%%%%%%%%%%%%%%%%%%%%%%%%%%%%

\usepackage{eurosym}
\usepackage{vmargin}
\usepackage{amsmath}
\usepackage{graphics}
\usepackage{epsfig}
\usepackage{enumerate}
\usepackage{multicol}
\usepackage{subfigure}
\usepackage{fancyhdr}
\usepackage{listings}
\usepackage{framed}
\usepackage{graphicx}
\usepackage{amsmath}
\usepackage{chngpage}

%\usepackage{bigints}



\usepackage{vmargin}

% left top textwidth textheight headheight

% headsep footheight footskip

\setmargins{2.0cm}{2.5cm}{16 cm}{22cm}{0.5cm}{0cm}{1cm}{1cm}

\renewcommand{\baselinestretch}{1.3}

\setcounter{MaxMatrixCols}{10}

\begin{document}

Higher Certificate, Paper II, 2004.  Question 5 
 
 
Part (i) 
 
(a) A type I error is to reject the null hypothesis, in favour of the alternative hypothesis, when in fact the null hypothesis is true. 
 
(b) A type II error is to fail to reject the null hypothesis when in fact the alternative hypothesis is true. 
 
(c) The level of significance of a test is the probability of rejecting the null hypothesis when in fact it is true, i.e. it is the probability of making a type I error.  It is conventionally denoted by α . 
 
(d) The power of a test is the probability of rejecting the null hypothesis, expressed as a function of the parameter (or equivalently, if it is not a test for a single parameter) being investigated.  So it is given by 1 – β , where β is the probability of making a type II error similarly expressed as a function. 
 
 
Part (ii) 
 
Let X represent the amount of coffee in a jar.  We have X ~ N( µ , 152).  The sample size is n = 9, so X ~ N( µ , 152/9).  Let Z ~ N(0, 1). 
 
 
(a) We have µ = 200. 
 () 190 200 190 2.0 15/3 P X P Z − < = < =−   = 0.02275. 
 () 210 200 210 2.0 15/3 P X P Z − > = > =   = 0.02275. 
 So the probability of committing a type I error is 0.02275 + 0.02275 = 0.0455. 
 
 
(b) Here µ = 216. 
 () 190 216 190 5.2 15/3 P X P Z − < = < =− =   ZERO to several decimal places. 
 () 210 216 210 1.2 0.1151 15/3 P X P Z − < = < =− =   . 
 So the total probability of accepting the output is 0.1151.  (This is the probability of a Type II error for this procedure, i.e. the value of β , for µ = 216.  Thus the power of the procedure when in fact µ = 216 is 1 – 0.1151 = 0.8849.)

\end{enumerate}
\end{document}

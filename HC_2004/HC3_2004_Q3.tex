\documentclass[a4paper,12pt]{article}

%%%%%%%%%%%%%%%%%%%%%%%%%%%%%%%%%%%%%%%%%%%%%%%%%%%%%%%%%%%%%%%%%%%%%%%%%%%%%%%%%%%%%%%%%%%%%%%%%%%%%%%%%%%%%%%%%%%%%%%%%%%%%%%%%%%%%%%%%%%%%%%%%%%

\usepackage{eurosym}
\usepackage{vmargin}
\usepackage{amsmath}
\usepackage{graphics}
\usepackage{epsfig}
\usepackage{enumerate}
\usepackage{multicol}
\usepackage{subfigure}
\usepackage{fancyhdr}
\usepackage{listings}
\usepackage{framed}
\usepackage{graphicx}
\usepackage{amsmath}
\usepackage{chngpage}

%\usepackage{bigints}



\usepackage{vmargin}

% left top textwidth textheight headheight

% headsep footheight footskip

\setmargins{2.0cm}{2.5cm}{16 cm}{22cm}{0.5cm}{0cm}{1cm}{1cm}

\renewcommand{\baselinestretch}{1.3}

\setcounter{MaxMatrixCols}{10}

\begin{document}

Higher Certificate, Paper III, 2004.  Question 3 
\begin{framed}
 
 3. A drug which was thought to affect the plasma cholesterol level of humans was tested in two experiments by comparing it with a placebo.  (A placebo is a chemically inert substance which is known to have no physical effect but which is similar in appearance to a conventional medicine.)  In experiment 1, 10 subjects chosen at random were given the drug on one day and the placebo a week later, by which time it was thought that any effect of the drug would have worn off.  Experiment 2 used 20 different subjects, also chosen at random.  Of these, ten were randomly allocated to the drug and the other ten were given the placebo at the same time.  Blood samples were taken from all subjects two hours after they had taken the drug or placebo, and the cholesterol levels were obtained. 
 
The table below shows the cholesterol levels of the 40 blood samples in mg/100ml. 
 
\begin{center}
\begin{tabular}{ccccccc}
Subject	&	Drug	&	Placebo	&	Subject	&	Drug	&	Subject	&	Placebo	\\ \hline 
A	&	196	&	192	&	K	&	203	&	U	&	168	\\ \hline 
B	&	190	&	187	&	L	&	197	&	V	&	174	\\ \hline 
C	&	155	&	149	&	M	&	210	&	W	&	205	\\ \hline 
D	&	199	&	200	&	N	&	153	&	X	&	251	\\ \hline 
E	&	190	&	183	&	O	&	197	&	Y	&	160	\\ \hline 
F	&	203	&	203	&	P	&	225	&	Z	&	173	\\ \hline 
G	&	237	&	242	&	Q	&	157	&	AA	&	162	\\ \hline 
H	&	202	&	194	&	R	&	236	&	BB	&	179	\\ \hline 
I	&	228	&	223	&	S	&	171	&	CC	&	202	\\ \hline 
J	&	212	&	207	&	T	&	22	&	DD	&	199	\\ \hline 
\end{tabular}
\end{center}
 
(i) Which of these two do you think is the better experimental design, and why? (3) 
 
(ii) Calculate a 95 per cent confidence interval for the difference between the mean cholesterol levels of subjects on drug and placebo using the results of experiment 1, stating any assumptions that you make. (7) 
\end{framed}

 
(i) By using the same subjects on both occasions, experiment 1 should give more precise results than experiment 2;  subject-to-subject variation has been designed out.  This assumes, of course, that any effect of the drug would indeed have worn off within the week. 
 
 
(ii) n = 10.  Differences di (drug – placebo) are 4, 3, 6, –1, 7, 0, –5, 8, 5, 5.  So we have 23.2, 16.40 d ds ==.  The required 95\% confidence interval is given by () 3.2 2.262 16.40/10 ±× where 2.262 is the double-tailed 5\% point of t9, i.e. the interval is (0.30, 6.10). We must assume that the differences are Normally distributed. 
 

\newpage
\begin{framed}
(iii) Calculate a 95 per cent confidence interval for the difference between the mean cholesterol levels of subjects on drug and placebo from the results of experiment 2, stating any assumptions that you make. (7) 
 
(iv) Does the drug appear to increase cholesterol level?  Justify your answer. 

\end{framed} 
(iii) Let x refer to the drug and y to the placebo.  We have nx = ny = 10.  Sample means and variances are 2197.1, 816.322 xsx == and 2187.3, 770.233 ysy ==.  We must assume that the two samples are from Normal distributions with the same variance. 
 
The pooled estimate of this common variance is 793.278.  The required 95\% confidence interval is given by () () () 11 10 10197.1 187.3 2.101 793.278 − ± × + where 2.101 is the double-tailed 5\% point of t18, i.e. the interval is (–16.66, 36.26). 
 
 
(iv) The interval in part (ii) does not contain zero;  both its end-points are positive.  This gives some evidence that there is an increase due to the drug.  The interval in part (iii) is uninformative, being very wide and well spread in both directions about zero;  subject-to-subject variation has not been designed out and thus inflates the estimate of variance. 
 \end{document}

Higher Certificate, Paper I, 2004. Question 5
(i) f (t ) =λ e−λ t , t > 0; λ > 0
(a) Sketch of f (t).
[NOTE. The curve should of course appear as a smooth decaying exponential;
it might not do so, due to the limits of electronic reproduction.]
(b) C.d.f. is ( ) ( ) 0
0
1 1
t
F t P T t tλ e λ vdv λ e λ v e λ t
λ
= ≤ = − = − −  = − −   ∫ .
(c) P(a < T ≤ b) = F (b) − F (a) = e−λ a − eλ b .
(ii) Assume all settlements of invoices are independent.
P(50 in first week) = { ( )} ( )50 50 F 1 = 1− e−λ , because T ≤ 1 for all these 50.
Likewise, 1 < T ≤ 2 for the 35 in the second week, so we have P(35 in second week) =
{ ( ) ( )}35 F 2 − F 1 = ( )e−λ − e−2λ 35 .
The remaining 15 have T > 2, which has probability 1 – P(T ≤ 2) = e−2λ , and thus
P(15 after week 2) = ( )e−2λ 15 .
The likelihood is therefore the product
( ) ( ) ( ) ( ) L λ = k 1− e−λ 50 e−λ − e−2λ 35 e−2λ 15
where k is a constant of proportionality.
Continued on next page
f (t)
t
Taking logarithms (base e),
log L(λ ) = log k + 50log(1− e−λ )+ 35log{e−λ (1− e−λ )}+15log (e−2λ )
= log k +85log (1− e−λ )− (35 + 30)λ = log k + 85log (1− e−λ )− 65λ .
log 85 65 85 65
1 1
d L e
d e e
λ
λ λ λ
−
− ∴ = − = −
− −
.
Equating to zero, 85 = 65(eλ −1) or eλ =150 / 65, so that λˆ = log (150 / 65) = 0.836 .
[It is easy to check that this is indeed a maximum; e.g. ( )
2
2 2
log 85 0
1
d L
dλ eλ
= − <
−
.]
(iii) 1− e−0.836 = 0.5666; e−0.836 − e−1.672 = 0.43344 − 0.18787 = 0.2456 . Hence, out
of 100 invoices, 56.66, 24.56 and 18.78 would be expected to be paid, on this model,
in weeks 1, 2 and later. The actual numbers were 50, 35 and 15. The prediction for
the second week is a long way from what happened, balanced by smaller
discrepancies in the other two periods. This does not seem very satisfactory.

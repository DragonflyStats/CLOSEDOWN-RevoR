\documentclass[a4paper,12pt]{article}

%%%%%%%%%%%%%%%%%%%%%%%%%%%%%%%%%%%%%%%%%%%%%%%%%%%%%%%%%%%%%%%%%%%%%%%%%%%%%%%%%%%%%%%%%%%%%%%%%%%%%%%%%%%%%%%%%%%%%%%%%%%%%%%%%%%%%%%%%%%%%%%%%%%

\usepackage{eurosym}
\usepackage{vmargin}
\usepackage{amsmath}
\usepackage{graphics}
\usepackage{epsfig}
\usepackage{enumerate}
\usepackage{multicol}
\usepackage{subfigure}
\usepackage{fancyhdr}
\usepackage{listings}
\usepackage{framed}
\usepackage{graphicx}
\usepackage{amsmath}
\usepackage{chngpage}

%\usepackage{bigints}



\usepackage{vmargin}

% left top textwidth textheight headheight

% headsep footheight footskip

\setmargins{2.0cm}{2.5cm}{16 cm}{22cm}{0.5cm}{0cm}{1cm}{1cm}

\renewcommand{\baselinestretch}{1.3}

\setcounter{MaxMatrixCols}{10}

\begin{document}Higher Certificate, Paper III, 2004.  Question 8 
 
 
(i) Observed and expected (on the null hypothesis of no association between class of degree and sex) frequencies, and the individual contributions of each cell to the X 2 statistic, are as follows. 
 
18 17   21.26 13.74   0.4999 0.7735 90 50   85.03 54.97   0.2905 0.4494 12 18   18.22 11.78   2.1234 3.2842 61 32   56.49 36.51   0.3601 0.5571 
 The test statistic is () 2 2 OEX E − =∑ = 8.34.  Refer this to 2 3χ .  The upper 5% point is 7.815, so the result is significant at the 5% level.  We have evidence to reject the null hypothesis  –  it seems there is a relation. 
 The individual contributions to X 2 show that the main contributions come from the cells for the 3rd class degree (the third row of the table), where we find fewer males and more females than would be expected.  This is also the case for 1st class degrees, balanced by the opposite being true for 2nd class and Pass degrees, but these cells do not make such a marked contribution.  These comments would be the substance of the report. 
 
 
(iii) We have ˆ p = 35/298 = 0.117 for this organisation.  The national population value of p is 0.083.  We want to test the null hypothesis that the proportion in this organisation is the same as the national value, against the alternative that this organisation's value is higher.  With a sample of size as large as 298, even with p as small as 0.083, a Normal approximation should be adequate.  So the test statistic (without continuity correction) is 
 
() ( )( ) ˆ 0.117 0.083 0.034 2.13 0.0161 / 0.083 0.917 /298 pp p p n −− = = = − , 
 
which is referred to N(0, 1) in a one-sided interpretation.  The result is significant at the 5% level (critical point 1.645) and approaching significance at the 1% level (critical point 2.326).  There is considerable evidence that this organisation's proportion is higher than the national value. 
 
The proportions of males and females with first class degrees are likely to be different, so to pool all the data into a single binomial distribution and test is not strictly correct. 
 
 
 
 
 
Continued on next page 

 
 
(iii) For this organisation, ˆ ˆ 90/181 0.497, 50/117 0.427 MF pp = = = =.  The estimated variance of ˆˆ MFp p− is given by 
 0.497 0.503 0.427 0.573 0.003472 181 117 ×× += . 
 
So the (Normal approximation) test statistic for testing the null hypothesis that the true values of pM and pF are equal is 
 0.497 0.427( 0) 1.19 0.003472 −− = . 
 
Referring this to N(0, 1), the result is not significant  –  there is no evidence to suggest that pM and pF are not equal. 
 
 
 
An alternative method for this part is a 2 × 2 contingency table.  The observed frequencies are 
 
 2nd class Other Male 90 91 Female 50 67 
 
and the expected frequencies if there is no association are 
 
 2nd class Other Male 85.034 95.966 Female 54.966 62.034 
 These give X 2 test statistic 1.39 (without use of Yates' correction) which, on reference to 2 1χ , is not significant. 

\documentclass[a4paper,12pt]{article}

%%%%%%%%%%%%%%%%%%%%%%%%%%%%%%%%%%%%%%%%%%%%%%%%%%%%%%%%%%%%%%%%%%%%%%%%%%%%%%%%%%%%%%%%%%%%%%%%%%%%%%%%%%%%%%%%%%%%%%%%%%%%%%%%%%%%%%%%%%%%%%%%%%%%%%%%%%%%%%%%%%%%%%%%%%%%%%%%%%%%%%%%%%%%%%%%%%%%%%%%%%%%%%%%%%%%%%%%%%%%%%%%%%%%%%%%%%%%%%%%%%%%%%%%%%%%

\usepackage{eurosym}
\usepackage{vmargin}
\usepackage{amsmath}
\usepackage{graphics}
\usepackage{epsfig}
\usepackage{enumerate}
\usepackage{multicol}
\usepackage{subfigure}
\usepackage{fancyhdr}
\usepackage{listings}
\usepackage{framed}
\usepackage{graphicx}
\usepackage{amsmath}
\usepackage{chngpage}

%\usepackage{bigints}
\usepackage{vmargin}

% left top textwidth textheight headheight

% headsep footheight footskip

\setmargins{2.0cm}{2.5cm}{16 cm}{22cm}{0.5cm}{0cm}{1cm}{1cm}

\renewcommand{\baselinestretch}{1.3}

\setcounter{MaxMatrixCols}{10}

\begin{document}
Higher Certificate, Paper I, 2004. Question 7
\begin{framed}
%%%%%%%%%%%%%%%%%%%%%%%%%%%%%%%%%
7. The lengths X of offcuts of timber in a carpenter's workshop follow the continuous
uniform distribution with probability density function (pdf)
f (x) 1 , 0 x θ
θ
= ≤ ≤ ,
where θ (> 0) is an unknown parameter.
(i) Find the mean and variance of X.


\end{framed}
\begin{enumerate}
\item (i) ( ) 2
0
0
1 1 1
2 2

%%%%%%%%%%%%%%%%%%%%%%%%%%%%%%%%%%%
\begin{eqnarray*}
E(X)  &=& \int^{\theta}_{0}  \frac{x}{\theta} dx \\
&=& \frac{1}{\theta} \left[ \frac{x^{2}}{2} \right]^{\theta}_{0}\\
&=& \frac{ \theta}{2}
\end{eqnarray*}

\begin{eqnarray*}
E(X^2)  &=& \int^{\theta}_{0} \frac{x^{2} }{\theta} dx \\
&=& \frac{1}{\theta} \left[ \frac{x^{3}}{3} \right]^{\theta}_{0}\\
&=& \frac{ \theta^2}{3}
\end{eqnarray*}

\begin{eqnarray*}
Var(X)  &=& E(X^2)  - \left[E(X)\right]^2 \\
&=& \frac{ \theta^2}{3}- \left[\frac{ \theta}{2}\right]^2  \\    
&=& \frac{ \theta^2}{3} - \frac{ \theta^2}{4}  \\   
&=& \frac{ \theta^2 }{12}
\end{eqnarray*}

\begin{eqnarray*}
F_X(x) &=& P(X \leq x)\\
&=& \int^{x}_{y}  f(u) du\\
% &=& \int^{\infty}_{1} x \frac{k}{x^2}f(x) dx\\
&=& \int^{\infty}_{1} x \frac{k}{x^{k+1}}f(x) dx\\
\end{eqnarray}


\item  P(longest offcut is $\leq x$) = P(all n offcuts are $\leq x$).
\begin{itemize}
    \item The c.d.f. for each Xi is ( ) ( ) 0
0
x



%%%%%%%%%%%%%%%%%%%%%%%%%%%%%%%%%%%%%%%%%%%%%%%%%%%%%
\begin{framed}
(5)
(ii) The carpenter takes a random sample of offcuts with lengths 1 2 , , ..., n X X X .
Explain why
(length of longest offcut in sample ) , 0
n P x x x θ
θ
≤ =   ≤ ≤  
 
,
and deduce the pdf of the sample maximum, (n) X say. Show that
( ) ( ) 1 n
E X n
n
= θ
+
and
( ) ( 1) ( 2)
Var 2
2
( ) + +
=
n n
X n n
θ .
Write down a multiple of (n) X which is an unbiased estimator of θ , and obtain
its variance.
(11)

\end{framed}
\item 


 Xi are all
independent. 
\item Therefore P(all n offcuts are \leq x) = { ( )}
n
F x n x
θ
=    
 
, and this is also
\item P(longest offcut is $\leq x$), i.e. the c.d.f. of the sample maximum (n) X .
\item Thus the p.d.f. of
(n) X is the derivative of this, i.e. ${ \displaystyle \frac{nx^{n–1}}{\theta^n}} $. 
This is for the interval $(0, \theta )$.
%%%%%%%%%%%%%%%%%%%%%%%%%%%%%%%%%%%
\begin{eqnarray*}
E(X_{n})  &=& \int^{\theta}_{0}  \frac{nx^{n} }{\theta^{n}} dx \\
&=& \frac{n}{\theta^{n}} \left[ \frac{x^{n+1}}{n+1} \right]^{\theta}_{0}\\
&=& \frac{n \theta}{n+1}
\end{eqnarray*}

\begin{eqnarray*}
E(X_{n}^2)  &=& \int^{\theta}_{0} \frac{nx^{n+1} }{\theta^{n}} dx \\
&=& \frac{n}{\theta^{n}} \left[ \frac{x^{n+2}}{n+2} \right]^{\theta}_{0}\\
&=& \frac{n \theta^2}{n+2}
\end{eqnarray*}

\begin{eqnarray*}
Var(X_{n})  &=& E(X_{n}^2)  - \left[E(X_{n})\right]^2 \\
&=& \frac{n^2 \theta}{n+2} - \left[\frac{n \theta}{n+1}\right]^2  \\    
&=& \frac{n^2 \theta}{n+2} - \frac{n^2 \theta^2}{(n+1)^2}  \\   
&=&  n \theta^2 \frac{[1 \times (n+1)^2] - [n \times (n+2)] }{(n+2)(n+1)^2}\\
&=& \frac{ n \theta^2 }{(n+2)(n+1)^2}
\end{eqnarray*}
%%%%%%%%%%%%%%%%%%%%%%%%%%%%%%%%%%%%%%%%%%%%%%%%%%%%%%%%
\item 
E X nx dx n x n
n n
θ
θ θ
θ θ
+  + 
= =   =  +  + ∫ .
( ) ( ) ( ) ( ) ( ) { ( )} ( )
2 2 2 2
2
2 
\end{itemize}
Var
2 1 n n n
X E X E X n n
n n
∴ = − = θ − θ
+ +
( ) ( )
( )( ) ( )( )
2 2
2
2 2
1 2
2 1 1 2
n n n n n
n n n n
θ θ
 + − + 
=   =
 + +  + +  
.
\begin{itemize}
%%%%%%%%%%%%%%%%%%%%%%%%%%%%%%%%%%%%%%%%%%%%
\newpage
\begin{framed}
(iii) Show that Σ=
n
i
i X
n 1
2 is the method of moments estimator of θ, and obtain the
variance of this estimator.
(4)
9
\end{framed}
\item Immediately we have ( )
1
n
E n X
n
 +  =θ
 
, 

\item so ( )
1
n
n X
n
+ is an unbiased estimator of θ.


\item ( ) ( )
( ) ( ) ( )
( )( ) ( )
2 2 2 2
2 2 2
1 1 1 Var Var
1 2 2 n
n n n n X n X
n n n n n n n
 +  = + = + θ = θ   +   + +
.
\end{itemize}
\item We have (see part (i)) that E(X) = θ /2. 
\item Thus the method of moments estimator
of θ /2 is X , and so the method of moments estimator of θ is 2X or 2
i X
n Σ as
required.
( ) ( ) ( )
2 4 4 2 2 Var Var 2 4Var Var .
12 3 iX X X X
n n n n
  = = = = θ =θ  
  Σ .
\end{enumerate}
\end{document}

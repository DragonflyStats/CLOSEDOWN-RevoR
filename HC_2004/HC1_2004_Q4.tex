\documentclass[a4paper,12pt]{article}

%%%%%%%%%%%%%%%%%%%%%%%%%%%%%%%%%%%%%%%%%%%%%%%%%%%%%%%%%%%%%%%%%%%%%%%%%%%%%%%%%%%%%%%%%%%%%%%%%%%%%%%%%%%%%%%%%%%%%%%%%%%%%%%%%%%%%%%%%%%%%%%%%%%%%%%%%%%%%%%%%%%%%%%%%%%%%%%%%%%%%%%%%%%%%%%%%%%%%%%%%%%%%%%%%%%%%%%%%%%%%%%%%%%%%%%%%%%%%%%%%%%%%%%%%%%%

\usepackage{eurosym}
\usepackage{vmargin}
\usepackage{amsmath}
\usepackage{graphics}
\usepackage{epsfig}
\usepackage{enumerate}
\usepackage{multicol}
\usepackage{subfigure}
\usepackage{fancyhdr}
\usepackage{listings}
\usepackage{framed}
\usepackage{graphicx}
\usepackage{amsmath}
\usepackage{chngpage}

%\usepackage{bigints}
\usepackage{vmargin}

% left top textwidth textheight headheight

% headsep footheight footskip

\setmargins{2.0cm}{2.5cm}{16 cm}{22cm}{0.5cm}{0cm}{1cm}{1cm}

\renewcommand{\baselinestretch}{1.3}

\setcounter{MaxMatrixCols}{10}

\begin{document}

Higher Certificate, Paper I, 2004. Question 4

\begin{enumerate}
\item  The binomial distribution with parameters n, p can be approximated by
$N(np, np(1 – p))$ when n is large and p is not too near to 0 or 1. 
\begin{itemize}
    \item As a "rule of thumb",
the approximation is likely to be good if both np and np(1 – p) are > 10.
\item Let $X ~ B(n, p)$ and let Φ denote the c.d.f. of N(0, 1). 

\item Using a continuity correction,
( )
( )
1
2
1
P X x x np
np p
 + −  ≤ ≈ Φ 
 −   
and ( )
( )
1
2
1
P X x x np
np p
 − −  < ≈ Φ 
 −   
.
\end{itemize}

The 95\% confidence interval for p uses the estimated variance $$\frac{\hat{p} (1− \hat{p} )}{n}$$ , giving the
approximate interval
ˆ (1 ˆ ) ˆ (1 ˆ )
ˆ 1.96 ˆ 1.96
p p p p
p p p
n n
− −
− < < +
The estimate of p is ˆ 30 0.6
50
p= = , so ˆ (1 ˆ ) (0.6)(0.4)
0.0693
50
p p
n
−
= = . Thus
the approximate interval is
\[0.6 − (1.96×0.0693) , 0.6 + (1.96×0.0693)\]
i.e. (0.464, 0.736).
\item  \[P(\mbox{neither hits}) = (1 – p)^2.\] Therefore 
\begin{eqnarray*}
P(at least 1 hit) &=& 1 – (1 – p)2 \\&=& p(2 – p).
\end{eqnarray*}
We estimate this by $(0.6)(2 – 0.6) = 0.84$.
\item  When p = 0.464 (lower limit of interval in part (i)), we have $p(2 – p) = 0.713$.

%=================%
\begin{itemize}
\item Similarly, when p = 0.736, we have p(2 – p) = 0.930. 
\item Thus (0.713, 0.930) is the
required interval.
\end{itemize}
%=================%
\item  When n pairs are fired, 
\[P(all miss) = [(1 – p)2]^n,\] estimated by (0.16)n. 
\begin{itemize}
    \item Hence
(0.16)n < 0.0005 is required. 
\item Solving this by taking logarithms to base 10, we have
n log10(0.16) < log10(0.0005), i.e. –0.79588n < –3.30103 which gives n > 4.148. 
\item so n
must be at least 5.
\end{itemize}

\end{enumerate}
\end{document}
Higher Certificate, Paper I, 2004. Question 4
(i) The binomial distribution with parameters n, p can be approximated by
N(np, np(1 – p)) when n is large and p is not too near to 0 or 1. As a "rule of thumb",
the approximation is likely to be good if both np and np(1 – p) are > 10.
Let X ~ B(n, p) and let Φ denote the c.d.f. of N(0, 1). Using a continuity correction,
( )
( )
1
2
1
P X x x np
np p
 + −  ≤ ≈ Φ 
 −   
and ( )
( )
1
2
1
P X x x np
np p
 − −  < ≈ Φ 
 −   
.
The 95% confidence interval for p uses the estimated variance pˆ (1− pˆ )/ n , giving the
approximate interval
ˆ (1 ˆ ) ˆ (1 ˆ )
ˆ 1.96 ˆ 1.96
p p p p
p p p
n n
− −
− < < +
The estimate of p is ˆ 30 0.6
50
p= = , so ˆ (1 ˆ ) (0.6)(0.4)
0.0693
50
p p
n
−
= = . Thus
the approximate interval is
0.6 − (1.96×0.0693) , 0.6 + (1.96×0.0693)
i.e. (0.464, 0.736).
(ii) P(neither hits) = (1 – p)2. Therefore P(at least 1 hit) = 1 – (1 – p)2 = p(2 – p).
We estimate this by (0.6)(2 – 0.6) = 0.84.
(iii) When p = 0.464 (lower limit of interval in part (i)), we have p(2 – p) = 0.713.
Similarly, when p = 0.736, we have p(2 – p) = 0.930. Thus (0.713, 0.930) is the
required interval.
(iv) When n pairs are fired, P(all miss) = [(1 – p)2]n, estimated by (0.16)n. Hence
(0.16)n < 0.0005 is required. Solving this by taking logarithms to base 10, we have
n log10(0.16) < log10(0.0005), i.e. –0.79588n < –3.30103 which gives n > 4.148. so n
must be at least 5.

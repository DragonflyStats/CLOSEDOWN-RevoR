Higher Certificate, Paper I, 2004. Question 3
Actual volume X ~ N(1010, 82). Let Z ~ N(0,1).
(i) ( 1000) 1000 1010 ( 1.25) 0.1056
8
P X < = P Z < −  = P Z < − =
 
.
(ii) Let Y be the total volume in a 6-pack.
We have Y ~ N(6 × 1010, 64 + 64 + 64 + 64 + 64 + 64), i.e. Y ~ N(6060, 384).
( 6000) 6000 6060 ( 3.06) 0.0011
384
P Y P Z −  P Z < =  <  = < − =
 
.
(Alternatively, could use X ~ N(1010, 64/6) and calculate P(X <1000).)
This probability is considerably smaller than that in part (i). In practical terms, this is
because there will be a tendency for heavier and lighter cartons in a 6-pack to balance
each other out. Alternatively, in terms of probability distributions, consider X and X :
X has the same mean as X but only one-sixth of the variance, so less of the lower tail
of the distribution of X is below the nominal volume of 1000.
(iii) The new volume W ~ N(μ, 42), where μ is the new mean. So we have that
( 1000) 1000
4
P W < = P Z < −μ 
 
. We require that this probability must be no greater
than 0.1056. Thus the cut-off point for Z is to be z = –1.25 (as before). Hence
1000 1.25
4
−μ = − , giving μ = 1005.
This means that 5 ml per carton could be saved, i.e. a cost saving per carton of
5
1000
× £1. To recover the £200, the number of cartons required is 200
5/1000
= 40000.

\documentclass[a4paper,12pt]{article}

%%%%%%%%%%%%%%%%%%%%%%%%%%%%%%%%%%%%%%%%%%%%%%%%%%%%%%%%%%%%%%%%%%%%%%%%%%%%%%%%%%%%%%%%%%%%%%%%%%%%%%%%%%%%%%%%%%%%%%%%%%%%%%%%%%%%%%%%%%%%%%%%%%%

\usepackage{eurosym}
\usepackage{vmargin}
\usepackage{amsmath}
\usepackage{graphics}
\usepackage{epsfig}
\usepackage{enumerate}
\usepackage{multicol}
\usepackage{subfigure}
\usepackage{fancyhdr}
\usepackage{listings}
\usepackage{framed}
\usepackage{graphicx}
\usepackage{amsmath}
\usepackage{chngpage}

%\usepackage{bigints}



\usepackage{vmargin}

% left top textwidth textheight headheight

% headsep footheight footskip

\setmargins{2.0cm}{2.5cm}{16 cm}{22cm}{0.5cm}{0cm}{1cm}{1cm}

\renewcommand{\baselinestretch}{1.3}

\setcounter{MaxMatrixCols}{10}

\begin{document}

Higher Certificate, Paper II, 2004. Question 2
%%%%%%%%%%%%%%%%%%%%%%%%%%%%%%%%%%%%%%%%%%%%%%%%%%%%%%%%%%%%%%%%%%%%%%% 

 
\begin{table}[ht!]
 
\centering
 
\begin{tabular}{|p{15cm}|}
 
\hline  

A supermarket has a policy of only buying tomatoes from growers who can supply tomatoes that have a mean diameter of 3.0 cm and a standard 
deviation of no more than 0.5 cm.  A representative of the supermarket goes to visit a potential new supplier and selects a random sample of 
16 tomatoes from the tomato grower's greenhouse.  

The diameter of each tomato is measured and the data are as follows, recorded for convenience in ascending order. 
    \[\{2.2   2.3   2.5   2.6   2.6   2.7   2.9   3.0   3.2   3.3   3.4   3.6   3.6   3.8   3.8   3.9 \}\]
By constructing suitable confidence intervals, analyse these data to establish whether the tomatoes provided by the grower 
will meet the supermarket's requirements, clearly stating any assumptions on which your analysis depends.  


Write a short report to the board of directors outlining your recommendations concerning whether or not to use this 
tomato grower to supply tomatoes for sale in the supermarket.  
\\ \hline
  
\end{tabular}

\end{table} 
%-----------------------------%
 
\begin{table}[ht!]
 
\centering
 
\begin{tabular}{|p{15cm}|}
 
\hline  



The supermarket representative suggests that the simplest way to select the sample would be to pick two tomato plants at random 
and select eight tomatoes at random from each.  Comment on the suitability of this method.  


\\ \hline
  
\end{tabular}

\end{table} 
%%%%%%%%%%%%%%%%%%%%%%%%%%%%%%%%%%%%%%%%%%%%%%%%%%%%%%%%%%%%%%%%%%%%%%% 

\begin{tabular}{ccccc}
$n = 16$ &  $\sum (x_i) = 49.4$ & $\sum (x_i)^2 = 157.3$ & $\bar{x} = 3.0875$ & $s^2 = 0.3185$.
\end{tabular}

\begin{itemize}
    \item We need to assume that diameters are Normally distributed.
A 95\% confidence interval for the true mean of this grower's tomatoes is given by
$x \pm t \frac{s}{\sqrt{16}}$ where $t$ is the double-tailed 5\% point of $t_{15}$, i.e. 2.131. 
\item So the interval is
\[3.0875 \pm 2.131 \frac{ 0.3185}{ \sqrt{16} }\] , i.e. $3.0875 \pm 0.3007$ , i.e. $(2.787, 3.388)$.
As the specified mean of 3.0 is within this interval, it seems this grower could be
accepted.
\item It is also specified that the true variance σ 2 should not be greater than $(0.5)^2$, which is
0.25. A 95\% confidence interval for $\sigma^2$ is given by
\[ frac{(n-1)s^2}{\chi^2_{L}} \;<\; \sigma^2  \;<\;  frac{(n-1)s^2}{\chi^2_{U}} \]

where $\chi^2_{L}$ and $\chi^2_{U}$ are the

L χ and 2
U χ are the lower and upper 2.5\% points of 2
1 χn− , i.e. of 2
15 χ , which are
6.262 and 27.488. 
\item Thus the interval is $0.1738 < \sigma^2 < 0.7629$, which is equivalent to
$0.42 < \sigma < 0.87$. 
\item This interval does contain the specified greatest value of 0.5 for $\sigma$,
but caution is suggested by the fact that the upper limit is well above 0.5; the grower
might well not be acceptable on this basis. 

\item Note that the comparatively large value
of $s^2$ has also affected the confidence interval for the mean calculated above – it is,
relatively speaking, rather a wide interval.)
\item The short report should say that although the mean diameter in the sample is near to
3.0, the material is so variable that the specified greatest value of 0.5 for the standard
deviation is quite likely to be exceeded, perhaps by a substantial amount. 
\item If the
directors are still interested, they should examine a larger sample.
The variability could well contain a large between-plant component, so a method
which mainly measures within-plant variation is not a good one – however quick and
easy it may be.
\end{itemize}
\end{enumerate}
\end{document}

\documentclass{article}
\usepackage[utf8]{inputenc}
\usepackage{enumerate}
\author{kobriendublin }
\date{December 2018}

\begin{document}

%- Higher Certificate, Module 5, 2009. Question 1
\section{Introduction}
\begin{enumerate}[(i)]
\item Higher Certificate, Module 5, 2009. Question 3
(i) ()1111!!kkkkkPXkCeCekkλλλλ∞∞∞−−===⎛⎞====⎜⎟⎝⎠ΣΣΣ
()()11CeeCeλλλ−−=−=−
(11Ceλ−− ∴=− as required.

%%%%%%%%%%%%%%%%%%%%%%%%%%%%%%%%%%%%%%%%%%%%%%
(ii) The likelihood is ()()()11!nnxniiiieeLPXxxλλλλ−−−Σ=−===ΠΠ.
∴Log likelihood is ()()()log1loglog!iinenxxλλλλ−=−−+Σ−Π��.
The maximum likelihood estimator ˆλ satisfies the equation 0ddλ=��, i.e. it satisfies 01ixneneλλλ−−Σ−−+=−.

%%%%%%%%%%%%%%%%%%%%%%%%%%%%%%%%%%%%%%%%%%%%%%
(iii) ()()()222211ieneneexddeλλλλλλλ−−−−−−−−Σ=−−�� ()221ixneeλλλ−−Σ=−−.
Now, ()()11!11kkeEXkkeeλλλλλ−∞−−===−−Σ.
()()()()222221111neednenEdeeeλλλλλλλλλλλ−−−−−−−−⎛⎞∴−=−+=⎜⎟−⎝⎠−−��.
��()()()2ˆˆˆˆ1Varˆ1eneeλλλλλλ−−−−∴≈−−.

%%%%%%%%%%%%%%%%%%%%%%%%%%%%%%%%%%%%%%%%%%%%%%
(iv) For n = 10 and ΣXi = 30, we have 1030101dedeλλλλ−−−=−+−��. The values of this at the stated points are:
λ
2.0
2.5
3.0
3.5
/dd�
3.43
1.11
–0.52
–1.74
See the graph below. From the graph, 0ddλ=�� at approximately 2.8λ=.
So 2.8 is (approximately) the required value of the maximum likelihood estimator (consideration of the gradient of /dd� shows that this is indeed a maximum).
Derivative of log(likelihood) versus λ
–2
–1
4.0
2.0
2.5
3.5
0
1
2
3
4
dd�
λ
\end{document}
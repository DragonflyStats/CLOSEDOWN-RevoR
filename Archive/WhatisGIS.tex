
A geographic information system (GIS) lets us visualise, question, analyse, interpret, and understand data in many ways that reveal relationships, patterns, and trends in the form of maps, globes, reports, and charts.

%-------------------------------------------------------------%

ntroduction

GIS map layersA geographic information system, or GIS, is a computerized data management system used to capture, store, manage, retrieve, analyze, and display spatial information. Data captured and used in a GIS commonly are represented on paper or other hard-copy maps. A GIS differs from other graphics systems in several respects. First, data are georeferenced to the coordinates of a particular projection system. This allows precise placement of features on the earth’s surface and maintains the spatial relationships between mapped features. As a result, commonly referenced data can be overlaid to determine relationships between data elements. For example, soils and wetlands for an area can be overlaid and compared to determine the correspondence between hydric soils and wetlands. Similarly, land use data for multiple time periods can be overlaid to determine the nature of changes that may have occurred since the original mapping. This overlay function is the basis of change detection studies across landscapes.

Second, GIS software use relational database management technologies to assign a series of attributes to each spatial feature. Common feature identification keys are used to link the spatial and attribute data between tables. A soil polygon, for example, can be linked to a series of database tables that define its mineral and chemical composition, crop yield, land use suitability, slope, and other characteristics.

Third, GIS provide the capability to combine various data into a composite data layer that may become a base layer in a database. For example, slope, soils, hydrography, demography, wetlands, and land use can be combined to develop a single layer of suitable hazardous waste storage sites. These data, in turn, may be incorporated into the permanent database of a local government and used for regulatory and planning decisions.

GIS software generally allow for two types of data. Some use raster data (i.e., discrete cells in a rigid row by column format), such as satellite imagery or aerial photography, while others use vectors (points, lines and polygons) to represent features on the earth’s surface. Most systems allow for full integration of both types of data. In either case, a fully functioning GIS allows the user to enter or digitize data that are georeferenced; link specific attributes to each feature using relational database management system technology; analyze relationship between various geographic features using a wide range of spatial operations and functions; and produce high-resolution images or graphics on color monitors or plotters.

\end{document}

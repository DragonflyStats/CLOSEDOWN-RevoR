\documentclass[a4paper,12pt]{article}
%%%%%%%%%%%%%%%%%%%%%%%%%%%%%%%%%%%%%%%%%%%%%%%%%%%%%%%%%%%%%%%%%%%%%%%%%%%%%%%%%%%%%%%%%%%%%%%%%%%%%%%%%%%%%%%%%%%%%%%%%%%%%%%%%%%%%%%%%%%%%%%%%%%%%%%%%%%%%%%%%%%%%%%%%%%%%%%%%%%%%%%%%%%%%%%%%%%%%%%%%%%%%%%%%%%%%%%%%%%%%%%%%%%%%%%%%%%%%%%%%%%%%%%%%%%%
\usepackage{eurosym}
\usepackage{vmargin}
\usepackage{amsmath}
\usepackage{graphics}
\usepackage{epsfig}
\usepackage{enumerate}
\usepackage{multicol}
\usepackage{subfigure}
\usepackage{fancyhdr}
\usepackage{listings}
\usepackage{framed}
\usepackage{graphicx}
\usepackage{amsmath}
\usepackage{chngpage}
%\usepackage{bigints}

\usepackage{vmargin}
% left top textwidth textheight headheight
% headsep footheight footskip
\setmargins{2.0cm}{2.5cm}{16 cm}{22cm}{0.5cm}{0cm}{1cm}{1cm}
\renewcommand{\baselinestretch}{1.3}
%- Higher Certificate, Module 2, 2009. Question 2
\setcounter{MaxMatrixCols}{10}
\begin{document}

\begin{enumerate}
    \item $X \sim N(52, 1)$ and $Y \sim N(26, 0.5625)$.
So the distribution of the total contents of a bottle, X + Y, is $N(78, 1.5625)$.
\item  ()(7578752.41.25PXY−⎛⎞+<=Φ=Φ−⎜⎟⎝⎠ = 0.0082
(where, as usual, $\Phi$ represents the standard Normal cdf).
\item  We want ()2.22.20XPPXY⎛⎞>=−>⎜⎟⎝⎠ .
Now, $X − 2.2Y \sim N(52 − (2.2×26)$, 1 + (2.22×0.5625)), i.e. N(−5.2, 3.7225).
()(0(5.2)2.20112.69523.7225PXY−−⎛⎞ ∴−>=−Φ=−Φ⎜⎟⎝⎠ = 0.00352 (approx).
Similarly, (1.81.80XPPXYY⎛⎞<=−<⎜⎟⎝⎠ and we have
$X − 1.8Y \sim N(52 − (1.8×26), 1 + (1.82×0.5625))$, i.e. $N(5.2, 2.8225)$.
()(05.21.803.09522.8225PXY−⎛⎞ ∴−<=Φ=Φ−⎜⎟⎝⎠ = 0.00098 (approximately).
P(ratio differs from 2 to 1 by more than 10\%) = P(X/Y < −1.8 or X/Y > 2.2)
= sum of the above two probabilities = 0.0045 approximately.
\item  Using the final answer of part (ii), the exact distribution of the number of bottles in 1000 with ratios different from 2 to 1 by more than 10\% is the binomial distribution B(1000, 0.0045).
A suitable approximation is Poisson(4.5).

From the cumulative Poisson tables, the probability of 10 or more such bottles is 1 – 0.9829 = 0.017 to 3 decimal places.

\end{enumerate}

\end{document}

\documentclass[a4paper,12pt]{article}
%%%%%%%%%%%%%%%%%%%%%%%%%%%%%%%%%%%%%%%%%%%%%%%%%%%%%%%%%%%%%%%%%%%%%%%%%%%%%%%%%%%%%%%%%%%%%%%%%%%%%%%%%%%%%%%%%%%%%%%%%%%%%%%%%%%%%%%%%%%%%%%%%%%%%%%%%%%%%%%%%%%%%%%%%%%%%%%%%%%%%%%%%%%%%%%%%%%%%%%%%%%%%%%%%%%%%%%%%%%%%%%%%%%%%%%%%%%%%%%%%%%%%%%%%%%%
\usepackage{eurosym}
\usepackage{vmargin}
\usepackage{amsmath}
\usepackage{graphics}
\usepackage{epsfig}
\usepackage{enumerate}
\usepackage{multicol}
\usepackage{subfigure}
\usepackage{fancyhdr}
\usepackage{listings}
\usepackage{framed}
\usepackage{graphicx}
\usepackage{amsmath}
\usepackage{chngpage}
%\usepackage{bigints}

\usepackage{vmargin}
% left top textwidth textheight headheight
% headsep footheight footskip
\setmargins{2.0cm}{2.5cm}{16 cm}{22cm}{0.5cm}{0cm}{1cm}{1cm}
\renewcommand{\baselinestretch}{1.3}

\setcounter{MaxMatrixCols}{10}

\section{Introduction}
%- Higher Certificate, Module 2, 2007. Question 4


The probability of observing k events in an interval is given by the equation 
\[ {\displaystyle P(k{\text{ events in interval}})=e^{-\lambda }{\frac {\lambda ^{k}}{k!}}} \]


Then ${\displaystyle \lambda =rt} $
 (with 
$ {\displaystyle r} $
 in units of 1/time), and 
 \[{\displaystyle P(k{\text{ events in interval }}t)=e^{-rt}{\frac {(rt)^{k}}{k!}}} \]

\begin{enumerate}[(i)]
    \item  Since A1, A2, …, Ak are mutually exclusive, ()()()1kjjjPBPBAPA==Σ.
()()()()()()()()()()1iiiiiikjjjPBAPAPBAPAPABPABPBPBPBAPA=∩∴===Σ.

%%%%%%%%%%%%%%%%%%%%%%%%%%
\item P(P) = 1/6, P(Q) = 1/3, P(R) = 1/2. P, Q, and R are mutually exclusive and exhaustive purchases.
We have Poisson distributions with PQR3,2,1λλλ===.
()22222flawsQ22!ePe−−== and ()121.112flawsR2!2ePe−−==.
\begin{itemize}
\item We have ()()()()()P,Q,R2flawsQQQ2flaws2flawsiPPPPiP==Σ
and similarly ()()()()()P,Q,R2flawsRRR2flaws2flawsiPPPPiP==Σ .

\item So the ratio ()()()()()()2112Q2flaws2flawsQQ830.981113R2flaws2flawsRR22ePPPePPPe−−×===× .
so it is (very slightly) more likely to have come from R.
\end{itemize}
%%%%%%%%%%%%%%%%%%%%%%%%%%%%%5
\item  (){}()2222flawsQ2P−= is the probability that two (independent) lengths from Q are faulty, and similarly the corresponding probability for R is (){}22112flawsR2Pe−⎛⎞=⎜⎟⎝⎠.
Now we need the ratio
(){}()(){}()222flawsbothboth the same2flawsbothboth the samePQPQPRPR
422219111369414111369424446470.9624991144eeee−−×××===××++++ .
\begin{itemize}
\item Again supplier R is slightly more likely to be the source than supplier Q in these circumstances. 
\item In fact R is slightly more likely to be the supplier in this case than in the case of part (ii).
\item These results might be counter-intuitive in that Q's material is more faulty than R's. However, this has to be balanced against the fact that more material comes from R than from Q.
\end{enumerate}
\end{enumerate}

\end{document}

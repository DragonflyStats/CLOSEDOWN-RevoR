\documentclass{article}
\usepackage[utf8]{inputenc}
\usepackage{framed}
\usepackage{enumerate}
\title{RSS_Jan_2019_Mod2}
\author{kobriendublin }
\date{December 2018}

\begin{document}

\maketitle

\section{Introduction}

\begin{enumerate}
    \item 
Higher Certificate, Module 2, 2007. Question 4
(i) Since A1, A2, …, Ak are mutually exclusive, ()()()1kjjjPBPBAPA==Σ.
()()()()()()()()()()1iiiiiikjjjPBAPAPBAPAPABPABPBPBPBAPA=∩∴===Σ.
(ii) P(P) = 1/6, P(Q) = 1/3, P(R) = 1/2. P, Q, and R are mutually exclusive and exhaustive purchases.
We have Poisson distributions with PQR3,2,1λλλ===.
()22222flawsQ22!ePe−−== and ()121.112flawsR2!2ePe−−==.
We have ()()()()()P,Q,R2flawsQQQ2flaws2flawsiPPPPiP==Σ
and similarly ()()()()()P,Q,R2flawsRRR2flaws2flawsiPPPPiP==Σ .
So the ratio ()()()()()()2112Q2flaws2flawsQQ830.981113R2flaws2flawsRR22ePPPePPPe−−×===× .
so it is (very slightly) more likely to have come from R.
Solution continued on next page
(iii) (){}()2222flawsQ2P−= is the probability that two (independent) lengths from Q are faulty, and similarly the corresponding probability for R is (){}22112flawsR2Pe−⎛⎞=⎜⎟⎝⎠.
Now we need the ratio
(){}()(){}()222flawsbothboth the same2flawsbothboth the samePQPQPRPR
422219111369414111369424446470.9624991144eeee−−×××===××++++ .
Again supplier R is slightly more likely to be the source than supplier Q in these circumstances. In fact R is slightly more likely to be the supplier in this case than in the case of part (ii).
These results might be counter-intuitive in that Q's material is more faulty than R's. However, this has to be balanced against the fact that more material comes from R than from Q.
\end{enumerate}

\end{document}
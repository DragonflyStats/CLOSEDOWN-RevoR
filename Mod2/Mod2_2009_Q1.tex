\documentclass[a4paper,12pt]{article}
%%%%%%%%%%%%%%%%%%%%%%%%%%%%%%%%%%%%%%%%%%%%%%%%%%%%%%%%%%%%%%%%%%%%%%%%%%%%%%%%%%%%%%%%%%%%%%%%%%%%%%%%%%%%%%%%%%%%%%%%%%%%%%%%%%%%%%%%%%%%%%%%%%%%%%%%%%%%%%%%%%%%%%%%%%%%%%%%%%%%%%%%%%%%%%%%%%%%%%%%%%%%%%%%%%%%%%%%%%%%%%%%%%%%%%%%%%%%%%%%%%%%%%%%%%%%
\usepackage{eurosym}
\usepackage{vmargin}
\usepackage{amsmath}
\usepackage{graphics}
\usepackage{epsfig}
\usepackage{enumerate}
\usepackage{multicol}
\usepackage{subfigure}
\usepackage{fancyhdr}
\usepackage{listings}
\usepackage{framed}
\usepackage{graphicx}
\usepackage{amsmath}
\usepackage{chngpage}
%\usepackage{bigints}

\usepackage{vmargin}
% left top textwidth textheight headheight
% headsep footheight footskip
\setmargins{2.0cm}{2.5cm}{16 cm}{22cm}{0.5cm}{0cm}{1cm}{1cm}
\renewcommand{\baselinestretch}{1.3}

\setcounter{MaxMatrixCols}{10}
\begin{document}

%- Higher Certificate, Module 2, 2009. Question 1

\section{Introduction}

\begin{enumerate}
    \item The number of different poker hands is
    \[ {52 \choose 5}  = \frac{52!}{5! \times 47!}  = \frac{52 \times 51 \times 50 \times 49 \times 48 \times 47!}{5! \times 47!} \]\
   
= 2598960.
%%%%%%%%%%%%%%%%%%%%%%%%%%%%%%%%%
\item The face value of the pair can be chosen in 13 ways, and when this has been done the face value of the triple can be chosen in 12 ways. 
\begin{itemize}
\item Since "AABBB" and "AAABB" are different, the total number of combinations of face values yielding different full house hands is $13\times 12 = 156$.

\item For any one of these combinations, the suits of the pair can be chosen in = 6 ways and the suits of the triple can be chosen in = 4 ways. 

\item Hence there are $6\times 4 = 24$ ways of choosing the suits for a given combination of face values. It follows that there are $24 \times 156 = 3744$ possible different full house hands.
\end{itemize}

⎟⎟⎠⎞⎜⎜⎝⎛24⎟⎟⎠⎞⎜⎜⎝⎛34
As all hands are equiprobable, the chance of a full house is therefore 4165625989603744= = 0.00144 to 3 significant figures.
%%%%%%%%%%%%%%%%%%%%%%%%%%%%%%%%
\item  As in part (ii), the face value of the pair can be chosen in 13 ways. The suits of the pair can then be chosen in = 6 ways. ⎟⎟⎠⎞⎜⎜⎝⎛24
The face values of the remaining 3 cards can be chosen in = 220 ways. The suits of each of these can be chosen in = 4 ways, so altogether there are 4⎟⎟⎠⎞⎜⎜⎝⎛312⎟⎟⎠⎞⎜⎜⎝⎛343 = 64 different sets of three cards with a given set of different face values.

Putting these results together, we have that there are $13×6×220×64 (= 1098240)$ possible different "one pair" hands. 

As all hands are equiprobable, the chance of a "one pair" hand is
\[136220643522598960833×××= = 0.423\] to 3 significant figures.\end{enumerate}

\end{document}

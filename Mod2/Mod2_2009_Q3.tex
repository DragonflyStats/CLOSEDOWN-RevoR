\documentclass[a4paper,12pt]{article}
%%%%%%%%%%%%%%%%%%%%%%%%%%%%%%%%%%%%%%%%%%%%%%%%%%%%%%%%%%%%%%%%%%%%%%%%%%%%%%%%%%%%%%%%%%%%%%%%%%%%%%%%%%%%%%%%%%%%%%%%%%%%%%%%%%%%%%%%%%%%%%%%%%%%%%%%%%%%%%%%%%%%%%%%%%%%%%%%%%%%%%%%%%%%%%%%%%%%%%%%%%%%%%%%%%%%%%%%%%%%%%%%%%%%%%%%%%%%%%%%%%%%%%%%%%%%
\usepackage{eurosym}
\usepackage{vmargin}
\usepackage{amsmath}
\usepackage{graphics}
\usepackage{epsfig}
\usepackage{enumerate}
\usepackage{multicol}
\usepackage{subfigure}
\usepackage{fancyhdr}
\usepackage{listings}
\usepackage{framed}
\usepackage{graphicx}
\usepackage{amsmath}
\usepackage{chngpage}
%\usepackage{bigints}

\usepackage{vmargin}
% left top textwidth textheight headheight
% headsep footheight footskip
\setmargins{2.0cm}{2.5cm}{16 cm}{22cm}{0.5cm}{0cm}{1cm}{1cm}
\renewcommand{\baselinestretch}{1.3}

\setcounter{MaxMatrixCols}{10}
\begin{document}

%- Higher Certificate, Module 2, 2009. Question 3
\section{Introduction}


\[ {\displaystyle {\frac {\lambda ^{k}e^{-\lambda }}{k!}}} \]

\begin{enumerate}
    \item Important Identity
    \[()()()()2011!xxeEXEXXXEXxxx\lambda\lambda−∞==−+=+−⎡⎤⎣⎦Σ
()()2222201.2!!xyxyeexy\lambda\lambda\lambda\lambda\lambda\lambda\lambda\lambda\lambda\lambda\lambda\lambda−∞∞−−===+=+=+=+−ΣΣ\]
\[\therefore Var(X) = E(X^ 2) - {E(X)}^2 = \lambda(\lambda + 1) − \lambda^2 = \lambda.\]

%%%%%%%%%%%%%%%%%%%%%%%%%%%%%%%%%%%%
\item Since X and Y are independent, we have, for w = 0, 1, 2, …,
()()()()00.!!xwxwwxxeePWwPXxPYwxxwx\lambda\mu\lambda\mu−−======−=−ΣΣ
()()()()0!!!wxwxwxeexwxw\lambda\mu\lambda\mu\lambda\mu\lambda\mu−−+−+=+==−Σ,
noting that
()( 0!!!xwxwwxwxwx\lambda\mu\lambda\mu−=+−Σ
011xwwxwx\lambda\lambda\lambda\mu\lambda\mu−=⎛⎞⎛⎞⎛⎞ =−=⎜⎟⎜⎟⎜⎟++⎝⎠⎝⎠⎝⎠Σ
(by consideration of the binomial distribution).

\begin{itemize}
\item Hence $W \sim$ Poisson$(\lambda + \mu)$. Thus, from the question and part (i), $E(W) = Var(W) = \lambda + \mu$.
\item 
Since $V$ is also the sum of independent Poison(\lambda) and Poisson($\mu$) variables, $V \sim$ Poisson$(\lambda + \mu)$ by the above argument, i.e. $V$ and $W$ have the same distribution.
\end{itemize}
%%%%%%%%%%%%%%%%%%%%%%%%%%%%%%%%%%%%
\item  Since $W (= X + Y)$ and Z are independent, $T = W - Z$ is the difference of two independent Poisson variables, i.e. \[Poisson(\lambda + \mu) - Poisson(\lambda). \]

\begin{itemize}
\item It follows that $E(T) = \lambda + \mu − \lambda = \mu$, $Var(T) = Var(W) + Var(Z) = \lambda + \mu + \lambda = 2\lambda + \mu$.
\item However, $U = V - Z = Y + Z - Z = Y$, so $U \sim Poisson(\mu)$ and therefore $E(U) = Var(U) = \mu$.
\item $P(U < 0) = 0$ because a Poisson variable cannot be negative, but the difference of two independent Poisson variables can be negative; for example, $X + Y = 1$ and $Z = 2$ arises with positive probability $22ee\lambda\lambda\mu\lambda−−−×$, and this gives $T = −1$.
\end{itemize}
%%%%%%%%%%%%%%%%%%%%%%%%%%%%%%%%%
\end{enumerate}
\end{document}

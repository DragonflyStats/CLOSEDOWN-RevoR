\documentclass[a4paper,12pt]{article}
%%%%%%%%%%%%%%%%%%%%%%%%%%%%%%%%%%%%%%%%%%%%%%%%%%%%%%%%%%%%%%%%%%%%%%%%%%%%%%%%%%%%%%%%%%%%%%%%%%%%%%%%%%%%%%%%%%%%%%%%%%%%%%%%%%%%%%%%%%%%%%%%%%%%%%%%%%%%%%%%%%%%%%%%%%%%%%%%%%%%%%%%%%%%%%%%%%%%%%%%%%%%%%%%%%%%%%%%%%%%%%%%%%%%%%%%%%%%%%%%%%%%%%%%%%%%
\usepackage{eurosym}
\usepackage{vmargin}
\usepackage{amsmath}
\usepackage{graphics}
\usepackage{epsfig}
\usepackage{enumerate}
\usepackage{multicol}
\usepackage{subfigure}
\usepackage{fancyhdr}
\usepackage{listings}
\usepackage{framed}
\usepackage{graphicx}
\usepackage{amsmath}
\usepackage{chngpage}
%\usepackage{bigints}

\usepackage{vmargin}
% left top textwidth textheight headheight
% headsep footheight footskip
\setmargins{2.0cm}{2.5cm}{16 cm}{22cm}{0.5cm}{0cm}{1cm}{1cm}
\renewcommand{\baselinestretch}{1.3}

\setcounter{MaxMatrixCols}{10}

\section{Higher Certificate, Module 2, 2008. Question 4 }
\begin{enumerate}

%%%%%%%%%%%%%%%%%%%%%%%%%%%%%%%%%%
\item  \[()()()111!1!1xxXXeepxpxxxxx\lambda\lambda\lambda\lambda\lambda\lambda−+−+===+++.\]
This can be used recursively to find the probability mass function. Start with \[(0)Xpe\lambda−=; then (1)(0)XXpp \lambda\lambda\lambda−==, 2(2)(/2)(1)(/2)XXpp \lambda\lambda\lambda−==\], and so on.
%%%%%%%%%%%%%%%%%%%%%%%%%%%%%%%%%%
\item Expected Value  \[E(X) = ()()10110!1!1!!xxxyxxxyeeeexxxxy\lambda\lambda\lambda\lambda\lambda\lambda\lambda\lambda\lambda\lambda\lambda−−−−−∞∞∞∞=======−−ΣΣΣΣ ,\]
putting y = x − 1 in the last summation and noticing that this re-creates the probability mass function. Similarly,
\[E[X(X − 1)] =()()()220221!2!2!x\lambda − ,\]
putting y = x − 2 in the last summation.
Hence \[Var(X) = E[X(X − 1)] + E(X) − {E(X)}2 = \lambda^2 + \lambda − \lambda^2 = \lambda\], as required.
%%%%%%%%%%%%%%%%%%%%%%%%%%%%%%%%%%
\item P(W = w) = ()()()()00!!!wxwxwwxwxxxweeeexxwxww\lambdaμ\lambdaμ\lambdaμ \lambdaμ\lambdaμ\lambdaμ−+−−−−+−==+⎛⎞×==⎜⎟−⎝⎠ΣΣ,
confirming that $W ~ Poisson(\lambda + μ)$. Since the general parameter \lambda has been shown in part (ii) to represent the mean, it follows that $E(W) = \lambda + μ$.
%%%%%%%%%%%%%%%%%%%%%%%%%%%%%%%%%%
    \item (a) 
\begin{eqnarray} 
P(exactly one breakdown)
&=& P(\mbox{A fails once}, \mbox{B does not fail}}) + P(\mbox{B fails once}, \mbox{A does not fail}})\\
&=& P(\mbox{A fails once}) × P(\mbox{B does not fail}})
+ P(\mbox{B fails once}) × P(\mbox{A does not fail}})\\
&=& (\lambdae−\lambda×e−μ) + (μe−μ×e−\lambda) = (\lambda + μ)e−(\lambda + μ).
\end{eqnarray}
∴the required conditional probability is
()( fails once, does not fail}})()PABe\lambdaμ\lambdaμ−++
()()()2eee\lambdaμ\lambdaμ\lambda\lambda\lambdaμ\lambdaμ−−−+×==++ = 0.8.
Solution continued on next page
%%%%%%%%%%%%%%%%%%%%%%%%%%%%%%%%%%
    \item W = total number of breakdowns ~ Poisson(2.5).
∴ \[P(W > 2) = 1 − P(W ≤ 2)\]
()2.52112.5(2.5/2)e−=−++
2.516.625e−=− = 1 – 0.5438 = 0.456
(alternatively, this can be obtained from the cumulative Poisson probabilities in the Society's Statistical tables for use in examinations).
%%%%%%%%%%%%%%%%%%%%%%%%%%%%%%%%%%
    \item $T ~ Poisson(50×2.5)$ or $Poisson(125)$, which we approximate by $N(125, 125)$.
The upper 5% point of N(125, 125) is 125 + 1.6449√125 = 143.4.
Since T0.95 must be an integer and the question says "will be exceeded on at most 5% of days", we round up to T0.95 = 144.
\end{enumerate}

\end{document}

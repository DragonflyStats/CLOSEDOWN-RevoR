\documentclass{article}
\usepackage[utf8]{inputenc}
\usepackage{framed}
\usepackage{enumerate}
\title{RSS_Jan_2019_Mod2}
\author{kobriendublin }
\date{December 2018}

\begin{document}

\maketitle

\section{Introduction}

\begin{enumerate}
    \item 
\end{enumerate}
Higher Certificate, Module 2, 2008. Question 4
%%%%%%%%%%%%%%%%%%%%%%%%%%%%%%%%%%
    \item  ()()()111!1!1xxXXeepxpxxxxxλλλλλλ−+−+===+++.
This can be used recursively to find the probability mass function. Start with (0)Xpeλ−=; then (1)(0)XXpp λλλ−==, 2(2)(/2)(1)(/2)XXpp λλλ−==, and so on.
%%%%%%%%%%%%%%%%%%%%%%%%%%%%%%%%%%
    \item  E(X) = ()()10110!1!1!!xxxyxxxyeeeexxxxyλλλλλλλλλλλ−−−−−∞∞∞∞=======−−ΣΣΣΣ ,
putting y = x − 1 in the last summation and noticing that this re-creates the probability mass function. Similarly,
E[X(X − 1)] =()()()220221!2!2!xλ − ,
putting y = x − 2 in the last summation.
Hence Var(X) = E[X(X − 1)] + E(X) − {E(X)}2 = λ2 + λ − λ2 = λ, as required.
(iii) P(W = w) = ()()()()00!!!wxwxwwxwxxxweeeexxwxwwλμλμλμ λμλμλμ−+−−−−+−==+⎛⎞×==⎜⎟−⎝⎠ΣΣ,
confirming that W ~ Poisson(λ + μ). Since the general parameter λ has been shown in part (ii) to represent the mean, it follows that E(W) = λ + μ.
(iv) (a) P(exactly one breakdown)
= P(A fails once, B does not fail) + P(B fails once, A does not fail)
= P(A fails once) × P(B does not fail)
+ P(B fails once) × P(A does not fail)
= (λe−λ×e−μ) + (μe−μ×e−λ) = (λ + μ)e−(λ + μ).
∴the required conditional probability is
()( fails once, does not fail)()PABeλμλμ−++
()()()2eeeλμλμλλλμλμ−−−+×==++ = 0.8.
Solution continued on next page
(b) W = total number of breakdowns ~ Poisson(2.5).
∴ P(W > 2) = 1 − P(W ≤ 2)
()2.52112.5(2.5/2)e−=−++
2.516.625e−=− = 1 – 0.5438 = 0.456
(alternatively, this can be obtained from the cumulative Poisson probabilities in the Society's Statistical tables for use in examinations).
(c) T ~ Poisson(50×2.5) or Poisson(125), which we approximate by N(125, 125).
The upper 5% point of N(125, 125) is 125 + 1.6449√125 = 143.4.
Since T0.95 must be an integer and the question says "will be exceeded on at most 5% of days", we round up to T0.95 = 144.
\end{enumerate}

\end{document}
\end{document}
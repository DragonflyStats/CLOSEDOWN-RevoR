\documentclass{article}
\usepackage[utf8]{inputenc}
\usepackage{framed}
\usepackage{enumerate}
\title{RSS_Jan_2019_Mod2}
\author{kobriendublin }
\date{December 2018}

\begin{document}

%- Higher Certificate, Module 2, 2007. Question 2
\section{Introduction}

\begin{enumerate}
    \item 

(),0,xfxex\lambda\lambda\lambda− =≥>
%%%%%%%%%%%%%%%%%%%%%%%%%%%%%%%%%%%
\item  ()00001()0xxxxeEXxedxxeedx\lambda\lambda\lambda\lambda\lambda\lambda\lambda∞−∞∞∞−−−⎡⎤⎡⎤==−−−=+⎢⎥⎣⎦−⎣⎦∫∫ .
()2220002xxEXxedxxexedx\lambda\lambda\lambda∞∞∞−−⎡⎤==−+⎣⎦∫∫
()2220EX\lambda\lambda=+=.
Hence ()[]2222211Var()()XEXEX \lambda\lambda\lambda=−=−=, and SD(X) = 1.\lambda
%%%%%%%%%%%%%%%%%%%%%%%%%%%%%%%%%%%
\item  ()xxcccPXcedxee\lambda\lambda\lambda\lambda∞∞−−⎡⎤>==−=⎣⎦∫ .
For x > c, ()()(){}()()()xcPXxXcPXxePXxXcPXcPXce\lambda\lambda−−>∩>>>>===>>
()xce\lambda−−=, as required.
Thus the conditional cdf of X, given that X > c, is ()1xce\lambda−−−, and by differentiation we get that the conditional pdf is ().xce\lambda\lambda−−

Therefore X – c has an exponential distribution with parameter \lambda or, putting it another way, X has the exponential distribution but with the origin shifted to c.
%%%%%%%%%%%%%%%%%%%%%%%%%%%%%%%%%%%
\item  The sample mean is 2.0 and the sample variance is 2120.074.3891⎛⎞−⎜⎟⎝⎠ = 3.82, so the sample standard deviation is 3.821.95.=

The sample mean and standard deviation are very nearly equal. This supports the exponential model, as the exponential distribution has equal mean and standard deviation (see part (i)).
\end{enumerate}
\end{document}

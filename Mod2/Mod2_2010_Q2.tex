\documentclass[a4paper,12pt]{article}
%%%%%%%%%%%%%%%%%%%%%%%%%%%%%%%%%%%%%%%%%%%%%%%%%%%%%%%%%%%%%%%%%%%%%%%%%%%%%%%%%%%%%%%%%%%%%%%%%%%%%%%%%%%%%%%%%%%%%%%%%%%%%%%%%%%%%%%%%%%%%%%%%%%%%%%%%%%%%%%%%%%%%%%%%%%%%%%%%%%%%%%%%%%%%%%%%%%%%%%%%%%%%%%%%%%%%%%%%%%%%%%%%%%%%%%%%%%%%%%%%%%%%%%%%%%%
\usepackage{eurosym}
\usepackage{vmargin}
\usepackage{amsmath}
\usepackage{graphics}
\usepackage{epsfig}
\usepackage{enumerate}
\usepackage{multicol}
\usepackage{subfigure}
\usepackage{fancyhdr}
\usepackage{listings}
\usepackage{framed}
\usepackage{graphicx}
\usepackage{amsmath}
\usepackage{chngpage}
%\usepackage{bigints}

\usepackage{vmargin}
% left top textwidth textheight headheight
% headsep footheight footskip
\setmargins{2.0cm}{2.5cm}{16 cm}{22cm}{0.5cm}{0cm}{1cm}{1cm}
\renewcommand{\baselinestretch}{1.3}

\setcounter{MaxMatrixCols}{10}
\begin{document}
Higher Certificate, Module 2, 2010. Question 2

\section{Introduction}

\begin{enumerate}
    \item 
W ~ N(24, 1)
\begin{enumerate}[(a)]
\item  \[P(W > 25) = 1 − ⎟⎠⎞⎜⎝⎛−\Phi12425 = 1 − \Phi(1) = 1 − 0.8413 = 0.1587.\]
\item  (a) \[P(C = 0) = P(W ≤ 25) = \Phi(1) = 0.8413.\]
P(C = 5) = P(25 < W ≤ 26)
\[= ⎟⎠⎞⎜⎝⎛−\Phi12426− \Phi(1) = \Phi(2) − \Phi(1) = 0.9772 − 0.8413 = 0.1359.\]
P(C = 10) = P(26 < W ≤ 27)
\[= ⎟⎠⎞⎜⎝⎛−\Phi12427− \Phi(2) = \Phi(3) − \Phi(2) = 0.9987 − 0.9772 = 0.0215.\]
\end{enumerate}
\item  We have the following for C.
c
0
5
10
15
P(C = c)
0.8413
0.1359
0.0215
0.0013
c P(C = c)
0
0.6795
0.215
0.0195
c2 P(C = c)
0
3.3975
2.15
0.2925
So E(C) = row total for the "c P(C = c)" row = 0.914.
Also, E(C2) = row total for the "c2 P(C = c)" row = 5.84,
and therefore Var(C) = 5.84 − 0.9142 = 5.005.
\item  E(CT) = 91400, Var(CT) = 500500.
We use a Normal approximation to the distribution of CT.
The upper 95\% point of this is \[91400 + (1.6449 \times \sqrt{500500}) = 92564\].
\item  Independence may hold for people travelling separately but is most unlikely to hold for families or other groups – they may, for example, try to equalise their loads to minimise the excess cost – and this will affect the variance of CT.
"100 000" must be a rough figure for the total number of passengers.
The distribution of C is clearly positively skew, so a large number of passengers is needed to validate the use of a Normal approximation for CT. The given total of 100 000 (even as a rough figure) is probably large enough.

\end{enumerate}

\end{document}

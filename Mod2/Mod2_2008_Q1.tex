\documentclass[a4paper,12pt]{article}
%%%%%%%%%%%%%%%%%%%%%%%%%%%%%%%%%%%%%%%%%%%%%%%%%%%%%%%%%%%%%%%%%%%%%%%%%%%%%%%%%%%%%%%%%%%%%%%%%%%%%%%%%%%%%%%%%%%%%%%%%%%%%%%%%%%%%%%%%%%%%%%%%%%%%%%%%%%%%%%%%%%%%%%%%%%%%%%%%%%%%%%%%%%%%%%%%%%%%%%%%%%%%%%%%%%%%%%%%%%%%%%%%%%%%%%%%%%%%%%%%%%%%%%%%%%%
\usepackage{eurosym}
\usepackage{vmargin}
\usepackage{amsmath}
\usepackage{graphics}
\usepackage{epsfig}
\usepackage{enumerate}
\usepackage{multicol}
\usepackage{subfigure}
\usepackage{fancyhdr}
\usepackage{listings}
\usepackage{framed}
\usepackage{graphicx}
\usepackage{amsmath}
\usepackage{chngpage}
%\usepackage{bigints}

\usepackage{vmargin}
% left top textwidth textheight headheight
% headsep footheight footskip
\setmargins{2.0cm}{2.5cm}{16 cm}{22cm}{0.5cm}{0cm}{1cm}{1cm}
\renewcommand{\baselinestretch}{1.3}

\setcounter{MaxMatrixCols}{10}

\section{Introduction}
Higher Certificate, Module 2, 2008. Question 1
\begin{enumerate}

\item (a) The number of PINs with four different digits is \[10 \times 9 \times 8 \times 7 = 5040.\]

%%%%%%%%%%%%%
(b) We require exactly three different digits. We can choose the face value of the pair in 10 ways. We can then choose two other different digits in = 36 ways. The number of distinguishable linear arrangements of two like and two unlike objects is ⎟⎟⎠⎞⎜⎜⎝⎛294!2!1!1!×× = 12, so the total number of 4-digit PINs with exactly three different digits is 12×10×36 = 4320.
%%%%%%%%%%%%%%%%%%5
\item  We require two different digits, each occurring twice. We can choose the face values of the two pairs in = 45 ways. The number of distinguishable linear arrangements of two (different) pairs of like objects is ⎟⎟⎠⎞⎜⎜⎝⎛2104!2!2!× = 6, so the total number of 4-digit PINs with two pairs of (different) like digits is 6×45 = 270.
%%%%%%%%%%%%%%%%%%55
\item We require exactly three digits the same. We can choose the face values of the triple and of the singleton in 10×9 ways (note that aaab and bbba are different PINs). The number of distinguishable linear arrangements is 4 (corresponding to 4 different places for the singleton), hence there are 4×904 = 360 possible PINs.
%%%%%%%%%%%%%%%%%%%%%%%%%%%%5
\item  (a) There are altogether = 210 ways of choosing the 4 digits of the second PIN, each being equally likely with probability 1/210. ⎟⎟⎠⎞⎜⎜⎝⎛410
Now consider the number of ways of choosing 4 digits for the second PIN such that k of them (for k = 0, 1, 2, 3, 4) are in common with digits in an arbitrary given PIN of four different digits. There are ways of choosing the k digits that are in common and ways of choosing the 4−k digits that are not in common. So the total number of ways is . ⎟⎟⎠⎞⎜⎜⎝⎛k4⎟⎟⎠⎞⎜⎜⎝⎛−k46⎟⎟⎠⎞⎜⎜⎝⎛k4⎟⎟⎠⎞⎜⎜⎝⎛−k46
So the required probability is 464104kk⎛⎞⎛⎞⎜⎟⎜⎟−⎝⎠⎝⎠⎛⎞⎜⎟⎝⎠.

\item  
\begin{itemize}
\item P(X = 0) = 460415110210144⎛⎞⎛⎞⎜⎟⎜⎟⎝⎠⎝⎠==⎛⎞⎜⎟⎝⎠;
\item P(X = 1) = ⎟⎟⎠⎞⎜⎜⎝⎛⎟⎟⎠⎞⎜⎜⎝⎛⎟⎟⎠⎞⎜⎜⎝⎛4103614= 21821080=;
\item P(X = 2) =46229031021074⎛⎞⎛⎞⎜⎟⎜⎟⎝⎠⎝⎠==⎛⎞⎜⎟⎝⎠;
\item P(X = 3) = 463124410210354⎛⎞⎛⎞⎜⎟⎜⎟⎝⎠⎝⎠==⎛⎞⎜⎟⎝⎠;
\item P(X = 4) = 46401102104⎛⎞⎛⎞⎜⎟⎜⎟⎝⎠⎝⎠=⎛⎞⎜⎟⎝⎠.
\end{itemize}

E(X) = 15809024101334210210210210210⎛⎞⎛⎞⎛⎞⎛⎞⎛×+×+×+×+×⎜⎟⎜⎟⎜⎟⎜⎟⎜⎝⎠⎝⎠⎝⎠⎝⎠⎝
= 336210 = 1.6.
\end{enumerate}
\end{document}

\documentclass[a4paper,12pt]{article}
%%%%%%%%%%%%%%%%%%%%%%%%%%%%%%%%%%%%%%%%%%%%%%%%%%%%%%%%%%%%%%%%%%%%%%%%%%%%%%%%%%%%%%%%%%%%%%%%%%%%%%%%%%%%%%%%%%%%%%%%%%%%%%%%%%%%%%%%%%%%%%%%%%%%%%%%%%%%%%%%%%%%%%%%%%%%%%%%%%%%%%%%%%%%%%%%%%%%%%%%%%%%%%%%%%%%%%%%%%%%%%%%%%%%%%%%%%%%%%%%%%%%%%%%%%%%
\usepackage{eurosym}
\usepackage{vmargin}
\usepackage{amsmath}
\usepackage{graphics}
\usepackage{epsfig}
\usepackage{enumerate}
\usepackage{multicol}
\usepackage{subfigure}
\usepackage{fancyhdr}
\usepackage{listings}
\usepackage{framed}
\usepackage{graphicx}
\usepackage{amsmath}
\usepackage{chngpage}
%\usepackage{bigints}

\usepackage{vmargin}
% left top textwidth textheight headheight
% headsep footheight footskip
\setmargins{2.0cm}{2.5cm}{16 cm}{22cm}{0.5cm}{0cm}{1cm}{1cm}
\renewcommand{\baselinestretch}{1.3}

\setcounter{MaxMatrixCols}{10}
\begin{document}

\maketitle

\section{Introduction}

\begin{enumerate}
    \item Higher Certificate, Module 2, 2009. Question 4
    \item  X ~ Poisson(2).
(a) P(X = 0) = 0.1353 from tables (or as e-2).
(b) P(X > 2) = 1 − P(X ≤ 2) = 1 - 0.6767 from tables, or by use of
1−222122e−⎛⎞++⎜⎟⎝⎠
= 0.3233.

%%%%%%%%%%%%%%%%%%%%%%%%%%
    \item  With obvious notation, XA ~ Poisson(0.2) and XB ~ Poisson(0.3).
(a) P(no flaws) = P(XA = 0 and XB = 0)
= P(XA = 0).P(XB = 0) by independence
= e-0.2 e-0.3 = 0.8187 × 0.7408 = 0.6065.
[Alternative method: XA + XB ~ Poisson(0.5), so P(XA + XB = 0) = e-0.5 = 0.6065.]
(b) 


\begin{eqnarray}
P(\mbox{exactly one flaw}) &=& P{(XA = 0 and XB = 1) or (XB = 0 and XA = 1)}\\
&=& P(XA = 0 and XB = 1) + P(XB = 0 and XA = 1)\\
&=& P(XA = 0).P(XB = 1) + P(XB = 0).P(XA = 1) by independence\\
&=& e^{-0.2} × (0.3)e^{-0.3}+ e^{-0.3} × (0.2)e^{-0.2}\\
&=& 0.3033 [Or by use of the alterative method as above.]
\end{eqnarray}
%%%%%%%%%%%%%5
    \item  (a) ()()()()7flaws7flaws7flawsPAPPAP=
()()()()()()7flaws7flaws7flawsPAPAPAPAPBP=+
47476740.757!460.750.257!7!eee−−−×=⎛⎞⎛×+×⎜⎟⎜⎝⎠⎝ = 0.5647.
    \item  Repeating this calculation for 8 flaws:
()()()()8flaws8flaws8flawsPAPPAP=
()()()()()()8flaws8flaws8flawsPAPAPAPAPBP=+
48486840.758!460.750.258!8!eee−−−×=⎛⎞⎛×+×⎜⎟⎜⎝⎠⎝ = 0.4638.

\begin{itemize}
\item The rigging contains more rope from company A than from B; but the rope from B is less reliable than that from A. 
\item Thus, as we find increasingly many flaws in the rope, the probability that it came from A reduces to less than $1/2$.
\end{itemize}
\end{enumerate}

\end{document}

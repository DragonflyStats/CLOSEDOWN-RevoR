THE ROYAL STATISTICAL SOCIETY
HIGHER CERTIFICATE EXAMINATION
NEW MODULAR SCHEME
introduced from the examinations in 2007
MODULE 2
SPECIMEN PAPER A
AND SOLUTIONS
The time for the examination is 1½ hours. The paper contains four questions, of which candidates are to attempt three. Each question carries 20 marks. An indicative mark scheme is shown within the questions, by giving an outline of the marks available for each part-question. The pass mark for the paper as a whole is 50%.
The solutions should not be seen as "model answers". Rather, they have been written out in considerable detail and are intended as learning aids. For this reason, they do not carry mark schemes. Please note that in many cases there are valid alternative methods and that, in cases where discussion is called for, there may be other valid points that could be made.
While every care has been taken with the preparation of the questions and solutions, the Society will not be responsible for any errors or omissions.
The Society will not enter into any correspondence in respect of the questions or solutions.
© RSS 2006
1. In a hi-tech company, the members of three research groups (A, B and C) are individually invited to enter a prize competition for the best solution to a technical problem. Group A has 2 staff, B has 3 and C has 5. It is assumed that all staff decide independently whether or not to enter. Members of groups A, B and C enter with respective probabilities 1/2, 1/4 and 1/5.
(i) For each group separately, find the probability of (a) no entries, (b) one entry.
(8)
(ii) Given that there is just one entry in total, show that the probability that it comes from a member of group A is 8/17.
(6)
(iii) Explain (but without doing the calculations) the steps that are needed to calculate the probability that there are exactly two entries in total.
(6)
2. The continuous random variable X has probability density function given by
()()221,010,elsewhere.kxxxfx⎧−≤⎪=⎨⎪⎩
(i) Find k and sketch the graph of f (x).
(7)
(ii) Find E(X) and Var(X), and show that 11738PX⎛⎞≤=⎜⎟⎝⎠ .
(8)
(iii) A random sample of size 5 is taken from this distribution. Find, correct to 4 decimal places, the probability that all 5 observations exceed 1/3.
(3)
(iv) Find, correct to 4 decimal places, the variance of the mean of a random sample of size 5.
(2)
3. My cycle journey to work is 3 km, and my cycling time (in minutes) if there are no delays is distributed N(15, 1), i.e. Normally with mean 15 and variance 1.
(i) Find the probability that, if there are no delays, I get to work in at most 17 minutes.
(2)
(ii) On my route there are three sets of traffic lights. Each time I meet a red traffic light, I am delayed by a random time that is distributed N(0.7, 0.09). These lights operate independently. Find the probability of my getting to work in at most 17 minutes
(a) if just one light is set at red when I reach it,
(b) if just two lights are set at red when I reach them,
(c) if all three lights are set at red when I reach them.
(9)
(iii) Suppose that, for each set of lights, the chance of delay is 0.5. Deduce that the mean value of T, my total journey time, is 16.05 minutes.
(4)
(iv) Given that Var(T ) = 1.5025, use a suitable approximation to calculate the probability that, over 10 journeys, my average journey time to work is at most 17 minutes.
(5)
4. The random variable X follows the binomial B(n, p) distribution with probability mass function
(),0,1,...,,01xnxnfxpqxnpx−⎛⎞==⎜⎟⎝⎠ ,
where q = 1 − p. Show that E(X) = np and Var(X) = npq.
(5)
A mathematics class in a school is divided into set A with 12 students and set B with 25 students. Both groups are given a test consisting of 16 short questions. For any student in set A, the score (that is, the number of correct answers) is distributed as B(16, 0.75); for any student in set B, the score is distributed as B(16, 0.5). All students answer independently.
(i) Find the probability that
(a) a given set A student gets all 16 questions right,
(3)
(b) at least one student in set A gets all 16 questions right.
(3)
(ii) Use an appropriate approximation to find the probability that a given set B student scores more than a given set A student.
(5)
(iii) Let X and Y denote the mean scores of students in set A and set B respectively. Write down ()EX and ()EY, and show that ()()Var1/4 and Var4/25.XY==
(4)
SOLUTIONS
Question 1
(i) A: (a) P(0 entries) = 2110.2524⎛⎞==⎜⎟⎝⎠.
(b) P(1 entry) = 1112222××= = 0.5.
B: (a) P(0 entries) = 33270.4219464⎛⎞==⎜⎟⎝⎠.
(b) P(1 entry) = 2132734464⎛⎞××=⎜⎟⎝⎠ = 0.4219.
C: (a) P(0 entries) = 5410240.327753125⎛⎞==⎜⎟⎝⎠.
(b) P(1 entry) = 41425650.409655625⎛⎞⎛⎞==⎜⎟⎜⎟⎝⎠⎝⎠.
(ii) P(1 entry in total)
= P(1 from A, 0 from B and C) + P(1 from B, 0 from A and C)
+ P(1 from C, 0 from A and B) 12710242711024256127459264312564431256254643125=××+××+××=.
[If worked in decimals, this is 0.1469.]
P(1 from A | 1 in total) = P(1 from A and 1 in total) / P(1 in total)
= P(1 from A, 0 from B and C) / P(1 in total)
= 271024126431254593125817××=.
(iii) Denote the numbers of entries from A, B, C as (0, 0, 0) etc. Then we need P(2, 0, 0) + P(0, 2, 0) + P(0, 0, 2) + P(1, 1, 0) + P(1, 0, 1) + P(0, 1, 1). Since entries from each group are independent, we have, as an example, P(1, 1, 0) = P(1 from A).P(1 from B).P(0 from C).

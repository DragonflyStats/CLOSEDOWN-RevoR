\documentclass{article}
\usepackage[utf8]{inputenc}
\usepackage{framed}
\usepackage{enumerate}
\title{RSS_Jan_2019_Mod2}
\author{kobriendublin }
\date{December 2018}

\begin{document}

%- Higher Certificate, Module 2, 2007. Question 3
\section{Introduction}

\begin{enumerate}
    \item 

 Suppose the coin is tossed n times and there are x heads and therefore (n – x) tails. The number of possible orders in which this can happen is ( !or!!nnxxnx⎛⎞⎜⎟−⎝⎠. The probability of each of these orders for independent tosses is ()1nxxpp−×−, so the required overall probability is ()()1nxxnPXxppx−⎛⎞==−⎜⎟⎝⎠, for x$ = 0, 1, …, n$.

A Normal approximation with the same mean and variance as this binomial distribution, i.e. N(np, np(1 – p)), is satisfactory when n is fairly large and p is not near 0 or 1. As a guideline, the value of np (or of n(1 – p) if p is near 1) should be at least 5. A continuity correction should be used.
\item For n = 20 and p = 0.2, the Normal approximation is N(4, 3.2). Using a continuity correction, P(X ≤ 3) is approximated by the area under the pdf of the Normal distribution up to 3.5. Thus the required probability is
()3.540.279510.6100.3903.2−⎛⎞Φ=Φ−=−=⎜⎟⎝⎠.
The Society's "Statistical tables for use in examinations" give the exact probability from B(20, 0.2) as P(X ≤ 3) = 0.411. Hence the percentage error is 100(0.390 – 0.411)/0.411 which is 5.11% (in the negative direction). This is not very satisfactory. Here, n is not large and p is fairly small; the guideline np ≥ 5 is not satisfied. The binomial distribution is not well approximated by the Normal distribution (it will be noticeably positively skew as p < ½).
\item There must be a head at the final toss and (x – 1) heads in the other (n – 1) tosses. So the probability is
()()()()()111111111nxnxxnnPNnpppppxx −−−−−−−⎛⎞⎛⎞==×−=−⎜⎟⎜⎟−−⎝⎠⎝⎠,
for n = x, x + 1, x + 2, … . For p = 0.2, x = 3, n = 20, this gives
()()()()31731719!19.180.20.80.20.80.03082!17!2.1==.
From the tables, P(X = 3) for the B(20, 0.2) distribution is 0.4114 – 0.2061 = 0.2053. The previous probability is x/n times this.
\end{enumerate}

\end{document}
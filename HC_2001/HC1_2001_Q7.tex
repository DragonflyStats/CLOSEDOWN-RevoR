\documentclass[a4paper,12pt]{article}
%%%%%%%%%%%%%%%%%%%%%%%%%%%%%%%%%%%%%%%%%%%%%%%%%%%%%%%%%%%%%%%%%%%%%%%%%%%%%%%%%%%%%%%%%%%%%%%%%%%%%%%%%%%%%%%%%%%%%%%%%%%%%%%%%%%%%%%%%%%%%%%%%%%%%%%%%%%%%%%%%%%%%%%%%%%%%%%%%%%%%%%%%%%%%%%%%%%%%%%%%%%%%%%%%%%%%%%%%%%%%%%%%%%%%%%%%%%%%%%%%%%%%%%%%%%%
\usepackage{eurosym}
\usepackage{vmargin}
\usepackage{amsmath}
\usepackage{graphics}
\usepackage{epsfig}
\usepackage{enumerate}
\usepackage{multicol}
\usepackage{subfigure}
\usepackage{fancyhdr}
\usepackage{listings}
\usepackage{framed}
\usepackage{graphicx}
\usepackage{amsmath}
\usepackage{chngpage}
%\usepackage{bigints}

\usepackage{vmargin}
% left top textwidth textheight headheight
% headsep footheight footskip
\setmargins{2.0cm}{2.5cm}{16 cm}{22cm}{0.5cm}{0cm}{1cm}{1cm}
\renewcommand{\baselinestretch}{1.3}

\setcounter{MaxMatrixCols}{10}
\begin{document}
%- Higher Certificate, Paper I, 2001. Question 7
\begin{table}[ht!]
     \centering
     \begin{tabular}{|p{15cm}|}
     \hline        
\noindent Text
\\ \hline
      \end{tabular}
    \end{table}


\begin{framed}
%==========================================================================================%
7. (a) The continuous random variable X is distributed with probability density
function (pdf) f(x) and cumulative distribution function (cdf) F(x). A
random sample X1, X2, …, Xn is drawn from the distribution. Denote the
maximum value of the sample by Xmax and the minimum value by Xmin .
(i) Note that Xmax ≤ x if and only if X1 ≤ x, X2 ≤ x, …, Xn ≤ x. Hence
show that the cdf of Xmax is given by
( ) ( ) max
n
FX x = F x  .

(ii) By noting the condition under which Xmin ≥ x, show that
( ) ( ) min 1 1 n
X F x = −  − F x  .

(iii) Deduce formulae for the probability density functions of Xmax and
Xmin in terms of F(x) and f(x).


\end{framed}
\end{framed}
\begin{enumerate}
\item P( Xi \leq  x) = F (x) for i = 1, 2, ..., n .
(i) ( ) ( ) ( ) max 1 2
1
, , ...,
n
X n i
i
F x P X x X x X x P X x
=
  = \leq  \leq  \leq  =Π \leq 
( ) n = F x 

%%%%%%%%%%%%%%%%%%%%%%%%%%%%%%%%%%%%%%%%%%%%%%%%%%%%%%%%%%%%%%%%%%%%%%
\item  For $X_{min} X \geq  x$ , we require every $X_i$ to be $\geq  x$ for all $i$.
Now, ( ) ( ), so ( ) 1 ( ),  i P X \leq  x = F x P X \geq  x = - F x i
( ) ( ) ( ) ( ) X_{min}  X_{min} 1 1 1 ,..., X n 


\begin{eqnarray*}
[F_{X_{min}} &=& P X( \leq  x ) \\
  &=&  1 - P( X \geq  x ) \\
  &=&  1- P(X_1 \geq  x,X_1 \geq  x,\ldots  X_n \geq  x)\\
  &=&  1- \left[ 1- F(X) \right]^{n}x 
\end{eqnarray*}

%%%%%%%%%%%%%%%%%%%%%%%%%%%%%%%%%%%%%%%%%%%%%%%%%%%%%%%%%%%%%%%%%%%%%%
\item Pdfs are derivatives of cdfs:

\begin{equation*}
f_{X_{min}}= n \left[ F(X) \right]^{n-1}f(x)
\end{equation}
\begin{equation*}
f_{X_{min}}= n \left[ 1- F(X) \right]^{n-1}f(x)
\end{equation}

since $\frac{\partial F(X)}{\partial x } = f(x)\]
%%%%%%%%%%%%%%%%%%%%%%%%%%%%%%%%%%%%%%%%%%%%%%%%%%%%%%%

(b)
0
0
0.1
0.2
0.3
0.4
0.5
0.6
0.7
0.8
0.9
1
1 2 3 4 5 6 7
x
( )
0 ( ) 1 ( ) ( )
0
1 1 1 .
1 1 1
x
x F x du
u u x \alpha \alpha \alpha
\alpha
+
   
= = -  = -
+  +  +
  ∫
Median M is such that ( ) 1
2
F M = . So we have
%( ) ( )
%1 1 1 or 1 1 or 2 (1 ) or 2(1/ ) 1
%1 2 1 2
%M M
%M M
%\alpha \alpha
%\alpha \alpha - = = = + = -
%+ +
  .
( ) ( ) X_{min}
%%%%%%%%%%%%%%%%%%%%%%%%%%%%%%%%%%%%%%%%%%%%%%%%%%%%%
\begin{table}[ht!]
     \centering
     \begin{tabular}{|p{15cm}|}
     \hline        
\noindent Text
\\ \hline
      \end{tabular}
    \end{table}
    
    \begin{table}[ht!]
     \centering
     \begin{tabular}{|p{15cm}|}
     \hline        
\noindent (b) Suppose that the random variable X has the Pareto density given by
( ) ( ) ( )1 f x 1 x α α − + = + , x > 0 , α > 0.
Draw a graph of this density and show that the cdf is given by
F (x) 1 (1 x) −α = − + .
Deduce that the median of X is 21/α −1 .

Use result (a)(ii) above to obtain the cdf of the minimum of a random
sample of n observations distributed as is X, and verify that this cdf is also
of Pareto form but with a different parameter.

Taking α = 1, find the smallest value of n such that the median value of the
sample minimum is less than 0.1.
\\ \hline
      \end{tabular}
    \end{table}
\begin{framed}
Cumulative distribution function[edit]
From the definition, the cumulative distribution function of a Pareto random variable with parameters α and xm is 
\[ {\displaystyle F_{X}(x)={\begin{cases}1-\left({\frac {x_{\mathrm {m} }}{x}}\right)^{\alpha }&x\geq x_{\mathrm {m} },\\0&x<x_{\mathrm {m} }.\end{cases}}} 
\]
Probability density function[edit]
It follows (by differentiation) that the probability density function is \[ {\displaystyle f_{X}(x)={\begin{cases}{\frac {\alpha x_{\mathrm {m} }^{\alpha }}{x^{\alpha +1}}}&x\geq x_{\mathrm {m} },\\0&x<x_{\mathrm {m} }.\end{cases}}} 
\]
\end{framed}
%%%%%%%%%%%%%%%%%%%%%%%%%%%%%%%%%%%%%%%%%%
    Using (a)(ii), 1 1 1 1
1 1
n
X n F
x x \alpha \alpha
 
= -   = -
 +  +  
, also Pareto but with \alpha replaced
by $n\alpha$.
The median of $X_{min}$ is then
1
2n\alpha -1, which is 21/ n -1 if \alpha =1. We require
21/ n -1< 0.1 or 21/ n <1.1, i.e. 1 ln 2 ln1.1
n
< , giving ln 2 0.6931 7.27
ln1.1 0.0953
n> = = .
Hence $n \geq  8$.

\end{enumerate}
\end{document}

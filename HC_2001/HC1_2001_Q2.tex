\documentclass[a4paper,12pt]{article}
%%%%%%%%%%%%%%%%%%%%%%%%%%%%%%%%%%%%%%%%%%%%%%%%%%%%%%%%%%%%%%%%%%%%%%%%%%%%%%%%%%%%%%%%%%%%%%%%%%%%%%%%%%%%%%%%%%%%%%%%%%%%%%%%%%%%%%%%%%%%%%%%%%%%%%%%%%%%%%%%%%%%%%%%%%%%%%%%%%%%%%%%%%%%%%%%%%%%%%%%%%%%%%%%%%%%%%%%%%%%%%%%%%%%%%%%%%%%%%%%%%%%%%%%%%%%
\usepackage{eurosym}
\usepackage{vmargin}
\usepackage{amsmath}
\usepackage{graphics}
\usepackage{epsfig}
\usepackage{enumerate}
\usepackage{multicol}
\usepackage{subfigure}
\usepackage{fancyhdr}
\usepackage{listings}
\usepackage{framed}
\usepackage{graphicx}
\usepackage{amsmath}
\usepackage{chngpage}
%\usepackage{bigints}

\usepackage{vmargin}
% left top textwidth textheight headheight
% headsep footheight footskip
\setmargins{2.0cm}{2.5cm}{16 cm}{22cm}{0.5cm}{0cm}{1cm}{1cm}
\renewcommand{\baselinestretch}{1.3}

\setcounter{MaxMatrixCols}{10}
\begin{document}
Higher Certificate, Paper I, 2001. Question 2
\begin{enumerate}
    \item (a) Fix the position of M1 (suppose him to be the host).
M1 Label the positions clockwise.
(1)
(6) (2) M2, M3, W1, W2, W3 may be arranged in 5! =
120 ways.
(5) (3)
(4)
   \item M2, M3 must occupy (3) and (5); W1, W2, W3 may occupy the other
places in 3! ways, making $2×3!$ arrangements. The probability is then
\[ \frac{12}{120} =\frac{1}{10}\]
   \item M2, M3 must occupy (2) and (6) or (2) and (3) or (5) and (6). In each
case, M2, M3 can be placed in two orders, making 6 positions altogether for
the three men. The women may again fill the remaining places in 3! ways.
The probability is 
\[ \frac{6 \times 6}{120} =\frac{36}{120} = \frac{3}{10}\]

   \item EITHER 1 1 3 3
10 10 5
− − = , because this is the only other arrangement
possible besides (i) and (ii);
OR by having M2 in (2), M3 in (4) or (5); M2 in (6), M3 in (3) or (4); M2 in
(3), M3 in (4); M2 in (4), M3 in (5); or any of these with M2, M3 interchanged,
giving 12 positionings of the men. There are again 3! orders for the women,
so the probability is 
\[ \frac{6 \times 12}{120} =\frac{72}{120} = \frac{3}{5}\]

\item Event D is "has disease", T is "tests positive".

\begin{framed}

\[ {P(D|T) = \frac{P(D \cap T)}{P(T)}\]

Remark 
\[ P(D \cap T)  = P(D|T) \times P(D) = P(T|D) \times P(T)\]

\[ P(D|T) = \frac{P(D \cap T)}{P(T)} = \frac{P(D|T) \times P(D)}{P(T)}\]

\end{framed}

Total Probability
\[P(T) = P(T \cap D) + P(T \cap D^{C})\]


\begin{eqnarray*}
P(D|T) &=&   \frac{P(D|T) \times P(D)}{P(T)}\\
&=& \frac{P(D|T) \times P(D)}{P(T \cap D) + P(T \cap D^{C})}\\
&=& \frac{P(D|T) \times P(D)}{P(T|D) \times P(D) + P(T|D) \times P(D^{C})}\\
=& \frac{p_1 \times p_0}{\left[p_1 \times p_0\right] + \left[(1- p_2 )(1- p_0)\right]}\\
\end{eqnarray*}


\item 

\[\frac{0.95 0.005 }{(0.95 0.005) (0.05 0.995) } = \frac{0.00475 }{0.0545} = 0.0872.\]

The error rates in the clinical tests are large compared to the chance of having
the disease, so the calculated probability is very small.
\end{enumerate}

\end{document}

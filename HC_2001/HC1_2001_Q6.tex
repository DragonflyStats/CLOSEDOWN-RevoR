\documentclass[a4paper,12pt]{article}
%%%%%%%%%%%%%%%%%%%%%%%%%%%%%%%%%%%%%%%%%%%%%%%%%%%%%%%%%%%%%%%%%%%%%%%%%%%%%%%%%%%%%%%%%%%%%%%%%%%%%%%%%%%%%%%%%%%%%%%%%%%%%%%%%%%%%%%%%%%%%%%%%%%%%%%%%%%%%%%%%%%%%%%%%%%%%%%%%%%%%%%%%%%%%%%%%%%%%%%%%%%%%%%%%%%%%%%%%%%%%%%%%%%%%%%%%%%%%%%%%%%%%%%%%%%%
\usepackage{eurosym}
\usepackage{vmargin}
\usepackage{amsmath}
\usepackage{graphics}
\usepackage{epsfig}
\usepackage{enumerate}
\usepackage{multicol}
\usepackage{subfigure}
\usepackage{fancyhdr}
\usepackage{listings}
\usepackage{framed}
\usepackage{graphicx}
\usepackage{amsmath}
\usepackage{chngpage}
%\usepackage{bigints}

\usepackage{vmargin}
% left top textwidth textheight headheight
% headsep footheight footskip
\setmargins{2.0cm}{2.5cm}{16 cm}{22cm}{0.5cm}{0cm}{1cm}{1cm}
\renewcommand{\baselinestretch}{1.3}

\setcounter{MaxMatrixCols}{10}
\begin{document}
Higher Certificate, Paper I, 2001. Question 6
\begin{table}[ht!]
     \centering
     \begin{tabular}{|p{15cm}|}
     \hline        
\noindent Text
\\ \hline
      \end{tabular}
    \end{table}

\begin{table}[ht!]
     \centering
     \begin{tabular}{|p{15cm}|}
     \hline        
\noindent Text
\\ \hline
      \end{tabular}
    \end{table}
    
    \begin{table}[ht!]
     \centering
     \begin{tabular}{|p{15cm}|}
     \hline        
\noindent Text
\\ \hline
      \end{tabular}
    \end{table}

\begin{framed}
%==========================================================================================%
6. The random variable X is distributed with the geometric probability mass function
p(x) = qx−1 p , x = 1, 2, 3, …
where 0 < p < 1 and q = 1 − p. A random sample x1, x2, …, xn is taken from this
distribution.
Write down the likelihood function L(p) based on these data, and show that the
maximum likelihood estimate of p is given by
pˆ =1/ x
where x is the sample mean.

By using the approximation
( ) 2
2
Var ˆ 1
ln
p
E d L
dp
≈
 
 − 
 
,
or otherwise, show that
( ) ( ) 2 1
Var ˆ
p p
p
n
−
≈ .

[ Note. You may assume that E(X) = 1/p. ]
Question 6 continued on next page
9
Turn over
A boy counts the number of times that he has to roll a given die in order to obtain
a six. His results, summarised, are as follows.
Number of rolls
to obtain six
x: 1 2 3 4 5 6 7 8 9 10
Frequency f: 7 7 6 4 5 4 2 3 1 2
so that Σ fx = 448. Find pˆ from these data and use the above result to estimate
the variance of ˆp . Assuming that values of ˆp are approximately Normally
distributed, calculate an approximate 95% confidence interval for the true
probability p of getting a six on a roll of this die.

Finally, suppose that the die is fair, so that p = 1/6. Assuming that values of
pˆ =1/ x are approximately distributed as N(p, p2(1 − p)/n), where p = 1/6, find
the approximate probability of the boy obtaining an estimate as low as or lower
than that given by the data above.
Comment on your answer.

x 11 13 16 17 20 22 25 33
f 2 2 3 1 2 3 1 1

\end{framed}


\begin{enumerate}
\item ( ) ( 1 ) ( )
1
1
1
i i
n n
x n x n nx
i
L p q p p q p p
p
− −
=
Σ   = = =   −  −  Π
\[ \ln L = n \ln p − n\ln (1− p) + nx \ln(1− p)\]
(ln ) 0 when 1 1 which gives ˆ 1 as m.l.estimate.
1 1 ˆ 1 ˆ
L n n nx x p
p p p p p p x
∂ = + − = = − + =
∂ − − −
( ) ( )
( )
2
2 2 2
ln 1
which is < 0, confirming the maximum.
1
L n n x
p p p
∂ −
= − −
∂ −
( ) ( ) ( ) ( )
2
2 2 2 2 2 2
ln 1 1 1
1 1 1
E L n n E X n n n n
p p p p p p p p p
∂        = +  −  = +  −  = +  ∂  − −   −
2 (1 )
n
p p
=
−
\item Hence ( ) ( ) 2 1
Var ˆ
p p
p
n
−
≈ .
448, 56, ˆ 56 0.125
448
Σ fx = Σ f = p = = .
( ) ( )
0.1252 0.875 Var ˆ 0.0002441, SE ˆ 0.015625
56
p = × = p = .
Approximate 95\% confidence interval for p is pˆ ±1.96SE(pˆ ) , which is
0.125 ±1.96×0.015625, i.e. 0.125 ± 0.030625 or (0.0944, 0.1556).
\item When 1 , we have 1 ~ N 1 , 5
6 6 216 56
p
X
=    ×   
, i.e. $N(0.1667, 0.00041336)$; therefore
the probability of obtaining pˆ ≤ 0.125 is approximately
0.125 0.1667 ( 2.0496) 0.0202
0.020331
Φ −  = Φ − = −  
 
.
\item The confidence interval for p did not include 1
6
; also now the probability being very
small is consistent with rejecting a null hypothesis that p = 1
6
, i.e. that the die is fair.
\end{enumerate}
\end{document}

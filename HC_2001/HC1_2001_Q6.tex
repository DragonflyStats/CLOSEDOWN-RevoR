\documentclass[a4paper,12pt]{article}
%%%%%%%%%%%%%%%%%%%%%%%%%%%%%%%%%%%%%%%%%%%%%%%%%%%%%%%%%%%%%%%%%%%%%%%%%%%%%%%%%%%%%%%%%%%%%%%%%%%%%%%%%%%%%%%%%%%%%%%%%%%%%%%%%%%%%%%%%%%%%%%%%%%%%%%%%%%%%%%%%%%%%%%%%%%%%%%%%%%%%%%%%%%%%%%%%%%%%%%%%%%%%%%%%%%%%%%%%%%%%%%%%%%%%%%%%%%%%%%%%%%%%%%%%%%%
\usepackage{eurosym}
\usepackage{vmargin}
\usepackage{amsmath}
\usepackage{graphics}
\usepackage{epsfig}
\usepackage{enumerate}
\usepackage{multicol}
\usepackage{subfigure}
\usepackage{fancyhdr}
\usepackage{listings}
\usepackage{framed}
\usepackage{graphicx}
\usepackage{amsmath}
\usepackage{chngpage}
%\usepackage{bigints}

\usepackage{vmargin}
% left top textwidth textheight headheight
% headsep footheight footskip
\setmargins{2.0cm}{2.5cm}{16 cm}{22cm}{0.5cm}{0cm}{1cm}{1cm}
\renewcommand{\baselinestretch}{1.3}

\setcounter{MaxMatrixCols}{10}
\begin{document}
Higher Certificate, Paper I, 2001. Question 6
\begin{table}[ht!]
     \centering
     \begin{tabular}{|p{15cm}|}
     \hline        
\noindent 6. The random variable X is distributed with the geometric probability mass function
\[p(x) = qx−1 p ,\qquad x = 1, 2, 3, \ldots\]
where $0 < p < 1$ and $q = 1 − p$. A random sample $\{x_1, x_2, \ldots, x_n\}$ is taken from this
distribution.
Write down the likelihood function $L(p)$ based on these data, and show that the
maximum likelihood estimate of p is given by
\[\hat{p} =1/ x\]
where $\bar{x}$ is the sample mean.

By using the approximation
( ) 2
2
Var ˆ 1
ln
p
E d L
dp
≈
 
 − 
 
,
or otherwise, show that
( ) ( ) 2 1
Var ˆ
p p
p
n
−
≈ .

[ Note. You may assume that E(X) = 1/p. ]

\\ \hline
      \end{tabular}
    \end{table}



\begin{enumerate}
\item ( ) ( 1 ) ( )
1
1
1
i i
n n
x n x n nx
i
L p q p p q p p
p
− −
=
Σ   = = =   −  −  Π
\[ \ln L = n \ln p − n\ln (1− p) + nx \ln(1− p)\]
$ \ln L = n\ln p - n \ln(1-p) + n\bar{x}\ln(1-p)$

\[\frac{\partial \ln L}{\partial p} = \frac{n}{p}  + \frac{n}{(1-p)} - \frac{n(\bar{x})}{1-p} \]

This is equal to 0 when
$ {\displaystyle \frac{1}{\hat{p}} = \frac{-1 + \bar{x}}{1 - \hat{p}} }$, which gives $ {\displaystyle\hat{p} = \frac{1}{x} }$ as a maximum likelihood estimate.

\[\frac{\partial^2 \ln L}{\partial p^2} = -\frac{n}{p^2} - \frac{n(\bar{x} - 1)}{(1-p)^2}\]

Tthis is less than 0, hence confirming the maximum.


\begin{eqnarray*}
E\left[ \frac{\partial \ln L}{\partial p^2} \right] 
&=& \frac{n}{p^2} + \frac{n}{(1-p)^2} \left[ E(\bar{X}) - 1 \right] \\
&=& \frac{n}{p^2} + \frac{n}{(1-p)^2} \left[ \frac{1}{p} - 1\right] \\
&=& \frac{n}{p^2} + \frac{n}{(1-p)^2} \left[ \frac{1-p}{p} \right] \\
&=& \frac{n}{p^2} + \frac{n}{p(1-p)} \\
&=& \frac{n(1-p)}{p^2(1-p)} + \frac{np}{p^2(1-p)} \\
&=& \frac{n}{p^2(1-p)}  \\
\end{eqnarray*}

Hence $\operatorname{Var}(\hat{p}) \approx \frac{p^2(1-p)}{n} $

\item Hence ( ) ( ) 2 1
Var ˆ
p p
p
n
−
≈ .
448, 56, ˆ 56 0.125
448
Σ fx = Σ f = p = = .
( ) ( )
0.1252 0.875 Var ˆ 0.0002441, SE ˆ 0.015625
56
p = × = p = .
Approximate 95\% confidence interval for p is $\hat{p} \pm 1.96SE(\hat{p})$ , which is
$0.125 \pm (1.96 \times 0.015625)$, i.e. $0.125 \pm 0.030625$ or (0.0944, 0.1556).

\begin{table}[ht!]
     \centering
     \begin{tabular}{|p{15cm}|}
     \hline        
\noindent A boy counts the number of times that he has to roll a given die in order to obtain
a six. His results, summarised, are as follows.
Number of rolls
to obtain six
x: 1 2 3 4 5 6 7 8 9 10
Frequency f: 7 7 6 4 5 4 2 3 1 2
so that Σ fx = 448. Find $\hat{p}$ from these data and use the above result to estimate
the variance of $\hat{p}$ . Assuming that values of $\hat{p}$ are approximately Normally
distributed, calculate an approximate 95\% confidence interval for the true
probability p of getting a six on a roll of this die.
\\ \hline
      \end{tabular}
    \end{table}
 
\newpage    
    \begin{table}[ht!]
     \centering
     \begin{tabular}{|p{15cm}|}
     \hline        
\noindent 
Finally, suppose that the die is fair, so that p = 1/6. Assuming that values of
$\hat{p}$ =1/ x are approximately distributed as $N(p, p^2(1 - p)/n)$, where p = 1/6, find
the approximate probability of the boy obtaining an estimate as low as or lower
than that given by the data above.

\begin{center}
\begin{tabular}{|c|c|c|c|c|c|c|c|c|} \hline
x & 11 & 13 & 16 & 17 & 20 & 22 & 25 & 33 \\ \hline
f & 2 & 2 & 3 & 1 & 2 & 3 & 1 & 1 \\ \hline
\end{tabular}
\end{center}

\\ \hline
      \end{tabular}
    \end{table}


\item When 1 , we have \[ \frac{1}{\bar{X}} \sim N\left( \frac{1}{6} , \frac{5}{216 \times 56} \right)\]
, i.e. $N(0.1667, 0.00041336)$; therefore
the probability of obtaining $\hat{p} \leq 0.125$ is approximately
When $p = \frac{1}{6}$,w 

\[ \frac{5}{216 \times 56}  = 0.000413\]
\[ \sqrt{\frac{5}{216 \times 56}}  = \sqrt{0.000413} = 0.0203\]


\[ \phi\left( \frac{0.125 - 0.1667}{0.0203} \right) = \Phi (-2.05) = 0.0202\]

\item The confidence interval for p did not include 1
6
; also now the probability being very
small is consistent with rejecting a null hypothesis that 
$p = \frac{1}{6}$
, i.e. that the die is fair.
\end{enumerate}
\end{document}

\documentclass[a4paper,12pt]{article}
%%%%%%%%%%%%%%%%%%%%%%%%%%%%%%%%%%%%%%%%%%%%%%%%%%%%%%%%%%%%%%%%%%%%%%%%%%%%%%%%%%%%%%%%%%%%%%%%%%%%%%%%%%%%%%%%%%%%%%%%%%%%%%%%%%%%%%%%%%%%%%%%%%%%%%%%%%%%%%%%%%%%%%%%%%%%%%%%%%%%%%%%%%%%%%%%%%%%%%%%%%%%%%%%%%%%%%%%%%%%%%%%%%%%%%%%%%%%%%%%%%%%%%%%%%%%
  \usepackage{eurosym}
\usepackage{vmargin}
\usepackage{amsmath}
\usepackage{graphics}
\usepackage{epsfig}
\usepackage{enumerate}
\usepackage{multicol}
\usepackage{subfigure}
\usepackage{fancyhdr}
\usepackage{listings}
\usepackage{framed}
\usepackage{graphicx}
\usepackage{amsmath}
\usepackage{chngpage}
%\usepackage{bigints}

\usepackage{vmargin}
% left top textwidth textheight headheight
% headsep footheight footskip
\setmargins{2.0cm}{2.5cm}{16 cm}{22cm}{0.5cm}{0cm}{1cm}{1cm}
\renewcommand{\baselinestretch}{1.3}

\setcounter{MaxMatrixCols}{10}
\begin{document}
%%%%%%%%%%%%%%%%%%%%%%%%%%%%%%%%%%%%%%%%%%%%%%%%%%%
Higher Certificate, Paper II, 2001. Question 8
%%%%%%%%%%%%%%%%%%%%%%%%%%%%%%%%%%%%%%%%%%%%%%%%%%%%%%%%%%%%%%%%%%%%%%%%%%%%%%%%%%%%%%%%%%%%%%%%%%%%%%%%%%%%%%%%%%%%%%%%%%%%%%%%%%%%%%%%%%%



 
 
 

%%%%%%%%%%%%%%%%%%%%%%%%%%%%%%%%%%%%%%%%%%%%%%%%%%%%%%%%%%%%%%%%%%%%%%%%%%%%%%%%%%%%%%%%%%%%%%%%%%%%%%%%%%%%%%%%%%%%%%%%%%%%%%%%%%%%%%%%%%%

%%%%%%%%%%%%%%%%%%%%%%%%%%%%%%%%%%%%%%%%%%%%%%%%%%%%%%%%%%%%%%%%%%%%%%%%%%%%%%%%%%%%%%%%%%%%%%%%%%%%%%%%%%%%%%%%%%%%%%%%%%%%%%%%%%%%%%%%%%% 
\begin{table}[ht!]
 
\centering
 
\begin{tabular}{|p{15cm}|}
 
\hline  

8. A specialist music school entered all its final year students for a national piano examination.  
The examination consisted of a written section and a practical section.  
The marks for each section together with the total mark achieved by each student are given in the following table. 
 Practical (out of 150) Written (out of 150) 
Total 
106 107 213 127 110 237 100   97 197 125 120 245 108 115 223 124 114 238 111 106 217   
96 104 200 115 105 220 134 100 234 145 145 290 107 103 210 140 133 273 105   98 203 110   96 206 
 
(i) Draw a box and whisker plot for the total examination mark and hence comment on the distribution. (8) 
 
\\ \hline
  
\end{tabular}

\end{table}



\begin{enumerate}[(a)]
\item  Total marks ranked in order of size from lowest:
\[\{197, 200, 203, 206, 210, 213, 217, 220, 223, 234, 237, 238, 245, 273, 290\}\]
q M Q
Median = 220. Lower quartile = 206 (or, with an alternative definition, ½(206 + 210)
= 208); upper quartile 238 (or 237 ½).
An alternative display marks the whisker ending at 273, with 290 shown as *. The
distribution is skew, since the median is not in the centre of the box and there is a very
long whisker to the right – although this is largely caused by the top two observations.

\newpage

\begin{table}[ht!]
 
\centering
 
\begin{tabular}{|p{15cm}|}
 
\hline  


(ii) A scatter plot of the marks obtained in the practical and written sections is given below.  
The product-moment correlation coefficient between the two sets of marks is 0.747.  
Calculate the Spearman rank correlation coefficient between the two sets of marks.  
What do these coefficients indicate about the association between these two sets of marks? (12) 
95 105 115 125 135 145
95
105
115
125
135
145
Practical
\\ \hline
  
\end{tabular}

\end{table} 
\item To compare the rank orders of Practical and Written, the 15 students' marks
need to be ranked and the difference in ranks, d, found. Then Spearman's coefficient
is ( )
2
2
6
1
1
d
n n
−
−
\sigma .
Student (1) (2) (3) (4) (5) (6) (7) (8) (9) (10) (11) (12) (13) (14) (15)

P 4 12 2 11 6 10 8 1 9 13 15 5 14 3 7
W 9 10 2 13 12 11 8 6 7 4 15 5 14 3 1
d −5 2 0 −2 −6 −1 0 −5 2 9 0 0 0 0 6
$\sigma_d^2 = 216$ . 1 6 216 0.614
15 224 S r = − × =
×
.
Both this and the product-moment coefficient are significant at the 1\% level, so there
is firm evidence that the practical and written marks increase together. The ranking
pattern is disturbed by (10), so reducing S r . The diagram shows a basically linear
relation with some noticeable scatter.


\end{enumerate}
\end{document}


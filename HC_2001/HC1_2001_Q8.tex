\documentclass[a4paper,12pt]{article}
%%%%%%%%%%%%%%%%%%%%%%%%%%%%%%%%%%%%%%%%%%%%%%%%%%%%%%%%%%%%%%%%%%%%%%%%%%%%%%%%%%%%%%%%%%%%%%%%%%%%%%%%%%%%%%%%%%%%%%%%%%%%%%%%%%%%%%%%%%%%%%%%%%%%%%%%%%%%%%%%%%%%%%%%%%%%%%%%%%%%%%%%%%%%%%%%%%%%%%%%%%%%%%%%%%%%%%%%%%%%%%%%%%%%%%%%%%%%%%%%%%%%%%%%%%%%
\usepackage{eurosym}
\usepackage{vmargin}
\usepackage{amsmath}
\usepackage{graphics}
\usepackage{epsfig}
\usepackage{enumerate}
\usepackage{multicol}
\usepackage{subfigure}
\usepackage{fancyhdr}
\usepackage{listings}
\usepackage{framed}
\usepackage{graphicx}
\usepackage{amsmath}
\usepackage{chngpage}
%\usepackage{bigints}

\usepackage{vmargin}
% left top textwidth textheight headheight
% headsep footheight footskip
\setmargins{2.0cm}{2.5cm}{16 cm}{22cm}{0.5cm}{0cm}{1cm}{1cm}
\renewcommand{\baselinestretch}{1.3}

\setcounter{MaxMatrixCols}{10}
\begin{document}
Higher Certificate, Paper I, 2001. Question 8
\begin{enumerate}
\item  yi =α + β xi +ε i
where y is the response (observation) and x the value of the explanatory variable on
that unit; { } i
ε is a set of independent, identically distributed random variables with
mean 0 and the same varianceσ 2 . Usually they are assumed Normal as a basis for
inference. i x is assumed "fixed", not "random".
%%%%%%%%%%%%%%%%%%%%%%%%%%%%%%%%%%%%%%%%%%%%%%%%%%%%%%%%%%%%%%
\item  (a) There is an increasing trend, and the relationship between y and x
appears curvilinear.
\begin{itemize}
\item In simple regression $R^2$ is the square of the correlation, r, between x
and y. In general it is the proportion of the variance of y which can be
explained by the dependence of y on all explanatory variables { } i x in the
model; hence it is the square of the correlation between ˆy and y.
\item In the ANOVA , regression SS 269.33 0.803, or 80.3\%
total SS 335.37
= = .
\item As x (\% operating capacity) increases by 1 unit so y (profit) increases
by 0.31562 units.
A 95\% confidence interval is 10 0.31562 ± t ×0.04942 , which is
0.31562 ± 0.11011or (0.2055, 0.4257) .
\item Values of profit have been predicted for capacity 25\%, 50\% and 75%.
The confidence intervals for these predictions are those given; but note that
25\% is far outside the range of available data (hence the remark about extreme
x values). A 95\% confidence interval is an interval which should cover the
true y at a given x with probability 0.95, based on the fitted linear regression.
\end{itemize}

%%%%%%%%%%%%%%%%%%%%%%%%%%%%%%%%%%%%%%%%%%%%%%%%%%%%%%%%%%%%%%
\item  (a) The logarithmic plot shows that a linear regression in these units is a much better fit. There is still an increasing trend.
%%%%%%%%%%%%%%%%%%%%
\begin{itemize}
\item log10(profits) = −0.519 + 0.0177(capacity)
i.e. profits =10−0.519+0.0177(capacity) .
\item For capacity = 25, the 95\% limits are –0.2731 and +0.1210, and the
actual prediction is –0.0760, in log10 units. Anti-logging these (i.e. raising 10
to these powers) we find the 95\% limits are 0.5332 and 1.3213.
The 'prediction' is 0.8395.
\end{itemize}
%%%%%%%%%%%%%%%%%%%%
These limits do not overlap the limits on the previous model.
The prediction now is for a small profit, compared with a loss on the previous
model.
\item  The scatter plots indicate that the logarithmic model is preferred, and so do the
plots of residuals which show a random pattern (as compared with a
systematic, curved one for the previous model).
\begin{itemize}
\item R2 also higher (92.4\%) on the
log model.
\item But a log model cannot predict negative profits – i.e. losses – which are quite
possible in general though not for these data if used within the range of x
values given.
\item Extrapolation down to 25\% is well outside the data and so is not reliable on
any model.
\end{itemize}

%%%%%%%%%%%%%%%%%%%%%%%%%%%%%%%%%%%%%%%%%%%%%%%%%%%%%%%%%%%%%%

\end{enumerate}
\end{document}

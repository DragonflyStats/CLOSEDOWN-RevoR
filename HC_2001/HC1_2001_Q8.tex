\documentclass[a4paper,12pt]{article}

%%%%%%%%%%%%%%%%%%%%%%%%%%%%%%%%%%%%%%%%%%%%%%%%%%%%%%%%%%%%%%%%%%%%%%%%%%%%%%%%%%%%%%%%%%%%%%%%%%%%%%%%%%%%%%%%%%%%%%%%%%%%%%%%%%%%%%%%%%%%%%%%%%%%%%%%%%%%%%%%%%%%%%%%%%%%%%%%%%%%%%%%%%%%%%%%%%%%%%%%%%%%%%%%%%%%%%%%%%%%%%%%%%%%%%%%%%%%%%%%%%%%%%%%%%%%

\usepackage{eurosym}
\usepackage{vmargin}
\usepackage{amsmath}
\usepackage{graphics}
\usepackage{epsfig}
\usepackage{enumerate}
\usepackage{multicol}
\usepackage{subfigure}
\usepackage{fancyhdr}
\usepackage{listings}
\usepackage{framed}
\usepackage{graphicx}
\usepackage{amsmath}
\usepackage{chngpage}

%\usepackage{bigints}
\usepackage{vmargin}

% left top textwidth textheight headheight

% headsep footheight footskip

\setmargins{2.0cm}{2.5cm}{16 cm}{22cm}{0.5cm}{0cm}{1cm}{1cm}

\renewcommand{\baselinestretch}{1.3}

\setcounter{MaxMatrixCols}{10}

\begin{document}Higher Certificate, Paper I, 2004. Question 8
\begin{enumerate}[(a)]
\item Trainee's time (y)
0
10
20
30
40
50
60
0 10 20 30 40 50
Benchmark time (x)
Simple linear regression analysis seems quite suitable.
\item  The model is yi = α + β xi + ei, where {ei} are uncorrelated with zero mean and
(constant) variance σ 2 (independent identically distributed $N(0, \sigma^2)$ for the purpose of
undertaking statistical tests, as in part (iii)). \begin{itemize}
\item Estimating by the method of least squares
gives
ˆ xy
xx
S
S
β = , αˆ = y −βˆ x ,
where (standard notation)
( )( ) i i
xy i i i i
x y
S x x y y xy
n
= Σ − − = − Σ Σ Σ ,
( ) ( )2
2 2 i
xx i i
x
S x x x
n
= − = − Σ Σ Σ .
\item We have
( )
( 2 )
ˆ 4440 150 220 /10 1140 1.20
3200 150 /10 950
xy
xx
S
S
β
− ×
= = = =
−
and αˆ = 22 − (1.20×15) = 4 ,
\item so the line is
y = 4 + 1.2x.
\item 
The total sum of squares is ( ) ( )2
2 2 1440
10
i
yy i i
y
S = y − y = y − = Σ Σ Σ .

\item The sum of squares for regression is ˆ
xy β S (or 2 / xy xx S S ) = 1368.
\item Therefore the residual sum of squares is 1440 – 1368 = 72.
\item This has 8 degrees of freedom, so the residual mean square ($\hat{\sigma}^2$ ) is 72/8 = 9.
\item The coefficient of determination $R^2 = 1368/1440 = 0.95$ (usually given as 95\%).
\end{itemize}

\item  The estimated variance of ˆβ is 9/950 = 0.009474. 
\begin{itemize}
    \item So the test statistic for
testing the null hypothesis β = 1 is 1.2 1
0.009474
− = 2.05, which we refer to t8.
\item This is not significant at the 5\% level, so the null hypothesis β = 1 cannot be rejected.
\end{itemize}

\item  The model here is yi = bxi + ei.
Estimating b by least squares, we minimise ( )2
1
n
i i
i
y bx
=
Ω =Σ − .
\begin{itemize}
    \item Differentiating with respect to b, we have 2 ( ) i i i
d y bx x
db
Ω = − Σ − .
\item Setting this equal to zero gives ˆ 2
i i i Σx y = bΣx , i.e. ˆ / 2 i i i b = Σx y Σx .
\end{itemize}

(Note that
2
2
2 2 0 i
d x
db
Ω = Σ > , so this is a minimum.)
Thus we have bˆ = 4440/3200 = 1.3875.
\end{enumerate}
\end{document}

\documentclass[a4paper,12pt]{article}
%%%%%%%%%%%%%%%%%%%%%%%%%%%%%%%%%%%%%%%%%%%%%%%%%%%%%%%%%%%%%%%%%%%%%%%%%%%%%%%%%%%%%%%%%%%%%%%%%%%%%%%%%%%%%%%%%%%%%%%%%%%%%%%%%%%%%%%%%%%%%%%%%%%%%%%%%%%%%%%%%%%%%%%%%%%%%%%%%%%%%%%%%%%%%%%%%%%%%%%%%%%%%%%%%%%%%%%%%%%%%%%%%%%%%%%%%%%%%%%%%%%%%%%%%%%%
\usepackage{eurosym}
\usepackage{vmargin}
\usepackage{amsmath}
\usepackage{graphics}
\usepackage{epsfig}
\usepackage{enumerate}
\usepackage{multicol}
\usepackage{subfigure}
\usepackage{fancyhdr}
\usepackage{listings}
\usepackage{framed}
\usepackage{graphicx}
\usepackage{amsmath}
\usepackage{chngpage}
%\usepackage{bigints}

\usepackage{vmargin}
% left top textwidth textheight headheight
% headsep footheight footskip
\setmargins{2.0cm}{2.5cm}{16 cm}{22cm}{0.5cm}{0cm}{1cm}{1cm}
\renewcommand{\baselinestretch}{1.3}

\setcounter{MaxMatrixCols}{10}
\begin{document}
Higher Certificate, Paper I, 2001. Question 8
\begin{table}[ht!]
     \centering
     \begin{tabular}{|p{15cm}|}
     \hline        
\noindent %==========================================================================================%
8. State the model for simple linear regression analysis on one explanatory variable.
What are the assumptions usually made regarding the stochastic term?

Data relating to percentage operating capacity and profits (£) per unit of output are
collected for 12 factories producing similar domestic refrigerators in the previous
year, as follows.
% operating
capacity
50 57 61 68 77 80 82 85 89 91 95 99
profit per unit
of output
2.5 4.0 3.1 4.6 7.3 6.2 6.1 11.6 10.0 14.2 16.1 19.5
These data are entered into Minitab, with profit in column 1 (i.e. c1) and
% operating capacity in column 2 (i.e. c2), and analysed as shown in the edited
output. Use the output to answer the following questions, noting the explanations
incorporated in it. c1 is named 'Profits' and c2 is named 'OpCapcty'.
(i) Consider the plot of profits against capacity and the subsequent regression
analysis.
(a) Comment on this plot.
(1)
(b) Explain what is meant by the statement R-Sq = 80.3\%. How might
this have been calculated from the Analysis of Variance
immediately following?
\\ \hline
      \end{tabular}
    \end{table}




\begin{enumerate}
\item  yi =α + β xi +ε i
where y is the response (observation) and x the value of the explanatory variable on
that unit; { } i
ε is a set of independent, identically distributed random variables with
mean 0 and the same varianceσ 2 . Usually they are assumed Normal as a basis for
inference. i x is assumed "fixed", not "random".
%%%%%%%%%%%%%%%%%%%%%%%%%%%%%%%%%%%%%%%%%%%%%%%%%%%%%%%%%%%%%%
\newpage
\begin{table}[ht!]
     \centering
     \begin{tabular}{|p{15cm}|}
     \hline        
\noindent
(c) What is the practical meaning of the slope parameter estimate
0.31562 ? Construct a 95\% confidence interval (CI) for the slope
parameter.

(d) The subcommand SUBC> predict … on page 13 produces the
lines starting 'Predicted Values' and ending 'very extreme X values'.
Noting that, for example, −15.799 + (0.31562)(25) = −7.909 and
that the 95\% CI is obtained as −7.909 ± 2.714 t10, 0.025 , explain the
meaning of this section of the output.

\\ \hline
      \end{tabular}
    \end{table}

\item  (a) There is an increasing trend, and the relationship between y and x
appears curvilinear.
\begin{itemize}
\item In simple regression $R^2$ is the square of the correlation, r, between x
and y. In general it is the proportion of the variance of y which can be
explained by the dependence of y on all explanatory variables { } i x in the
model; hence it is the square of the correlation between ˆy and y.
\item In the ANOVA , regression SS 269.33 0.803, or 80.3\%
total SS 335.37
= = .
\item As x (\% operating capacity) increases by 1 unit so y (profit) increases
by 0.31562 units.
A 95\% confidence interval is 10 0.31562 ± t ×0.04942 , which is
0.31562 ± 0.11011or (0.2055, 0.4257) .
\item Values of profit have been predicted for capacity 25\%, 50\% and 75%.
The confidence intervals for these predictions are those given; but note that
25\% is far outside the range of available data (hence the remark about extreme
x values). A 95\% confidence interval is an interval which should cover the
true y at a given x with probability 0.95, based on the fitted linear regression.
\end{itemize}
\begin{table}[ht!]
     \centering
     \begin{tabular}{|p{15cm}|}
     \hline        
\noindent Question 8 continued on next page.
Minitab output follows on pages 13 and 14.
12
(ii) In the analysis of Model 2, column 3 (i.e. c3) is constructed as
log10(profits).
(a) Compare the plot of log10(profits) against % operating capacity
with that of profits against % operating capacity.

(b) Use the regression of log10(profits) on % operating capacity to
express profits as a function of % operating capacity.
\\ \hline
      \end{tabular}
    \end{table}
    
    \begin{table}[ht!]
     \centering
     \begin{tabular}{|p{15cm}|}
     \hline        
\noindent 

(c) Convert the predicted value of log10(profits) for 25% operating
capacity, and its CI, to a corresponding prediction and CI for
profits. Compare your answers with the corresponding values
indicated in the output referred to in part (i)(d) above.

(iii) Discuss with reasons which of the two regressions you consider provides
the better summary of the data. Indicate any limitations which should be
borne in mind when using your preferred model.
\\ \hline
      \end{tabular}
    \end{table}
\begin{framed}

%%%%%%%%%%%%%%%%%%%%%%%%%%%%%%%%%%%%%%%%%%%%%%%%%%%%%%%%%%%%%%
\item  (a) The logarithmic plot shows that a linear regression in these units is a much better fit. There is still an increasing trend.
%%%%%%%%%%%%%%%%%%%%
\begin{itemize}
\item log10(profits) = −0.519 + 0.0177(capacity)
i.e. profits =10−0.519+0.0177(capacity) .
\item For capacity = 25, the 95\% limits are –0.2731 and +0.1210, and the
actual prediction is –0.0760, in log10 units. Anti-logging these (i.e. raising 10
to these powers) we find the 95\% limits are 0.5332 and 1.3213.
The 'prediction' is 0.8395.
\end{itemize}
%%%%%%%%%%%%%%%%%%%%
These limits do not overlap the limits on the previous model.
The prediction now is for a small profit, compared with a loss on the previous
model.
\item  The scatter plots indicate that the logarithmic model is preferred, and so do the
plots of residuals which show a random pattern (as compared with a
systematic, curved one for the previous model).
\begin{itemize}
\item R2 also higher (92.4\%) on the
log model.
\item But a log model cannot predict negative profits – i.e. losses – which are quite
possible in general though not for these data if used within the range of x
values given.
\item Extrapolation down to 25\% is well outside the data and so is not reliable on
any model.
\end{itemize}

%%%%%%%%%%%%%%%%%%%%%%%%%%%%%%%%%%%%%%%%%%%%%%%%%%%%%%%%%%%%%%

\end{enumerate}
%%%%%%%%%%%%%%%%%%%%%%%%%%%%%%%%%%%%%%%%%%%%%%%%%%%%%%%%%%%%%%
\newpage
\begin{framed}
Two pages of Minitab output follow
13 Turn over
MTB > set c1 # profit in £1000s, for 12 factories
DATA> 2.5 4.0 3.1 4.6 7.3 6.2 6.1 11.6 10.0 14.2 16.1 19.5 DATA> end
MTB > set c2 # corresponding % operating capacity for 12 factories as above
DATA> 50 57 61 68 77 80 82 85 89 91 95 99 DATA> end
MTB > name c1 'Profits' c2 'OpCapcty'
(i) Analysis of Model 1
MTB > plot c1 c2
-
- *
18.0+
-
Profits - *
- *
-
12.0+ *
-
- *
-
- *
6.0+ * *
- *
- * *
- *
----+---------+---------+---------+---------+---------+--OpCapcty
50 60 70 80 90 100
MTB > regress c1 1 c2;
SUBC> residual c3;
SUBC> predict c1 for c2 = 25; SUBC> predict c1 for c2 = 50; SUBC> predict c1 for c2 = 75.
The regression equation is Profits = - 15.8 + 0.316 OpCapcty
Predictor Coef StDev T P
Constant -15.799 3.918 -4.03 0.002
OpCapcty 0.31562 0.04942 6.39 0.000
S = 2.570 R-Sq = 80.3% R-Sq(adj) = 78.3%
Analysis of Variance
Source DF SS MS F P
Regression 1 269.33 269.33 40.78 0.000
Residual Error 10 66.04 6.60
Total 11 335.37
Predicted Values
Fit StDev Fit 95.0% CI
-7.909 2.714 ( -13.957, -1.860) XX
Fit StDev Fit 95.0% CI
-0.018 1.563 ( -3.500, 3.464)
Fit StDev Fit 95.0% CI
7.872 0.755 ( 6.190, 9.554)
X denotes use of X values away from the centre
XX denotes use of very extreme X values
MTB > let c4=c1-c3 # c4 is the fit
MTB > gstd
* NOTE * Standard Graphics are enabled. Professional Graphics are disabled.
MTB > plot c3 c4 # c3 contains the residuals from regression, c4 is the fit
-
c3 - *
-
3.0+
- *
- * *
- *
- *
0.0+
- *
- * *
-
- *
-3.0+ *
-
- *
--+---------+---------+---------+---------+---------+----c4
0.0 3.0 6.0 9.0 12.0 15.0
14
(ii) Analysis of Model 2
MTB > let c3=logten(c1)
MTB > name c3 ‘log Prof’
MTB > plot c3 c2
- *
-
1.20+ *
- *
log Prof- *
- *
-
0.90+
- *
- **
-
- *
0.60+ *
-
- *
- *
-
----+---------+---------+---------+---------+---------+--OpCapcty
50 60 70 80 90 100
MTB > regress c3 1 c2;
SUBC> residual c4;
SUBC> predict c3 for c2 = 25; SUBC> predict c3 for c2 = 50; SUBC> predict c3 for c2 = 75.
The regression equation is log Prof = - 0.519 + 0.0177 OpCapcty
Predictor Coef StDev T P
Constant -0.5185 0.1276 -4.06 0.002
OpCapcty 0.017699 0.001610 10.99 0.000
S = 0.08372 R-Sq = 92.4% R-Sq(adj) = 91.6%
Analysis of Variance
Source DF SS MS F P
Regression 1 0.84694 0.84694 120.83 0.000
Residual Error 10 0.07009 0.00701
Total 11 0.91703
Predicted Values
Fit StDev Fit 95.0% CI
-0.0760 0.0884 ( -0.2731, 0.1210) XX
Fit StDev Fit 95.0% CI
0.3664 0.0509 ( 0.2530, 0.4799)
Fit StDev Fit 95.0% CI
0.8089 0.0246 ( 0.7541, 0.8637)
X denotes a row with X values away from the centre
XX denotes a row with very extreme X values
MTB > let c5=c3-c4 # c4 contains the residuals from regression, c5 is the fit
MTB > plot c4 c5
-
- *
0.10+
- *
c4 - * *
- * *
- *
0.00+
- *
-
- * *
-
-0.10+ *
-
- *
--------+---------+---------+---------+---------+--------c5
0.48 0.64 0.80 0.96 1.12



\end{framed}

\end{document}

\documentclass[a4paper,12pt]{article}
%%%%%%%%%%%%%%%%%%%%%%%%%%%%%%%%%%%%%%%%%%%%%%%%%%%%%%%%%%%%%%%%%%%%%%%%%%%%%%%%%%%%%%%%%%%%%%%%%%%%%%%%%%%%%%%%%%%%%%%%%%%%%%%%%%%%%%%%%%%%%%%%%%%%%%%%%%%%%%%%%%%%%%%%%%%%%%%%%%%%%%%%%%%%%%%%%%%%%%%%%%%%%%%%%%%%%%%%%%%%%%%%%%%%%%%%%%%%%%%%%%%%%%%%%%%%
  \usepackage{eurosym}
\usepackage{vmargin}
\usepackage{amsmath}
\usepackage{graphics}
\usepackage{epsfig}
\usepackage{enumerate}
\usepackage{multicol}
\usepackage{subfigure}
\usepackage{fancyhdr}
\usepackage{listings}
\usepackage{framed}
\usepackage{graphicx}
\usepackage{amsmath}
\usepackage{chngpage}
%\usepackage{bigints}

\usepackage{vmargin}
% left top textwidth textheight headheight
% headsep footheight footskip
\setmargins{2.0cm}{2.5cm}{16 cm}{22cm}{0.5cm}{0cm}{1cm}{1cm}
\renewcommand{\baselinestretch}{1.3}

\setcounter{MaxMatrixCols}{10}
\begin{document}
%%%%%%%%%%%%%%%%%%%%%%%%%%%%%%%%%%%%%%%%%%%%%%%%%%%
Higher Certificate, Paper II, 2001. Question 1
%%%%%%%%%%%%%%%%%%%%%%%%%%%%%%%%%%%%%%%%%%%%%%%%%%%%%%%%%%%%%%%%%%%%%%%%%%%%%%%%%%%%%%%%%%%%%%%%%%%%%%%%%%%%%%%%%%%%%%%%%%%%%%%%%%%%%%%%%%% 
\begin{table}[ht!]
 
\centering
 
\begin{tabular}{|p{15cm}|}
 
\hline  

1. (i) State the central limit theorem and briefly explain its practical importance. (6) 


\\ \hline
  
\end{tabular}

\end{table}


 
%%%%%%%%%%%%%%%%%%%%%%%%%%%%%%%%%%%%%%%%%%%%%%%%%%%%%%%%%%%%%%%%%%%%%%%%%%%%%%%%%%%%%%%%%%%%%%%%%%%%%%%%%%%%%%%%%%%%%%%%%%%%%%%%%%%%%%%%%%%
\begin{enumerate}[(a)]
\item If random samples are taken from a non-Normal population, whose mean and
standard deviation are known to be $\mu$ and $\sigma$ , then when the sample size n is large the
sample mean ( X , say) will be approximately Normal with mean $\mu$  and standard
deviation
n
$\sigma$ . The approximation improves as n increases, and is adequate for
moderate size n if the X distribution is not very skew. The total of n observations has
a similar distribution (scaled up by a factor n). Thus estimators that are averages or
totals can often be taken as approximately Normal.
\begin{table}[ht!]
 
\centering
 
\begin{tabular}{|p{15cm}|}
 
\hline  
(ii) A fruit grower wishes to test a new spray that a manufacturer claims will reduce the amount of fruit lost due to damage by a certain insect.  To test the claim, the grower sprays 100 trees with the new spray and 100 other trees with his standard spray.  The yield of fruit was measured, in kg, for each tree.  Summary statistics were as follows. 
 
 New spray Standard spray Sample yield per tree 249 237 Sample variance 490 410 
 
Construct a 95\% confidence interval for the difference between the mean yields for the two sprays and interpret your findings. (7) 
 
 

\\ \hline
  
\end{tabular}

\end{table}



%%%%%%%%%%%%%%%%%%%%%%%%%%%%%%%%%%%%%%%%%%%%%%%%%%%%%%%%%
\item Let N, S be the new and standard sprays respectively; then the mean XN − XS
can be taken as N(249 − 237, 490 410
100 100
+ ) approximately (inserting the sample
estimates for the mean and variance), i.e. N(12,9).
A 95% confidence interval then is 12 \pm  (1.96) 9 = 12 \pm  5.88
or 6.12 to 17.88 kg for N S \mu −\mu .
The unbiased estimate of ( ) N S \mu −\mu is 12 kg, and the interval (6.12, 17.88) contains
the true value of ( ) N S \mu −\mu with probability 0.95. There is strong evidence to say
that N is better than S.
%%%%%%%%%%%%%%%%%%%%%%%%%%%%%%%%%%%%%%%%%%%%%%%%%%%%%%%%%
\newpage

\begin{table}[ht!]
 
\centering
 
\begin{tabular}{|p{15cm}|}
 
\hline  

(iii) The manufacturer of the new spray also claims that it can be used to prevent the loss due to insect damage of tender seedlings.  To test this claim, the grower sprays 50 tomato seedlings with the new spray and his remaining 100 tomato seedlings with his standard spray.  After six weeks, the fruit grower counts the number of healthy plants with the following results. 
\begin{center}
\begin{tabular}{ccc} 
 & New spray& Standard spray \\
 Number of seedlings  sprayed & 50 & 100 \\
 Number of healthy plants at six weeks&  40&  70 \\
\end{tabular}
\end{center}
Construct an approximate 95\% confidence interval for the difference in the proportion of healthy plants six weeks after spraying between the two groups. 
 

\\ \hline
  
\end{tabular}

\end{table} 
\item  ˆ 40 0.8
50 N p = = , the proportion of healthy plants on N; and ˆ 70 0.7
100 S p = = on
S. Also 50 N n = and 100 S n = . 

Using a Normal approximation to the binomial distribution, the true difference ˆ (1 ˆ ) ˆ (1 ˆ )
~ N ˆ ˆ ,
50 100
N N S S
N S N S
p p p p
p p p p
 − − 
−  − + 
 
A 95\% confidence interval for N S p − p is (0.8 0.7) 1.96 0.8 0.2 0.7 0.3
50 100
− \pm  × + ×
i.e. 0.1\pm 1.96×0.0728 or 0.1\pm  0.143 , i.e. (−0.04, 0.24).
It is possible that S may be better by 4%, but the upper limit is for N to be better by
24%.
\end{enumerate}
\end{document}


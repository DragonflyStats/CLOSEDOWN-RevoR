\documentclass[a4paper,12pt]{article}
%%%%%%%%%%%%%%%%%%%%%%%%%%%%%%%%%%%%%%%%%%%%%%%%%%%%%%%%%%%%%%%%%%%%%%%%%%%%%%%%%%%%%%%%%%%%%%%%%%%%%%%%%%%%%%%%%%%%%%%%%%%%%%%%%%%%%%%%%%%%%%%%%%%%%%%%%%%%%%%%%%%%%%%%%%%%%%%%%%%%%%%%%%%%%%%%%%%%%%%%%%%%%%%%%%%%%%%%%%%%%%%%%%%%%%%%%%%%%%%%%%%%%%%%%%%%
\usepackage{eurosym}
\usepackage{vmargin}
\usepackage{amsmath}
\usepackage{graphics}
\usepackage{epsfig}
\usepackage{enumerate}
\usepackage{multicol}
\usepackage{subfigure}
\usepackage{fancyhdr}
\usepackage{listings}
\usepackage{framed}
\usepackage{graphicx}
\usepackage{amsmath}
\usepackage{chngpage}
%\usepackage{bigints}

\usepackage{vmargin}
% left top textwidth textheight headheight
% headsep footheight footskip
\setmargins{2.0cm}{2.5cm}{16 cm}{22cm}{0.5cm}{0cm}{1cm}{1cm}
\renewcommand{\baselinestretch}{1.3}
%- Higher Certificate, Paper I, 2001. Question 3
\setcounter{MaxMatrixCols}{10}
\begin{document}
\begin{enumerate}
\item  ( 2300) 1 ( 2300) 1 2300 2000
300
P S ≥ = − P S < = −\Phi − 
 
= 1− \Phi(1) = 1− 0.8413 = 0.1587 .
( 2300) 1 2300 2500 1 ( 1.6)
125
P H ≥ = −\Phi −  = − \Phi −
 
= 1− 0.0548 = 0.9452 .
\item  


X \sim N(2000,300^2)

P(S \geq 2300) = 1- P(S \leq 2300)

\begin{framed}
\noindent \textbf{Z Score for $S = 2300$}

\[z_{2300}  = \frac{2300 - 2000}{ 300}  = \frac{300}{300} = 1\]
\end{framed}

%-----------%
$P(S >H) = P(S-H>0)$ where (S-H)  is N(-500, (300^2) + (125^2)

\[P(S >H) = \Phi\left( \frac{-500}{325} \right) = \phi (-1.5385) = 0.0620\]

%-----------%
%%%%%%%%%%%%%%%%%%%%%%%%%%%%%%%5
\item The lifetime X is S with probability 0.6 and H with probability 0.4.
Hence \[E[X ] = 0.6×2000 + 0.4× 2500 = 2200 \mbox{hrs} .\]
\begin{eqnarray*}
P( X > 2600) &=& P(X > 2600 | S)P(S ) + P( X > 2600 | H )P(H )\\
 &=& 600 0.6 100 0.4
300 125\\
&=& \Phi − × + \Phi − ×    \\
&=& 0.6 \Phi(−2) + 0.4 \Phi(−0.8) = 0.6×0.02275 + 0.4×0.2119\\
&=& 0.01365 + 0.08476 \\ 
&=& 0.09841.
\end{eqnaray*}
   
(using the appropriate tail areas from Normal tables)




%%%%%%%%%%%%%%%%%%%%%%%%%%%%%%%%%%%%%%%%%%%%%%%%%%

% HC3 2001 Q3
% Normal



\begin{eqnarray*}
P(X > 2600) &=& P(X \geq 2600|S)P(S) + P(X > 2600|H)P(H) \\
 &=& \Phi \left( - frac{600}{300}\right) + \Phi \left( - frac{600}{300}\right) \\
 &=& 0.01363 + 0.08476 \\
 &=& 0.09841 \\
\end{eqnarray*}

%%%%%%%%%%%%%%%%%%%%%%%%%%%%%%%
(ii) ( ) ( ) ( )
( )
2600 | 0.08476 | 2600 0.8613
2600 0.09841
P X H P H
P H
P X
>
> = = =
>
.
\item X will have mean 2200. It is not Normally distributed but we way
apply the Central Limit Theorem if we know its variance. In large samples we
may take X as approximately N(2200, 346.82), so that
( 2300) 1 2300 2200 1 100
346.8/ 100 34.68
P X > = −\Phi −  = −\Phi     
= \Phi(−2.8835) = 0.002 .
\end{enumerate}

\end{document}
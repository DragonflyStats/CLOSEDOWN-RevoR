\documentclass[a4paper,12pt]{article}
%%%%%%%%%%%%%%%%%%%%%%%%%%%%%%%%%%%%%%%%%%%%%%%%%%%%%%%%%%%%%%%%%%%%%%%%%%%%%%%%%%%%%%%%%%%%%%%%%%%%%%%%%%%%%%%%%%%%%%%%%%%%%%%%%%%%%%%%%%%%%%%%%%%%%%%%%%%%%%%%%%%%%%%%%%%%%%%%%%%%%%%%%%%%%%%%%%%%%%%%%%%%%%%%%%%%%%%%%%%%%%%%%%%%%%%%%%%%%%%%%%%%%%%%%%%%
\usepackage{eurosym}
\usepackage{vmargin}
\usepackage{amsmath}
\usepackage{graphics}
\usepackage{epsfig}
\usepackage{enumerate}
\usepackage{multicol}
\usepackage{subfigure}
\usepackage{fancyhdr}
\usepackage{listings}
\usepackage{multicol}
\usepackage{framed}
\usepackage{graphicx}
\usepackage{amsmath}
\usepackage{chngpage}
%\usepackage{bigints}

\usepackage{vmargin}
% left top textwidth textheight headheight
% headsep footheight footskip
\setmargins{2.0cm}{2.5cm}{16 cm}{22cm}{0.5cm}{0cm}{1cm}{1cm}
\renewcommand{\baselinestretch}{1.3}
%- Higher Certificate, Paper I, 2001. Question 3
\setcounter{MaxMatrixCols}{10}
\begin{document}


\begin{table}[ht!]
 \centering
 \begin{tabular}{|p{15cm}|}
 \hline
\noindent A manufacturer produces components of two quality grades:
\begin{itemize}
    \item Standard quality, with lifetimes distributed as N(2000, 90000), i.e.
 Normally with mean 2000 hours and standard deviation $\sqrt{90000}$ = 300
hours.
\item High quality, with lifetimes distributed as $N(2500, 15625)$.
\end{itemize}
\\
\noindent \textbf{Part (a)}\\
Find the probability that a randomly chosen standard component
lasts at least 2300 hours, and find the corresponding probability for
a high quality component.
\\ \hline
  \end{tabular}
\end{table}




\begin{enumerate}[(a)]
\item
\begin{multicols}{2}
\begin{eqnarray*}
P( S \geq 2300) &=& 1- P (S \leq 2300)\\
&=& 1-\Phi(1)\\
&=& 1- 0.8413 \\
&=& 0.1587 \\
\end{eqnarray*}
\begin{framed}
\noindent \textbf{Z Score for $S = 2300$}
\[z_{2300}  = \frac{2300 - 2000}{ 300}  = \frac{300}{300} = 1\]
\end{framed}
\begin{eqnarray*}
P( H \geq 2300) &=& 1- P (H \leq 2300)\\
&=& 1-\Phi(-1.6)\\
&=& \Phi(1.6) \\
&=& 0.9452 \\
\end{eqnarray*}
\begin{framed}
\noindent \textbf{Z Score for $H = 2300$}
\[z_{2300}  = \frac{2300 - 2500}{125}  = -\frac{200}{125} = -1.6\]
\end{framed}
\end{multicols}


\item  $X \sim N(2000,300^2)$
%%%%%%%%%%%%%%%%%%%%%%%%%%%%%%%5

\begin{table}[ht!]
 \centering
 \begin{tabular}{|p{15cm}|}
 \hline
\noindent  Find the probability that a randomly chosen standard component
lasts longer than a randomly chosen high quality component.
\\ \hline
  \end{tabular}
\end{table}

$P(S >H) = P(S-H>0)$ where (S-H)  is $N(-500, (300^2) + (125^2)$

\[P(S >H) = \Phi\left( \frac{-500}{325} \right) = \phi (-1.5385) = 0.0620\]

%%%%%%%%%%%%%%%%%%%%%%%%%%%%%%%5

\begin{table}[ht!]
 \centering
 \begin{tabular}{|p{15cm}|}
 \hline
\noindent  Due to a machine malfunction, a large batch of components is produced, of
which 60\% are standard and 40\% high quality; however, these
components are unlabelled and indistinguishable in appearance. A single
component is chosen at random from this batch.
(i) Find the expectation of its lifetime and the probability that it lasts
at least 2600 hours.
\\ \hline
  \end{tabular}
\end{table}
%%%%%%%%%%%%%%%%%%%%%%%%%%%%%%%5
\item The lifetime X is S with probability 0.6 and H with probability 0.4.
Hence \[E[X ] = 0.6×2000 + 0.4× 2500 = 2200 \mbox{hrs} .\]
\begin{eqnarray*}
P( X > 2600) &=& P(X > 2600 | S)P(S ) + P( X > 2600 | H )P(H )\\
 &=& 600 0.6 100 0.4
300 125\\
\end{eqnarray*}


\begin{framed}
\noindent \textbf{Z Score for $H = 2300$}
\[z_{2300}  = \frac{2300 - 2500}{125}  = -\frac{200}{125} = -1.6\]
\end{framed}

\begin{eqnarray*} 
P(X \geq 2600) &=& \left(0.6 \times \Phi(-2)\right) + \left(0.4 \times \Phi(-0.8)\right)\\
&=& \left(0.6 \times 0.02275 \right) + \left(0.4 \times 0.2119\right)\\
 &=& 0.01365 + 0.08476 \\ 
 &=& 0.09841.
\end{eqnarray*}

(using the appropriate tail areas from Normal tables)


%%%%%%%%%%%%%%%%%%%%%%%%%%%%%%%%%%%%%%%%%%%%%%%%%%



\begin{eqnarray*}
P(X > 2600) &=& P(X \geq 2600|S)P(S) + P(X > 2600|H)P(H) \\
 &=& \Phi \left( - \frac{600}{300}\right) + \Phi \left( - \frac{600}{300}\right) \\
 &=& 0.01363 + 0.08476 \\
 &=& 0.09841 \\
\end{eqnarray*}

   
%%%%%%%%%%%%%%%%%%%%%%%%%%%%%%%
\newpage
\begin{table}[ht!]
 \centering
 \begin{tabular}{|p{15cm}|}
 \hline
\noindent Given that a component from this batch lasts at least 2600 hours,
what is the probability that it is of the high quality type?
\\ \hline
  \end{tabular}
\end{table}
\item  ( ) ( ) ( )
( )
2600 |  | 2600 
2600 
\begin{eqnarray*}
P (X  H) &=& \frac{P(H>2600)}{P(X>2600)}\\
&=& \frac{0.08476}{0.8613}\\
&=& 0.09841
\end{eqnarray*}.

%%%%%%%%%%%%%%%%%%%%%%%%%%%%%%%%%%%%%%%%%%%%%%%%%%%%%%%%%%%%%%%%%
\newpage
\begin{table}[ht!]
 \centering
 \begin{tabular}{|p{15cm}|}
 \hline
\noindent (iii) Given that the standard deviation of the lifetime of a component
taken at random from this batch is 346.8 hours, find the
approximate probability that the mean lifetime of 100 such
components exceeds 2300 hours.
\\ \hline
  \end{tabular}
\end{table}
\item X will have mean 2200. It is not Normally distributed but we way
apply the Central Limit Theorem if we know its variance. In large samples we
may take X as approximately N(2200, 346.82), so that $P(\bar{X}\geq 2300)$ = 
  
\[\bar{X} \sim n(\mu = 2200, \sigma^2 = (346.8)^2\]
   
 \[Z_{2300}  =   \frac{2300-2200}{   \left(\frac{346.8}{\sqrt{100}} \right)} = 2.8835\]



\begin{eqnarray*} 
P(\bar{X}\geq 2300) &=& 1- \Phi(2.8835)\\  &=& 1- 0.9980 \\&=& 0.02
\end{eqnarray*}

\end{enumerate}

%%%%%%%%%%%%%%%%%%%%%%%%%%%%%%%%%%%%%%%%%%
  

  

\end{document}

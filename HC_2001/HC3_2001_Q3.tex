\documentclass[a4paper,12pt]{article}
%%%%%%%%%%%%%%%%%%%%%%%%%%%%%%%%%%%%%%%%%%%%%%%%%%%%%%%%%%%%%%%%%%%%%%%%%%%%%%%%%%%%%%%%%%%%%%%%%%%%%%%%%%%%%%%%%%%%%%%%%%%%%%%%%%%%%%%%%%%%%%%%%%%%%%%%%%%%%%%%%%%%%%%%%%%%%%%%%%%%%%%%%%%%%%%%%%%%%%%%%%%%%%%%%%%%%%%%%%%%%%%%%%%%%%%%%%%%%%%%%%%%%%%%%%%%
  \usepackage{eurosym}
\usepackage{vmargin}
\usepackage{amsmath}
\usepackage{graphics}
\usepackage{epsfig}
\usepackage{enumerate}
\usepackage{multicol}
\usepackage{subfigure}
\usepackage{fancyhdr}
\usepackage{listings}
\usepackage{framed}
\usepackage{graphicx}
\usepackage{amsmath}
\usepackage{chngpage}
%\usepackage{bigints}

\usepackage{vmargin}
% left top textwidth textheight headheight
% headsep footheight footskip
\setmargins{2.0cm}{2.5cm}{16 cm}{22cm}{0.5cm}{0cm}{1cm}{1cm}
\renewcommand{\baselinestretch}{1.3}

\setcounter{MaxMatrixCols}{10}
\begin{document}
%%%%%%%%%%%%%%%%%%%%%%%%%%%%%%%%%%%%%%%%%%%%%%%%%%%
Higher Certificate, Paper III, 2001. Question 3
%%%%%%%%%%%%%%%%%%%%%%%%%%%%%%%%%%%%%%%%%%%%%%%%%%%%%%%%%%%%%%%%%%%%%%%%%%%%%%%%%%%%%%%%%

\begin{table}[ht!]
     


\centering
     


\begin{tabular}{|p{15cm}|}
     


\hline 

3. A group of astronomers carried out a study of the relationship between light intensity and surface temperature.  
Data gathered on 24 stars in the cluster CYG OB1 are given in the table below.  Note that there are three outlying points indicated by an asterisk (*). 
 
Log surface temperature (x) 
Log light intensity  (y) 
Log surface temperature (x) 
Log light intensity  (y) 
Log surface temperature (x) 
Log light intensity  (y) 4.37 5.23 4.56 5.74 4.23 3.94 4.26 4.93 4.56 5.74 4.23 4.18 4.30 5.19 4.46 5.46 4.29 4.38  3.48* 6.05 4.57 5.27 4.42 4.42 4.26 5.57 4.37 5.12 4.42 4.18  3.49* 5.73 4.43 5.45  3.49* 5.89 4.48 5.42 4.43 5.57 4.29 4.22 4.29 4.26 4.42 4.58 4.49 4.85 * indicates outlying point 
 
(i) A regression analysis of the full data set was performed using a statistical package and produced the following output. 
 
     The regression equation is      \[Log (light intensity)  =  7.74  –  0.628 × Log (surface temperature) \]
 
Predictor Coeff St dev t p   Constant 7.74 1.73   4.48 <0.001   Slope −0.628 0.403 −1.56 0.134 s = 0.6207 R 2 = 9.9% 
 
A quick look at the data suggests that there is a positive relationship between surface temperature and light intensity.  
However, the estimate of the slope is negative.  Why is this?  
 (ii) The astronomers decided to asses the impact of the three outlying data points by deleting them and then calculating the following summaries. 
 92.13x =∑   103.70 y=∑ 2 404.4303 x =∑ 2 519.0608y =∑ 455.7101 xy=∑  
 
Estimate the slope of the regression line after removing the outlying points and test the hypothesis that the slope is zero.  
 (iii) Construct a scatter plot of the full data set.  Explain why the estimated slopes in (i) and (ii) have different signs.  
 (iv) What conclusions do you draw from these analyses?  

 
\\ \hline



\end{tabular}
    


\end{table}


%%%%%%%%%%%%%%%%%%%%%%%%%%%%%%%%%%%%%%%%%%%%%%%%%%%%%%%%%%%%%%%%%%%%%%%%%%%%%%%%%%%%%%%%%

\begin{enumerate}[(a)]
\item  The graph below shows that the three points in the top left corner have a large
influence in the calculation of slope.
3.6
4
4.4
4.8
5.2
5.6
6
6.4
3.3 3.5 3.7 3.9 4.1 4.3 4.5 4.7
log surface temperature
(ii) Without the three outlying points, N = 21 and x = 4.387, y = 4.938 .
455.7101 1 (92.13)(103.7) 0.76339
21 XY S = − =
92.132 404.4303 0.24283
21 XX S = − =
Hence ˆ XY 3.144
XX
b S
S
= = .
(iii) This bˆ is much larger, and positive, as there is a strong tendency for log y and
log x to increase together, except for the three outlying points already mentioned.
To test the null hypothesis b = 0 we need the Analysis of Variance:
Total
103.72 SS 519.0608 6.98032
21 YY = S = − = .
Regression
2
SS XY 2.39989
XX
S
S
= = .
SOURCE DF SS MS
Regression 1 2.39989
Deviations 19 4.58043 0.24108 = \sigmaˆ 2
TOTAL 20 6.98032
( ) ( ) ˆ ˆ 2 ˆ Var 0.9928, 0.996
XX
b SE b
S
= \sigma = = .
Value of test statistic is 3.144/0.996 = 3.16. Comparing with t19, this is significant at
the 1% level. Reject the null hypothesis b = 0.
(iv) Including the outliers gives a slope which is negative, but not significantly
different from 0; and very little variation (9.9%) is explained. Removing them allows
34.4% of variation to be explained, with a slope that is clearly not zero.
The three points are the highest surface temperature values. They do not seem to be
of the same population as the rest; perhaps some different mechanism is operating at
high temperatures.
\end{enumerate}
\end{document}


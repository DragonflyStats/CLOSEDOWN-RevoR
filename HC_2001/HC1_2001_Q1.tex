\documentclass[a4paper,12pt]{article}
%%%%%%%%%%%%%%%%%%%%%%%%%%%%%%%%%%%%%%%%%%%%%%%%%%%%%%%%%%%%%%%%%%%%%%%%%%%%%%%%%%%%%%%%%%%%%%%%%%%%%%%%%%%%%%%%%%%%%%%%%%%%%%%%%%%%%%%%%%%%%%%%%%%%%%%%%%%%%%%%%%%%%%%%%%%%%%%%%%%%%%%%%%%%%%%%%%%%%%%%%%%%%%%%%%%%%%%%%%%%%%%%%%%%%%%%%%%%%%%%%%%%%%%%%%%%
\usepackage{eurosym}
\usepackage{vmargin}
\usepackage{amsmath}
\usepackage{graphics}
\usepackage{epsfig}
\usepackage{enumerate}
\usepackage{multicol}
\usepackage{subfigure}
\usepackage{fancyhdr}
\usepackage{listings}
\usepackage{framed}
\usepackage{graphicx}
\usepackage{amsmath}
\usepackage{chngpage}
%\usepackage{bigints}

\usepackage{vmargin}
% left top textwidth textheight headheight
% headsep footheight footskip
\setmargins{2.0cm}{2.5cm}{16 cm}{22cm}{0.5cm}{0cm}{1cm}{1cm}
\renewcommand{\baselinestretch}{1.3}

\setcounter{MaxMatrixCols}{10}
\begin{document}
\large
% Higher Certificate, Paper I, 2001. Question 1
%%%%%%%%%%%%%%%%%%%%%%%%%%%%%%%%%%%%%%%%%%%%%%%%%%%%%%
\begin{table}[ht!]
 \centering
 \begin{tabular}{|p{15cm}|}
 \hline
\noindent 1. The events A, B and C are such that
A is independent of B,
A is independent of C,
and
( )
( )
( )
( )
( )
23
1
2
35
1
4
1
10
,
,
,
,
.
P A
P B
P C
P A B C
P A B C
=
=
=
\cap \cap =
\cap \cap =
\\ \hline
  \end{tabular}
\end{table}
%%%%%%%%%%%%%%%%%%%%%%%%%%%%%%%%%%%%%%%%%%%%%%%%%%%%%%
\begin{table}[ht!]
 \centering
 \begin{tabular}{|p{15cm}|}
 \hline
\noindent Find

(i) $P(A \cap B)$ ,
\\ \hline
  \end{tabular}
\end{table}

%%%%%%%%%%%%%%%%%%%%%%%%%%%%%%%%%%%%%%%%%%%%%%%%%%%%%%

\begin{itemize}
\item[(a)] A and B are independent. So $P(A\cap B) = P(A)P(B)$ .

\begin{itemize}
\item[$\bullet$] $P (\bar{A}) = 1 - P(A) = 1/3 $
\item[$\bullet$] $P (B) = 1/2$
\item[$\bullet$] $P (A\cap B)= 1/6$
\end{itemize}
\end{itemize}
%%%%%%%%%%%%%%%%%%%%%%%%%%%%%%%%%%%%%%%%%%%%%%%%%%%%%%%%%%%%%%%%%%%%%%%%%%

\begin{table}[ht!]
 \centering
 \begin{tabular}{|p{15cm}|}
 \hline
\noindent Find
(ii) $P(A \cap B \cap C)$ ,
\\ \hline
  \end{tabular}
\end{table}

\begin{itemize}
\item[(b)] \[A\cap B = [(A\cap B)\cap C]\cup[(A\cap B)\cap C],\]
with the two events $[(A\cap B)\cap C]$ and $[(A\cap B)\cap C]$ being disjoint .
Hence 
\begin{eqnarray*}
P A\cap B \cap C &=& P A\cap B - P A\cap B \cap C \\ #
&=& \frac{1}{6} - \frac{1}{10}\\
&=& \frac{5}{30} - \frac{3}{30}\\
&=&  \frac{1}{15}\\
\end{eqnarray*}
\end{itemize}
%%%%%%%%%%%%%%%%%%%%
\begin{table}[ht!]
 \centering
 \begin{tabular}{|p{15cm}|}
 \hline
\noindent Find
(iii) $P(B \cap C)$ ,


\\ \hline
  \end{tabular}
\end{table}

\begin{itemize}
\item[(c)]  \[B\cap C = (A\cap B\cap C)\cup(\bar{A} \cap B\cap C), \] these being disjoint.
\begin{itemize}
\item[$\bullet$] Further, \[(A\cap B \cap C)\cup(A\cap B \cap C) = (A\cap B) .\]
\item[$\bullet$] Hence \[P(B \cap C) = P(A\cap B \cap C) + P(\bar{A} \cap B) - P(\bar{A} \cap B \cap \bar{C})\]
\[P(B \cap C) =  \frac{1}{4}+ \left( \frac{1}{3} \times \frac{1}{2} \right) 
- P (\bar{A} \cap \bar{C} ) - P (\bar{A} \cap \bar{B} \cap \bar{C}) \]
since A,B are independent.

\item[$\bullet$] But $A$, $C$ are also independent and so are $\bar{A}$,$\bar{C}$ .
\end{itemize}

Therefore 
\begin{eqnarray*}
P( B \cap C)  &=& \frac{1}{4} +\frac{1}{6} -\left( \frac{1}{3} \times \frac{2}{5} \right) + \frac{1}{15} \\
&=& \frac{3}{12} +\frac{2}{12} -\left(  \frac{2}{15} \right) + \frac{1}{15}
\\
&=& \frac{5}{12} -\frac{1}{15}\\
&=& \frac{25}{60} -\frac{4}{60}\\
&=& \frac{21}{60} = \frac{7}{20}\\
\end{eqnarray*}

\end{itemize}

%%%%%%%%%%%%%%%%%%%%%%%%%%%%%%%%%%%%%%%%%%%%%%%%%%%%%%%%%%%%%%%%%%%%

\begin{table}[ht!]
 \centering
 \begin{tabular}{|p{15cm}|}
 \hline
\noindent 
$P (B | C)$ ,
\\ \hline
  \end{tabular}
\end{table}
\begin{itemize}
\item[(d)] Compute $P (B | C)$

\[
P (B | C) = \frac{P( B \cap C)}{P (C)} = \frac{7/20}{3/5} = \frac{7/20}{12/20} = \frac{7}{12}\]
%%%%%%%%%%%%%%%%%%%%%%%%%%%%%%%%%%%%%%%%%%%%%%%%%%%%%%%%%%%%%%%%%%%%
\begin{table}[ht!]
 \centering
 \begin{tabular}{|p{15cm}|}
 \hline
\noindent  $P( A B\cap C)$ ,


\\ \hline
  \end{tabular}
\end{table}

\item Compute $P (A | B  \cap C)$ 
\[
P (A | B  \cap C) = \frac{P( A \cap B \cap C)}{P (B \cap C)} = \frac{1/4}{7/20} = \frac{20}{28} = \frac{5}{7}\]

\begin{table}[ht!]
 \centering
 \begin{tabular}{|p{15cm}|}
 \hline
\noindent 

 $P( A\cap B A\cap C)$ .
\\ \hline
  \end{tabular}
\end{table}


\item Compute $P (A \cap B | A \cap C)$ 
\[
P (A \cap B | A \cap C) = \frac{P( A \cap B \cap C)}{P (A \cap C)} = \frac{1/4}{2/3 \times 3/5} = \frac{1/4}{6/15} = \frac{15}{24} = \frac{5}{8}\]
(A, C are independent).
\end{itemize}
\end{document}

\documentclass[a4paper,12pt]{article}
%%%%%%%%%%%%%%%%%%%%%%%%%%%%%%%%%%%%%%%%%%%%%%%%%%%%%%%%%%%%%%%%%%%%%%%%%%%%%%%%%%%%%%%%%%%%%%%%%%%%%%%%%%%%%%%%%%%%%%%%%%%%%%%%%%%%%%%%%%%%%%%%%%%%%%%%%%%%%%%%%%%%%%%%%%%%%%%%%%%%%%%%%%%%%%%%%%%%%%%%%%%%%%%%%%%%%%%%%%%%%%%%%%%%%%%%%%%%%%%%%%%%%%%%%%%%
\usepackage{eurosym}
\usepackage{vmargin}
\usepackage{amsmath}
\usepackage{graphics}
\usepackage{epsfig}
\usepackage{enumerate}
\usepackage{multicol}
\usepackage{subfigure}
\usepackage{fancyhdr}
\usepackage{listings}
\usepackage{framed}
\usepackage{graphicx}
\usepackage{amsmath}
\usepackage{chngpage}
%\usepackage{bigints}

\usepackage{vmargin}
% left top textwidth textheight headheight
% headsep footheight footskip
\setmargins{2.0cm}{2.5cm}{16 cm}{22cm}{0.5cm}{0cm}{1cm}{1cm}
\renewcommand{\baselinestretch}{1.3}

\setcounter{MaxMatrixCols}{10}
\begin{document}
\large
% Higher Certificate, Paper I, 2001. Question 1
%%%%%%%%%%%%%%%%%%%%%%%%%%%%%%%%%%%%%%%%%%%%%%%%%%%%%%
\begin{table}[ht!]
 \centering
 \begin{tabular}{|p{15cm}|}
 \hline
 \large
\noindent %1. 
The events A, B and C are such that 

\begin{itemize} 
\item A is independent of B, 
\item A is independent of C, 
\end{itemize} 
 and 
\begin{multicols}{2}
\begin{itemize}
\item ${ \displaystyle P(A ) = \frac{2}{3} }$
\item ${ \displaystyle P(B ) =  \frac{1}{2} }$
\item ${ \displaystyle P(C ) = \frac{3}{5} }$
\item ${ \displaystyle P(A \cap B \cap C) = \frac{1}{4} }$
\item ${ \displaystyle P(\bar{A} \cap \bar{B} \cap C)  = \frac{1}{10} }$
\end{itemize} 
\end{multicols} \\
\\ \hline
  \end{tabular}
\end{table}

\bigskip
\begin{table}[ht!]
 \centering
 \begin{tabular}{|p{15cm}|}
 \hline
 \large
 \\ \large
Find the following
\begin{multicols}{3}
\begin{enumerate}[(a)]
\item ${ \displaystyle P(\bar{A} \cap \bar{B} ) }$  

 
\item ${ \displaystyle P(\bar{A} \cap \bar{B} \cap \bar{C}   }$  
 
 
\item ${ \displaystyle P (B \cap  C) }$
 
\item ${ \displaystyle P (B |  C) }$

\item ${ \displaystyle P (A| B \cap  C) }$
 

\item ${ \displaystyle P (A  \cap  B| B \cap  C) }$
\end{enumerate} 
\end{multicols}
\\ \hline
  \end{tabular}
\end{table}
%%%%%%%%%%%%%%%%%%%%%%%%%%%%%%%%%%%%%%%%%%%%%%%%%%%%%%
\newpage
\begin{table}[ht!]
 \centering
 \begin{tabular}{|p{15cm}|}
 \hline
 \large
\noindent \textbf{Part a} \\
\large
Find $P(\bar{A} \cap \bar{B} )$ ,
\\ \hline
  \end{tabular}
\end{table}

%%%%%%%%%%%%%%%%%%%%%%%%%%%%%%%%%%%%%%%%%%%%%%%%%%%%%%
\large
\begin{itemize}
\item[(a)] A and B are independent. So $P(A\cap B) = P(A)P(B)$ .

\begin{itemize}
\item[$\bullet$] $P (\bar{A}) = 1 - P(A) = 1/3 $
\item[$\bullet$] $P (\bar{B}) = 1/2$
\item[$\bullet$] $P (\bar{A} \cap \bar{B} )= 1/6$
\end{itemize}
\end{itemize}
%%%%%%%%%%%%%%%%%%%%%%%%%%%%%%%%%%%%%%%%%%%%%%%%%%%%%%%%%%%%%%%%%%%%%%%%%%
\newpage
\begin{table}[ht!]
 \centering
 \begin{tabular}{|p{15cm}|}
 \hline
 \large
\noindent \textbf{Part b} \\
\large Find
$P(\bar{A} \cap \bar{B} \cap{C})$ ,
\\ \hline
  \end{tabular}
\end{table}

\begin{itemize}
\item[(b)] \[A\cap B = [(\bar{A} \; \cap \; \bar{B})\cap C]\;\cup\;[(\bar{A} \;\cap\; \bar{B})\;\cap\; \bar{C}],\]
with the two events $[(A\cap B)\cap C]$ and $[(A\cap B)\cap C]$ being disjoint .
Hence 
\begin{eqnarray*}
P( A\cap B \cap C) &=& P (A\cap B - P A\cap B \cap C) \\ 
&=& \frac{1}{6} - \frac{1}{10}\\
&=& \frac{5}{30} - \frac{3}{30}\\
&=&  \frac{1}{15}\\
\end{eqnarray*}
\end{itemize}
%%%%%%%%%%%%%%%%%%%%
\newpage
\begin{table}[ht!]
 \centering
 \begin{tabular}{|p{15cm}|}
 \hline
\noindent Find
(iii) $P(B \cap C)$ ,


\\ \hline
  \end{tabular}
\end{table}

\begin{itemize}
\item[(c)]  \[B\cap C = (A\cap B\cap C)\;\cup\;(\bar{A} \;\cap\; B \;\cap\; C), \] these being disjoint.
\begin{itemize}
\item[$\bullet$] Further, \[(A\cap B \cap C)\cup(A\cap B \cap C) = (A\cap B) .\]
\item[$\bullet$] Hence \[P(B \cap C) = P(A\cap B \cap C) + P(\bar{A} \cap B) - P(\bar{A} \cap B \cap \bar{C})\]
\[P(B \cap C) =  \frac{1}{4}+ \left( \frac{1}{3} \times \frac{1}{2} \right) 
- P (\bar{A} \cap \bar{C} ) - P (\bar{A} \cap \bar{B} \cap \bar{C}) \]
since A,B are independent.

\item[$\bullet$] But $A$, $C$ are also independent and so are $\bar{A}$,$\bar{C}$ .
\end{itemize}

Therefore 
\begin{eqnarray*}
P( B \cap C)  &=& \frac{1}{4} +\frac{1}{6} -\left( \frac{1}{3} \times \frac{2}{5} \right) + \frac{1}{15} \\
&=& \frac{3}{12} +\frac{2}{12} -\left(  \frac{2}{15} \right) + \frac{1}{15}
\\
&=& \frac{5}{12} -\frac{1}{15}\\
&=& \frac{25}{60} -\frac{4}{60}\\
&=& \frac{21}{60} = \frac{7}{20}\\
\end{eqnarray*}

\end{itemize}

%%%%%%%%%%%%%%%%%%%%%%%%%%%%%%%%%%%%%%%%%%%%%%%%%%%%%%%%%%%%%%%%%%%%
\newpage
\begin{table}[ht!]
 \centering
 \begin{tabular}{|p{15cm}|}
 \hline
\noindent 
$P (B | C)$ ,
\\ \hline
  \end{tabular}
\end{table}
\begin{itemize}
\item[(d)] Compute $P (B | C)$

\begin{eqnarray*}
P (B | C) &=& \frac{P( B \cap C)}{P (C)} \\ &=& \frac{7/20}{3/5}\\ &=& \frac{7/20}{12/20} \\ &=& \frac{7}{12} \\\end{eqnarray*}
%%%%%%%%%%%%%%%%%%%%%%%%%%%%%%%%%%%%%%%%%%%%%%%%%%%%%%%%%%%%%%%%%%%%
\newpage
\begin{table}[ht!]
 \centering
 \begin{tabular}{|p{15cm}|}
 \hline
\noindent  $P( A B\cap C)$ ,
\\ \hline
 \end{tabular}
\end{table}

\item Compute $P (A | B  \cap C)$ 
\begin{eqnarray*}
P (A | B  \cap C) &=& \frac{P( A \cap B \cap C)}{P (B \cap C)} \\ 
&=& \frac{1/4}{7/20} \\ 
&=&  \frac{20}{28} \\ 
&=&  \frac{5}{7} \\
\\\end{eqnarray*}


%%%%%%%%%%%%%%%%%%%%%%%%%%%%%%%%%%%%%%%%%%%%%%%%%%%%%%%%%%%%%%%%%%%%
\newpage
\begin{table}[ht!]
 \centering
 \begin{tabular}{|p{15cm}|}
 \hline
\noindent 

\large Find $P (A  \cap  B| B \cap  C))$ .
\\ \hline
  \end{tabular}
\end{table}


\item Compute $P (A \cap B | A \cap C)$ 
\begin{eqnarray*}
P (A \cap B | A \cap C) = \frac{P( A \cap B \cap C)}{P (A \cap C)} \\ 
&=&  \frac{1/4}{2/3 \times 3/5} \\ 
&=& \frac{1/4}{6/15} \\ 
&=&  \frac{15}{24} \\ 
&=&  \frac{5}{8} \\
\\\end{eqnarray*}
(A, C are independent).
\end{itemize}
\end{document}

\documentclass[a4paper,12pt]{article}
%%%%%%%%%%%%%%%%%%%%%%%%%%%%%%%%%%%%%%%%%%%%%%%%%%%%%%%%%%%%%%%%%%%%%%%%%%%%%%%%%%%%%%%%%%%%%%%%%%%%%%%%%%%%%%%%%%%%%%%%%%%%%%%%%%%%%%%%%%%%%%%%%%%%%%%%%%%%%%%%%%%%%%%%%%%%%%%%%%%%%%%%%%%%%%%%%%%%%%%%%%%%%%%%%%%%%%%%%%%%%%%%%%%%%%%%%%%%%%%%%%%%%%%%%%%%
  \usepackage{eurosym}
\usepackage{vmargin}
\usepackage{amsmath}
\usepackage{graphics}
\usepackage{epsfig}
\usepackage{enumerate}
\usepackage{multicol}
\usepackage{subfigure}
\usepackage{fancyhdr}
\usepackage{listings}
\usepackage{framed}
\usepackage{graphicx}
\usepackage{amsmath}
\usepackage{chngpage}
%\usepackage{bigints}

\usepackage{vmargin}
% left top textwidth textheight headheight
% headsep footheight footskip
\setmargins{2.0cm}{2.5cm}{16 cm}{22cm}{0.5cm}{0cm}{1cm}{1cm}
\renewcommand{\baselinestretch}{1.3}

\setcounter{MaxMatrixCols}{10}
\begin{document}
%%%%%%%%%%%%%%%%%%%%%%%%%%%%%%%%%%%%%%%%%%%%%%%%%%%%%%%%%%%%%%%%%%%%%%%%%%%%%%%%%%%%%%%%%%%%%%%%%%

Higher Certificate, Paper II, 2001. Question 7
\begin{enumerate}[(a)]
\item  yij = \mu +\alpha i +ε ij , where ij y is the observation on the jth unit receiving
treatment i (or lying in group i), \mu is a grand mean, i
\alpha  is an 'effect' (departure from
\mu ) due to treatment (or group) i and ij ε is a random, Normally distributed, residual
'error' term, all { } ij ε independent and all having variance \sigma 2 .
The model is "additive", constructed by adding quantities rather than (for example)
multiplying them; this and the properties of { } ij ε are necessary assumptions for the
analysis.
\item
Woodland n \sigmax \sigmax2 x (\sigmax)2/n
A 10 664 45780 66.400 44089.600
B 8 313 13453 39.125 12246.125
C 6 402 27938 67.000 26934.000
N=24 1379=G 87171 83269.725
2
G 79235.042
N
=
Total corrected SS
2
87171 G 7935.958
N
= − =
Woodlands SS
2
83269.725 G 4034.683
N
= − =
Analysis of Variance
ITEM DF SUM OF SQUARES MEAN SQUARE
Woodlands 2 4034.683 2017.342 F2,21 = 10.86
Residual 21 3901.275 185.775=s2
TOTAL 23 7935.958
Comparing 10.86 with F2,21, the null hypothesis "all i
\alpha  are zero" can be rejected at
the 0.1% significance level.
From the values of x , we can immediately see that the reason for this is that B is
different from the other two.
Compare B and C:
xC − xB = 27.875 , SE of difference 2 1 1
8 6
= s  + 
 
.
Hence the 21 t test statistic is 27.875 3.79
7.361
= which is significant at the 0.1% level.
Alternatively, a 95% confidence interval for ( C B \mu −\mu ) is
27.875 \pm  2.080×7.361 i.e. (12.56 to 43.19) .
The precision of the results is poor, as shown by the wide interval.
There is a similar result for B versus A.
\end{enumerate}
\end{document}

\documentclass[a4paper,12pt]{article}
%%%%%%%%%%%%%%%%%%%%%%%%%%%%%%%%%%%%%%%%%%%%%%%%%%%%%%%%%%%%%%%%%%%%%%%%%%%%%%%%%%%%%%%%%%%%%%%%%%%%%%%%%%%%%%%%%%%%%%%%%%%%%%%%%%%%%%%%%%%%%%%%%%%%%%%%%%%%%%%%%%%%%%%%%%%%%%%%%%%%%%%%%%%%%%%%%%%%%%%%%%%%%%%%%%%%%%%%%%%%%%%%%%%%%%%%%%%%%%%%%%%%%%%%%%%%
  \usepackage{eurosym}
\usepackage{vmargin}
\usepackage{amsmath}
\usepackage{graphics}
\usepackage{epsfig}
\usepackage{enumerate}
\usepackage{multicol}
\usepackage{subfigure}
\usepackage{fancyhdr}
\usepackage{listings}
\usepackage{framed}
\usepackage{graphicx}
\usepackage{amsmath}
\usepackage{chngpage}
%\usepackage{bigints}

\usepackage{vmargin}
% left top textwidth textheight headheight
% headsep footheight footskip
\setmargins{2.0cm}{2.5cm}{16 cm}{22cm}{0.5cm}{0cm}{1cm}{1cm}
\renewcommand{\baselinestretch}{1.3}

\setcounter{MaxMatrixCols}{10}
\begin{document}
Higher Certificate, Paper I, 2001. Question 5
\begin{table}[ht!]
 \centering
 \begin{tabular}{|p{15cm}|}
 \hline
\noindent 5. In an experiment, the number of events happening in any unit time interval is a
Poisson random variable X with probability mass function
( ) !
x
p x e
x
= −λ λ , x = 0, 1, 2, … , λ > 0 .
Show that
( 1) ( )
1
px px
x
+ = λ
+
, x = 0, 1, 2, … .
(1)
Draw graphs of p(x) for the cases λ = ½ and λ = 2.

Obtain the moment generating function (mgf) of X and hence show that
E(X) = Var(X) = λ. Show also that E[(X − λ)3] = λ.
\\ \hline
  \end{tabular}
\end{table}

%%%%%%%%%%%%%%%%%%%%%%%%%%%%%%%%%%%%5
\begin{framed}
\noindent The moment-generating function of a random variable X is 
\[{\displaystyle M_{X}(t):=\operatorname{E} \left[e^{tX}\right],\quad t\in \mathbb{R} ,} \]

\noindent wherever this expectation exists. In other words, the moment-generating function is the expectation of the random variable 
${\displaystyle e^{tX}}. $
\end{framed}

\begin{enumerate}
\item ( )
( ) ( )
1 1 ! .
1 ! 1
x
x
p x e x
p x x e x
\lambda
\lambda
\lambda \lambda
\lambda
− +
  −
+
  = =
  + +
  for x = 0, 1, 2, … .
For 1 , (0) 0.60653; so (1) 0.30327, (2) 0.07582, (3) 0.01264
2
\lambda = p = p = p = p = .
Graph of p(x) for $$\lambda = 1/2$.
For $\lambda = 2$, p(0) = 0.13534;

so
\begin{itemize}
\item p(1) = 0.27067,
\item p(2) = 0.27067,
\item p(3) = 0.18045, 
\item p(4) = 0.09022,
\item p(5) = 0.03609.
\end{itemize}

Graph of p(x) for $\lambda = 2$.
0
0.1
0.2
0.3
0.4
0.5
0.6
0.7
0 1 2 3
0
0.1
0.2
0.3
0 1 2 3 4 5
( ) ( ) ( { })
0 0
. exp 1
! !
  t
  
\noindent The moment-generating function of a random variable X is 
\begin{eqnarray*}   
M_{X}(t) &=& \operatorname {E} \left[e^{tX}\right] \\
&=& \sum^{\infty}_{x=0} \frac{e^{xt}e^{-\lambda} \lambda^x}{x!}\\
&=&  e^{-\lambda} \sum^{\infty}_{x=0} \frac{e (\lambda^{t})^x}{x!}\\
&=& e^{-\lambda} \times e^{lambda e^{t} } \\
&=& e^{\lambda(e^{lambda t} - 1)\\
\end{eqnarray*}

\smallskip

\begin{eqnarray*} 
\frac{\partialM}{\partial t} &=&  \left(e^{\lambda(e^{lambda t} - 1) } \right)
\end{eqnarray*}

 put 0 and this is $\lambda$ , which is therefore $E(X)$.

\begin{eqnarray*} 
\frac{\partial^2M}{\partial t^2} &=& \left( \lambda^2 e^{2t} + \lambda e^{t} \right) \times \left(e^{\lambda(e^{lambda t} - 1) } \right)
\end{eqnarray*}

 put 0 and this is + ,
et M e t et e t
t
\lambda \lambda \lambda \lambda \lambda ∂ = +  −  =  
∂  
but it is also E X 2  .
\begin{itemize}
\item Hence 
\begin{eqnarray*}
Var(X) &=& E[X^2]  − E X(X)^2 \\
&=&  \lambda^2 +\lambda − (\lambda)^2 \\
&=& \lambda\\
\end{eqnarray*}

\begin{eqnarray*} \frac{\partial^3M}{\partial t^3} &=& \left( \lambda^3 e^{3t} + 3 \lambda^2 e^{2t} + \lambda e^{t} \right) \times \left(e^{\lambda(e^{lambda t} - 1) } \right)
&=&  \lambda^3 + 3\lambda^2 + \lambda \\
\end{eqnarray*}

3 ( ) ( t )
3 3 2 2 1 3 2 3
3 3 = 3 at 0.


\begin{table}[ht!]
 \centering
 \begin{tabular}{|p{15cm}|}
 \hline
\noindent 
In successive unit time intervals, the numbers of events X1, X2, …, Xn are
independent and each has the same distribution as X. Obtain the mgf of
Y = X1 + X2 + … + Xn , deduce the form of the distribution of Y and write down
E(Y) and Var(Y).
\\ \hline
  \end{tabular}
\end{table}

\begin{table}[ht!]
 \centering
 \begin{tabular}{|p{15cm}|}
 \hline
\noindent 
In the case λ = ½, n = 50, use an appropriate approximation to find P(Y ≥ 40), and
state with a reason whether you would expect your answer to be greater than or
less than the true value.
\\ \hline
  \end{tabular}
\end{table}
\begin{table}[ht!]
 \centering
 \begin{tabular}{|p{15cm}|}
 \hline
\noindent Text
\\ \hline
  \end{tabular}
\end{table}
\item This is M e t e t et e e t E X
t
% \lambda \lambda \lambda \lambda \lambda \lambda % \lambda ∂ = + +  −  + + =       ∂  
.


\item Now
\begin{eqnarray*}  
E (X −\lambda^3 ) &=&  
E [X^3] − 3\lambda E[X^2] + 3\lambda^2E[X]−\lambda^3 \\
&=&  \left[\lambda^3 + 3\lambda^2 +\lambda \right]− 3\lambda \left[ (\lambda^2 +\lambda ) \right]+ 3\lambda^2 left[\lambda \right] −\lambda^3 \\ 
&=&  \lambda 
\end{eqnarray*}

For Y = X1 + X2 + … + Xn we have
( ) ( )
1
E i ( ) exp 1 .
n
Y t X t n t
X
i
\[E e e M t \lambda n e
=
    =Π   = = \] − 
\end{itemize}
%%%%%%



\begin{itemize}
\item This is the mgf of a Poisson distribution with parameter $\lambda\;n$and so 
\[E[Y] = Var(Y) = \lambda n.\]

\[P (Y \geq 40) = 1− P(Y \leq 39) \approx 1 - \Phi(\frac{39.5-25}{5})\]

using continuity correction, and μ = \lambda n = 25. 
\item This is $1−\Phi(2.9) = 0.00187$.
With a positively skew distribution, the Normal approximation is likely to
underestimate the probability in the right hand tail and so we expect this answer to be
less than the true value.
\end{itemize}
\end{enumerate}
\end{document}

\documentclass[a4paper,12pt]{article}
%%%%%%%%%%%%%%%%%%%%%%%%%%%%%%%%%%%%%%%%%%%%%%%%%%%%%%%%%%%%%%%%%%%%%%%%%%%%%%%%%%%%%%%%%%%%%%%%%%%%%%%%%%%%%%%%%%%%%%%%%%%%%%%%%%%%%%%%%%%%%%%%%%%%%%%%%%%%%%%%%%%%%%%%%%%%%%%%%%%%%%%%%%%%%%%%%%%%%%%%%%%%%%%%%%%%%%%%%%%%%%%%%%%%%%%%%%%%%%%%%%%%%%%%%%%%
  \usepackage{eurosym}
\usepackage{vmargin}
\usepackage{amsmath}
\usepackage{graphics}
\usepackage{epsfig}
\usepackage{enumerate}
\usepackage{multicol}
\usepackage{subfigure}
\usepackage{fancyhdr}
\usepackage{listings}
\usepackage{framed}
\usepackage{graphicx}
\usepackage{amsmath}
\usepackage{chngpage}
%\usepackage{bigints}

\usepackage{vmargin}
% left top textwidth textheight headheight
% headsep footheight footskip
\setmargins{2.0cm}{2.5cm}{16 cm}{22cm}{0.5cm}{0cm}{1cm}{1cm}
\renewcommand{\baselinestretch}{1.3}

\setcounter{MaxMatrixCols}{10}
\begin{document}
%%%%%%%%%%%%%%%%%%%%%%%%%%%%%%%%%%%%%%%%%%%%%%%%%%%

Higher Certificate, Paper III, 2001. Question 1
%%%%%%%%%%%%%%%%%%%%%%%%%%%%%%%%%%%%%%%%%%%%%%%%%%%%%%%%%%%%%%%%%%%%%%%%%%%%%%%%%%%%%%%%%

\begin{table}[ht!]
     


\centering
     


\begin{tabular}{|p{15cm}|}
     


\hline 


1. The following data are blood cholesterol levels (in mg/100ml) of 10 heart attack patients one and two weeks after the attack, together with the difference between the two levels. 
 
Patient One week after Two weeks after 
Difference Total 
1   142   116   −26   258 
2   360   352     −8   712 
3   244   269   +25   513 
4   186   190     +4   376 
5   210   215     +5   425 
6   236   242     +6   478 
7   288   248   −40   536 
8   276   220   −56   496 
9   224   200   −24   424 
10   311   302     −9   613 
Total 2477 2354 −123 4831 
 
 
(i) Perform a paired t test on these data. 
 
 
(ii) Calculate the sums of squares (SSs) for "patients" and "week" and perform the analysis of variance (ANOVA) for a randomised block design, with "patients" corresponding to blocks.  Note that the (corrected) total SS = 74498.95.  
 
(iii) Perform a Wilcoxon signed-rank test on these data. 
 
 
(iv) State the null and alternative hypotheses being tested in parts (i), (ii) and (iii) and the assumptions the tests involved make.  What conclusions do you draw from each of these tests?  Give reasons for your answers.  
 
(v) What is the relationship between the paired t value of part (i) and the F value for "week" of part (ii)?  
 
 
\\ \hline



\end{tabular}
    


\end{table}


%%%%%%%%%%%%%%%%%%%%%%%%%%%%%%%%%%%%%%%%%%%%%%%%%%%%%%%%%%%%%%%%%%%%%%%%%%%%%%%%%%%%%%%%%

\begin{enumerate}[(a)]
\item  Using the differences, test the null hypothesis "mean difference = 0", assuming
Normality of the distribution of differences.
12.3, 2 (24.3176)2 ; so test statistic is 12.3 0 1.60
24.3176 / 10 d d = − s = − − = − , which is not
significant as an observation from t9.
(ii) Correction term = 48312 / 20 =1166928.05 .
SS for weeks 1 (24772 23542 ) correction 1167684.50 1166928.05
10
= + − = −
= 756.45
SS for patients 1 (2582 ... 6132 ) correction = 1238009.50 1166928.05
2
= ++ − −
=71081.45
Analysis of Variance
ITEM DF SS MS
Patients 9 71081.45 7897.94 F9,9 = 26.71 sig at 0.1%
Weeks 1 756.45 756.45 F1,9 = 2.56 not significant
Residual 9 2661.05 295.67
TOTAL 19 74498.95
(iii)
Ranking of |diff| 4 5 6 8 9 24 25 26 40 56
(1) (2) (3) (4) (5) (6) (7) (8) (9) (10)
Sign + + + − − − + − − −
Sum of + ranks is S+ = 13; S− = 42. Tables show that for n = 10 and at the 5% level
in a two-tail test, the smaller of S+ and S− should be 8 or less for significance.
(iv) The null hypothesis for (i) and (ii) is as stated in (i). The alternative
hypothesis is "mean difference ≠ 0". Normality of the data would be required in (ii),
not just of the differences. A dot-plot would in either case cast serious doubt on this
assumption. The Wilcoxon test does not require any distributional assumption, only
that the + and − rankings are randomly placed in the set. In each case we must not
reject the null hypothesis because we do not have any statistically significant test
results.
(v) 2
9 1,9 t = F .


\end{enumerate}
\end{document}

\documentclass[a4paper,12pt]{article}
%%%%%%%%%%%%%%%%%%%%%%%%%%%%%%%%%%%%%%%%%%%%%%%%%%%%%%%%%%%%%%%%%%%%%%%%%%%%%%%%%%%%%%%%%%%%%%%%%%%%%%%%%%%%%%%%%%%%%%%%%%%%%%%%%%%%%%%%%%%%%%%%%%%%%%%%%%%%%%%%%%%%%%%%%%%%%%%%%%%%%%%%%%%%%%%%%%%%%%%%%%%%%%%%%%%%%%%%%%%%%%%%%%%%%%%%%%%%%%%%%%%%%%%%%%%%
  \usepackage{eurosym}
\usepackage{vmargin}
\usepackage{amsmath}
\usepackage{graphics}
\usepackage{epsfig}
\usepackage{enumerate}
\usepackage{multicol}
\usepackage{subfigure}
\usepackage{fancyhdr}
\usepackage{listings}
\usepackage{framed}
\usepackage{graphicx}
\usepackage{amsmath}
\usepackage{chngpage}
%\usepackage{bigints}

\usepackage{vmargin}
% left top textwidth textheight headheight
% headsep footheight footskip
\setmargins{2.0cm}{2.5cm}{16 cm}{22cm}{0.5cm}{0cm}{1cm}{1cm}
\renewcommand{\baselinestretch}{1.3}

\setcounter{MaxMatrixCols}{10}
\begin{document}
%%%%%%%%%%%%%%%%%%%%%%%%%%%%%%%%%%%%%%%%%%%%%%%%%%%

Higher Certificate, Paper II, 2001. Question 3\begin{table}[ht!]
 
\centering
 
\begin{tabular}{|p{15cm}|}
 
\hline  

3. A pharmaceutical company needs to determine whether a new drug successfully lowers blood cholesterol levels for patients with heart disease.  To investigate this, 26 patients suffering from heart disease were randomised to receive the new drug or a placebo for a period of 6 weeks.  At the end of this period, the patients had their blood cholesterol levels measured in suitable units with the following results. 
 
       Placebo 251  242  281  246  270  292  285  255  266  294  299 
        Drug  282  230  271  282  233  227  257  240  225  250  271  263  275  262  280 
 
(i) Draw a dot-plot of these data and comment on the distribution of the observations in each group. (4) 
 
\\ \hline
  
\end{tabular}

\end{table}  

\begin{table}[ht!]
 
\centering
 
\begin{tabular}{|p{15cm}|}
 
\hline  

(ii) Using a suitable one-tailed non-parametric test, investigate whether the drug is more successful than the placebo in reducing blood cholesterol level in patients with heart disease. (7) 
 

\\ \hline
  
\end{tabular}

\end{table} 

\begin{table}[ht!]
 
\centering
 
\begin{tabular}{|p{15cm}|}
 
\hline  

(iii) It is suggested that a parametric test would be more appropriate to analyse these data.  Repeat your analysis using a suitable parametric test, stating any assumptions necessary for this analysis to be valid. (7) 
 
(iv) Comment on the comparison of the results obtained in (ii) and (iii). 
(2) 
\\ \hline
  
\end{tabular}

\end{table} 

\begin{enumerate}[(a)]
\item  Both data sets fairly symmetrical, but not clustered round mean.
Cholesterol Level
200 220 240 260 280 300 320
Placebo (11obs)
Drug (15obs)
Placebo on average somewhat higher than Drug.
\item 
Rank 1 2 3 4 5 6 7 8 9 10 11 12 13
Obs. 225 227 230 233 240 242 246 250 251 255 257 262 263
Trt. D D D D D P P D P P D D D
14 15 16 17 18 19 20 21 22 23 24 25 26
266 270 271 271 275 280 281 282 282 285 292 294 299
P P D D D D P D D P P P P
%(________) (_________)
A Mann-Whitney U test may be applied.
Sum of ranks of P = 179. ( ) ( ) P
11 15 1 11 12 179
2
U = × + × −
= 52
The 5% one-sided critical value is 44 for n1=11, n2=15.
Therefore on these data there is no evidence for claiming that the drug reduces blood
pressure.
\item If we assume the data to be Normally distributed, with the same \sigma 2 in each
distribution, a t test can be applied.
Placebo: ( )x = 271.00, s2 = 20.489 2
Drug: ( )x = 256.53, s2 = 20.908 2
clearly s2
P, s2
D can be pooled to give:
s2 = ( ) ( ) ( )
2 2
2 10 20.489 14 20.908
429.9172 20.734
24
× + ×
= = .
0 D P 1 D P H :\mu =\mu , H :\mu < \mu
( )
P D
24
P D
271.00 256.53 14.47 1.758
1 1 8.23 20.734
11 15
t x x
SE x x
= − = − = =
−
+
which is greater than the one-tail 5% value which is 1.711.
Hence there is evidence to claim a reduction using the drug.
\item The assumption of Normality increases the power of the t test compared with
Mann-Whitney which makes no distributional assumption.
\end{enumerate}
\end{document}

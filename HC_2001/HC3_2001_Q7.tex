\documentclass[a4paper,12pt]{article}
%%%%%%%%%%%%%%%%%%%%%%%%%%%%%%%%%%%%%%%%%%%%%%%%%%%%%%%%%%%%%%%%%%%%%%%%%%%%%%%%%%%%%%%%%%%%%%%%%%%%%%%%%%%%%%%%%%%%%%%%%%%%%%%%%%%%%%%%%%%%%%%%%%%%%%%%%%%%%%%%%%%%%%%%%%%%%%%%%%%%%%%%%%%%%%%%%%%%%%%%%%%%%%%%%%%%%%%%%%%%%%%%%%%%%%%%%%%%%%%%%%%%%%%%%%%%
  \usepackage{eurosym}
\usepackage{vmargin}
\usepackage{amsmath}
\usepackage{graphics}
\usepackage{epsfig}
\usepackage{enumerate}
\usepackage{multicol}
\usepackage{subfigure}
\usepackage{fancyhdr}
\usepackage{listings}
\usepackage{framed}
\usepackage{graphicx}
\usepackage{amsmath}
\usepackage{chngpage}
%\usepackage{bigints}

\usepackage{vmargin}
% left top textwidth textheight headheight
% headsep footheight footskip
\setmargins{2.0cm}{2.5cm}{16 cm}{22cm}{0.5cm}{0cm}{1cm}{1cm}
\renewcommand{\baselinestretch}{1.3}

\setcounter{MaxMatrixCols}{10}
\begin{document}
%%%%%%%%%%%%%%%%%%%%%%%%%%%%%%%%%%%%%%%%%%%%%%%%%%%


Higher Certificate, Paper III, 2001. Question 7
\begin{enumerate}[(a)]
\item  Non-response is a problem introduced by people refusing to reply to a
survey (or being genuinely unavailable), because they may be different in
some ways from those who do reply. People may not have the information to
reply to questions, may have different working or leisure habits, may be away
more frequently, may live in shared accommodation which is less easy to
locate, may be of one particular age-group.
For example, those who play sports would be more likely to be out evenings
or weekends, but their views on facilities would be different from those who
do not.
(ii) (1) Pilot testing of questionnaires to check clarity,
understandability, avoid giving offence by wording or by
including sensitive questions.
(2) Use interviewers who are well-trained, understand the aim and
purpose of the survey, and the meaning of questions, and are
able to put people at their ease.
(3) Give advance notice where appropriate, e.g. to the area or
group of people being surveyed.
(4) Revisit those unable to be interviewed through absence.
(b) Light engineering: n = 125, P(improve) = 67 0.536
125
= .
Banking & finance: n = 200, P(improve) = 126 0.630
200
= .
(i) Using a Normal approximation (1 )
,
p p
p
n
 − 
 
 
for each proportion, the
variance of their difference is ( ) ( ) 1 1 2 2 3
1 2
1 1
3.1551 10
p p p p
n n
− − −
+ = × .
An approximate 95% confidence interval for the true difference ( ) 2 1 π −π is
( ) 3
2 1 p − p \pm 1.96 3.1551×10− , i.e. (−0.016, 0.204) .
(ii) Strictly we cannot say they are different because this interval
includes 0; but the lower limit is only just below 0 so we might investigate
further, if possible, using larger samples.
\end{enumerate}
\end{document}
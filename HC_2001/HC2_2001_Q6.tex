\documentclass[a4paper,12pt]{article}
%%%%%%%%%%%%%%%%%%%%%%%%%%%%%%%%%%%%%%%%%%%%%%%%%%%%%%%%%%%%%%%%%%%%%%%%%%%%%%%%%%%%%%%%%%%%%%%%%%%%%%%%%%%%%%%%%%%%%%%%%%%%%%%%%%%%%%%%%%%%%%%%%%%%%%%%%%%%%%%%%%%%%%%%%%%%%%%%%%%%%%%%%%%%%%%%%%%%%%%%%%%%%%%%%%%%%%%%%%%%%%%%%%%%%%%%%%%%%%%%%%%%%%%%%%%%
  \usepackage{eurosym}
\usepackage{vmargin}
\usepackage{amsmath}
\usepackage{graphics}
\usepackage{epsfig}
\usepackage{enumerate}
\usepackage{multicol}
\usepackage{subfigure}
\usepackage{fancyhdr}
\usepackage{listings}
\usepackage{framed}
\usepackage{graphicx}
\usepackage{amsmath}
\usepackage{chngpage}
%\usepackage{bigints}

\usepackage{vmargin}
% left top textwidth textheight headheight
% headsep footheight footskip
\setmargins{2.0cm}{2.5cm}{16 cm}{22cm}{0.5cm}{0cm}{1cm}{1cm}
\renewcommand{\baselinestretch}{1.3}

\setcounter{MaxMatrixCols}{10}
\begin{document}
%%%%%%%%%%%%%%%%%%%%%%%%%%%%%%%%%%%%%%%%%%%%%%%%%%%


Higher Certificate, Paper II, 2001. Question 6
\begin{enumerate}[(a)]
\item 
Interval Midpoint, x f fx fx2 F
3.0 – 4.0 3.50 14 49.0 171.500 14
4.0 – 4.5 4.25 20 85.0 361.250 34
4.5 – 5.0 4.75 32 152.0 722.000 66
5.0 – 5.5 5.25 22 115.5 606.375 88
5.5 – 6.5 6.00 12 72.0 432.000 100
100 473.5 2293.125
2
4.735. 2 1 2293.125 473.5 0.5162; 0.718
99 100
x s s
 
= =  −  = =
 
.
Median 4.5 16 0.5 4.75
32
≈ + × = (approx − depending on accuracy of measurement).
The data appear to be approximately symmetrical (x ≈ median) and, from the
histogram, could be assumed Normally distributed.
\item With these assumptions, and based on the given data, a 9\% confidence
interval for the true mean is x 1.96 s
n
\pm 
i.e. 4.375 \pm  1.96×0.0718, or 4.735 \pm  0.141, i.e. (4.59 to 4.88) .

\end{enumerate}
\end{document}

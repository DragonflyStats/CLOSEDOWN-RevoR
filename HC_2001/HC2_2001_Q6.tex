\documentclass[a4paper,12pt]{article}
%%%%%%%%%%%%%%%%%%%%%%%%%%%%%%%%%%%%%%%%%%%%%%%%%%%%%%%%%%%%%%%%%%%%%%%%%%%%%%%%%%%%%%%%%%%%%%%%%%%%%%%%%%%%%%%%%%%%%%%%%%%%%%%%%%%%%%%%%%%%%%%%%%%%%%%%%%%%%%%%%%%%%%%%%%%%%%%%%%%%%%%%%%%%%%%%%%%%%%%%%%%%%%%%%%%%%%%%%%%%%%%%%%%%%%%%%%%%%%%%%%%%%%%%%%%%
  \usepackage{eurosym}
\usepackage{vmargin}
\usepackage{amsmath}
\usepackage{graphics}
\usepackage{epsfig}
\usepackage{enumerate}
\usepackage{multicol}
\usepackage{subfigure}
\usepackage{fancyhdr}
\usepackage{listings}
\usepackage{framed}
\usepackage{graphicx}
\usepackage{amsmath}
\usepackage{chngpage}
%\usepackage{bigints}

\usepackage{vmargin}
% left top textwidth textheight headheight
% headsep footheight footskip
\setmargins{2.0cm}{2.5cm}{16 cm}{22cm}{0.5cm}{0cm}{1cm}{1cm}
\renewcommand{\baselinestretch}{1.3}

\setcounter{MaxMatrixCols}{10}
\begin{document}
%%%%%%%%%%%%%%%%%%%%%%%%%%%%%%%%%%%%%%%%%%%%%%%%%%%


Higher Certificate, Paper II, 2001. Question 6

\begin{table}[ht!]
 
\centering
 
\begin{tabular}{|p{15cm}|}
 
\hline  

6. In a study of houseflies, a biologist measured the wing lengths of a random sample of 100 houseflies.  The data obtained are given in the following table. 
 
Wing length in millimetres Number of houseflies < 3.0   0 ≥ 3.0 but < 4.0 14 ≥ 4.0 but < 4.5 20 ≥ 4.5 but < 5.0 32 ≥ 5.0 but < 5.5 22 ≥ 5.5 but < 6.5 12 ≥ 6.5   0 
 
(i) Draw a histogram of the above data and find approximate values of the median, mean and standard deviation of the data.  What do the data and your statistics reveal about the distribution of the wing lengths of houseflies? (14) 

\\ \hline
  
\end{tabular}

\end{table} 



\begin{enumerate}[(a)]
\item 
Interval Midpoint, x f fx fx2 F
3.0 – 4.0 3.50 14 49.0 171.500 14
4.0 – 4.5 4.25 20 85.0 361.250 34
4.5 – 5.0 4.75 32 152.0 722.000 66
5.0 – 5.5 5.25 22 115.5 606.375 88
5.5 – 6.5 6.00 12 72.0 432.000 100
100 473.5 2293.125
2
4.735. 2 1 2293.125 473.5 0.5162; 0.718
99 100
x s s
 
= =  −  = =
 
.
Median 4.5 16 0.5 4.75
32
≈ + × = (approx − depending on accuracy of measurement).
The data appear to be approximately symmetrical (x ≈ median) and, from the
histogram, could be assumed Normally distributed.

\begin{table}[ht!]
 
\centering
 
\begin{tabular}{|p{15cm}|}
 
\hline  
(ii) Construct a 95\% confidence interval for the mean wing length of a housefly, stating any assumptions that you make. (6) 
 
\\ \hline
  
\end{tabular}

\end{table} 
\item With these assumptions, and based on the given data, a 95\% confidence
interval for the true mean is x 1.96 s
n
\pm 
i.e. $4.375 \pm  1.96×0.0718$, or $4.735 \pm  0.141$, i.e. ($4.59 to 4.88$) .

\end{enumerate}
\end{document}

\documentclass[a4paper,12pt]{article}
%%%%%%%%%%%%%%%%%%%%%%%%%%%%%%%%%%%%%%%%%%%%%%%%%%%%%%%%%%%%%%%%%%%%%%%%%%%%%%%%%%%%%%%%%%%%%%%%%%%%%%%%%%%%%%%%%%%%%%%%%%%%%%%%%%%%%%%%%%%%%%%%%%%%%%%%%%%%%%%%%%%%%%%%%%%%%%%%%%%%%%%%%%%%%%%%%%%%%%%%%%%%%%%%%%%%%%%%%%%%%%%%%%%%%%%%%%%%%%%%%%%%%%%%%%%%
  \usepackage{eurosym}
\usepackage{vmargin}
\usepackage{amsmath}
\usepackage{graphics}
\usepackage{epsfig}
\usepackage{enumerate}
\usepackage{multicol}
\usepackage{subfigure}
\usepackage{fancyhdr}
\usepackage{listings}
\usepackage{framed}
\usepackage{graphicx}
\usepackage{amsmath}
\usepackage{chngpage}
%\usepackage{bigints}

\usepackage{vmargin}
% left top textwidth textheight headheight
% headsep footheight footskip
\setmargins{2.0cm}{2.5cm}{16 cm}{22cm}{0.5cm}{0cm}{1cm}{1cm}
\renewcommand{\baselinestretch}{1.3}

\setcounter{MaxMatrixCols}{10}
\begin{document}
%%%%%%%%%%%%%%%%%%%%%%%%%%%%%%%%%%%%%%%%%%%%%%%%%%%


Higher Certificate, Paper III, 2001. Question 2
%%%%%%%%%%%%%%%%%%%%%%%%%%%%%%%%%%%%%%%%%%%%%%%%%%%%%%%%%%%%%%%%%%%%%%%%%%%%%%%%%%%%%%%%%

\begin{table}[ht!]
\centering
\begin{tabular}{|p{15cm}|}
\hline 
2. The following data are from a two-factor experiment on sugar beet.  The two factors are nitrogen (0 kg, 50 kg and 100 kg sulphate of ammonia per acre) and depth of winter ploughing (8 cm and 12 cm). 
 
 
Yield of sugar (kg per acre) per treatment combination 
 
Nitrogen 0 kg 50 kg 100 kg 0 kg 50 kg 100 kg Depth of ploughing 8 cm 8 cm 8 cm 12 cm 12 cm 12 cm 1054 1218 1406 1177 1374 1554 1099 1258 1423 1160 1350 1572 1080 1279 1468 1151 1351 1536 1093 1273 1430 1145 1362 1561 Means 1081.50 1257.00 1431.75 1158.25 1359.25 1555.75 
 
 
 
The analysis of variance (ANOVA) table is 

\begin{center}
    \begin{tabular}{|c|c|c|c|c|}
 SOURCE & DF &  SS & MS & \\ \hline
Nitrogen & 2 & ? & ? & ? \\ \hline
Depth & ? & ? & ? & \\ \hline
N×D & 2 & ? & 1118.650 & \\ \hline
Residual & ? & ? & 397.875 & \\ \hline
TOTAL & 23 & 629744.50  & & \\ \hline
    
    \end{tabular}
\end{center} 
% Source DF SS MS 
% Nitrogen 2 **** **** 
% Depth **** **** **** 
% Nitrogen x Depth **** 2237.3 **** 
% Error **** **** 397.9 
% Total 23 629744.5  
 
(i) Complete the ANOVA table and use it to assess what evidence the experiment provides regarding the six treatment combinations.
\\ \hline
\end{tabular}
\end{table}


%%%%%%%%%%%%%%%%%%%%%%%%%%%%%%%%%%%%%%%%%%%%%%%%%%%%%%%%%%%%%%%%%%%%%%%%%%%%%%%%%%%%%%%%%

Totals are:
N0 50 100
Depth 8 cm 4326 5028 5727 15081 12 cm 4633 5437 6223 16293
8959 10465 11950 31374
Correction term
313742 41013661.50
24
= = .
SS for depths 1 (150812 162932 ) 41074867.50
12
= + = .
SS for nitrogen 1 (89592 104652 119502 ) 41572800.75
8
= + + = .

\begin{enumerate}[(a)]
\item 
(i) Completed ANOVA is:

\begin{center}
    \begin{tabular}{|c|c|c|c|c|}
 SOURCE & DF &  SS & MS & \\ \hline
Nitrogen & 2 & 559139.25 & 279569.625& \\ \hline
Depth & 1 & 61206.00 & 61206.000 & \\ \hline
N×D & 2 & 2237.30 & 1118.650 & \\ \hline
Residual & 18 & 7161.75 & 397.875 & \\ \hline
TOTAL & 23 & 629744.50  & & \\ \hline
    
    \end{tabular}
\end{center}

Both the nitrogen effect and the depth effect are obviously very highly significant.
Depth 12 gives a yield very significantly greater than depth 8. There is a very
significant nitrogen effect.
There is no evidence of interaction (F2,18 test statistic is 2.81).
\newpage

\begin{table}[ht!]
\centering
\begin{tabular}{|p{15cm}|}
\hline
(ii) Draw a simple diagram using the six mean values which illustrates the effects of nitrogen, ploughing depth and their interaction.  
 
(iii) Summarise your conclusions in non-technical language that the experimenter would understand.  
 
\\ \hline
\end{tabular}
\end{table}
(ii)
(iii) There will be considerably higher yield if 100kg of sulphate of ammonia per
acre is used, as compared with 50 or with none at all; and 12cm depth of winter
ploughing will give better results than 8cm. (The benefit of greater depth is about the
same whichever level of nitrogen is used.)
1000
1100
1200
1300
1400
1500
1600
N0 N50 N100
DEPTH
12 cm
8 cm
\end{enumerate}
\end{document}


\documentclass[a4paper,12pt]{article}
%%%%%%%%%%%%%%%%%%%%%%%%%%%%%%%%%%%%%%%%%%%%%%%%%%%%%%%%%%%%%%%%%%%%%%%%%%%%%%%%%%%%%%%%%%%%%%%%%%%%%%%%%%%%%%%%%%%%%%%%%%%%%%%%%%%%%%%%%%%%%%%%%%%%%%%%%%%%%%%%%%%%%%%%%%%%%%%%%%%%%%%%%%%%%%%%%%%%%%%%%%%%%%%%%%%%%%%%%%%%%%%%%%%%%%%%%%%%%%%%%%%%%%%%%%%%
\usepackage{eurosym}
\usepackage{vmargin}
\usepackage{amsmath}
\usepackage{graphics}
\usepackage{epsfig}
\usepackage{enumerate}
\usepackage{multicol}
\usepackage{subfigure}
\usepackage{fancyhdr}
\usepackage{listings}
\usepackage{framed}
\usepackage{graphicx}
\usepackage{amsmath}
\usepackage{chngpage}
%\usepackage{bigints}

\usepackage{vmargin}
% left top textwidth textheight headheight
% headsep footheight footskip
\setmargins{2.0cm}{2.5cm}{16 cm}{22cm}{0.5cm}{0cm}{1cm}{1cm}
\renewcommand{\baselinestretch}{1.3}

\setcounter{MaxMatrixCols}{10}
\begin{document}
Higher Certificate, Paper I, 2001. Question 4
An easy method is to consider X as ΣXi , where Xi are a set of n Bernoulli variables
with ( 1) , ( 0) (1 ). i i P X = = p P X = = − p Then E[ ] , so [ ] i X = p E X = np .
Also E 2 , so Var ( ) 2 and Var ( ) ( 2 ) . i i X  = p X = p − p X = n p − p = npq
ALTERNATIVELY: [ ] ( ) ( ) 0 1
!
1 ! !
n n x n x
x n x
x x
n n p q E X x p q
x x n x
−
−
= =
 
=   =   − −
Σ Σ
= 1 ( 1) ( 1)
1
1
1
n
x n x
x
n
np p q np
x
− − − −
=
 − 
  =  − 
Σ .
Similarly, ( ) ( ) [ ] ( [ ])2 Var X =E X X −1  +E X - E X , and we have
( ) ( ) ( )
0 2
1 1 1
n n
x n x x n x
x x
n n
E X X x x p q x x p q
x x
− −
= =
   
 −  = −   = −  
   
Σ Σ
( ) 2 2 ( 2) ( 2) ( ) 2
2
2
1 1
2
n
x n x
x
n
n n p p q n n p
x
− − − −
=
 − 
= −   = −  − 
Σ ,
and hence Var (X) = n(n −1) p2 + np − n2 p2 = np − np2 = npq .
PGFs or MGFs could also be used.
(i) (a) 0.7516 ≈ 0.0100226 = 0.0100 approx.
(b) 1 − P(no one gets all 16 right), probability is { }1− 1− 0.7516 12
= 1 − {0.9899774}12 = 0.1139.
(ii) P(B − A > 0) can be studied using a Normal approximation to the difference
B − A, i.e. N(16{0.5 − 0.75}, 16{(0.5×0.5) + (0.75×0.25)}), i.e. N(−4,7) .
The probability is found as 1
2
P B − A > 
 
using a continuity correction since B − A
takes discrete values.
Hence it is ( ) ( ) 0.5 4 1 4.5 1.7008 0.0445
7 7
 − −    −Φ  = Φ −  = Φ − ≈
   
.
[Note: this would be 0.0653 without the continuity correction.]
(iii) E X  = E[X ] = np =16×0.75 =12 in set A .
Similarly, E Y  =16×0.5 = 8 in set B.
There are 12 students in A and 25 in B, so that
Var ( ) 16 0.75 0.25 1
12 4
X = × × = in set A
Var ( ) 16 0.5 0.5 4
25 25
Y = × × = in set B.

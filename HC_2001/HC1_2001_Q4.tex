\documentclass[a4paper,12pt]{article}
%%%%%%%%%%%%%%%%%%%%%%%%%%%%%%%%%%%%%%%%%%%%%%%%%%%%%%%%%%%%%%%%%%%%%%%%%%%%%%%%%%%%%%%%%%%%%%%%%%%%%%%%%%%%%%%%%%%%%%%%%%%%%%%%%%%%%%%%%%%%%%%%%%%%%%%%%%%%%%%%%%%%%%%%%%%%%%%%%%%%%%%%%%%%%%%%%%%%%%%%%%%%%%%%%%%%%%%%%%%%%%%%%%%%%%%%%%%%%%%%%%%%%%%%%%%%
\usepackage{eurosym}
\usepackage{vmargin}
\usepackage{amsmath}
\usepackage{graphics}
\usepackage{epsfig}
\usepackage{enumerate}
\usepackage{multicol}
\usepackage{subfigure}
\usepackage{fancyhdr}
\usepackage{listings}
\usepackage{framed}
\usepackage{graphicx}
\usepackage{amsmath}
\usepackage{chngpage}
%\usepackage{bigints}

\usepackage{vmargin}
% left top textwidth textheight headheight
% headsep footheight footskip
\setmargins{2.0cm}{2.5cm}{16 cm}{22cm}{0.5cm}{0cm}{1cm}{1cm}
\renewcommand{\baselinestretch}{1.3}

\setcounter{MaxMatrixCols}{10}
\begin{document}
Higher Certificate, Paper I, 2001. Question 4

\begin{table}[ht!]
 \centering
 \begin{tabular}{|p{15cm}|}
 \hline
\noindent 
4. The random variable X follows the binomial B(n, p) distribution with probability
mass function
( ) x n x , 0,1, ..., , 0 1 n
f x p q x n p
x
−  
=   = < <
 
,
where q = 1 − p. Show that E(X) = np and Var(X) = npq.
\\ \hline
  \end{tabular}
\end{table}



\begin{enumerate}[(a)]
\item An easy method is to consider X as ΣXi , where Xi are a set of n Bernoulli variables
with ( 1) , ( 0) (1 ). i i P X = = p P X = = − p
\begin{itemize}
\item Then E[ ] , so [ ] i X = p E X = np .
\item Also E 2 , so Var ( ) 2 and Var ( ) ( 2 ) . i i X  = p X = p − p X = n p − p = npq
\item ALTERNATIVELY: [ ] ( ) ( ) 0 1

\end{itemize}

\begin{table}[ht!]
 \centering
 \begin{tabular}{|p{15cm}|}
 \hline
\noindent 
A mathematics class in a school is divided into set A with 12 students and set B
with 25 students. Both groups are given a test consisting of 16 short questions.
For any student in set A, the score (that is, the number of correct answers) is
distributed as B(16, 0.75); for any student in set B, the score is distributed as
B(16, 0.5). All students answer independently.
(i) Find the probability that
(a) a given set A student gets all 16 questions right,

(b) at least one student in set A gets all 16 questions right.

\\ \hline
  \end{tabular}
\end{table}
%==========================================================================================%
!
1 ! !
n n x n x
x n x
x x
n n p q E X x p q
x x n x
−
−
= =
 
=   =   − −
Σ Σ
= 1 ( 1) ( 1)
1
1
1
n
x n x
x
n
np p q np
x
− − − −
=
 − 
  =  − 
Σ .
Similarly, ( ) ( ) [ ] ( [ ])2 Var X =E X X −1  +E X - E X , and we have
( ) ( ) ( )
0 2
1 1 1
n n
x n x x n x
x x
n n
E X X x x p q x x p q
x x
− −
= =
   
 −  = −   = −  
   
Σ Σ
( ) 2 2 ( 2) ( 2) ( ) 2
2
2
1 1
2
n
x n x
x
n
n n p p q n n p
x
− − − −
=
 − 
= −   = −  − 
Σ ,
and hence 
\begin{eqnarray*}
Var (X) &=& n(n −1) p2 + np − n2 p2 \\
&=& np − np2\\ 
&=& np(1-p) \\
&=& npq \\
\end{eqnarray*}
PGFs or MGFs could also be used.
\item  $0.75^{16} \approx 0.0100226 = 0.0100$ approx.
%%%%%%%%%%%%%%%%%%%%%%%%%%%%%%%%%%%%%%%%%%%%%%%%%%%%%%%%%%%%%%%%%%%
\newpage
\begin{table}[ht!]
 \centering
 \begin{tabular}{|p{15cm}|}
 \hline
\noindent (ii) Use an appropriate approximation to find the probability that a given set B
student scores more than a given set A student.



\\ \hline
  \end{tabular}
\end{table}


\item 1 − P(no one gets all 16 right), probability is { }1− 1− 0.7516 12
= 1 − {0.9899774}12 = 0.1139.
\item $P(B − A > 0)$ can be studied using a Normal approximation to the difference
B − A, i.e. N(16{0.5 − 0.75}, 16{(0.5×0.5) + (0.75×0.25)}), i.e. N(−4,7) .
\begin{itemize}
\item The probability is found as 1
2
P B − A > 
 
using a continuity correction since B − A
takes discrete values.
\item Hence it is ( ) ( ) 0.5 4 1 4.5 1.7008 0.0445
7 7
 − −    −Φ  = Φ −  = Φ − ≈
   
.
\item Note: this would be 0.0653 without the continuity correction.
\end{itemize}
%%%%%%%%%%%%%%%%%%%%%%%%%%%%%%%%%%%%%%%%%%%%%%%%%%%%%%%%%%%%%%%%%
\newpage

\begin{table}[ht!]
 \centering
 \begin{tabular}{|p{15cm}|}
 \hline
\noindent (iii) Let X and Y denote the mean scores of students in set A and set B
respectively. Write down E (X ) and E(Y ) , and show that
Var(X ) =1/ 4 and Var(Y ) = 4/ 25.
\\ \hline
  \end{tabular}
\end{table}
\item 
\begin{eqnarray*}
E X  &=& E[X ] \\
&=& np \\
&=& 16\times 0.75 \\
&=& 12 
\end{eqnarray*}

in set A .

\begin{itemize}
\item Similarly, E Y  =16×0.5 = 8 i$n$set B.
\item There are 12 students in A and 25 in B, so that
\item $Var (X)  = 16 \times 0.75 \times 0.25 = 3$
\end{itemize}

 1
12 4
X = × × = in set A

Var ( ) 16 0.5 0.5 4
25 25
Y = × × = in set B.

\end{enumerate}
\end{document}

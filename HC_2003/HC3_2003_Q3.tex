\documentclass[a4paper,12pt]{article}
%%%%%%%%%%%%%%%%%%%%%%%%%%%%%%%%%%%%%%%%%%%%%%%%%%%%%%%%%%%%%%%%%%%%%%%%%%%%%%%%%%%%%%%%%%%%%%%%%%%%%%%%%%%%%%%%%%%%%%%%%%%%%%%%%%%%%%%%%%%%%%%%%%%%%%%%%%%%%%%%%%%%%%%%%%%%%%%%%%%%%%%%%%%%%%%%%%%%%%%%%%%%%%%%%%%%%%%%%%%%%%%%%%%%%%%%%%%%%%%%%%%%%%%%%%%%
\usepackage{eurosym}
\usepackage{vmargin}
\usepackage{amsmath}
\usepackage{graphics}
\usepackage{epsfig}
\usepackage{enumerate}
\usepackage{multicol}
\usepackage{subfigure}
\usepackage{fancyhdr}
\usepackage{listings}
\usepackage{framed}
\usepackage{graphicx}
\usepackage{amsmath}
\usepackage{chngpage}
%\usepackage{bigints}

\usepackage{vmargin}
% left top textwidth textheight headheight
% headsep footheight footskip
\setmargins{2.0cm}{2.5cm}{16 cm}{22cm}{0.5cm}{0cm}{1cm}{1cm}
\renewcommand{\baselinestretch}{1.3}

\setcounter{MaxMatrixCols}{10}
\begin{document}
Higher Certificate, Paper III, 2003. Question 3
\begin{enumerate}[(a)]
\item  The likelihood of the sample of data is L =
1 !
n xi
i i
e
x
−λλ
= Π
= n xi / !
i e− λλ Σ Π x
log ( )log log ( !) i i ∴ L = −nλ + Σx λ − Σ x .
(log ) i d L n x
dλ λ
= − + Σ . Solving d (log L) 0
dλ
= gives λˆ = x .
( )
2
2 2 log i d L x
dλ λ
= − Σ which is < 0 since all xi > 0 (and so λˆ > 0 ). Hence this
gives a maximum.
%%%%%%%%%%%%%%%%%%%%%%%%%%%%%%%%%%%%%%%%%%%%%%%%
\item (a) If a Poisson distribution gives a good fit, its mean is estimated by the
sample mean x , since we have no more specific information about λ.
(0 18) (1 25) (2 13) (3 10) (4 6) (5 3) 120 1.6
75 75
x fx
f
Σ × + × + × + × + × + × = = = =
Σ
.
Probabilities expected with λ ≡ λˆ = x are 1.6 (1.6) / ! x e− x for x = 0, 1, 2, … .
Expected frequencies are 75 times these.
P(0) = e−1.6 = 0.2019 P(1) =1.6e−1.6 = 0.3230 P(2) = 0.2584
P(3) = 0.1378 P(≥ 4) = 0.0788 .
(Each quoted probability is accurate to 4 decimal places.)
Hence we have
x 0 1 2 3 ≥ 4 Total
Observed frequency 18 25 13 10 9 75
Expected frequency 15.14 24.23 19.38 10.34 5.91
A chi-squared goodness of fit test has 3 degrees of freedom, since there are 5
categories of data and 1 parameter estimated.
Test statistic = ( ) ( ) ( ) ( ) ( ) 2 2 2 2 2 2.86 0.77 6.38 0.34 3.09
15.14 24.23 19.38 10.34 5.92
+ + + + = 4.29, which
is not significant (the 5% point of 2
1 χ is 7.81). Hence a Poisson distribution is
an acceptable model for these data.
%%%%%%%%%%%%%%%%%%%%%%%%%%%%%%%%%%%%%%%%%%%%%%%%
\item This is a reasonably large sample of data, although the mean value of
1.6 is somewhat low for using a Normal-approximation confidence interval.
An approximate 95% confidence interval for λ is
λˆ ±1.96 λˆ / 75 , i.e. 1.6 ± 0.286 or (1.31, 1.89).
[If the Poisson distribution is NOT assumed, use Σfx2 = 338, giving
( )2
2 1 120 338 1.9730
74 75
s
 
=  −  =
 
 
, so s = 1.4046. Thus the interval is
1.6 ± (1.96×1.4046 / 75) = 1.6 ± 0.318 , i.e. (1.28, 1.92).]
\end{enumerate}
\end{document}

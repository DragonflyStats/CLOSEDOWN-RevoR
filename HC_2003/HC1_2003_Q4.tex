\documentclass[a4paper,12pt]{article}
%%%%%%%%%%%%%%%%%%%%%%%%%%%%%%%%%%%%%%%%%%%%%%%%%%%%%%%%%%%%%%%%%%%%%%%%%%%%%%%%%%%%%%%%%%%%%%%%%%%%%%%%%%%%%%%%%%%%%%%%%%%%%%%%%%%%%%%%%%%%%%%%%%%%%%%%%%%%%%%%%%%%%%%%%%%%%%%%%%%%%%%%%%%%%%%%%%%%%%%%%%%%%%%%%%%%%%%%%%%%%%%%%%%%%%%%%%%%%%%%%%%%%%%%%%%%
\usepackage{eurosym}
\usepackage{vmargin}
\usepackage{amsmath}
\usepackage{graphics}
\usepackage{epsfig}
\usepackage{enumerate}
\usepackage{multicol}
\usepackage{subfigure}
\usepackage{fancyhdr}
\usepackage{listings}
\usepackage{framed}
\usepackage{graphicx}
\usepackage{amsmath}
\usepackage{chngpage}
%\usepackage{bigints}

\usepackage{vmargin}
% left top textwidth textheight headheight
% headsep footheight footskip
\setmargins{2.0cm}{2.5cm}{16 cm}{22cm}{0.5cm}{0cm}{1cm}{1cm}
\renewcommand{\baselinestretch}{1.3}

\setcounter{MaxMatrixCols}{10}
\begin{document}
Higher Certificate, Paper I, 2003. Question 4
f (x) = kx2 (1− x) , 0 ≤ x ≤1
(i) ( )
3 4 1 1 2 3
0
0
1 1
3 4 3 4 12
k x x dx k x x k k
    − =  −  =  −  =     ∫ ,
which must be equal to 1. So k = 12.
(ii) ( ) (12 2 12 3 ) 24 36 2 12 (2 3 )
df x d x x x x x x
dx dx
= − = − = −
which is zero for 2 – 3x = 0 [and for x = 0, but this is clearly not the mode (i.e.
not the maximum of f (x)], i.e. x = 2/3. To check that this is the maximum (i.e.
the mode), we can consider the second derivative:-
2 ( )
2 24 72
d f x
x
dx
= − , which is clearly < 0 at x = 2/3.
Hence the mode is at x = 2/3, and the graph of f (x) is as shown. [NOTE. The
curve should of course appear smooth; it might not do so, due to the limits of
electronic reproduction.]
[At the mode, f (x) = 12(2/3)2(1/3) = 16/9.]
Continued on next page
f (x)
1 x
(iii) ( ) ( ) ( )
4 5 1 1 1 3 4
0 0
0
12 12
4 5
E X x f x dx x x dx x x
 
= = − =  − 
  ∫ ∫
12 1 1 12 3
4 5 20 5
=  −  = =  
 
.
( ) ( ) ( )
5 6 1 2 1 2 1 4 5
0 0
0
12 12
5 6
E X x f x dx x x dx x x
 
= = − =  − 
  ∫ ∫
12 1 1 12 2
5 6 30 5
=  −  = =  
 
.
So Var(X) = E(X2) – [E(X)]2 = 2 9
5 25
− = 1
25
.
(iv) The cumulative distribution function is ( ) ( 2 3 )
0
12 x F x = ∫ u −u du
( )
3 4
3 4 3
0
12 4 3 4 3 , for 0 1
3 4
x u u x x x x x
  
=   −  = − = − ≤ ≤
  
.
The mean is 35
and the standard deviation is 1
5 . We require ( 2 4 )
5 5 P < X < .
This can be found by integrating the pdf between 2
5 and 4
5 or, directly, as
3 3 4 2 4 8 2 14 648 814 16
5 5 5 5 5 5 625 25
F   − F   =     −    = × − × =
           
.

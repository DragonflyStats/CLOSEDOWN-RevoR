\documentclass[a4paper,12pt]{article}
%%%%%%%%%%%%%%%%%%%%%%%%%%%%%%%%%%%%%%%%%%%%%%%%%%%%%%%%%%%%%%%%%%%%%%%%%%%%%%%%%%%%%%%%%%%%%%%%%%%%%%%%%%%%%%%%%%%%%%%%%%%%%%%%%%%%%%%%%%%%%%%%%%%%%%%%%%%%%%%%%%%%%%%%%%%%%%%%%%%%%%%%%%%%%%%%%%%%%%%%%%%%%%%%%%%%%%%%%%%%%%%%%%%%%%%%%%%%%%%%%%%%%%%%%%%%
  \usepackage{eurosym}
\usepackage{vmargin}
\usepackage{amsmath}
\usepackage{graphics}
\usepackage{epsfig}
\usepackage{enumerate}
\usepackage{multicol}
\usepackage{subfigure}
\usepackage{fancyhdr}
\usepackage{listings}
\usepackage{framed}
\usepackage{graphicx}
\usepackage{amsmath}
\usepackage{chngpage}
%\usepackage{bigints}

\usepackage{vmargin}
% left top textwidth textheight headheight
% headsep footheight footskip
\setmargins{2.0cm}{2.5cm}{16 cm}{22cm}{0.5cm}{0cm}{1cm}{1cm}
\renewcommand{\baselinestretch}{1.3}

\setcounter{MaxMatrixCols}{10}
\begin{document}
Higher Certificate, Paper I, 2003. Question 5
\begin{framed}
The random variable $X$ follows a Poisson distribution with probability mass function
\[f(x) = \frac{e^{-\lambda} \lambda^{x}}{x!}\]
with $X \in \{0,1,2,3,4\ldots\}$ and $\lambda >0$.


Sketch $f(x)$ for the cases $\lambda = 0.5$ and $\lambda = 0.2$, and state the expectation and
variance of $X$.
\end{framed}


\begin{enumerate}
\item Poisson distribution: 
  \[f(x) = \frac{e^{-\lambda} \lambda^{x}}{x!}\]

Expectation = variance = $\lambda$.

\begin{enumerate}[(i)]
    \item 
$\lambda = 0.5$ : $f (0) = e^{-0.5} = 0.6065, f (1) = 0.3033, f (2) = 0.0758, … $.\\
Expectation = variance = 0.5.
\item $\lambda = 2$ : $f (0) = 0.1353, f (1) = 0.2707, f (2) = 0.2707, f (3) = 0.1804,
f (4) = 0.0902, \ldots $. \\

Expectation = variance = 2.
\end{enumerate}

Sketches are as shown.
%%%%%%%%%%%%%%%%%%%%%%%%%%%%%%%%%%%%%%%%%%%%%%%%%%%%%%%%%%%%%
\item Likelihood 

\[ L = \prod^n_{i=1} \frac{e^{-\lambda} \lambda^{x_i}}{(x_i)!} = \frac{}{\prod^n_{i=1}(x_i)!} \]

\begin{framed}
\[\prod^n_{i=1} \frac{a_i \times b_i }{(c_i)}  = \frac{(\prod^n_{i=1}a_i) \times (\prod^n_{i=1}b_i) }{\prod^n_{i=1}(c_i)}\]
\end{framed}

\begin{itemize}
\item Taking logarithms to base e,
\[log L = - n\lambda + \sum (x_i) log (\lambda ) - log( \prod x_i !)  \]   

\item Differentiating, log i d L n x
\[d\lambda \lambda
= - + Σ ; \] setting this equal to 0 gives the solution
\[1
ˆ 1 n
ML i
i
x x
n
\lambda
=
  = Σ = . \]We have
2
2
log i 0 d L x
d\lambda \lambda 2
= - Σ < , confirming that this is
a maximum.

%%%%%%%%%%%%%%%%%%%%%%%%%%%%%%%%%%%%%%%%%%%%%%%%%%%%%%%%%%%%%
\newpage
\begin{framed}
(ii) Given a random sample $x_1, x_2, \ldots , x_n$ from this distribution, obtain the
maximum likelihood estimator of $\hat{\lambda}$, $\hat{\lambda}_{ML}$  say. State a suitable approximation
for the distribution of $\hat{\lambda}_{ML}$ assuming that the sample size $n$ is large, and use this
approximation to deduce an approximate 95\% confidence interval for $\lambda$.
\end{framed}


\item The central limit theorem gives $X \sim N(\lambda , \lambda /n )$, so we have
(approximately)
\[P \left(\left[ \lambda - 1.96 \sqrt{ \frac{\lambda}{n}} \right] \leq \; \bar{X} \; \leq \; \left[ \lambda + 1.96 \sqrt{ \frac{\lambda}{n}}\right]\right)  =  0.95\]

\item 
\[P (\bar{X} - 1.96 \sqrt{ \frac{\lambda}{n}} \leq \; \lambda \; \leq \; \bar{X} + 1.96 \sqrt{ \frac{\lambda}{n}})  =  0.95\]

\end{itemize} 

Hence, inserting the observed value $\bar{x}$ and, further, using $\hat{\lambda}_{ML} = \bar{x}$ as an
estimate for the underlying variance, an approximate 95\% confidence interval
for $\lambda$ is

\[  \bar{x} \pm 1.96 \sqrt{ \frac{\bar{x} }{n}}\]
%%%%%%%%%%%%%%%%%%%%%%%%%%%%%%%%%%%%%%%%%%%%%%%%%%%%%%%%%%%%%
\newpage

\begin{framed}
A random sample of 400 observations yields $\sum x_i = 2500$. Calculate an
approximate 95\% confidence interval for $\lambda$. Given further that $\sum x^2_i= 25600$,
calculate the sample variance of the given sample. Recalculate the confidence
interval for $\lambda$ using the central limit theorem but without assuming that the data
are Poisson distributed. Compare this interval with that found in part (ii) and
comment briefly.

\end{framed}
\item $n=400$, \sum^{n}_{i=1}(x_i)=2500$ 2500; $\sum^{n}_{i=1}(x_i)^2=25600$.

\begin{itemize}
\item So the approximate 95\% confidence interval is


\[   6.25\pm 1.96 \sqrt{ \frac{6.25}{400} } = \left(6.005, 6.495\right)\]

\item Now using $\sum^{n}_{i=1}(x_i)^2=25600$, we have that the sample variance $s^2$ is

\[  s^2 = \frac{1}{399} \left( 25600 - \frac{(2500)^2}{400} \right)  = \frac{9975}{399} = 25.00 \]

\item Using $s^2$ in the confidence interval gives the interval as

\[   6.25\pm 1.96 \sqrt{ \frac{25.00}{400} }\]

\item This interval is twice as wide – because $s^2$ is four times the size of x – which
suggests that a Poisson assumption is not valid.
\end{itemize}
%%%%%%%%%%%%%%%%%%%%%%%%%%%%%%%%%%%%%%%%%%%%%%%%%%%%%%%%%%%%%
\end{enumerate}
\end{document}

\documentclass[a4paper,12pt]{article}
%%%%%%%%%%%%%%%%%%%%%%%%%%%%%%%%%%%%%%%%%%%%%%%%%%%%%%%%%%%%%%%%%%%%%%%%%%%%%%%%%%%%%%%%%%%%%%%%%%%%%%%%%%%%%%%%%%%%%%%%%%%%%%%%%%%%%%%%%%%%%%%%%%%%%%%%%%%%%%%%%%%%%%%%%%%%%%%%%%%%%%%%%%%%%%%%%%%%%%%%%%%%%%%%%%%%%%%%%%%%%%%%%%%%%%%%%%%%%%%%%%%%%%%%%%%%
  \usepackage{eurosym}
\usepackage{vmargin}
\usepackage{amsmath}
\usepackage{graphics}
\usepackage{epsfig}
\usepackage{enumerate}
\usepackage{multicol}
\usepackage{subfigure}
\usepackage{fancyhdr}
\usepackage{listings}
\usepackage{framed}
\usepackage{graphicx}
\usepackage{amsmath}
\usepackage{chngpage}
%\usepackage{bigints}

\usepackage{vmargin}
% left top textwidth textheight headheight
% headsep footheight footskip
\setmargins{2.0cm}{2.5cm}{16 cm}{22cm}{0.5cm}{0cm}{1cm}{1cm}
\renewcommand{\baselinestretch}{1.3}

\setcounter{MaxMatrixCols}{10}
\begin{document}
Higher Certificate, Paper I, 2003. Question 5
\begin{enumerate}
\item Poisson distribution: ( ) !
\[f(x) = \frac{e^{-lambda} \lambda^{x}{x!}\]

Expectation = variance = $\lambda$.
(i) 
$\lambda = 0.5$ : f (0) = e-0.5 = 0.6065, f (1) = 0.3033, f (2) = 0.0758, … .
Expectation = variance = 0.5.
$\lambda = 2$ : f (0) = 0.1353, f (1) = 0.2707, f (2) = 0.2707, f (3) = 0.1804,
f (4) = 0.0902, … . Expectation = variance = 2.
Sketches are as shown.
%%%%%%%%%%%%%%%%%%%%%%%%%%%%%%%%%%%%%%%%%%%%%%%%%%%%%%%%%%%%%
\item Likelihood 

\[ L = \product^n_{i=1} \frac{e^{-\lambda} \lambda^{x_i}}{(x_i)!} = \frac{}{\product^n_{i=1}(x_i)!} \]

\begin{framed}
\[\product^n_{i=1} \frac{a_i \times b_i }{(c_i)}  = \frac{(\product^n_{i=1}a_i) \times (\product^n_{i=1}b_i) }{\product^n_{i=1}(c_i)}\]
\end{framed}


\begin{itemize}
\item Taking logarithms to base e,
log ( )log log( !) i i L = − n\lambda + Σx \lambda − Π x .

\item Differentiating, log i d L n x
d\lambda \lambda
= − + Σ ; setting this equal to 0 gives the solution
1
ˆ 1 n
ML i
i
x x
n
\lambda
=
  = Σ = . We have
2
2
log i 0 d L x
d\lambda \lambda 2
= − Σ < , confirming that this is
a maximum.
\item The central limit theorem gives $X \sim N(\lambda , \lambda /n )$, so we have
(approximately)
P 1.96 X 1.96 0.95
n n
\lambda \lambda \lambda \lambda
 
 − ≤ ≤ +  =
\end{itemize}   
%%%%%%%%%%%%%%%%%%%%%%%%%%%%%%%%%%%%%%%%%%%%%%%%%%%%%%%%%%%%%
f (x) f (x)
0 x 0 x 2 … 2 4 …
or
P X 1.96 X 1.96 0.95
n n
\lambda \lambda \lambda
 
 − ≤ ≤ +  =
   
.
Hence, inserting the observed value x and, further, using ˆ
ML \lambda = x as an
estimate for the underlying variance, an approximate 95% confidence interval
for \lambda is
x 1.96 x , x 1.96 x
n n
− + .
%%%%%%%%%%%%%%%%%%%%%%%%%%%%%%%%%%%%%%%%%%%%%%%%%%%%%%%%%%%%%
\item 400, 2500; 6.25 i n = Σx = x = .

\begin{itemize}
\item So the approximate 95% confidence interval is
6.25 1.96 6.25 , 6.25 1.96 6.25
400 400
− +
  i.e. 6.005 , 6.495 .
\item Now using Σxi
2 = 25600, we have that the sample variance s2 is
( )2
2 1 2500 9975 25600 25.00
399 400 399
s
 
=  −  = =
   
 
.
\item Using s2 in the confidence interval gives the interval as
6.25 1.96 25.00 , 6.25 1.96 25.00
400 400
− +
  i.e. 5.76 , 6.74 .
\item This interval is twice as wide – because s2 is four times the size of x – which
suggests that a Poisson assumption is not valid.
\end{itemize}
%%%%%%%%%%%%%%%%%%%%%%%%%%%%%%%%%%%%%%%%%%%%%%%%%%%%%%%%%%%%%
\end{enumerate}
\end{document}

\documentclass[a4paper,12pt]{article}
%%%%%%%%%%%%%%%%%%%%%%%%%%%%%%%%%%%%%%%%%%%%%%%%%%%%%%%%%%%%%%%%%%%%%%%%%%%%%%%%%%%%%%%%%%%%%%%%%%%%%%%%%%%%%%%%%%%%%%%%%%%%%%%%%%%%%%%%%%%%%%%%%%%%%%%%%%%%%%%%%%%%%%%%%%%%%%%%%%%%%%%%%%%%%%%%%%%%%%%%%%%%%%%%%%%%%%%%%%%%%%%%%%%%%%%%%%%%%%%%%%%%%%%%%%%%
\usepackage{eurosym}
\usepackage{vmargin}
\usepackage{amsmath}
\usepackage{graphics}
\usepackage{epsfig}
\usepackage{enumerate}
\usepackage{multicol}
\usepackage{subfigure}
\usepackage{fancyhdr}
\usepackage{listings}
\usepackage{framed}
\usepackage{graphicx}
\usepackage{amsmath}
\usepackage{chngpage}
%\usepackage{bigints}

\usepackage{vmargin}
% left top textwidth textheight headheight
% headsep footheight footskip
\setmargins{2.0cm}{2.5cm}{16 cm}{22cm}{0.5cm}{0cm}{1cm}{1cm}
\renewcommand{\baselinestretch}{1.3}

\setcounter{MaxMatrixCols}{10}
\begin{document}
Higher Certificate, Paper III, 2003. Question 5
\begin{enumerate}[(a)]
\item 
The relation is not linear. It began by showing an exponential-type fall over
the first 100 or so and then levelled off, with a suggestion of a further curved
downward trend after about 200.
%%%%%%%%%%%%%%%%%%%%%%%%%%%%%%%%%%%%%%%%%%%%%%%%
\item HRS versus logNBR gives the largest percentage of variation explained
(R2 = 96%). (HRS versus 1/NBR is also fairly good, but some way behind this
one.)
%%%%%%%%%%%%%%%%%%%%%%%%%%%%%%%%%%%%%%%%%%%%%%%%
\item HRS = 1306 – 181 logNBR. The constant is a baseline or average cost
estimated from these data. The coefficient of logNBR ("log" here implies to
base e) gives the reduction (in this case) in HRS (thousands of man-hours) for
every increase of 1 in logNBR, i.e. every 2.71828 along the number scale.
This is an average reduction.
If the residuals were available, we would look at them to see if they were
"random" or if they showed some pattern (for example, all the middle ones
were of one sign and the first and last were of the other sign). If so, this would
suggest that the model could still be improved, even given the high value of
R2.
(iv) Perhaps logHRS against logNBR might show improvement.
0
200
400
600
800
1000
1200
0 50 100 150 200 250 300
NBR
HRS
\end{enumerate}
\end{document}

\documentclass[a4paper,12pt]{article}
%%%%%%%%%%%%%%%%%%%%%%%%%%%%%%%%%%%%%%%%%%%%%%%%%%%%%%%%%%%%%%%%%%%%%%%%%%%%%%%%%%%%%%%%%%%%%%%%%%%%%%%%%%%%%%%%%%%%%%%%%%%%%%%%%%%%%%%%%%%%%%%%%%%%%%%%%%%%%%%%%%%%%%%%%%%%%%%%%%%%%%%%%%%%%%%%%%%%%%%%%%%%%%%%%%%%%%%%%%%%%%%%%%%%%%%%%%%%%%%%%%%%%%%%%%%%
  \usepackage{eurosym}
\usepackage{vmargin}
\usepackage{amsmath}
\usepackage{graphics}
\usepackage{epsfig}
\usepackage{enumerate}
\usepackage{multicol}
\usepackage{subfigure}
\usepackage{fancyhdr}
\usepackage{listings}
\usepackage{framed}
\usepackage{graphicx}
\usepackage{amsmath}
\usepackage{chngpage}
%\usepackage{bigints}

\usepackage{vmargin}
% left top textwidth textheight headheight
% headsep footheight footskip
\setmargins{2.0cm}{2.5cm}{16 cm}{22cm}{0.5cm}{0cm}{1cm}{1cm}
\renewcommand{\baselinestretch}{1.3}

\setcounter{MaxMatrixCols}{10}
\begin{document}
Higher Certificate, Paper I, 2003. Question 6

\[E(Y) = np Var(Y) = np(1 – p)\]
\begin{enumerate}
\item Binomial with n = 48, p = ¼.
\item Score is distributed 36 + B(12, ¼).

(a) Hence mean correct is 36 + (12/4) = 39 and variance is 12 × ¼ × ¾ = 9/4.
(b) Number wrong is distributed B(12, ¾).
(c) The required probability is $1 – P(0) – P(1) – P(2)$ based on the B(12, ¼)
distribution. This is
12 11 2 10 1 3 12 1 3 12 11 1 3
4 4 4 2 4 4
−   −    − ×             
        
= 1− 0.031676 − 0.126705 − 0.232293 = 0.6093 .
\item Number of correct answers for A is distributed as 27 + B(21, ¼).
Number of correct answers for B is distributed as 28 + B(20, ¼).
Number of correct answers for C is distributed as 30 + B(18, ¼).

\begin{itemize}
\item Means are 27 + (21/4) = 32¼, 28 + (20/4) = 33, 30 + (18/4) = 34½
respectively.
\item  Variances are (21)(¼)(¾) = 63/16, (20)(¼)(¾) = 60/16 = 15/4, (18)(¼)(¾) =
  54/16 = 27/8 respectively.
\item So overall mean is 1 (32.25 33 34.5)
3
+ + = 33.25,
and variance of overall mean is 1 63 15 27
9 16 4 8
 + +   
 
= 1.2292.
%%%%%%%%%%%%%%%%%%%%%%%%%%%%%%%%%%%
( ) ( ) ( )
( ) ( ) ( )
, ,
29 29 ; 1 for , ,
29 3
i A B C
P APA
PA Pi i A B C
P iPi
=
  = = = Σ .
( ) ( )
19 2
1
4
29 B 21, 2 21 20 3 1
2 4 4
P A = P  =  = ×        
( ) ( )
19
1
4
29 B 20, 1 20 3 1
4 4
P B = P  =  =        
( ) ( 1 )
4 P 29 C = P B 18, = −1 = 0
\item  Note. C must get at least the 30 he knows, so it must follow that
  P(C 29) = 0 , which is true if P(29 C) = 0 .
So ( )
19 2
19 2 19
1 .21.20 3 1
29 2 4 4
1 .21.20 3 1 20 3 1
2 4 4 4 4
P A
   
   
=    
    +            
       
21
32 21 0.7241 21 1 29
32 4
= = =
  +
  and similarly
( )
1
29 4 8 0.2759 21 1 29
32 4
P B = = =
  +
  (and P(C 29) = 0 , see above).
\end{itemize}

%%%%%%%%%%%%%%%%%%%%%%
\end{enumerate}
\end{document}

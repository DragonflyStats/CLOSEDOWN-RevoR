\documentclass[a4paper,12pt]{article}
%%%%%%%%%%%%%%%%%%%%%%%%%%%%%%%%%%%%%%%%%%%%%%%%%%%%%%%%%%%%%%%%%%%%%%%%%%%%%%%%%%%%%%%%%%%%%%%%%%%%%%%%%%%%%%%%%%%%%%%%%%%%%%%%%%%%%%%%%%%%%%%%%%%%%%%%%%%%%%%%%%%%%%%%%%%%%%%%%%%%%%%%%%%%%%%%%%%%%%%%%%%%%%%%%%%%%%%%%%%%%%%%%%%%%%%%%%%%%%%%%%%%%%%%%%%%
  \usepackage{eurosym}
\usepackage{vmargin}
\usepackage{amsmath}
\usepackage{graphics}
\usepackage{epsfig}
\usepackage{enumerate}
\usepackage{multicol}
\usepackage{subfigure}
\usepackage{fancyhdr}
\usepackage{listings}
\usepackage{framed}
\usepackage{graphicx}
\usepackage{amsmath}
\usepackage{chngpage}
%\usepackage{bigints}

\usepackage{vmargin}
% left top textwidth textheight headheight
% headsep footheight footskip
\setmargins{2.0cm}{2.5cm}{16 cm}{22cm}{0.5cm}{0cm}{1cm}{1cm}
\renewcommand{\baselinestretch}{1.3}

\setcounter{MaxMatrixCols}{10}
\begin{document}
Higher Certificate, Paper I, 2003. Question 6
\begin{framed}
6. The random variable Y follows a binomial distribution with probability mass function
given by
( ) (1 ) , 0,1, ..., ; 0 1 y n y n
fy p p y n p
y
  −
=   − = < <
 
.
Write down the mean and variance of Y.
\end{framed}

\begin{enumerate}[(a)]
\item 
\[E(Y) = np Var(Y) = np(1 - p)\]

%%%%%%%%%%%%%%%%%%%%%%%%%%%%%%%%%%%%%%%%%%%%%%%%%%%%%%%
\begin{framed}
(i) An intelligence test consists of 48 multiple-choice questions. For each
question, four possible answers are presented but only one is correct. If a
student answers all the questions independently by random guesswork, what
will be the distribution of the number of questions they get right?
\end{framed}




\item Binomial with n = 48, p = 0.25.


%%%%%%%%%%%%%%%%%%%%%%%%%%%%%%%%%%%%%%%%%%%%%

\newpage
\begin{framed}
(ii) Assume that, for any question, if a student knows the answer he writes it down
correctly, and otherwise he guesses at random. If he knows the answer to 36
questions, find
(a) the mean and variance of the number of questions he gets right,
(b) the distribution of the number of questions he gets wrong,
(c) the probability that he gets more than two questions right by chance.
\end{framed}


\item Score is distributed 36 + B(12, 0.25).

\begin{description}
\item[(a)] Hence mean correct is 36 + (12/4) = 39 and variance is $12 \times 0.25 \times 0.75 = 9/4$.
\item[(b)] Number wrong is distributed $B(12, 0.75)$.
\item[(c)] The required probability is $1 - P(0) - P(1) - P(2)$ based on the B(12, 0.25)
distribution. 
\end{description}

This is
\begin{eqnarray*}
1 - P(0) - P(1) - P(2) &=& 1- \left[\left(\frac{3}{4}\right)^{12}\right] - \left[12\left(\frac{1}{4}\right) \left(\frac{3}{4}\right)^{11} \right] - \left[\frac{12 \times 11}{2} \left(\frac{1}{4}\right)^{2} \left(\frac{3}{4}\right)^{10}\right]\\ \\ \bigskip
&=& 1- 0.031676 - 0.126705 - 0.232293 \\ \\ \smallskip
&=& 0.6093 .
\end{eqnarray*}

%%%%%%%%%%%%%%%%%%%%%%%%%%%%%%%%%%%%%%%%%%%%%%%%%%%%%%%%
\newpage

\begin{framed}
The test is given separately to students A, B and C in a tutor-group who know
the answers to 27, 28 and 30 questions respectively. Find the mean and
variance of the average number of questions they get right. Given that an
unnamed test paper (which is a priori equally likely to be from any one of
these students) has 29 questions right, find the respective probabilities that this
paper was written by A, B or C.

\end{framed}

\item 
\begin{itemize}
\item Number of correct answers for A is distributed as 27 + B(21, 0.25).
\item Number of correct answers for B is distributed as 28 + B(20, 0.25).
\item Number of correct answers for C is distributed as 30 + B(18, 0.25).
\end{itemize}
%%%%%%%%%%%%%%%%%%%%%%%%%%%%%%%%%%%%%%%%
\begin{itemize}
\item Means are 
\begin{itemize}
\item 27 + (21/4) = 32.25, 
\item 28 + (20/4) = 33, 
\item 30 + (18/4) = 34.5
\end{itemize}  
respectively.

%%%%%%%%%%%%%%%%%%%%%%%%%%%%%%%%%%%%%%%%
\item  Variances are 
\begin{itemize}
\item (21)(0.25)(0.75) = 63/16, 
\item (20)(0.25)(0.75) = 60/16 = 15/4, 
\item (18)(0.25)(0.75) = 54/16 = 27/8 
\end{itemize}  
  respectively.
 
%%%%%%%%%%%%%%%%%%%%%%%%%%%%%%%%%%%%%%%%
\item So overall mean is 
\[ \frac{1}{3}  (32.25 + 33  + 34.5) = 33.25,\]
and variance of overall mean is 
\[ \frac{1}{9}  (\frac{63}{16} + \frac{14}{4}  + \frac{27}{8}) = 1.2292\] 



\[ P(A|29) = \frac{P(29|A) \times P(A)}{ \sum_{i=A,B,C} P(29|i) \times P(i)}\]
with $P(i) = 1/3$ for $i = A,B,C$.


\[ P(29|A) =  P \left[ B(21,1/4)] = 2  \right]  = \frac{21 \times 20}{2} \left(  \frac{3}{4}\right)^{19} \times \left(  \frac{1}{4}\right)^2\] 

\[ P(29|B) =  P \left[ B(20,1/4)] = 1  \right]  = 20 \left(  \frac{3}{4}\right)^{19} \times \left(  \frac{1}{4}\right)^2\]      

\[ P(29|C) =  P \left[ B(18,1/4)] = -1  \right]  = 0 \]   

Note. C must get at least the 30 he knows, so it must follow that
  $P(C| 29) = 0$ , which is true if $P(29|C) = 0$ .
%%%%%%%%%%%%%%%%%%%%%%%%%%%%%%%%%%%%%%%%%%%%%%%%%%%%%%%%%%%%%%%%%%%


\item  
So ( )
19 2
19 2 19
1 .21.20 3 1
29 2 4 4
1 .21.20 3 1 20 3 1
2 4 4 4 4
P A
   
   
=    
    +            
       
21
32 21 0.7241 21 1 29
32 4
= = =
  +
  and similarly
( )
1
29 4 8 0.2759 21 1 29
32 4
P B = = =
  +
  (and P(C 29) = 0 , see above).
\end{itemize}

%%%%%%%%%%%%%%%%%%%%%%
\end{enumerate}
\end{document}

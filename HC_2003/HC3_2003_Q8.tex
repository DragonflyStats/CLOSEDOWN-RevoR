\documentclass[a4paper,12pt]{article}
%%%%%%%%%%%%%%%%%%%%%%%%%%%%%%%%%%%%%%%%%%%%%%%%%%%%%%%%%%%%%%%%%%%%%%%%%%%%%%%%%%%%%%%%%%%%%%%%%%%%%%%%%%%%%%%%%%%%%%%%%%%%%%%%%%%%%%%%%%%%%%%%%%%%%%%%%%%%%%%%%%%%%%%%%%%%%%%%%%%%%%%%%%%%%%%%%%%%%%%%%%%%%%%%%%%%%%%%%%%%%%%%%%%%%%%%%%%%%%%%%%%%%%%%%%%%
\usepackage{eurosym}
\usepackage{vmargin}
\usepackage{amsmath}
\usepackage{graphics}
\usepackage{epsfig}
\usepackage{enumerate}
\usepackage{multicol}
\usepackage{subfigure}
\usepackage{fancyhdr}
\usepackage{listings}
\usepackage{framed}
\usepackage{graphicx}
\usepackage{amsmath}
\usepackage{chngpage}
%\usepackage{bigints}

\usepackage{vmargin}
% left top textwidth textheight headheight
% headsep footheight footskip
\setmargins{2.0cm}{2.5cm}{16 cm}{22cm}{0.5cm}{0cm}{1cm}{1cm}
\renewcommand{\baselinestretch}{1.3}

\setcounter{MaxMatrixCols}{10}
\begin{document}


Higher Certificate, Paper III, 2003. Question 8
\begin{enumerate}[(a)]
\item Main points which could be made include the following.
Total number of employees remained fairly steady, around 22 million, but the
ratio of males to females decreased from 13.4/9.4 = 1.43 in 1978 (and 1.35 in
1981) to 11.5/11.3 = 1.02 in 1997 (1.07 in 1991).
Percentage of males in category H remained much the same, as did that of
females, but the average percentage for males was about 18 and for females
about 40 (presumably nursing and medical services are included in H, which
would provide an explanation).
Percentage of workers overall (both sexes) in category B fell sharply between
1978/1981 and 1991/1997.
Percentage of workers in category C increased for both sexes between
1978/1981 and 1991/1997.
In D, E and F the percentages over time remained similar, with more males
than females.
In G, the number of employees had dropped sharply by 1997.
Some calculations of actual numbers (bottom row × appropriate percentages) would
help to emphasise the drop in numbers of workers (both sexes) in manufacturing, and
the increases in numbers for financial and business services.
A combination of percentages and actual numbers would indicate a noticeable
increase in male employment in category A. For females, numbers increase though
not percentages.
Graphs of "time series" for the two sexes and four years, a single graph for each
category, would help to show the changes, in categories B and C particularly.
Bar charts could be used to show actual numbers or percentages over time, one for
each year. Because of the presence of several categories with small percentages,
annual pie-charts would be slightly less easy to appreciate (but quite valid).
Some of the categories are rather broad, and explanations of changes therefore not
always possible even for a UK commentator.
\end{enumerate}
\end{document}

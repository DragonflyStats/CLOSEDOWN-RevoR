\documentclass[a4paper,12pt]{article}
%%%%%%%%%%%%%%%%%%%%%%%%%%%%%%%%%%%%%%%%%%%%%%%%%%%%%%%%%%%%%%%%%%%%%%%%%%%%%%%%%%%%%%%%%%%%%%%%%%%%%%%%%%%%%%%%%%%%%%%%%%%%%%%%%%%%%%%%%%%%%%%%%%%%%%%%%%%%%%%%%%%%%%%%%%%%%%%%%%%%%%%%%%%%%%%%%%%%%%%%%%%%%%%%%%%%%%%%%%%%%%%%%%%%%%%%%%%%%%%%%%%%%%%%%%%%
\usepackage{eurosym}
\usepackage{vmargin}
\usepackage{amsmath}
\usepackage{graphics}
\usepackage{epsfig}
\usepackage{enumerate}
\usepackage{multicol}
\usepackage{subfigure}
\usepackage{fancyhdr}
\usepackage{listings}
\usepackage{framed}
\usepackage{graphicx}
\usepackage{amsmath}
\usepackage{chngpage}
%\usepackage{bigints}

\usepackage{vmargin}
% left top textwidth textheight headheight
% headsep footheight footskip
\setmargins{2.0cm}{2.5cm}{16 cm}{22cm}{0.5cm}{0cm}{1cm}{1cm}
\renewcommand{\baselinestretch}{1.3}

\setcounter{MaxMatrixCols}{10}
\begin{document}

Higher Certificate, Paper III, 2003. Question 7

%%%%%%%%%%%%%%%%%%%%%%%%%%%%%%%%%%%%%%%%%%%%%%%%%%%%%%%%%%%%%%%%%%%%%%%%
\begin{framed}
7.
(i)
(a)
Explain why non-response is a problem in social surveys.
(4)
(b)
(ii)
Suggest how non-response might be reduced in a mail survey by
appropriate follow-up procedures.
(4)
An interview survey of heads of household is to be undertaken and it has been
estimated that 600 completed interviews are needed.
(a)
Comment on the following two strategies (A) and (B) for achieving
600 completed interviews.
(A)
Give the team of interviewers contact details of a large sample
of heads of household and give instructions that interviewing should
stop once 600 interviews have been completed.
(B)
Give the team of interviewers contact details of a sample of 600
heads of household and give instructions that only one attempt is to be
made to interview each of these heads. If no interview is obtained from
m of these 600 heads of household then give the team contact details of
a further sample of m heads of household and give instructions that
several attempts are to be made to interview the members of this
further sample.
(7)
(b)
A simple random sample of n heads of household is to be selected. It is
thought that the probability of any particular head of household
responding is 0.75. What is the smallest value of n such that there is a
probability of at least 0.99 of obtaining 600 or more responding
households?
(5)
8

\end{framed}
%%%%%%%%%%%%%%%%%%%%%%%%%%%%%%%%%%%%%%%%%%%%%%%%%%%%%%%%%%%%%%%%%%%%%%%%
\begin{enumerate}[(a)]
\item  (a) Non-respondents tend not to be a random part of the whole population,
but instead particular types of person are more likely to fail, or refuse,
to reply. Their responses, if known, would quite likely be different
from other parts of the population. Unless they are represented, the
results of the survey will not validly apply to the whole population.
Besides introducing this bias, intended sample size is reduced by nonresponse
and so precision suffers.
(b) Some possible procedures are as follows.
• Send the questionnaire again, once or twice more
• Send reminder letters (without the questionnaire)
• Telephone people who have not responded
• Visit those who have not responded, perhaps only a sample of them
These all require identification of non-responders, usually by means of
a number on the questionnaire which is kept separate from the answers
to preserve anonymity.
%%%%%%%%%%%%%%%%%%%%%%%%%%%%%%%%%%%%%%%%%%%%%%%%
\item (a) Strategy A will lead to a sample consisting of those who are easiest to
locate, so even if the list is constructed in a properly random way the
actual members used will not have been selected at random from the
list. Any who refuse at first request will be ignored rather than any
attempt being made to persuade them. Since there is a team of
interviewers, the more efficient of these may carry out a higher
proportion of the 600 (more quickly). Or, alternatively, quickly
completed interviews may not have been done so thoroughly. Why
stop at 600 instead of attempting to get as many as possible of the
originally selected list?
Strategy B will also lead to willing and easily available heads of
household being selected, so that the first sample of (600 – m) could
suffer considerable bias. The second sample of m ought to be more
representative, but the quality of the final data will be affected by how
large m is (the larger the better to avoid bias). Office work is also
increased by this method.
Continued on next page
(b) The number R of respondents will be Binomial(n, ¾). Since n is bound
to be a large number, we may use a Normal approximation:
N 3 , 3
4 16
R  n n 
 
 
∼ .
Thus, approximately,
( )
3
4
3
16
0,1
n
n
R
Z N
−
= ∼ ,
and the upper 99% point of Z is 2.326.
Hence
3
4
1
4
600 2400 3 2.326 , i.e. 2.326
3 3
n n
n n
≤
− − ≤ .
Solve this for equality, using only the upper value for n:-
2.326 3n = 2400 − 3n , or 3n + 2.326 3n − 2400 = 0 .
Write 3n = x , so we have x2 + 2.326x − 2400 = 0 , and the roots are
( ) ( )
2 2.326 2.326 9600 1 2.326 98.0072
2 2
x
− ± +
= =− ±
= –50.1666 or +47.8406,
giving
1 2 ( )
3 n = x = 838.9 or 762.9 . Take n = 839.
\end{enumerate}
\end{document}

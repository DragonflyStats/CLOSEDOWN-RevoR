\documentclass[a4paper,12pt]{article}
%%%%%%%%%%%%%%%%%%%%%%%%%%%%%%%%%%%%%%%%%%%%%%%%%%%%%%%%%%%%%%%%%%%%%%%%%%%%%%%%%%%%%%%%%%%%%%%%%%%%%%%%%%%%%%%%%%%%%%%%%%%%%%%%%%%%%%%%%%%%%%%%%%%%%%%%%%%%%%%%%%%%%%%%%%%%%%%%%%%%%%%%%%%%%%%%%%%%%%%%%%%%%%%%%%%%%%%%%%%%%%%%%%%%%%%%%%%%%%%%%%%%%%%%%%%%
\usepackage{eurosym}
\usepackage{vmargin}
\usepackage{amsmath}
\usepackage{graphics}
\usepackage{epsfig}
\usepackage{enumerate}
\usepackage{multicol}
\usepackage{subfigure}
\usepackage{fancyhdr}
\usepackage{listings}
\usepackage{framed}
\usepackage{graphicx}
\usepackage{amsmath}
\usepackage{chngpage}
%\usepackage{bigints}

\usepackage{vmargin}
% left top textwidth textheight headheight
% headsep footheight footskip
\setmargins{2.0cm}{2.5cm}{16 cm}{22cm}{0.5cm}{0cm}{1cm}{1cm}
\renewcommand{\baselinestretch}{1.3}

\setcounter{MaxMatrixCols}{10}
\begin{document}
Higher Certificate, Paper II, 2003. Question 6
%%%%%%%%%%%%%%%%%%%%%%%%%%%%%%%%%%%%%%%%%%%%%%%%%%%%%%%%%%%%%
\begin{table}[ht!]
 
\centering
 
\begin{tabular}{|p{15cm}|}
 
\hline  


6. (i) State and explain a linear model that can be used as the basis for a one-way analysis of variance.  Explain clearly what each term in the model represents and state any assumptions required for the analysis to be valid. (6) 
 

 


\\ \hline
  
\end{tabular}

\end{table}
\begin{table}[ht!]
 
\centering
 
\begin{tabular}{|p{15cm}|}
 
\hline 
 (ii) (a) A farmer is considering which of a range of fertilisers to use on his potato crop.  To help him decide, he set up an experiment in which a field containing seed potatoes was divided into 20 equal plots, and one of four fertilisers A, B, C or D was randomly allocated to each plot, as in the diagram below, in which north is at the top. 
 
D A D A B C C A B B A D C D C C B D B A 
 
Having grown the crop, the yield of potatoes, in kilograms, obtained from each plot was recorded as follows. 
 
A B C D 25 30 23 23 20 33 25 28 21 32 26 26 22 35 24 23 24 31 21 24 
 
 
Carry out a suitable analysis of these data and write a report for the farmer, who is not trained in statistics, clearly stating your recommendations about which fertiliser he should use if he wants to maximise his potato yield. (10) 
 
  (b) The amount of water in the soil can affect the yield of potato crops.  If the farmer had suspected that the drainage in the field varies in an eastwest direction, how might the design of the experiment be altered to take account of this? (4) 
 
 
\\ \hline
  
\end{tabular}

\end{table}
 

%%%%%%%%%%%%%%%%%%%%%%%%%%%%%%%%%%%%%%%%%%%%%%%%%%%%%%%%%%%%%%%%%%
\begin{enumerate}[(a)]
\item (i) yij = m + ti + eij ,
where yij is the measurement made on the jth unit receiving the ith treatment,
m is the underlying population mean of all observations and {eij} are
independent, Normally distributed, residual (natural) variation terms with
mean 0 and common variance σ 2.
This model separates the total variation among the {yij} into a systematic
component due to "treatments" ti and a random component represented by eij.
The number of replicates of treatment i is ri [i.e. we have j = 1, 2, …, ri for
each i], and Σri = N, the total number of experimental units.
(For the usual form of analysis, {eij} are assumed to be Normal, although
randomisation theory validates the inferences usually made from a one-way
analysis.)
(ii) (a) ri = 5 for i = A, B, C, D. N = 20. ΣΣyij
2 = 13666.
Treatment (fertiliser) totals are A 112, B 161, C 119, D 124.
Grand total G = 516.
Corrected total SS = ΣΣyij
2 – (G2/N) = 13666 – {(516)2/20}
= 13666 – 13312.8 = 353.2.
SS for fertilisers
2 2 1122 1612 1192 1242 2
5
i
i
T G G
r N N
= Σ − = + + + −
=
68002 2
5
G
N
− = 13600.4 – 13312.8 = 287.6.
Analysis of variance
ITEM df Sum of Squares Mean Square F ratio
Fertilisers 3 287.6 95.87 23.38
Residual 16 65.6 4.10
Total 19 353.2
The F ratio is referred to F3,16 and is very highly significant, leading us
to reject a null hypothesis that there are no differences between
fertiliser mean yields.
Continued on next page
Fertiliser means are A 22.4, B 32.2, C 23.8, D 24.8 (kg).
σ 2 ("residual variation" or "experimental error") is estimated as 4.10.
Significant differences can be claimed between any pair of means
differing by at least
2
(16) (16) (16)
2 ˆ 1.64 or 1.281
5
t σ = t t where (16) t
is the two-tailed 5\% point of t16, i.e. 2.120. Thus (16) 1.281t = 2.71.
\begin{itemize}
    \item Clearly B is different from all the others and there are no differences
between A, C, D. 
\item Assuming that all the four fertilisers were applied in
the same way, at the same time, it is reasonable to claim that B is best.
\item The farmer simply needs to be told that statistical analysis very
strongly suggests that the four fertilisers did not all give similar results,
and that after allowing for the natural variations among the crop we
can say that B is clearly better than A, C, D.
\end{itemize}

(b) Five randomised blocks, the columns in the diagram on the question
paper, should be used. This will remove an "east–west" trend. In each
column, one replicate of each treatment (fertiliser) should be set out in
random order (different randomisations being used for each block).
\end{enumerate}
\end{document}

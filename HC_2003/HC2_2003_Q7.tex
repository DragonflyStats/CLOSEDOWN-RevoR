\documentclass[a4paper,12pt]{article}
%%%%%%%%%%%%%%%%%%%%%%%%%%%%%%%%%%%%%%%%%%%%%%%%%%%%%%%%%%%%%%%%%%%%%%%%%%%%%%%%%%%%%%%%%%%%%%%%%%%%%%%%%%%%%%%%%%%%%%%%%%%%%%%%%%%%%%%%%%%%%%%%%%%%%%%%%%%%%%%%%%%%%%%%%%%%%%%%%%%%%%%%%%%%%%%%%%%%%%%%%%%%%%%%%%%%%%%%%%%%%%%%%%%%%%%%%%%%%%%%%%%%%%%%%%%%
\usepackage{eurosym}
\usepackage{vmargin}
\usepackage{amsmath}
\usepackage{graphics}
\usepackage{epsfig}
\usepackage{enumerate}
\usepackage{multicol}
\usepackage{subfigure}
\usepackage{fancyhdr}
\usepackage{listings}
\usepackage{framed}
\usepackage{graphicx}
\usepackage{amsmath}
\usepackage{chngpage}
%\usepackage{bigints}

\usepackage{vmargin}
% left top textwidth textheight headheight
% headsep footheight footskip
\setmargins{2.0cm}{2.5cm}{16 cm}{22cm}{0.5cm}{0cm}{1cm}{1cm}
\renewcommand{\baselinestretch}{1.3}

\setcounter{MaxMatrixCols}{10}
\begin{document}




Higher Certificate, Paper II, 2003. Question 7
%%%%%%%%%%%%%%%%%%%%%%%%%%%%%%%%%%%%%%%%%%%%%%%%%%%%%%%%%%%%%%%%%%%%%%%%%%%%%%%%%%%%%%%%%%%%%%%%%%%%%%%%%%%%%%%%%%%%%%%%%%%%%%%%%%%
%%%%%%%%%%%%%%%%%%%%%%%%%%%%%%%%%%%%%%%%%%%%%%%%%%%%%%%%%%%%%%%%%%%%%%%%%%%%%%%%%%%%%%%%%%%%%%%%%%%%%%%%%%%%%%%%%%%%%%%%%%%%%%%%%
\begin{table}[ht!]
 
\centering
 
\begin{tabular}{|p{15cm}|}
 
\hline  
7. The table below is derived from Table 6.1 of the report on the Family Expenditure Survey 2000 − 2001.  It shows household expenditure in the United Kingdom during 1990 − 2001.  All expenditure figures are shown at 2000 − 2001 prices.  Using the information in this table, write an article for a serious newspaper on the distribution of and trends in household expenditure during this time.  Your article should incorporate such diagrams and such statistics calculated from the table as you think appropriate. (20) 
 

\\ \hline
  
\end{tabular}

\end{table}

\newpage

%%%%%%%%%%%%%%%%%%
\begin{table}[ht!]
 
\centering
 
\begin{tabular}{|p{15cm}|}
 
\hline   
 
Household expenditure 1990 to 2000-01, at 2000-01 prices. 
 Year 1990 1992 1994−95 1995−96 1996−97 1997−98 1998−99 1999−00 2000−01 Commodity or service Average weekly household expenditure (£) Housing 60.40 58.60 54.70 55.60 54.60 55.50 59.80 58.70 63.90 Fuel and Power 15.10 16.10 15.30 14.60 14.80 13.50 12.20 11.70 11.90 Food and Drink 74.50 72.70 73.90 75.70 77.20 76.90 76.20 77.20 76.90 Tobacco   6.50   6.70   6.60   6.70   6.90   6.80   6.10   6.20   6.10 Clothing and Footwear 21.80 20.30 20.20 20.30 20.90 21.90 22.70 21.60 22.00 Household goods and services 43.90 43.70 44.50 44.50 47.70 48.20 50.80 51.10 54.60 Personal goods and services 12.90 12.60 12.70 13.40 13.20 13.70 13.90 14.30 14.70 Motoring and travel 54.40 53.00 50.40 51.20 55.60 60.30 62.80 63.50 64.60 Leisure goods and services 44.60 50.60 53.10 53.90 56.60 61.30 62.50 64.40 70.30 Miscellaneous   1.90   2.20   2.70   1.40   1.10   1.20   1.30   1.50   0.70 Total 335.80 336.40 334.00 337.30 348.50 359.20 368.40 370.20 385.70 
 Note:  entries may not add to totals because of rounding. 

 

\\ \hline
  
\end{tabular}

\end{table}
 
%%%%%%%%%%%%%%%%%%%%%%%%%%%%%%%%%%%%%%%%%%%%%%%%%%%%%%%%%%%%%%%%%%%%%%%%%%%%%%%%%%%%%%%%%%%%%%%%%%%%%%%%%%%%%%%%%%%%%%%%%%%%%%%%%
 
\begin{enumerate}[(a)]
\item Suitable diagrams include the following:
pie charts, for chosen years between 1990 and 2001, showing the distribution
of spending between different categories of the household expenditure;
time series graphs for individual categories, expressing expenditure either as
an absolute figure or a percentage of the total;
bar charts, similar in purpose to pie charts, either as \% bar charts or totals to
show overall expenditure as well as components.
\begin{itemize}
    \item Because prices are given in terms of 2000/2001 levels, it may be less easy to track the
effects of price changes on consumption of individual items, or to see what may have
altered in the expenditure of items within each category.
\item Some points suggested by the raw data are:
total was steady early on, then began to increase, more quickly at the end;
housing showed a fall, then a rise, sharp at the end;
fuel and power fell, particularly from 1998 on;
household and personal goods and services increased in absolute value, as did
travel and leisure;
\item These latter four categories could certainly be explored as percentages of total, as well
as absolute values.
\item A newspaper article might emphasise the largest and smallest areas of spending, in
which areas spending increased, decreased or remained constant, and any apparent
relations in behaviour of the various categories (e.g. the four noted above).
\end{itemize}

\end{enumerate}
\end{document}

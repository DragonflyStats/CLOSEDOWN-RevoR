\documentclass[a4paper,12pt]{article}
%%%%%%%%%%%%%%%%%%%%%%%%%%%%%%%%%%%%%%%%%%%%%%%%%%%%%%%%%%%%%%%%%%%%%%%%%%%%%%%%%%%%%%%%%%%%%%%%%%%%%%%%%%%%%%%%%%%%%%%%%%%%%%%%%%%%%%%%%%%%%%%%%%%%%%%%%%%%%%%%%%%%%%%%%%%%%%%%%%%%%%%%%%%%%%%%%%%%%%%%%%%%%%%%%%%%%%%%%%%%%%%%%%%%%%%%%%%%%%%%%%%%%%%%%%%%
\usepackage{eurosym}
\usepackage{vmargin}
\usepackage{amsmath}
\usepackage{graphics}
\usepackage{epsfig}
\usepackage{enumerate}
\usepackage{multicol}
\usepackage{subfigure}
\usepackage{fancyhdr}
\usepackage{listings}
\usepackage{framed}
\usepackage{graphicx}
\usepackage{amsmath}
\usepackage{chngpage}
%\usepackage{bigints}

\usepackage{vmargin}
% left top textwidth textheight headheight
% headsep footheight footskip
\setmargins{2.0cm}{2.5cm}{16 cm}{22cm}{0.5cm}{0cm}{1cm}{1cm}
\renewcommand{\baselinestretch}{1.3}

\setcounter{MaxMatrixCols}{10}
\begin{document}
Higher Certificate, Paper II, 2003. Question 4
%%%%%%%%%%%%%%%%%%%%%%%%%%%%%%%%%%%%%%%%%%%%%%%%%%%%%%%%%%%%%%%%%%%%%%%%%%%%%%%%%%%%%%%%%%%%%%%%%%%%%%%%%%%%%%%%%%%%%%%%%%%%%%%%%

\begin{table}[ht!]
 
\centering
 
\begin{tabular}{|p{15cm}|}
 
\hline  

4. (i) A psychologist claims that visual memory is more effective than aural memory.  To test this claim, ten students are selected at random and examined for visual and aural memory using a standard memory test.  For each student, the psychologist notes whether his or her aural (A) or visual (V) memory score is the greater.  The results are as follows. 
 
Student 1 2 3 4 5 6 7 8 9 10 
Test A V V A V V V A V V 
 
Carry out a suitable analysis of these data to investigate the psychologist’s claim and comment on your results. (7) 

\\ \hline
  
\end{tabular}

\end{table}

\begin{table}[ht!]
 
\centering
 
\begin{tabular}{|p{15cm}|}
 
\hline  

\\ \hline
  
\end{tabular}

\end{table}

\begin{table}[ht!]
 
\centering
 
\begin{tabular}{|p{15cm}|}
 
\hline  
(ii) After completing this experiment, the psychologist decides to investigate whether aural memory scores can be improved by coaching.  A second experiment is conducted in which a random sample of 12 students perform an aural memory test before and after several sessions of coaching in skills believed to aid aural memory.  The results of the two tests are as follows. 
 
Before 53 59 61 48 39 56 75 45 73 60 69 66 After 60 57 67 52 63 71 70 46 76 65 62 65 
 
(a) It is required to test whether aural memory is improved, using a nonparametric test.  Explain why it is not satisfactory to use a test similar to the one used in part (i).  Carry out an appropriate non-parametric test. (8) 
 
(b) It is suggested that a parametric test would be more appropriate to analyse the data in (ii).  Without performing the analysis, state which test you would use and any assumptions necessary for this analysis to be valid.  Would these assumptions be reasonable in this case? (5) 
 
 

\\ \hline
  
\end{tabular}

\end{table}

%%%%%%%%%%%%%%%%%%%%%%%%%%%%%%%%%%%%%%%%%%%%%%%%%%%%%%%%%%%%%%%%%%%%%%%%%%%%%%%%%%%%%%%%%%%%%%%%%%%%%%%%%%%%%%%%%%%%%%%%%%%%%%%%%
\begin{enumerate}[(a)]
\item (i) A sign test is appropriate, the null hypothesis being that A and V are equally
likely to be the greater. Hence the number of As is binomial with parameters
10 and ½, as is the number of Vs, if the null hypothesis is true. The alternative
hypothesis claims that V is greater. A one-sided test is therefore needed.
The observed results are nA = 3, nB = 7.
If the null hypothesis is true, we have
( ) 10 10
10 10 10 7 1 1 120 45 10 1 0.172
7 8 9 2 2 V P n
       + + + ≥ =   +   +   +  = =
      
.
There is not enough evidence to reject the null hypothesis.
The sign test is not very powerful. The sample size here (8) is not really large
enough for its effective use.
\item  (a) The data provide information which the sign test would not use,
namely that measuring the change on a numerical scale. As well as the
sign of the change we should use the size. A Wilcoxon signed-rank
test is suitable. Difference A – B are as follows, and the absolute
values of the differences are ranked in size order, with tied values
given their average ranking.
(1) (2) (3) (4) (5) (6) (7) (8) (9) (10) (11) (12)
7 –2 6 4 24 15 –5 1 3 5 –7 –1
Rank 9½ 3 8 5 12 11 6½ 1½ 4 6½ 9½ 1½
The null hypothesis is that aural memory scores are not altered by
coaching, the alternative hypothesis is that coaching leads to
improvement. A one-sided test is required.
The sum of ranks of negative values is T– = 20½ and of positive values
is T+ = 57½. We use T = min(T–, T+) = 20½. Tables for n = 12 give 17
as the critical value for a one-sided 5% test, so there is not enough
evidence to reject the null hypothesis.
(b) A paired-samples t test using the differences would be appropriate if
the distribution of differences appeared to be approximately Normal.
The two large values for (5) and (6) make this unlikely, both being on
the same side (+). It would be unwise to use the t test in this case.
\end{enumerate}
\end{document}

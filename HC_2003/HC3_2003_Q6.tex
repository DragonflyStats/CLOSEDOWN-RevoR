\documentclass[a4paper,12pt]{article}
%%%%%%%%%%%%%%%%%%%%%%%%%%%%%%%%%%%%%%%%%%%%%%%%%%%%%%%%%%%%%%%%%%%%%%%%%%%%%%%%%%%%%%%%%%%%%%%%%%%%%%%%%%%%%%%%%%%%%%%%%%%%%%%%%%%%%%%%%%%%%%%%%%%%%%%%%%%%%%%%%%%%%%%%%%%%%%%%%%%%%%%%%%%%%%%%%%%%%%%%%%%%%%%%%%%%%%%%%%%%%%%%%%%%%%%%%%%%%%%%%%%%%%%%%%%%
\usepackage{eurosym}
\usepackage{vmargin}
\usepackage{amsmath}
\usepackage{graphics}
\usepackage{epsfig}
\usepackage{enumerate}
\usepackage{multicol}
\usepackage{subfigure}
\usepackage{fancyhdr}
\usepackage{listings}
\usepackage{framed}
\usepackage{graphicx}
\usepackage{amsmath}
\usepackage{chngpage}
%\usepackage{bigints}

\usepackage{vmargin}
% left top textwidth textheight headheight
% headsep footheight footskip
\setmargins{2.0cm}{2.5cm}{16 cm}{22cm}{0.5cm}{0cm}{1cm}{1cm}
\renewcommand{\baselinestretch}{1.3}

\setcounter{MaxMatrixCols}{10}
\begin{document}

Higher Certificate, Paper III, 2003. Question 6
\begin{enumerate}[(a)]
\item  84 126
ˆ 150 , 150. ˆ 200 , 200. A A B B p = n = p = n =
95% limits for the true value of (pA – pB) are estimated as
( ) ˆ (1 ˆ ) ˆ (1 ˆ )
ˆ ˆ 1.96 A A B B
A B
A B
p p p p
p p
n n
− −
− ± + ,
i.e. (0.56 0.63) 1.96 0.56 0.44 0.63 0.37
150 200
− ± × + ×
= −0.07 ±1.96 0.002808 = −0.07 ± 0.104 ,
i.e. (–0.174, 0.034).
The interval contains zero, so we should not claim that one journal is
significantly better than the other. However, with 95% confidence, we may
claim that the difference between them ranges from 3.4% in favour of A to
17.4% in favour of B.
%%%%%%%%%%%%%%%%%%%%%%%%%%%%%%%%%%%%%%%%%%%%%%%%
\item The two statements are alternatives, which together with (I) make up the
responses of the whole sample. Hence they are not independent, and the test
in part (i) assumes that they are.
%%%%%%%%%%%%%%%%%%%%%%%%%%%%%%%%%%%%%%%%%%%%%%%%
\item Although the statements relate to two different issues of the journal, they are
answered by (at least some of) the same people, and so once again the
responses do not come from independent samples. The two proportions will
most likely be correlated.
%%%%%%%%%%%%%%%%%%%%%%%%%%%%%%%%%%%%%%%%%%%%%%%%
\item Suppose n is the required sample size. Using ˆp = 0.63, as in part (i), Var( ˆp )
is estimated as (0.63)(0.37)/n. An approximate value for n is found by making
0.05 =1.96×SE( pˆ ) . This gives
( ) ( )( ) 0.05 2 0.63 0.37 Var ˆ
1.96
p
n
  = =  
 
,
( )( )
2 i.e. 0.63 0.37 1.96 358.2
0.05
n =   =
 
,
so at least 359 responses are needed.
Because this is a large sample, and p is not too far from ½, this approximation
will be satisfactory. Also we are told that there are a very large number of
subscribers, so that a "finite population correction" is unnecessary (even if we
knew N).
\end{enumerate}
\end{document}

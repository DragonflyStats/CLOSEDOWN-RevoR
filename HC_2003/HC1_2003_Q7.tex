\documentclass[a4paper,12pt]{article}
%%%%%%%%%%%%%%%%%%%%%%%%%%%%%%%%%%%%%%%%%%%%%%%%%%%%%%%%%%%%%%%%%%%%%%%%%%%%%%%%%%%%%%%%%%%%%%%%%%%%%%%%%%%%%%%%%%%%%%%%%%%%%%%%%%%%%%%%%%%%%%%%%%%%%%%%%%%%%%%%%%%%%%%%%%%%%%%%%%%%%%%%%%%%%%%%%%%%%%%%%%%%%%%%%%%%%%%%%%%%%%%%%%%%%%%%%%%%%%%%%%%%%%%%%%%%
  \usepackage{eurosym}
\usepackage{vmargin}
\usepackage{amsmath}
\usepackage{graphics}
\usepackage{epsfig}
\usepackage{enumerate}
\usepackage{multicol}
\usepackage{subfigure}
\usepackage{fancyhdr}
\usepackage{listings}
\usepackage{framed}
\usepackage{graphicx}
\usepackage{amsmath}
\usepackage{chngpage}
%\usepackage{bigints}




\usepackage{vmargin}
% left top textwidth textheight headheight
% headsep footheight footskip
\setmargins{2.0cm}{2.5cm}{16 cm}{22cm}{0.5cm}{0cm}{1cm}{1cm}
\renewcommand{\baselinestretch}{1.3}

\setcounter{MaxMatrixCols}{10}
\begin{document}
% Higher Certificate, Paper I, 2003. Question 7
\begin{framed}
\noindent The random variable X denotes the number of failures preceding the first
success in a series of independent Bernoulli trials, in each of which the
probability of success is p. Show that the probability mass function of $X$ is
\[{\displaystyle \Pr(X=k)=(1-p)^{k}p}\]
for k = 0, 1, 2, 3, .... 

and draw a graph of this function for the case p = 0.4.
\end{framed}

\begin{enumerate}[(a)]
\item Number of failures before the first
success

\begin{center}
\begin{tabular}{|l|c|c|} \hline 
Success at first attempt & k = 0 & S \\ \hline 
Success at second attempt & k = 1 & F S\\ \hline 
Success at third attempt & k = 2 & F F S\\ \hline 
Success at fourth attempt & k = 3 & F F F S\\ \hline 
\ldots & \ldots  &\ldots  \\ \hline 
\end{tabular}
\end{center}


\[P(X = k) = \underbrace{(1 - p)(1 - p)…(1 - p) }_\text{ k times }  p\] , for k = 0, 1, 2, … .


  When p = 0.4, P(0) = 0.4, P(1) = 0.24, P(2) = 0.144, P(3) = 0.0864,
P(4) = 0.0518, P(5) = 0.0311, P(6) = 0.0187, … .

%%%%%%%%%%%%%%%%%%%%%%%%%%%%%%%%%%%%%%%%%%%%%%%%%%%%%%%%%%%%%%%%%%%%%%
\newpage
\begin{framed}
Show that the probability generating function of $X$ is given by
( ) 1 (1 ) X
G s p
p s
=
− −
.
[for a suitable range of values of s].
Hence or otherwise obtain the mean and variance of Y.
\end{framed}
\begin{framed}
\noindent \textbf{Revision}\\
If X is a discrete random variable taking values in the non-negative integers \{0,1, ...\}, then the probability generating function of X is defined as \[{\displaystyle G(z)=\operatorname {E} (z^{X})=\sum _{x=0}^{\infty }p(x)z^{x},} \]
where p(x) is the probability mass function of X. 
\end{framed}

\item The probability generating function of X is

\begin{eqnarray*}
 E(s^x) &=& \sum_{x=0}^{\infty }pmf(x)s^{x}\\
 &=& \sum_{x=0}^{\infty }pmf(x)s^{x_i} p(1-p)^{x_{i}}\\
\end{eqnarray*}

\noindent Combining terms: $\qquad s^{x_i} \times (1-p)^{x_{i}} = t^{x_{i}}$ where $t = (1-p)\times s$

\begin{eqnarray*}
 E(s^x) &=& \sum_{x=0}^{\infty } p t^{x_{i}}\\ 
         &=& p \, \sum_{x=0}^{\infty }  t^{x_{i}}\\
         &=& p(1 + t +t^2 + \ldots)  \\
         &=& p(1-t)^{-1}    \mbox{(Geometric Series)}\\
         &=& \frac{p}{1-[s(1-p)]}
\end{eqnarray*} 

The mean is given by $G'(1)$ and the variance by $G''(1) + G'(1)- [G'(1)]2$, where
the differentiation is with respect to s.

\begin{eqnarray*} 
G^{\prime}(S) 
&=& \frac{p(1-p)}{[1-(1-p)^s]^2} \\
&=& \frac{1-p}{p}
\end{eqnarray*}


, so mean ( ) ( )
2


\begin{eqnarray*} 
G^{\prime}(1) 
&=& \frac{p(1-p)}{p^2} \\
&=& \frac{1-p}{p}
\end{eqnarray*}


\begin{eqnarray*} 
G^{\prime\prime}(S) 
&=& \frac{2p(1-p)^2}{[1-(1-p)^s]^3} \\
\end{eqnarray*}


\begin{eqnarray*} 
G^{\prime\prime}(1) 
&=& \frac{2p(1-p)^2}{p^3} \\
&=& \frac{1-p}{p^2}\\
\end{eqnarray*}
%%%%%%%%%%%%%%%%%%%%%%%%%%%%%%%%%%%%%%%%%%%%%%%%%%%%%%%%%%

Hence the variance is 

\begin{eqnarray*} 
\frac{2(1-p)^2}{p^2} + \frac{1-p}{p} + \frac{(1-p)^2}{p^2} &=& \frac{(1-p)^2}{p^2} - \frac{1-p}{p} \\
&=& \frac{1-p}{p^2}
\end{eqnarray*}


%%%%%%%%%%%%%%%%%%%%%%%%%%%%%%%%%%%%
\newpage
\begin{framed}
The random variable Y is defined as the number of trials up to and including
the first success in the series of Bernoulli trials referred to in part (a). Express
Y in terms of X, write down the probability mass function and the probability
generating function of Y and state the mean and variance of Y.

\end{framed}
\item We have Y = X + 1.
So $P(Y=y ) = p(1-p)^{y-i}$ , for $y= 1,2,3,...$.

The probability generating function of Y can be obtained by a similar method
to that used for X above, or it can be written down using the "linear
transformation" result for probability generating functions:
\begin{itemize}
\item[(i)] 
Pgf of Y is $s^bG(as)$ with a =1 and b =1, i.e. \[\frac{ps}{1-(1-p)s}.\]

\item[(ii)] 
\begin{eqnarray*}
\mbox{ Mean of } Y &=& (\mbox{ Mean of } X) + 1 \\ &=& 1 + \frac{1-p}{p} \\ &=& \frac{1}{p}\\
\end{eqnarray*}

\item[(iii)] Variance of Y = variance of X.
\end{itemize}
%%%%%%%%%%%%%%
\end{enumerate}
\end{document}

\documentclass[a4paper,12pt]{article}
%%%%%%%%%%%%%%%%%%%%%%%%%%%%%%%%%%%%%%%%%%%%%%%%%%%%%%%%%%%%%%%%%%%%%%%%%%%%%%%%%%%%%%%%%%%%%%%%%%%%%%%%%%%%%%%%%%%%%%%%%%%%%%%%%%%%%%%%%%%%%%%%%%%%%%%%%%%%%%%%%%%%%%%%%%%%%%%%%%%%%%%%%%%%%%%%%%%%%%%%%%%%%%%%%%%%%%%%%%%%%%%%%%%%%%%%%%%%%%%%%%%%%%%%%%%%
  \usepackage{eurosym}
\usepackage{vmargin}
\usepackage{amsmath}
\usepackage{graphics}
\usepackage{epsfig}
\usepackage{enumerate}
\usepackage{multicol}
\usepackage{subfigure}
\usepackage{fancyhdr}
\usepackage{listings}
\usepackage{framed}
\usepackage{graphicx}
\usepackage{amsmath}
\usepackage{chngpage}
%\usepackage{bigints}

\usepackage{vmargin}
% left top textwidth textheight headheight
% headsep footheight footskip
\setmargins{2.0cm}{2.5cm}{16 cm}{22cm}{0.5cm}{0cm}{1cm}{1cm}
\renewcommand{\baselinestretch}{1.3}

\setcounter{MaxMatrixCols}{10}
\begin{document}

Higher Certificate, Paper I, 2003. Question 7
\begin{enumerate}
\item P(X = x) = (1 – p)(1 – p)…(1 – p)p , for x = 0, 1, 2, … .
F F F S
----- x times -----
  When p = 0.4, P(0) = 0.4, P(1) = 0.24, P(2) = 0.144, P(3) = 0.0864,
P(4) = 0.0518, P(5) = 0.0311, P(6) = 0.0187, … .
\item The probability generating function of X is
( ) ( ) ( )
1 1 1
i i 1 i i X x x x x
i
i i i
G s E s s p s p p p t
∞ ∞ ∞
= = =
  = =Σ =Σ − = Σ where t = (1 – p)s
( ) ( )p 1 t t2 t3 ... p 1 t 1 − = + + + + = −
1 (1 )
p
p s
=
  − −
.
The mean is given by $G'(1)$ and the variance by $G''(1) + G'(1) – [G'(1)]2$, where
the differentiation is with respect to s.
( ) ( )
{ ( ) }2
1
'
1 1
p p
G s
p s
−
=
 − −
, so mean ( ) ( )
2
1 1  1
p p p G
p p
− − = = = .
( ) ( )
{ ( ) }
2
3
2 1
''
1 1
p p
G s
p s
−
=
− −
, so ( ) ( ) ( ) 2 2
3 2
2 1 2 1
'' 1
p p p
G
p p
− −
= = .
Hence the variance is ( ) ( ) ( ) 2 2 2
2 2 2 2
2 1 p 1 p 1 p 1 p 1 p 1 p
p p p p p p
− − − − − − + − = − = .
Continued on next page
0.4
0 1 2 3 4 5 6
P(x)
…
x
\item We have Y = X + 1.
So ( ) ( ) 1 1 , for 1,2,3,... y P Y y p p y − = = − = .

The probability generating function of Y can be obtained by a similar method
to that used for X above, or it can be written down using the "linear
transformation" result for probability generating functions:
\begin{itemize}
\item 
Pgf of Y is sbG(as) with a =1 and b =1, i.e. 1 (1 )
ps
− − p s
.
\item Mean of Y = (mean of X) + 1 = 1 1 p 1
p p
+ − = .
\item Variance of Y = variance of X.
\end{itemize}
%%%%%%%%%%%%%%
\end{enumerate}
\end{document}

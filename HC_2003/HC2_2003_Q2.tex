\documentclass[a4paper,12pt]{article}
%%%%%%%%%%%%%%%%%%%%%%%%%%%%%%%%%%%%%%%%%%%%%%%%%%%%%%%%%%%%%%%%%%%%%%%%%%%%%%%%%%%%%%%%%%%%%%%%%%%%%%%%%%%%%%%%%%%%%%%%%%%%%%%%%%%%%%%%%%%%%%%%%%%%%%%%%%%%%%%%%%%%%%%%%%%%%%%%%%%%%%%%%%%%%%%%%%%%%%%%%%%%%%%%%%%%%%%%%%%%%%%%%%%%%%%%%%%%%%%%%%%%%%%%%%%%
\usepackage{eurosym}
\usepackage{vmargin}
\usepackage{amsmath}
\usepackage{graphics}
\usepackage{epsfig}
\usepackage{enumerate}
\usepackage{multicol}
\usepackage{subfigure}
\usepackage{fancyhdr}
\usepackage{listings}
\usepackage{framed}
\usepackage{graphicx}
\usepackage{amsmath}
\usepackage{chngpage}
%\usepackage{bigints}

\usepackage{vmargin}
% left top textwidth textheight headheight
% headsep footheight footskip
\setmargins{2.0cm}{2.5cm}{16 cm}{22cm}{0.5cm}{0cm}{1cm}{1cm}
\renewcommand{\baselinestretch}{1.3}

\setcounter{MaxMatrixCols}{10}
\begin{document}

Higher Certificate, Paper II, 2003. Question 2

%%%%%%%%%%%%%%%%%%%%%%%%%%%%%%%%%%%%%%%%%%%%%%%%%%%%%%%%%%%%%%%%%%%%%%%%%%%%%%%%%%%%%%%%%%%%%%%%%%%%%%%%%%%%%%%%%%%%%%%%%%%%%%%%%

\begin{table}[ht!]
 
\centering
 
\begin{tabular}{|p{15cm}|}
 
\hline  
2. The telephonist answers telephone calls arriving at the switchboard of a particular organisation.  A random sample of 100 calls received at the switchboard on a particular day was monitored and the time taken for the telephonist to answer was recorded.  The data obtained are summarised in the following table. 
 
Time in seconds Number of Calls < 10     5 ≥ 10 but < 20   16 ≥ 20 but < 25   10 ≥ 25 but < 30   20 ≥ 30 but < 35   21 ≥ 35 but < 40   14 ≥ 40 but < 50   10 ≥ 50 but < 70     4 Total 100 
 
 
(i) Draw a histogram depicting the above data. 
(7) 
 

\\ \hline
  
\end{tabular}

\end{table}

\begin{table}[ht!]
 
\centering
 
\begin{tabular}{|p{15cm}|}
 
\hline  
(ii) Estimate the mean and median of the data.  What do the data and your statistics indicate about the distribution of the number of seconds it takes for the telephonist to answer a call? (6) 
 
(iii) Construct a 95\% confidence interval for the mean number of seconds for a call to be answered, stating any assumptions that you make. (7) 
 
\\ \hline
  
\end{tabular}

\end{table}
 


 
 

%%%%%%%%%%%%%%%%%%%%%%%%%%%%%%%%%%%%%%%%%%%%%%%%%%%%%%%%%%%%%%%%%%%%%%%%%%%%%%%%%%%%%%%%%%%%%%%%%%%%%%%%%%%%%%%%%%%%%%%%%%%%%%%%%
\begin{enumerate}[(a)]
\item Time (t) Frequency (f) Midpoint of
time interval
(x)
Cumulative
frequency (F)
fx fx2
0 – 10 5 5 5 25 125
10 – 20 16 15 21 240 3600
20 – 25 10 22.5 31 225 5062.5
25 – 30 20 27.5 51 550 13125
30 – 35 21 32.5 72 682.5 22181.25
35 – 40 14 37.5 86 525 19687.5
40 – 50 10 45 96 450 20250
50 – 70 4 60 100 240 14400
100 2937.5 100431.25
(i) Frequency density (per 5 seconds)
Time (seconds)
\item  Mean = 2937.5
100
= 29.38 seconds. Median = 30 – 1 5
20
 ×   
 
= 29.75 seconds.
The distribution is roughly symmetrical, perhaps a little skewed to the left.
Variance s2 = ( )2 1 2937.5 100431.25
99 100
 
 − 
 
 
= 142.8504, so s = 11.95.
\item  Assuming Normality of the distribution of times, and using the large-sample
formula, 95% limits are
29.38 1.96 11.95
100
± × , i.e. 29.38 ± 2.34 or (27.04, 31.72) seconds.
0 10 20 30 40 50 60 70
5
10
15
20
\end{enumerate}
\end{document}

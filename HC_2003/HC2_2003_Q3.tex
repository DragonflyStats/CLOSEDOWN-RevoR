\documentclass[a4paper,12pt]{article}
%%%%%%%%%%%%%%%%%%%%%%%%%%%%%%%%%%%%%%%%%%%%%%%%%%%%%%%%%%%%%%%%%%%%%%%%%%%%%%%%%%%%%%%%%%%%%%%%%%%%%%%%%%%%%%%%%%%%%%%%%%%%%%%%%%%%%%%%%%%%%%%%%%%%%%%%%%%%%%%%%%%%%%%%%%%%%%%%%%%%%%%%%%%%%%%%%%%%%%%%%%%%%%%%%%%%%%%%%%%%%%%%%%%%%%%%%%%%%%%%%%%%%%%%%%%%
\usepackage{eurosym}
\usepackage{vmargin}
\usepackage{amsmath}
\usepackage{graphics}
\usepackage{epsfig}
\usepackage{enumerate}
\usepackage{multicol}
\usepackage{subfigure}
\usepackage{fancyhdr}
\usepackage{listings}
\usepackage{framed}
\usepackage{graphicx}
\usepackage{amsmath}
\usepackage{chngpage}
%\usepackage{bigints}

\usepackage{vmargin}
% left top textwidth textheight headheight
% headsep footheight footskip
\setmargins{2.0cm}{2.5cm}{16 cm}{22cm}{0.5cm}{0cm}{1cm}{1cm}
\renewcommand{\baselinestretch}{1.3}

\setcounter{MaxMatrixCols}{10}
\begin{document}
Higher Certificate, Paper II, 2003. Question 3
 
%%%%%%%%%%%%%%%%%%%%%%%%%%%%%%%%%%%%%%%%%%%%%%%%%%%%%%%%%%%%%%%%%%%%%%%%%%%%%%%%%%%%%%%%%%%%%%%%%%%%%%%%%%%%%%%%%%%%%%%%%%%%%%%%%

\begin{table}[ht!]
 
\centering
 
\begin{tabular}{|p{15cm}|}
 
\hline 
3. One of the tasks routinely undertaken by a particular laboratory is to establish the potassium content of blood serum.  It has been established that, when apparatus is working properly, in repeated tests of the same blood serum the standard deviation should not exceed 0.05g (%). 
 
The manager of the laboratory decides to perform a quality control study of the two sets of apparatus used by the laboratory to measure the potassium content of various compounds.  A test sample is prepared in which the potassium content is known to be 10.5g (%) and each set of apparatus is used to make eight repeat analyses of the test sample.  The results in g (%) are as follows. 
 
Apparatus A 10.55 10.62 10.40 10.52 10.46 10.31 10.50 10.49 Apparatus B 10.30 10.25 10.35 10.30 10.28 10.35 10.24 10.43 
 
For each set of apparatus is there significant evidence that 
 
(i) the readings are more variable than the general laboratory standard? 
(8) 
 
(ii) the readings are biased? 
(8) 
 
Comment on how each set of apparatus should be altered to improve the accuracy of its measurements. (4) 
\\ \hline
  
\end{tabular}

\end{table}

 
%%%%%%%%%%%%%%%%%%%%%%%%%%%%%%%%%%%%%%%%%%%%%%%%%%%%%%%%%%%%%%%%%%%%%%%%%%%%%%%%%%%%%%%%%%%%%%%%%%%%%%%%%%%%%%%%%%%%%%%%%%%%%%%%%

\begin{enumerate}[(a)]
\item Null hypotheses to be tested are σ = 0.05 and μ = 10.5.
Summary statistics for the two sets of apparatus are:
A: x = 10.48, s2 = 0.008898 (s = 0.09433)
B: x = 10.31, s2 = 0.003876 (s = 0.06228)
n = 8 in both cases.
(i) Alternative hypothesis is σ 2 > (0.05)2 = 0.0025. Test statistic is ( ) 2
2
n 1 s
σ
−
,
refer to 2
1 χn− , i.e. 2
7 χ here: upper 5% point is 14.07, upper 1% point is 18.48.
For A, we get 7 0.008898
0.0025
× = 24.91, highly significant. For B, we get
7 0.003876
0.0025
× = 10.85, not significant.
The null hypothesis is rejected for A, but cannot be rejected for B. There is
evidence that A is more variable than standard, but not that B is more variable.
\item  Alternative hypothesis is μ ≠ 10.5. Test statistic is 10.5
/
x
s n
− , refer to tn–1, i.e.
t7 here.
For A, we get 0.02 8
0.09433
− = –0.60, not significant. For B, we get 0.19 8
0.06228
− =
–8.63, extremely highly significant.
There is very strong evidence that B is biased (downwards) but none that A is
biased.
\item  Probably B only needs a scale of measurement adjusted, if the complete
process is automated; more seriously there may be a fault in the way the
potassium content is measured. For A, there is too much variation, though the
mean value is acceptable, and this is likely to need adjustment to that part of
the process which can give rise to variability. In each case a laboratory
technician should be called in.
\end{enumerate}
\end{document}

\documentclass[a4paper,12pt]{article}
%%%%%%%%%%%%%%%%%%%%%%%%%%%%%%%%%%%%%%%%%%%%%%%%%%%%%%%%%%%%%%%%%%%%%%%%%%%%%%%%%%%%%%%%%%%%%%%%%%%%%%%%%%%%%%%%%%%%%%%%%%%%%%%%%%%%%%%%%%%%%%%%%%%%%%%%%%%%%%%%%%%%%%%%%%%%%%%%%%%%%%%%%%%%%%%%%%%%%%%%%%%%%%%%%%%%%%%%%%%%%%%%%%%%%%%%%%%%%%%%%%%%%%%%%%%%
\usepackage{eurosym}
\usepackage{vmargin}
\usepackage{amsmath}
\usepackage{graphics}
\usepackage{epsfig}
\usepackage{enumerate}
\usepackage{multicol}
\usepackage{subfigure}
\usepackage{fancyhdr}
\usepackage{listings}
\usepackage{framed}
\usepackage{graphicx}
\usepackage{amsmath}
\usepackage{chngpage}
%\usepackage{bigints}

\usepackage{vmargin}
% left top textwidth textheight headheight
% headsep footheight footskip
\setmargins{2.0cm}{2.5cm}{16 cm}{22cm}{0.5cm}{0cm}{1cm}{1cm}
\renewcommand{\baselinestretch}{1.3}

\setcounter{MaxMatrixCols}{10}
\begin{document}

Higher Certificate, Paper II, 2003. Question 1

%%%%%%%%%%%%%%%%%%%%%%%%%%%%%%%%%%%%%%%%%%%%%%%%%%%%%%%%%%%%%%%%%%%%%%%%%%%%%%%%%%%%%%%%%%%%%%%%%%%%%%%%%%%%%%%%%%%%%%%%%%%%%%%%%
\begin{table}[ht!]
 
\centering
 
\begin{tabular}{|p{15cm}|}
 
\hline  

 1. (i) A sports equipment company has commissioned an advertising agency to develop an advertising campaign for one of its new products.  They can choose between two particular television commercials, A and B.  To aid them in their decision, an experiment is performed in which 200 volunteers are randomly assigned to view one of the two commercials, 100 being assigned to each.  After seeing the commercial, each volunteer is asked to state whether they would consider buying the product, with the following results. 
 
  Commercial   A B No 70 80 Purchase product Yes 30 20 
 
 
Apply a chi-squared test to these data and comment on your results.  What recommendations, if any, would you make to the sports manufacturer concerning the choice of commercial for the proposed advertising campaign? (7) 

\\ \hline
  
\end{tabular}

\end{table}
\begin{table}[ht!]
 
\centering
 
\begin{tabular}{|p{15cm}|}
 
\hline  

(ii) A random sample of sportswear manufacturers was surveyed to determine whether they advertised on television and/or the internet.  The results are given in the following table. 
 
  Internet   No Yes No   3   5 Television Yes 15 17 
 
Apply McNemar's test to the above data and comment on your results. 
(7) 
\\ \hline
  
\end{tabular}

\end{table}


\begin{table}[ht!]
 
\centering
 
\begin{tabular}{|p{15cm}|}
 
\hline  

(iii) Distinguish carefully between chi-squared tests and McNemar's tests, as used to analyse data such as given in parts (i) and (ii) of this question, giving examples of when each would be preferred to the other. (6) 
 
\\ \hline
  
\end{tabular}

\end{table} 

%%%%%%%%%%%%%%%%%%%%%%%%%%%%%%%%%%%%%%%%%%%%%%%%%%%%%%%%%%%%%%%%%%%%%%%%%%%%%%%%%%%%%%%%%%%%%%%%%%%%%%%%%%%%%%%%%%%%%%%%%%%%%%%%%
\begin{enumerate}[(a)]
\item The chi-squared test will examine the null hypothesis that there is no relation
between "Commercial" and "Purchase", against the alternative hypothesis that
there is a relation.
Observed frequencies and (in brackets) expected frequencies on the null
hypothesis:
Commercial
A B Total
Purchase No 70 (75) 80 (75) 150 Yes 30 (25) 20 (25) 50
Total 100 100 200
Test statistic = ( ) ( ) ( ) ( ) 2 2 2 2 70 75 80 75 30 25 20 25
75 75 25 25
− − − −
+ + + = 2.667
[or 2.16 if calculated with Yates' correction so that "(O – E)2" becomes (4.5)2].
Refer to 2
1 χ : not significant. There is no evidence of a relation between
"Commercial" and "Purchase".
Hence a decision could be made on non-statistical grounds, such as cost of the
commercial or the potential size of the audience.
\item  McNemar's test for paired data tests similar hypotheses on association or
otherwise of the two classifications, in this case which advertising medium
each manufacturer uses. It does not use either "No–No" or "Yes–Yes"
manufacturers.
Test statistic = ( )2 5 15 100 5.00
5 15 20
−
= =
+
, refer to 2
1 χ , significant at 5%.
There is evidence against a null hypothesis of no preference of advertising
medium.
\item In part (i), two different random samples of responses are obtained, and the
problem is that of comparing the proportions giving a particular response in
the two samples. This is often the case in a chi-squared test, for example in
opinion surveys where the two samples are drawn from males and females, or
"young" and "old" age-groups, in otherwise similar populations. It also
applies in medical trials where different groups of patients (e.g. smokers and
non-smokers) are classified as having or not having a particular disease.
However, in part (ii) there are not two independent samples, and this may
occur more often in medical trials; for instance, suppose two drugs are used to
treat a chronic illness, each for a short period of time, on the same patients (or
at the least on patients who have been closely paired for age, sex and general
medical condition). No information is gained from patients (or pairs) where
both drugs worked, or both failed. The McNemar test examines whether, in
cases where only one worked, drug A was more successful than drug B or not.
It compares the proportions of preferences in this part of the data only. The
example in part (ii) uses the same manufacturers, so a McNemar test is valid.
\end{enumerate}
\end{document}

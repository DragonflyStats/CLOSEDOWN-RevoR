\documentclass[a4paper,12pt]{article}
%%%%%%%%%%%%%%%%%%%%%%%%%%%%%%%%%%%%%%%%%%%%%%%%%%%%%%%%%%%%%%%%%%%%%%%%%%%%%%%%%%%%%%%%%%%%%%%%%%%%%%%%%%%%%%%%%%%%%%%%%%%%%%%%%%%%%%%%%%%%%%%%%%%%%%%%%%%%%%%%%%%%%%%%%%%%%%%%%%%%%%%%%%%%%%%%%%%%%%%%%%%%%%%%%%%%%%%%%%%%%%%%%%%%%%%%%%%%%%%%%%%%%%%%%%%%
\usepackage{eurosym}
\usepackage{vmargin}
\usepackage{amsmath}
\usepackage{graphics}
\usepackage{epsfig}
\usepackage{enumerate}
\usepackage{multicol}
\usepackage{subfigure}
\usepackage{fancyhdr}
\usepackage{listings}
\usepackage{framed}
\usepackage{graphicx}
\usepackage{amsmath}
\usepackage{chngpage}
%\usepackage{bigints}

\usepackage{vmargin}
% left top textwidth textheight headheight
% headsep footheight footskip
\setmargins{2.0cm}{2.5cm}{16 cm}{22cm}{0.5cm}{0cm}{1cm}{1cm}
\renewcommand{\baselinestretch}{1.3}

\setcounter{MaxMatrixCols}{10}
\begin{document}
Higher Certificate, Paper I, 2003. Question 2
( ) 2 ( ) 1 ( ) 1
3 2 4
P A = P B = P C =
\begin{enumerate}
\item (i) (a) By the given independence,
( ) ( ) ( ) ( ) 2 1 1 1 1 1
3 2 4 4
P A∩B ∩C = P A P B P C =  −  −  =
  
.
(b) ( ) (( ) ( ))
( )
( )
( )
P A C A B P A C B
P A C A B
P A B P A B
∩ ∩ ∩ ∩ ∩
∩ ∩ = =
∩ ∩
( )
1
4
2 1
3 2
3
1 4
= =
−
.
\item 
Using pairwise independence, the value of P( A∩B) is P(A)P(B), etc, and
hence the values 1
3 x − , 18
− x and 1
6 − x are found. The others follow using
P(A), P(B) and P(C).
(a) ( ) 1
6
P A∩ B ∩C = + x from the diagram.
(b) ( ) (( ) ( ))
( )
( )
( )
P A C A B P A C B
P A C A B
P A B P A B
∩ ∩ ∩ ∩ ∩
∩ ∩ = =
∩ ∩
( ) ( )
16
1 1
6 6
3 1
2
x x
x x
= + = +
+ + −
.
(c) ( ) 19
24
P A∪ B∪C = + x .
Since all probabilities must lie in [0,1], we have 1
24 x ≥ and 18
x ≤ , i.e.
1 1
24 8
≤ x ≤ .
1
6 + x 1
3 − x 1
24 + x
x 18
− x 1
6 − x
1
24 x −
A B
C
\end{enumerate}
\end{document}
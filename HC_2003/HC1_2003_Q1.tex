\documentclass[a4paper,12pt]{article}
%%%%%%%%%%%%%%%%%%%%%%%%%%%%%%%%%%%%%%%%%%%%%%%%%%%%%%%%%%%%%%%%%%%%%%%%%%%%%%%%%%%%%%%%%%%%%%%%%%%%%%%%%%%%%%%%%%%%%%%%%%%%%%%%%%%%%%%%%%%%%%%%%%%%%%%%%%%%%%%%%%%%%%%%%%%%%%%%%%%%%%%%%%%%%%%%%%%%%%%%%%%%%%%%%%%%%%%%%%%%%%%%%%%%%%%%%%%%%%%%%%%%%%%%%%%%
\usepackage{eurosym}
\usepackage{vmargin}
\usepackage{amsmath}
\usepackage{graphics}
\usepackage{epsfig}
\usepackage{enumerate}
\usepackage{multicol}
\usepackage{subfigure}
\usepackage{fancyhdr}
\usepackage{listings}
\usepackage{framed}
\usepackage{graphicx}
\usepackage{amsmath}
\usepackage{chngpage}
%\usepackage{bigints}

\usepackage{vmargin}
% left top textwidth textheight headheight
% headsep footheight footskip
\setmargins{2.0cm}{2.5cm}{16 cm}{22cm}{0.5cm}{0cm}{1cm}{1cm}
\renewcommand{\baselinestretch}{1.3}

\setcounter{MaxMatrixCols}{10}
\begin{document}

Higher Certificate, Paper I, 2003. Question 1
\begin{enumerate}
\item Any of the digits $\{0, 1, \ldots, 9\}$ can occur in each of the six positions, so the number
is $10^6 = 1000000$.
\item $10 \times 9 \times 8 \times 7 \times 6 \times 5 = 151,200$, since no repetition is allowed.
[Alternatively, ( )
\[\frac{10!}{10- 6 !} =  \frac{10!}{4!}\] 

as above.]
\item There are ${10 \choose 6}$
choices of six different digits from ten, each of which
can be used in only one of its possible orders. So the number is
\[ {10 \choose 6} = \frac{10!}{6! \times (10-6)!} = \frac{10!}{6! \times 4!}
= \frac{10 \times 9 \times 8 \times 7 \times 6!}{6! \times 4!} = \frac{10 \times 9 \times 8 \times 7}{4 \times 3 \times 2 \times 1} = 210\]


\item  Here there are $ \displaymode{{10 \choose 3}}$
 
choices of digits, for each of which there are
$ \displaymode{\frac{6!}{2!2!2!}}$ = 90 orders

= orders, so the number is
$ \displaymode{{10 \choose 3}} \times 90 = \mbox{10800}$
 


\[ {10 \choose 3} = \frac{10!}{3! \times (10-3)!} = \frac{10!}{3! \times 7!}
= \frac{10 \times 9 \times 8 \times 7! }{3! \times 7!}\]

\item  

\begin{enumerate}[(i)]
    \item  There are $3! = 6$ possible orders for the first three digits, and 1 order
(the reverse order of the first three digits) for the last three. So there
are 6 codes.
\item If one digit is used 4 times in a palindromic code, another must be used
twice. These digits may be chosen in 3 and 2 ways respectively, i.e. in
6 ways for the pair. Once the digits have been chosen, only 3 patterns
are possible; for example, say the digits are 1 and 2, then the possible
patterns are 1 1 2 2 1 1, 1 2 1 1 2 1 and 2 1 1 1 1 2. So the total
number of codes is $6 \times 3 = 18$.
\item There are 3 choices of digit and only one possible pattern for each; so
there are 3 codes.
\item  Using the patterns from part (ii), there are
10
3
 
 
 
choices of digits for (a), each
giving 6 codes, i.e.

\[ { 10 \choose 3} \times 6 = 720\]

 
\times 6 = 720 altogether. For (b), there are
${ 10 \choose 2} = 45$ 
 
 
choices
of digits, i.e. 45, with 3 patterns as above, in which the two digits can be used
in 2 ways (4 of 1 and 2 of 2 or vice versa), giving $45 \times 3 \times 2 = 270$ ways. And
(c) can occur in 10 ways (because any digit can be used 6 times). Thus the
total is 720 + 270 + 10 = 1000 ways.
\end{enumerate}
ALTERNATIVELY, the first three positions may each be filled in 10 ways,
and then the whole sequence is determined, so there are $10^3 = 1000$ ways.
\end{enumerate}
\end{document}

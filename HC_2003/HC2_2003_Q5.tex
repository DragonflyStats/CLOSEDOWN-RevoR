\documentclass[a4paper,12pt]{article}
%%%%%%%%%%%%%%%%%%%%%%%%%%%%%%%%%%%%%%%%%%%%%%%%%%%%%%%%%%%%%%%%%%%%%%%%%%%%%%%%%%%%%%%%%%%%%%%%%%%%%%%%%%%%%%%%%%%%%%%%%%%%%%%%%%%%%%%%%%%%%%%%%%%%%%%%%%%%%%%%%%%%%%%%%%%%%%%%%%%%%%%%%%%%%%%%%%%%%%%%%%%%%%%%%%%%%%%%%%%%%%%%%%%%%%%%%%%%%%%%%%%%%%%%%%%%
\usepackage{eurosym}
\usepackage{vmargin}
\usepackage{amsmath}
\usepackage{graphics}
\usepackage{epsfig}
\usepackage{enumerate}
\usepackage{multicol}
\usepackage{subfigure}
\usepackage{fancyhdr}
\usepackage{listings}
\usepackage{framed}
\usepackage{graphicx}
\usepackage{amsmath}
\usepackage{chngpage}
%\usepackage{bigints}

\usepackage{vmargin}
% left top textwidth textheight headheight
% headsep footheight footskip
\setmargins{2.0cm}{2.5cm}{16 cm}{22cm}{0.5cm}{0cm}{1cm}{1cm}
\renewcommand{\baselinestretch}{1.3}

\setcounter{MaxMatrixCols}{10}
\begin{document}
Higher Certificate, Paper II, 2003. Question 5
%%%%%%%%%%%%%%%%%%%%%%%%%%%%%%%%%%%%%%%%%%%%%%%%%%%%%%%%%%%%%%%%%%%%%%%%%%%%%%%%%%%%%%%%%%%%%%%%%%%%%%%%%%%%%%%%%%%%%%%%%%%%%%%%%
\begin{table}[ht!]
 
\centering
 
\begin{tabular}{|p{15cm}|}
 
\hline  5. (i) Explain informally the central limit theorem and briefly explain its practical importance. (6) 
 
(ii) A pharmaceutical company needs to determine whether a new drug is effective in the treatment of elderly patients with insomnia.  To investigate this, 288 patients over the age of 65 suffering from insomnia were randomised to receive the new drug or a placebo for a period of 6 weeks.  At the end of this period, each patient was asked to complete a diary card for the next seven days, indicating the number of hours they had slept the preceding night.  The total number of hours slept by each patient during this assessment period was then recorded.  The following table gives summary statistics for the number of hours slept during the assessment period for the patients in each of the two groups. 
 
 New Drug Placebo Number of patients 144 144 Mean number of hours slept per patient 50.6 35.4 Variance of number of hours slept per patient 10.3 14.7 
 
Construct a 95\% confidence interval for the difference in the mean number of hours slept per patient between the two groups and interpret your findings. (7) 
 


\\ \hline
  
\end{tabular}

\end{table}
\begin{table}[ht!]
 
\centering
 
\begin{tabular}{|p{15cm}|}
 
\hline (iii) In a trial of anti-inflammatory drugs in the treatment of eczema, each member of a sample of 500 adults suffering from eczema was allocated at random to receive one of two treatments.  After one month, the patients were asked to state whether their eczema improved.  They replied as follows. 
 
 Improved Not Improved Treatment A 205 45 Treatment B 180 70 
 
Construct an approximate 95\% confidence interval for the difference in the proportion of eczema sufferers in the population who would report an improvement if given treatment A rather than treatment B. (7) 
 

\\ \hline
  
\end{tabular}

\end{table}
 
%%%%%%%%%%%%%%%%%%%%%%%%%%%%%%%%%%%%%%%%%%%%%%%%%%%%%%%%%%%%%%%%%%%%%%%%%%%%%%%%%%%%%%%%%%%%%%%%%%%%%%%%%%%%%%%%%%%%%%%%%%%%%%%%%
\begin{enumerate}[(a)]
\item (i) When a large sample {xi} of data is available from any distribution
(continuous or discrete) whose mean is μ and (finite) variance is σ 2, the total
ΣXi and the mean of the sample, X , are both approximately Normally
distributed with parameters (nμ, nσ 2) for the total and (μ, σ 2/n) for the mean.
The sample size n is required to be "large", but the implication of this for data
collection depends on the shape of the distribution; if it is not too
unsymmetrical n can be quite small, but for a highly skew distribution n needs
to be very large.
For example, data from agricultural plots, consisting of a large number of
individual plants, can usually be treated as approximately Normal, and so can
data from large-scale surveys. This allows statistical inference based on the
theory for the Normal distribution to be used. It also allows maximum
likelihood estimators based on large samples to be treated as Normal, for
example in constructing confidence intervals.
\item  2
For the new drug, n1 =144, x1 = 50.6, s1 =10.3 ;
2
2 2 2 For the placebo, n =144, x = 35.4, s =14.7 .
The variance of ( ) 1 2 X − X is ( 2 ) ( 2 )
1 1 2 2 σ / n + σ / n , and with large samples (as
these are for this type of measurement) we simply use 2
1 s and 2
2 s for 2
1 σ and
2
2 σ , and use the Normal approximation to obtain the interval
( )
2 2
1 2
1 2
1 2
x x 1.96 s s
n n
− ± + .
Thus we get
(50.6 35.4) 1.96 10.3 14.7
144
− ± + ,
i.e. 15.2 1.96 25
144
± , i.e. 15.2 1.96 5
12
± ×   
 
,
which is 15.2 ± 0.82 or (14.38, 16.02).
The new drug appears to give (with 95% confidence) between 14.38 and 16.02
extra hours of sleep in the week. This is a substantial improvement, of at least
2 hours per day, and significant at a high level since the interval does not
contain the value 0 (or go anywhere near it).
Continued on next page
\item  Treatment A: nA = 250, proportion improving ˆ A p = 205/250 = 0.82.
Treatment B: nB = 250, proportion improving ˆB p = 180/250 = 0.72.
These samples are large enough to use a Normal approximation to the
distributions of pˆ A and pˆB , and Var ( ˆ ˆ ) A B p − p can be estimated as
ˆ (1 ˆ ) ˆ (1 ˆ ) 0.82 0.18 0.72 0.28 0.1476 0.2016
250 250 250
A A B B
A B
p p p p
n n
− − × × + + = + =
= 0.0013968, so we have SE( ˆ ˆ ) A B p − p = 0.0374.
Also, we have ˆ ˆ A B p − p = 0.10, and therefore 95% limits for pA – pB are
(approximately)
0.10 ± (1.96×0.0374) , i.e. 0.10 ± 0.073 or (0.027, 0.173).
Treatment A gives better results than B, by an amount between 2.7% and
17.3% improvement (with 95% confidence). This is a significant
improvement since the interval does not contain 0.

\end{enumerate}
\end{document}




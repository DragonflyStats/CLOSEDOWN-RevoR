\documentclass[a4paper,12pt]{article}
%%%%%%%%%%%%%%%%%%%%%%%%%%%%%%%%%%%%%%%%%%%%%%%%%%%%%%%%%%%%%%%%%%%%%%%%%%%%%%%%%%%%%%%%%%%%%%%%%%%%%%%%%%%%%%%%%%%%%%%%%%%%%%%%%%%%%%%%%%%%%%%%%%%%%%%%%%%%%%%%%%%%%%%%%%%%%%%%%%%%%%%%%%%%%%%%%%%%%%%%%%%%%%%%%%%%%%%%%%%%%%%%%%%%%%%%%%%%%%%%%%%%%%%%%%%%
\usepackage{eurosym}
\usepackage{vmargin}
\usepackage{amsmath}
\usepackage{graphics}
\usepackage{epsfig}
\usepackage{enumerate}
\usepackage{multicol}
\usepackage{subfigure}
\usepackage{fancyhdr}
\usepackage{listings}
\usepackage{framed}
\usepackage{graphicx}
\usepackage{amsmath}
\usepackage{chngpage}
%\usepackage{bigints}

\usepackage{vmargin}
% left top textwidth textheight headheight
% headsep footheight footskip
\setmargins{2.0cm}{2.5cm}{16 cm}{22cm}{0.5cm}{0cm}{1cm}{1cm}
\renewcommand{\baselinestretch}{1.3}

\setcounter{MaxMatrixCols}{10}
\begin{document}
Higher Certificate, Paper III, 2003. Question 1
%%%%%%%%%%%%%%%%%%%%%%%%%%%%%%%%%%%%%%%%%%%%%%%%%%%%%%%%%%%%%%%%%%%%%%%%
\begin{framed}
1.
Three types of watch dial were tested on 21 subjects under simulated conditions. One
dial was assigned at random to each subject and the number of errors the subject made
in reading this dial during a standardised series of tests was recorded. The results are
shown in Table 1 below.
Table 1
1
42
30
21
47
34
22
42
38
Dial type
2
62
53
61
47
45
59
3
56
36
43
58
46
24
31
Incomplete results of a one-way analysis of variance of the data are shown in Table 2.
Table 2
One-way ANOVA: type 1, type 2, type 3
Analysis of Variance
Source
DF
SS
Dial type
2
1377
Error
18
1858
Total
20
3234
(i) Complete the analysis and interpret your results, stating any assumptions you
have made in reaching a conclusion.
(6)
(ii) Estimate the difference between the mean numbers of errors that would be
made by subjects reading dials of type 1 and of type 2, and find a 95%
confidence interval for this difference. Explain what is meant by describing
your interval as a "95% confidence interval". You may take it that any
assumptions needed for your analysis are satisfied.
(10)
(iii) How could you investigate the assumptions needed for the one-way analysis of
variance of the data in Table 1? If you were unwilling to accept these
assumptions, explain briefly how you might proceed. (Do not actually do so.)
(4)
2

\end{framed}
%%%%%%%%%%%%%%%%%%%%%%%%%%%%%%%%%%%%%%%%%%%%%%%%%%%%%%%%%%%%%%%%%%%%%%%%
\begin{enumerate}[(a)]
\item Analysis of variance
Source of variation df SS Mean Square F ratio
Dial type 2 1377 688.50 6.67
Residual (Error) 18 1858 103.22
Total 20 3234
The F ratio of 6.67 is referred to F2,18 and is very highly significant (p =
0.007). We reject the null hypothesis that the mean numbers of errors with the
three dial types are all the same. We deduce that at least one mean is different
from the other two.
We have assumed that all sets of data come from Normally distributed
populations with the same variance σ 2.
%%%%%%%%%%%%%%%%%%%%%%%%%%%%%%%%%%%%%%%%%%%%%%%%
\item We have x2 − x1 = 20.0 , and the standard error of this estimate is
2 2
1 2
s s
n n
+
= ( 1 1 )
8 6 103.22 + = 5.487. [Thus the difference between the means for dial
types 1 and 2 is very highly significant: test statistic is 20.0/5.487 = 3.645,
refer to t18.] The 5% critical value for t18 is 2.101, so 95% confidence limits
for the true population mean difference μ2 – μ1 are
20.0 ± (2.101 × 5.487) or 20.0 ± 11.53, i.e. (8.47, 31.53).
This means that, on the basis of these experimental data, we can say with 95%
confidence of being correct that the calculated interval does contain the true
value of μ2 – μ1.

%%%%%%%%%%%%%%%%%%%%%%%%%%%%%%%%%%%%%%%%%%%%%%%%
\item Residuals could be calculated for the 21 observations, and their pattern studied
either as a Normal probability plot or by plotting residuals against fitted
values.
The variances within the three dial types could be checked for equality, but no
good, sensitive, test exists for small amounts of data such as we have here.
Outliers, if any, could be checked for possible recording error or change in
background conditions. Any outliers could be removed from the data before
re-doing an analysis.
Sometimes a transformation (such as log) will make data behave more like
data from Normal homoscedastic distributions.
Summary: Dial type 1 2 3
Total 276 327 294
Number of tests ni 8 6 7 Total 21
Mean number of errors i x 34.5 54.5 42.0
\end{enumerate}
\end{document}

\documentclass{article}
\usepackage[utf8]{inputenc}
\usepackage{enumerate}

\author{kobriendublin }
\date{December 2018}

\begin{document}

%- Higher Certificate, Module 5, 2008. Question 1
\section{Introduction}
\begin{enumerate}[(i)]
\item 
\begin{table}[]
\begin{tabular}{lllll}
 &  &  &  &  \\
 &  &  &  &  \\
 &  &  &  &  \\
 &  &  &  & 
\end{tabular}
\end{table}

 ()()(),PXxYyPYyXxPXx======.
Table of P(X= x, Y = y).
Values of Y
1
2
3
4
Total
1
1/16
1/16
1/16
1/16
1/4
2
1/12
1/12
1/12
1/4
3
1/8
1/8
1/4
Values of X
4
1/4
1/4
Total
3/48
7/48
13/48
25/48
1
%%%%%%%%%%%%%%%%%%%%%%%%%%%%%
\item The marginal probability distribution of Y is as follows, copied from the margin of the table above.
()()()()3171321;2;3;44816484848PYPYPYPY⎛⎞========⎜⎟⎝⎠ .
()37132515613123448484848484EY⎛⎞⎛⎞⎛⎞⎛⎞=×+×+×+×==⎜⎟⎜⎟⎜⎟⎜⎟⎝⎠⎝⎠⎝⎠⎝⎠.
()237132554813714916484848484812EY⎛⎞⎛⎞⎛⎞⎛⎞=×+×+×+×==⎜⎟⎜⎟⎜⎟⎜⎟⎝⎠⎝⎠⎝⎠⎝⎠.
()21371341Var12448Y⎛⎞∴=−=⎜⎟⎝⎠.
(iii) ()11111123446161616161212EXY⎛⎞⎛⎞⎛⎞⎛⎞⎛⎞⎛=×+×+×+×+×+×⎜⎟⎜⎟⎜⎟⎜⎟⎜⎟⎜⎝⎠⎝⎠⎝⎠⎝⎠⎝⎠⎝ 11189121612884⎛⎞⎛⎞⎛⎞⎛+×+×+×+×⎜⎟⎜⎟⎜⎟⎜⎝⎠⎝⎠⎝⎠⎝ ()14203691216243254721924848=+++++++++=.
%%%%%%%%%%%%%%%%%%%%%%%%%%5
\item 
()()110123444EX=+++×= (and E(Y) = 13/4, see above).
()4201013Cov,4844XY⎛⎞∴=−×⎜⎟⎝⎠()1342039048488 =−==.
%%%%%%%%%%%%%%%%%%%%%%%%%%5
\item  U = X + Y.
()()121,116PUPXY=====
()()131,216PUPXY=====
()()()11741,32,2161248PUPXYPXY====+===+=
()()()11751,42,3161248PUPXYPXY====+===+=
()()()1110562,43,31284824PUPXYPXY====+===+==
()()173,48PUPXY=====
()()184,44PUPXY=====
No other values of U have non-zero probability.
\end{enumerate}
\end{document}
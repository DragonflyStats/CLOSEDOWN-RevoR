\documentclass{article}
\usepackage[utf8]{inputenc}
\usepackage{enumerate}

\author{kobriendublin }
\date{December 2018}

\begin{document}

%- Higher Certificate, Module 5, 2008. Question 1
\section{Introduction}
\begin{enumerate}[(i)]
%%%%%%%%%%%%%%%%%%%%%%%%%%%%%%%%%%%
\item  ()20011122xEXdxx\theta\theta\theta\theta\theta⎡⎤===⎢⎥⎣⎦∫.
           ()2230011133xEXdxx\theta\theta \theta\theta\theta⎡⎤===⎢⎥⎣⎦∫.
           ()()(){}2222111Var3212XEXEX \theta\theta\theta⎛⎞∴=−=−=⎜⎟⎝⎠.
%%%%%%%%%%%%%%%%%%%%%%%%%%%%%%%%%%%
\item  P(longest offcut is ≤ x) = P(all n offcuts are ≤ x).
           The c.d.f. for each Xi is ()()00xxduuxFxPXx\theta\theta\theta⎡⎤=≤===⎢⎥⎣⎦∫, and the Xi are all independent. 
          
Therefore P(all n offcuts are ≤ x) = (){}nnxFx\theta⎛⎞=⎜⎟⎝⎠, and this is also P(longest offcut is ≤ x), i.e. the c.d.f. of the sample maximum . 

Thus the p.d.f. of is the derivative of this, i.e. nx()nX()nXn–1/\theta n. 
This is for the interval $(0, \theta )$.
           ()()10011nnnnnnxnxnEXdxnn\theta\theta\theta\theta\theta+⎡⎤∴===⎢⎥++⎣⎦∫.
           ()()1220022nnnnnnxnxnEXdxnn\theta\theta \theta\theta\theta++⎡⎤===⎢⎥++⎣⎦∫.
           ()()()()()(){}()222222Var21nnnnnXEXEXnn\theta\theta∴=−=−++
             ()()()()()( 22222122112nnnnnnnnn\theta\theta⎛⎞+−+==⎜⎟⎜⎟++++⎝⎠.

Immediately we have ()1nnEXn\theta+⎛⎞=⎜⎟⎝⎠, so ()1nnXn+ is an unbiased estimator of \theta.
        ()()()()()()()()2222222111VarVar212nnnnnXnXnnnnnn\theta\theta+++⎛⎞===⎜⎟+⎝⎠++ .
        Solution continued on next page
%%%%%%%%%%%%%%%%%%%%%%%%%%%%%%%%%%%
\item  We have (see part (i)) that E(X) = \theta /2. Thus the method of moments estimator of \theta /2 is X, and so the method of moments estimator of \theta is 2X or 2iXnΣ as required.
        ()()()2224VarVar24VarVar.123iXXXXnn \theta\theta⎛⎞====⎜⎟⎝⎠Σ .
\end{enumerate}
\end{document}
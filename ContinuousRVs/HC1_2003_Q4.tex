\documentclass[a4paper,12pt]{article}
%%%%%%%%%%%%%%%%%%%%%%%%%%%%%%%%%%%%%%%%%%%%%%%%%%%%%%%%%%%%%%%%%%%%%%%%%%%%%%%%%%%%%%%%%%%%%%%%%%%%%%%%%%%%%%%%%%%%%%%%%%%%%%%%%%%%%%%%%%%%%%%%%%%%%%%%%%%%%%%%%%%%%%%%%%%%%%%%%%%%%%%%%%%%%%%%%%%%%%%%%%%%%%%%%%%%%%%%%%%%%%%%%%%%%%%%%%%%%%%%%%%%%%%%%%%%
  \usepackage{eurosym}
\usepackage{vmargin}
\usepackage{amsmath}
\usepackage{graphics}
\usepackage{epsfig}
\usepackage{enumerate}
\usepackage{multicol}
\usepackage{subfigure}
\usepackage{fancyhdr}
\usepackage{listings}
\usepackage{framed}
\usepackage{graphicx}
\usepackage{amsmath}
\usepackage{chngpage}
%\usepackage{bigints}

\usepackage{vmargin}
% left top textwidth textheight headheight
% headsep footheight footskip
\setmargins{2.0cm}{2.5cm}{16 cm}{22cm}{0.5cm}{0cm}{1cm}{1cm}
\renewcommand{\baselinestretch}{1.3}

\setcounter{MaxMatrixCols}{10}
\begin{document}
\large
%- Higher Certificate, Paper I, 2003. Question 4
\begin{framed}
\noindent The random variable $X$ has probability density function $f(x)$ given by
\[f(x) \;=\; kx^2 (1- x) , \qquad 0 \leq x \leq 1 \]
\medskip
\noindent \textbf{Part (a)}\\
Show that $k = 12$.

\end{framed}

\large

\begin{framed}
\noindent Total area under probability density function is 1 (between lower and upper limits)
{
\Large
\[\int^{U}_{L} f(x) dx = 1\]
}
\end{framed}
\newpage

\begin{enumerate}[(a)]
\item 
\[f(x) \;=\; kx^2 (1- x) , \qquad 0 \leq x \leq 1 \]
\begin{eqnarray*}
\mbox{ Area Under Curve } &=& \int^{1}_{0} f(x) dx \\
&=& \int^{1}_{0} k(x^2\;-\;x^3) dx \\
&=& \int^{1}_{0} kx^2\;-\;kx^3 dx \\
&=& k \left[ \frac{x^3}{3} - \frac{x^4}{4}\right]^1_0\\
&=& k \left[ \left( \frac{1}{3} \;-\; \frac{1}{4} \right) - \left( 0 \;-\; 0 \right)  \right]\\
&=& k \left[  \frac{1}{3} - \frac{1}{4} \right]\\
&=& k \left[  \frac{4}{12} - \frac{3}{12} \right]\\
1 &=&  \frac{k}{12} \\
\end{eqnarray*}

Necessarily $k = 12$.
%%%%%%%%%%%%%%%%%%%%%%%%%%%%%%%%%%%%%%%%%%%%%%%%%%%%%%%%
\newpage
\large
\begin{framed}
\noindent \textbf{Part (b)}\\
Show that the mode of $X$ is at $x = 2/3$.
and draw a graph of $f(x)$.
\end{framed}

\begin{figure}[h!]
\centering
\includegraphics[width=1.0\textwidth]{images/Curve-RSS-2003-Q4.jpeg}
\end{figure}
\item  
\begin{itemize}
\item Differentiating the probability density function
\begin{eqnarray*}
\frac{d\;f(x)}{dx} &=& \frac{d}{dx}\left( 12x^2-12x^3 \right)\\
&=&  24x-36x^2 \\
\end{eqnarray*}
which is zero for $2 - 3x = 0$.\\
Therefore $x = 2/3$.
\smallskip
\item \noindent \textbf{Remark:} \\ The derivative is zero for x = 0 also, but this is clearly not the mode, i.e.not the maximum of f (x).
\newpage
    \item 
To check that this is the maximum (i.e.the mode), we can consider the second derivative:-
\begin{eqnarray*}
\frac{d^2\;f(x)}{d^2x} &=& \frac{d}{dx}\left( 24x-36x^2 \right)\\
&=&  24-72x \\
\end{eqnarray*}
which is clearly $< 0$ at $x = 2/3$.
\item Hence the mode is at $x = 2/3$, and the graph of $f(x)$ is as shown above. 

% [NOTE. The  curve should of course appear smooth; it might not do so, due to the limits of
% electronic reproduction.]

%[At the mode, f (x) = 12(2/3)2(1/3) = 16/9.]
\end{itemize}
\large

%%%%%%%%%%%%%%%%%%%%%%%%%%%%%%%%%%%%%%%%%%%%%%%%%%%%%%%%%
%%%%%%%%%%%%%%%%%%%%%%%%%%%%%%%%%%%%%%%
\newpage
\begin{framed}
\large
\noindent \textbf{Part (c)}\\
Find the mean and variance of $X$.

\end{framed}
\large
\item  
\begin{eqnarray*}
E(X) &=& \int^{1}_{0} x \;f(x) dx \\
&=& \int^{1}_{0} x[12(x^2\;-\;x^3)] dx \\
&=& \int^{1}_{0} 12x^3\;-\;12x^4 dx \\
&=& 12 \left[ \frac{x^4}{4} \;-\; \frac{x^5}{5}\right]^1_0\\
&=& 12 \left[ \left( \frac{1}{4}- \frac{1}{5}\right) \;-\; \left( 0 - 0 \right)  \right]\\
&=& 12 \left[  \frac{5}{20} \;-\; \frac{4}{20} \right]\\
&=&  \frac{12}{20} \\
&=&  \frac{3}{5}\\
\end{eqnarray*}

\large
\begin{eqnarray*}
E(X^2) &=& \int^{1}_{0} x^2 \;f(x) dx \\
&=& \int^{1}_{0} x^2[12(x^2\;-\;x^3)] dx \\
&=& \int^{1}_{0} 12x^4\;-\;12x^5 dx \\
&=& 12 \left[ \frac{x^5}{5} \;-\; \frac{x^6}{6}\right]^1_0\\
&=& 12 \left[ \left( \frac{1}{5}-\frac{1}{6} \right) \;-\; \left(0 - 0 \right)  \right]\\
&=& 12 \left[  \frac{6}{30} \;-\; \frac{5}{30} \right]\\
&=&  \frac{12}{30} \\
& & \\
&=&  \frac{2}{5}\\
\end{eqnarray*}
So 

\begin{eqnarray*}
\operatorname{Var}(X) &=& E(X^2) - [E(X)]^2 \\
 &=& \frac{2}{5} - \left(\frac{3}{5}\right)^2 \\
 &=& \frac{10}{25} - \frac{9}{25} \\
 & & \\
 &=& \frac{1}{25} \\
\end{eqnarray*}
%%%%%%%%%%%%%%%%%%%%%%%%%%%%%%%%%%%%%%%
\newpage
\begin{framed}
\noindent \textbf{Part (d)}\\
 Find the cumulative distribution function of X and obtain the probability that $X$
lies within one standard deviation of its mean.

\end{framed}
\item  The cumulative distribution function is
\begin{eqnarray*}
F(X) &=& \int^{x}_{0} f(u) du \\
&=& \int^{x}_{0} 12(u^2\;-\;u^3)] du \\
&=& 12 \left[ \frac{u^3}{3} - \frac{u^4}{4}\right]^1_0\\
&=& 12 \left[ \left( \frac{x^3}{3}- \frac{x^4}{4} \right) - \left( 0 - 0 \right)  \right]\\
&=&  4x^3- 3x^4  \qquad 0 \leq x \leq 1 \\
\end{eqnarray*}
The mean is $3/5$
and the standard deviation is $1/5$ . 

We require  $P(2/5 < X < 4/5)$ .

This can be found by integrating the cdf between $2/5$ and $4/5$ or, directly, as

\begin{eqnarray*}
F\left(\frac{4}{5}\right) - F\left(\frac{2}{5}\right) &=& \left(\frac{4}{5}\right)^3\left(\frac{8}{5}\right) - \left(\frac{2}{5}\right)^3\left(\frac{14}{5}\right)\\
&=& \frac{(64 \times 8) - (8 \times 14)}{625} \\
&=& \frac{400}{625} \\
& & \\
&=& \frac{16}{25}
\end{eqnarray*}

\end{enumerate}
\end{document}

\documentclass{article}
\usepackage[utf8]{inputenc}
\usepackage{enumerate}

\author{kobriendublin }
\date{December 2018}

\begin{document}

%- Higher Certificate, Module 5, 2009. Question 4
\section{Introduction}
\begin{enumerate}[(i)]
\item 

%%%%%%%%%%%%%%%%%%%%%%%%%%%%%%%%%%%%%%%%%%%%%%
 Using the given probability generating function, ()()()211(1)dtppdtptπ−=−−.
()()()2111tdtpppEYdtppπ=−−∴===.
Also, ()()2232211(1)ppddtptπ−=−−.
()()2222312121tpppddtppπ=−− ∴==.
()()222221111VarppppYpppp−⎛⎞−−−∴=+−=⎜⎟⎝⎠.

%%%%%%%%%%%%%%%%%%%%%%%%%%%%%%%%%%%%%%%%%%%%%%

\item ()()1111niipEYEYnp=−===Σ , so Y is a biased estimator of 1p.
But ()11EYp+=, so 1Y+ is an unbiased estimator of 1p.
()()()()222111Var1VarVarinp pYYYnnp− −+====Σ
and this . 0asn→→

Since 1Y+ is unbiased for 1p (for all n) and its variance tends to zero as , n→∞1Y+ is a consistent estimator of 1p.
Solution continued on next page

%%%%%%%%%%%%%%%%%%%%%%%%%%%%%%%%%%%%%%%%%%%%%%
\item  The method of moments estimator of p is the solution, say, of ˆp()YEY=, i.e. we have 11ˆYp=−. 11ˆYp∴=+, i.e. 1ˆ1pY=+.
(iv) We have . So the distribution of W is B(n, p). ()()001iPYppp==−=
()EWnp∴=, and so is an unbiased estimator of p. /Wn
()()()22111VarVar1ppWWnppnnnn−⎛⎞==−=⎜⎟⎝⎠.

\end{enumerate}
\end{document}
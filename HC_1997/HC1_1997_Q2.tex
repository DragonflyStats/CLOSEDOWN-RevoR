\documentclass[a4paper,12pt]{article}
%%%%%%%%%%%%%%%%%%%%%%%%%%%%%%%%%%%%%%%%%%%%%%%%%%%%%%%%%%%%%%%%%%%%%%%%%%%%%%%%%%%%%%%%%%%%%%%%%%%%%%%%%%%%%%%%%%%%%%%%%%%%%%%%%%%%%%%%%%%%%%%%%%%%%%%%%%%%%%%%%%%%%%%%%%%%%%%%%%%%%%%%%%%%%%%%%%%%%%%%%%%%%%%%%%%%%%%%%%%%%%%%%%%%%%%%%%%%%%%%%%%%%%%%%%%%
\usepackage{eurosym}
\usepackage{vmargin}
\usepackage{amsmath}
\usepackage{graphics}
\usepackage{epsfig}
\usepackage{enumerate}
\usepackage{multicol}
\usepackage{subfigure}
\usepackage{fancyhdr}
\usepackage{listings}
\usepackage{framed}
\usepackage{graphicx}
\usepackage{amsmath}
\usepackage{chngpage}
%\usepackage{bigints}

\usepackage{vmargin}
% left top textwidth textheight headheight
% headsep footheight footskip
\setmargins{2.0cm}{2.5cm}{16 cm}{22cm}{0.5cm}{0cm}{1cm}{1cm}
\renewcommand{\baselinestretch}{1.3}

\setcounter{MaxMatrixCols}{10}
\begin{document}
\begin{table}[ht!]
     \centering
     \begin{tabular}{|p{15cm}|}
     \hline        
If a Ruritanian peasant farmer grows cereals, his profit $X_1$ in Ruritanian pounds (£R),  is Normally distributed with mean 1750 and standard deviation 300.  

If he grows beans his profit $X_2$ is Normally distributed with mean 2000 and standard deviation 400.  If he grows a proportion $p$ of cereals and a proportion $1−p$ of beans, the profit, Y, is pX p X 12 1 +− ()where $0\leq p\leq 1 $ .
\\
\noindent \textbf{Part (a)}\\
State the distribution of $Y$, assuming that $X_1$ and $X_2$ are independent.\\

 \hline
      \end{tabular}
    \end{table}
    


\begin{enumerate}[(a)]
\item $Y = pX_1 + (1 - p)X_2$ is

\begin{itemize}
    \item $\mu_Y \;=\; 1750p + 2000(1 - p) \;=\; 2000 - 250p$
    \item $\sigma^2_Y \;=\; [p \times 300]^2 + [(1 - p) \times 400]^2 \;=\; 10000 \left[9p^2 + 16(1 - p)^2\right]$
\end{itemize}


\[ Y \sim N \left(2000 - 250p \,;\, 10000\left[25p^2 - 32p + 16\right] \right)\].


%%%%%%%%%%%%%%%%%%%%%%%%%%%%%%%%%%%%%%5
\newpage
  \begin{table}[ht!]
     \centering
     \begin{tabular}{|p{15cm}|}
     \hline  
(ii) State the value of p, $p_1$ say, which maximises the farmer’s expected profit,$E(Y)$. 
\\ \hline 
      \end{tabular}
    \end{table}
    
    
\item $E[Y ] = 2000 - 250p$ and has maximum value (2000) for $p1 = 0$.
\item $V[Y]$ is minimized when 
\[ \frac{d}{dp}(25p^2 - 32p + 16) =0\]

\[i.e. 50p - 32 = 0\] or \[p2 = \frac{16}{25}=0.64\].
The second deviation is > 0, indicating a minimum.
\item  $E[Y |p1 = 0] = \$2000$. 
\begin{eqnarray*}
E\left[Y | p^2 = \frac{16}{25}\right] &=& 2000 - 250 \frac{16}{25}\\ 
&=& \$1840.
\end{eqnarray*}


%%%%%%%%%%%%%%%%%%%%%%%%%%%%%%%%%%%%%%%%%%%%%%%%5
\newpage
  \begin{table}[ht!]
     \centering
     \begin{tabular}{|p{15cm}|}
     \hline  

(iii) Find the value of $p$, $p_2$  say, which minimises the variance of the farmer’s profit, that is, minimises $V(Y)$. \\ \hline 
      \end{tabular}
    \end{table}
\item  (a) On p = p1 = 0, $Y \sim N(2000; 400^2)$:
\begin{eqnarray*}
P(Y < 1480) &=& P(Z < \left( \frac{1480-2000}{400}\right) \\
&=& P(Z < -1.30)\\ 
&=& 0.0968.\\
\end{eqnarray*}
%%%%%%%%%%%%%%%%%%%%%%%%%%%%%%%%%%%%%%%%%%%%%%%%%%%%
\begin{table}[ht!]
\centering
\begin{tabular}{|p{15cm}|}
\hline  
(iv) Calculate the expected profit when $p = p_1$ and when $p = p_2$.
(v) The farmer reckons that he will be ruined if his profit is less than \$1480. Calculate the probability that the farmer will be ruined (a) if he adopts $p = p_1$, (b) if he adopts $p = p_2$ .

Which course of action do you think is better for the farmer, and why?\\
\\ \hline
\end{tabular}
\end{table}
(b) On $p = p_2 = \frac{16}{25}$, 
\[Y \sim N(1840; 10^4\left[\frac{16^2}{25} - \frac{32 \times 16}{25} + 16\right] ).\]
i.e. \[Y \sim N \left( 1840, 16\times 10^4 \left[1- \frac{ 16}{25} \right] \right) = N(1840, 240^2)\]:


\begin{eqnarray*}
P(Y < 1480) &=& P\left(Z < \left( \frac{1480-1840}{240} \right)
\right) \\
&=& P(Z < -1.50)\\ 
&=& 0.0668.\\
\end{eqnarray*}
%%%%%%%%%%%%%%%%%%%%%%%%

\begin{itemize}
    \item Z stands for the standardized variate N(0; 1).
    \item Use mixed strategy (b) because
its probability of ruin is only two-thirds of that on (a).
\item His expectation
is lower, but variability is also lower, on (b).
\end{itemize}

\end{enumerate}
\end{document}

\documentclass[a4paper,12pt]{article}
%%%%%%%%%%%%%%%%%%%%%%%%%%%%%%%%%%%%%%%%%%%%%%%%%%%%%%%%%%%%%%%%%%%%%%%%%%%%%%%%%%%%%%%%%%%%%%%%%%%%%%%%%%%%%%%%%%%%%%%%%%%%%%%%%%%%%%%%%%%%%%%%%%%%%%%%%%%%%%%%%%%%%%%%%%%%%%%%%%%%%%%%%%%%%%%%%%%%%%%%%%%%%%%%%%%%%%%%%%%%%%%%%%%%%%%%%%%%%%%%%%%%%%%%%%%%
\usepackage{eurosym}
\usepackage{vmargin}
\usepackage{amsmath}
\usepackage{graphics}
\usepackage{epsfig}
\usepackage{enumerate}
\usepackage{multicol}
\usepackage{subfigure}
\usepackage{fancyhdr}
\usepackage{listings}
\usepackage{framed}
\usepackage{graphicx}
\usepackage{amsmath}
\usepackage{chngpage}
%\usepackage{bigints}

\usepackage{vmargin}
% left top textwidth textheight headheight
% headsep footheight footskip
\setmargins{2.0cm}{2.5cm}{16 cm}{22cm}{0.5cm}{0cm}{1cm}{1cm}
\renewcommand{\baselinestretch}{1.3}

\setcounter{MaxMatrixCols}{10}
\begin{document}
\begin{table}[ht!]
 \centering
 \begin{tabular}{|p{15cm}|}
 \hline  
Write an essay describing the use of residuals as  diagnostic tools in data analysis. You should define what residuals are and illustrate your explanation with sketches of the types of patterns one might expect to see when examining residual plots and indicate what the patterns reveal.
\\ \hline
  \end{tabular}
\end{table}
\begin{itemize}
    \item A set of data may be fitted by a statistical model, e.g. a linear regression yi =
a+bxi+ei or an experimental design model such as yij = ¹+ti+bj +eij for
randomized complete blocks. The terms feig or feijg are generally assumed
N(0; ¾2), independently of one another. After the parameters a; b or ¹, ftig,
19
fbjg have been estimated the fitted values ˆyi = ˆa+ˆbxi or ˆyij = ˆ¹+ˆti+ˆbj can
be found. Then the differences yi ¡ ˆyi or yij ¡ ˆyij , observed minus expected
(fitted), are the residuals. These residuals should be from the same N(0; ¾2)
population. 
\item Their sizes should bear no systematic relationship to the sizes
of the corresponding ˆyi, ˆyij , or (for example) to xi if x represents time in a
set of time-series data or if x is any variable on a quantitative scale such as
a level of fertilized application.
\item Clearly they should cluster around 0 and be
symmetrical. We may examine several of these properties in diagrams. A
useful one is to plot the residuals ei against corresponding fitted values yi.
\item If the wrong model has been fitted, e.g. a linear regression which should be
a curve, the residuals will show a regular patten, e.g.,
If the values of ¾2 is not constant, but increases as y increases, a ‘fan’ shape
may appear.
\item A skew distribution, rather than normal, will have all the largest residuals
on the same side of 0:
20
\item There may be outliers in the data, which will show up as isolated large values
(positive or negative) of ei:
However, this does not always happen (cf. Qu 2 where the two “odd” points
can be fitted quite well by the first line with negative slope). 
\item Normal probability
plotting can also be used to check the assumption of normality. The
residuals, ordered by size, are plotted against the expected values of normal
order statistics. Noticeable non-linearity is a warning that the assumption
may be valid.
\end{itemize}
\end{document}

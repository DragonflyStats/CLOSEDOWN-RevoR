\documentclass[a4paper,12pt]{article}
%%%%%%%%%%%%%%%%%%%%%%%%%%%%%%%%%%%%%%%%%%%%%%%%%%%%%%%%%%%%%%%%%%%%%%%%%%%%%%%%%%%%%%%%%%%%%%%%%%%%%%%%%%%%%%%%%%%%%%%%%%%%%%%%%%%%%%%%%%%%%%%%%%%%%%%%%%%%%%%%%%%%%%%%%%%%%%%%%%%%%%%%%%%%%%%%%%%%%%%%%%%%%%%%%%%%%%%%%%%%%%%%%%%%%%%%%%%%%%%%%%%%%%%%%%%%
\usepackage{eurosym}
\usepackage{vmargin}
\usepackage{amsmath}
\usepackage{graphics}
\usepackage{epsfig}
\usepackage{enumerate}
\usepackage{multicol}
\usepackage{subfigure}
\usepackage{fancyhdr}
\usepackage{listings}
\usepackage{framed}
\usepackage{graphicx}
\usepackage{amsmath}
\usepackage{chngpage}
%\usepackage{bigints}

\usepackage{vmargin}
% left top textwidth textheight headheight
% headsep footheight footskip
\setmargins{2.0cm}{2.5cm}{16 cm}{22cm}{0.5cm}{0cm}{1cm}{1cm}
\renewcommand{\baselinestretch}{1.3}

\setcounter{MaxMatrixCols}{10}
\begin{document}
%%%%%%%%%%%%%%%%%%%%%%%%%%%%%%%%%%%%%%%%%%%%%%%%%%%%%%%%%%%%%%%%%%%%%%%%%%%%%%%%%%%%%
\begin{table}[ht!]
     
\centering
     
\begin{tabular}{|p{15cm}|}
     
\hline 
1. The table below contains some data about hospital patients.
Descriptive statistics of 1383 hospital stays at Hospital del Mar, Barcelona,
in 1988 and 1990
\begin{center}
\begin{tabular}{|c|c|c|}
Variable 	&	1988	&	1990	\\ \hline
Total number of stays 	&	750	&	633	\\ \hline
Mean age at admission	&	53.4 years	&	55.3 years	\\ \hline
Standard deviation of age	&	19.7 years	&	19.5 years	\\ \hline
Males	&	349	&	321	\\ \hline
Females	&	401	&	312	\\ \hline
\end{tabular}
\end{center}

(Source: S.J. Gange et al, Applied Statistics, No 3, 1996)
You may take these patients to be simple random samples of admissions to this hospital in
1988 and 1990. Use appropriate statistical tests to examine whether, between 1988 and
1990:
i) the mean age at admission increased;
ii) the sex ratio changed.
In each case, if you reject the null hypothesis of no change, estimate the change and provide
a 95 per cent confidence interval for your estimate.
\\ \hline
      
\end{tabular}
    
\end{table}
%%%%%%%%%%%%%%%%%%%%%%%%%%%%%%%%%%%%%%%%%%%%%%%%%%%%%%%%%%%%%%%%%%%%%%%%%%%%%%%%%%%%%
\begin{enumerate}

\item  For 1988, $\bar{x}_1 = 53:4$, $s_1 = 19.7$; also n = 750;
for 1990, $\bar{x}_2 = 55.3$, s2 = 19:5; also n = 633.
If $\mu_1$; $\mu_2$ are the corresponding population means, H0 is \mu_1 = \mu_2 (or, strictly,
$\mu_1 ¸ \mu_2$) and H1, to be tested, is \mu_2 > \mu_1.

\[S.E. (\bar{x}_2 ¡ \bar{x}_1) = 

\sqrt{  \frac{s^2_1}{n_1} + \frac{s^2_2}{n_2} }
=  \sqrt{  \frac{19.7^2}{750} + \frac{19.5^2}{633} }= \sqrt{1:11816} = 1:057 \], SE .
As these are large samples of date we use a normal (r) test:
r = 55:3¡53:4
1:057 = 1:9
1:057 = 1:798.
\begin{itemize}
\item The form of $H_1$ requires a one-tail test, with critical value 1.645 at 5\%.
\item Hence we reject $H_0$.
\item A 95\% confidence interval for the increase is 1:9§1:96£1:057 = 1:9§2:07,
or (-0.17; 3.97).
\item If we are certain that there must have been an increase we may prefer to
quote this result as (0; 3:97).
\end{itemize}

%%%%%%%%%%%%%%%%%
\item 

\begin{itemize}
\item For 1988, $p_M = \frac{349}{750} = 0.4653$ and $p_F = 0.5347$; n = 750.
\item For 1990, $p_M = \frac{321}{633} = 0.5071$ and $p_F = 0.4929$; n = 633.
\end{itemize}
The hypotheses $H_0$: pM;1988 = pM;1990 and H1 : pM has changed can be
examined in a $2 \times 2$ table of ‘observed’ frequencies and ‘those expected on
$H_0$’.

\begin{center}
{
\begin{tabular}{c|c|c|c|}
  \multicolumn{2} {c} {} & \multicolumn{2}{c} {{OBSERVED(EXPECTED)}} \\
\cline{3-4}
\multicolumn{2}{c|}{} & 1988        & 1990        \\
 \cline{2-4}
\multirow  {{}}& Male & 349(363.34) & 321(306.66) \\
 \cline{2-4}
& Female &401(386.66)& 312(326.34)\\
\cline{2-4}

%\hline
\end{tabular}
}
\end{center}
OBSERVED(EXPECTED) 1988 1990 TOTAL
MALE   670
FEMALE   713
750 633 1383

\begin{eqnarray*}
\chi^2_{
(1)} &=& 
\frac{(349 - 363.34)^2}{
363.34} 
+ \frac{(321 - 306.66)^2}{
306.66} 
+   \frac{(401 - 386.66)^2}{
386.66} 
+ \frac{(349 - 
326.34)^2}{
326.34}\\
&=&  \frac{(-14.34)^2}{363.34} + \frac{(14.34)^2}{306.66} + \frac{(14.34)^2}{386.66} + \frac{(-14.34)^2}{326.34}  \\
&=&  \\
&=& 2.40n.s.\\
\end{eqnarray*}
\begin{itemize}
    \item There is no evidence of change.
\item An alternative method is to use normal approximations for pM: $N(p; \frac{p(1-p)}{n})$

in each year and consider the difference. 
\item This would be needed if confidence
intervals had been required. 
\end{itemize}

\end{enumerate}
\end{document}

\documentclass[a4paper,12pt]{article}
%%%%%%%%%%%%%%%%%%%%%%%%%%%%%%%%%%%%%%%%%%%%%%%%%%%%%%%%%%%%%%%%%%%%%%%%%%%%%%%%%%%%%%%%%%%%%%%%%%%%%%%%%%%%%%%%%%%%%%%%%%%%%%%%%%%%%%%%%%%%%%%%%%%%%%%%%%%%%%%%%%%%%%%%%%%%%%%%%%%%%%%%%%%%%%%%%%%%%%%%%%%%%%%%%%%%%%%%%%%%%%%%%%%%%%%%%%%%%%%%%%%%%%%%%%%%
\usepackage{eurosym}
\usepackage{vmargin}
\usepackage{amsmath}
\usepackage{graphics}
\usepackage{epsfig}
\usepackage{enumerate}
\usepackage{multicol}
\usepackage{subfigure}
\usepackage{fancyhdr}
\usepackage{listings}
\usepackage{framed}
\usepackage{graphicx}
\usepackage{amsmath}
\usepackage{chngpage}
%\usepackage{bigints}

\usepackage{vmargin}
% left top textwidth textheight headheight
% headsep footheight footskip
\setmargins{2.0cm}{2.5cm}{16 cm}{22cm}{0.5cm}{0cm}{1cm}{1cm}
\renewcommand{\baselinestretch}{1.3}

\setcounter{MaxMatrixCols}{10}
\begin{document}

%%%%%%%%%%%%%%%%%%%%%%%%%%%%%%%%%%%%%%%%%%%%%%%%%%%%%%%%%%%%%%%%%%%%%%%%%%%%%%%%%%%%%
\begin{table}[ht!]
     
\centering
     
\begin{tabular}{|p{15cm}|}
     
\hline  
3. Twenty employees of a company were selected for computer training, and divided into two
groups of ten. As numeracy was thought to be relevant, the employees were matched on this
variable before the training course began and assigned to groups randomly. Different
training methods were used for each group. After the training course was completed, an
aptitude test was given, with the following results.

\begin{center}
\begin{tabular}{|c|c|c|c|c|c|c|c|c|c|c|}
Matched Pair 	&	A 	&	B 	&	C 	&	D 	&	E 	&	F 	&	G 	&	H	&	 I 	&	J	\\ \hline
First group 	&	59	&	63	&	65	&	61	&	58	&	55	&	68	&	66	&	6	&	54	\\ \hline
Second group 	&	62	&	71	&	70	&	60	&	57	&	80	&	67	&	69	&	75	&	64	\\ \hline
\end{tabular}
\end{center}

Test the null hypothesis that the training method does not affect aptitude test scores using
(i) the sign test, and (ii) the Wilcoxon matched-pairs signed-ranks test. State and comment
on your findings. Under what circumstances would a parametric test have been more
appropriate?
\\ \hline
      
\end{tabular}
    
\end{table}
%%%%%%%%%%%%%%%%%%%%%%%%%%%%%%%%%%%%%%%%%%%%%%%%%%%%%%%%%%%%%%%%%%%%%%%%%%%%%%%%%%%%%


\begin{center}
\begin{tabular}{|c|c|c|c|c|c|c|c|c|c|c|c|}																				 \hline
Pair 	&	A 	&	B	&	 C 	&	D 	&	E 	&	F 	&	G 	&	H 	&	I 	&	J	&		\\
	&		&		&		&		&		&		&		&		&		&		&		\\ \hline
Sign  	&	 +	&	 + 	&	+ 	&	- 	&	- 	&	+ 	&	- 	&	+ 	&	+ 	&	+ 	&	(7"+", 3"-")	\\
(Gp.2 - Gp.1)	&		&		&		&		&		&		&		&		&		&		&		\\ \hline
Difference 	&	3	&	8	&	5	&	-1	&	-1 	&	25	&	-1 	&	3	&	19	&	10	&		\\
	&		&		&		&		&		&		&		&		&		&		&		\\ \hline
Rank 	&	$4\frac{1}{2}$	&	7	&	6	&	2	&	2	&	10	&	2	&	$4\frac{1}{2}$ 	&	9	&	8	&		\\
	&		&		&		&		&		&		&		&		&		&		&		\\ \hline
\end{tabular}
\end{center}

\begin{enumerate}

\item  The number of + signs should be binomial (n = 10; p = 1=2) on the Null
Hypotheses of no difference between groups (i.e. training methods). 
\begin{itemize}
\item Using
a continuity correction, find P(r ¸ 7) in N(5; 5=2):
r = 6 1
2
-5
p
2:5
= 1:5
1:581 = 0:949, n.s., so no evidence of difference.
\item The exact probability P(7)+P(8)+P(9)+P(10) in B(10; 1=2)  
\begin{eqnarray*} 
P(7)+P(8)+P(9)+P(9) &=& 
\left[ { 10 \choose 2}
\left( \frac{1}{2} \right) ^2 \left(\frac{1}{2}\right)^2 \right]+ 
\left[ { 10 \choose 3}
\left( \frac{1}{2} \right) ^3  \left(\frac{1}{2}\right)^1 \right]\\
& & +\left[ { 10 \choose 4}
\left( \frac{1}{2} \right) ^4  \left(\frac{1}{2}\right)^{0} \right]
+\left[ { 10 \choose 4}
\left( \frac{1}{2} \right) ^4  \left(\frac{1}{2}\right)^{0} \right]
\\  &=& 
+ \left( \frac{1}{2} \right) ^2  \left(\frac{1}{2}\right)^{3}
+ \left( \frac{1}{2} \right) ^3  \left(\frac{1}{2}\right)^{2}\\
& & + \left( \frac{1}{2} \right) ^4  \left(\frac{1}{2}\right)^{1}
+ \left( \frac{1}{2} \right) ^5  \left(\frac{1}{2}\right)^{0} 
\\ &=&  \frac{(120 + 45 + 10 + 1)}{1024}    \\
 &=& \frac{176}{1024} \\
 &=& 0.172.
\end{eqnarray*}
\item So the
probability of the given result in a 2-tail test (A. H. “there is a difference
between groups”, direction not specified) is 0.344. 
\item  Again no evidence of any
difference.
\end{itemize}
%%%%%%%%%%%%%%%%%%%%%%%
\item  The sum of the positive ranks is 49, and of negative 6. 
\item The value 6 is
(approximately) $N( \frac{n(n+1)}{4} ,\frac{1}{24}(n + 1)(n + 2)$), making no allowance for
the ties in the ranks (3 of -1 and 2 of +3). n = 10, the number of nonzero
differences, so $\frac{n(n+1)}{4}  = 27.5 $
 and $\frac{1}{24}(n + 1)(n + 2) = 96.25$. 
 \item Using a
continuity correction, \[r = \frac{6.5-27.5}{\sqrt{96.25} }= \frac{-21.0}{9.81} = -2.14.\]
\begin{itemize}
    \item At the 5\% level, there is significant evidence against the N. H. [Using the
Wilcoxon table, the critical number is 8, and 6, being less than this, is
significant at 5\%.]
\item This test uses the information on numerical sizes of differences, whereas the
sign test does not.
\item All the negative ones were very small.
\item If the differences had appeared to be normally distributed, a t-test (paired
version) would have been appropriate. 
\item This seems very unliablely, since there
is no clustering around a mean, and there are several large values.
\end{itemize}

\end{enumerate}
\end{document}

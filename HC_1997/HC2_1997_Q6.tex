\documentclass[a4paper,12pt]{article}
%%%%%%%%%%%%%%%%%%%%%%%%%%%%%%%%%%%%%%%%%%%%%%%%%%%%%%%%%%%%%%%%%%%%%%%%%%%%%%%%%%%%%%%%%%%%%%%%%%%%%%%%%%%%%%%%%%%%%%%%%%%%%%%%%%%%%%%%%%%%%%%%%%%%%%%%%%%%%%%%%%%%%%%%%%%%%%%%%%%%%%%%%%%%%%%%%%%%%%%%%%%%%%%%%%%%%%%%%%%%%%%%%%%%%%%%%%%%%%%%%%%%%%%%%%%%
\usepackage{eurosym}
\usepackage{vmargin}
\usepackage{amsmath}
\usepackage{graphics}
\usepackage{epsfig}
\usepackage{enumerate}
\usepackage{multicol}
\usepackage{subfigure}
\usepackage{fancyhdr}
\usepackage{listings}
\usepackage{framed}
\usepackage{graphicx}
\usepackage{amsmath}
\usepackage{chngpage}
%\usepackage{bigints}

\usepackage{vmargin}
% left top textwidth textheight headheight
% headsep footheight footskip
\setmargins{2.0cm}{2.5cm}{16 cm}{22cm}{0.5cm}{0cm}{1cm}{1cm}
\renewcommand{\baselinestretch}{1.3}

\setcounter{MaxMatrixCols}{10}
\begin{document}
%%%%%%%%%%%%%%%%%%%%%%%%%%%%%%%%%%%%%%%%%%%%%%%%%%%%%%%%%%%%%%%%%%%%%%%%%%%%%%%%%%%%%
\begin{table}[ht!]
     
\centering
     
\begin{tabular}{|p{15cm}|}
     
\hline 
A machine is supposed to produce cartons with a mean net weight of at least 400 grams and
a standard deviation not exceeding 8 grams. A simple random sample of the net weights
(grams) of 16 cartons is as follows:
\[\{408 394 403 383 402 383 395 392 404 394 382 402 395 402 389 410\}\]
Test whether the two production criteria are being met. Explain your conclusions in plain
English, including any assumptions on which your analysis depends. Determine whether
your conclusions would have been the same if the sample size had been four times larger,
with the same sample mean and standard deviation?
\\ \hline
      
\end{tabular}
    
\end{table}


%%%%%%%%%%%%%%%%%%%%%%%%%%%%%%%%%%%%%%%%%%%%%%%%%%%%
If we assume the data follow a normal distribution with mean $\mu$ and variance ¾2,
then we can use the data (16 observations) to test the Null Hypotheses that
(i) ¹ ¸ 400, (ii) ¾2 · 64 against the Alternatives ¹ < 400 and ¾2 > 64.
The mean of the sample is $\bar{x}$ = 396:125 and variance $s^2 = 77.9833$.


\begin{enumerate}
    \item For the mean, t(15) = p396:125¡400
77:9833=16
= ¡3:875
2:208 = ¡1:76, which is just on the
borderline of significance at 5\% in a 1-tail test. The evidence from these
data is that the mean is not likely to be ¸ 400.
\item For the variance, (n¡1)s2
¾2 » Â2(
n¡1) i.e. 15£77:9833
64 is Â2
(15) = 18:28, which is
less than the 5\% (upper) point of Â2
(15), so there is no evidence to reject the
hypothesis that ¾2 · 64, even though the observed value is above this.
\begin{itemize}
\item With a sample four times as large, i.e. $n = 64$, assuming the same estimates
$\bar{x}$ and $s^2$, t(63) would be
p
4, i.e. 2, times as large, providing very strong
evidence against the Null Hypothesis for ¹. 

The variance would be based
on 63 d.f., and the Â2 statistic would be 63£77:9833
64 = 76:76, which is not
significant and so there is still no evidence against the Null Hypothesis for
¾2.
\end{itemize}
\end{enumerate}
\end{document}

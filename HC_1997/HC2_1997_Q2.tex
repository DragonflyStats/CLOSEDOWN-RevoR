\documentclass[a4paper,12pt]{article}
%%%%%%%%%%%%%%%%%%%%%%%%%%%%%%%%%%%%%%%%%%%%%%%%%%%%%%%%%%%%%%%%%%%%%%%%%%%%%%%%%%%%%%%%%%%%%%%%%%%%%%%%%%%%%%%%%%%%%%%%%%%%%%%%%%%%%%%%%%%%%%%%%%%%%%%%%%%%%%%%%%%%%%%%%%%%%%%%%%%%%%%%%%%%%%%%%%%%%%%%%%%%%%%%%%%%%%%%%%%%%%%%%%%%%%%%%%%%%%%%%%%%%%%%%%%%
\usepackage{eurosym}
\usepackage{vmargin}
\usepackage{amsmath}
\usepackage{graphics}
\usepackage{epsfig}
\usepackage{enumerate}
\usepackage{multicol}
\usepackage{subfigure}
\usepackage{fancyhdr}
\usepackage{listings}
\usepackage{framed}
\usepackage{graphicx}
\usepackage{amsmath}
\usepackage{chngpage}
%\usepackage{bigints}

\usepackage{vmargin}
% left top textwidth textheight headheight
% headsep footheight footskip
\setmargins{2.0cm}{2.5cm}{16 cm}{22cm}{0.5cm}{0cm}{1cm}{1cm}
\renewcommand{\baselinestretch}{1.3}

\setcounter{MaxMatrixCols}{10}
\begin{document}
%%%%%%%%%%%%%%%%%%%%%%%%%%%%%%%%%%%%%%%%%%%%%%%%%%%%%%%%%%%%%%%%%%%%%%%%%%%%%%%%%%%%%
\begin{table}[ht!]
     
\centering
     
\begin{tabular}{|p{15cm}|}
     
\hline  
2. (a) State and explain the linear, fixed-effects additive model for one-way analysis of
variance.
(b) The time taken for a medicinal tablet to dissolve is important to the pharmaceutical
scientist. The data below show the effect of four different storage conditions on the
time it takes, in seconds, for the first 50% of each tablet in a sample to dissolve in
water.


\begin{center}
\begin{tabular}{|c|c|c|c|c|c|c|c|c|}
	&		&		&		&		&		&		&	Sum	&	 Sum of
squares	\\  \hline
Storage 1 	&	19	&	22	&	28	&	21	&	19	&	19	&	128	&	2792	\\  \hline
Storage 2 	&	21.5	&	20.5	&	18.4	&	19	&		&		&	79.4	&	1582.06	\\  \hline
Storage 3 	&	22	&	24.9	&	22.7	&	21.1	&		&		&	90.7	&	2064.51	\\  \hline
Storage 4	&	19	&	24	&	17.5	&		&		&		&	60.5	&	1243.25	\\  \hline
\end{tabular}
\end{center}
(Source. M.J. Crowder, Applied Statistics, No 3, 1996)

(i) Carry out a one-way analysis of variance of these data, state the assumptions you
have made and explain what you conclude as a result of your analysis.
(ii) How would your conclusions in this case be affected if you were now told that the
measurements in sample 1 had been rounded?
\\ \hline
      
\end{tabular}
    
\end{table}
%%%%%%%%%%%%%%%%%%%%%%%%%%%%%%%%%%%%%%%%%%%%%%%%%%%%%%%%%%%%%%%%%%%%%%%%%%%%%%%%%%%%%
. (a) $yij = ¹ + ¿i + \epsilon_{ij}$, where yij is the observation measured as the $j-$th of the
items receiving treatment i; ¹ is a grand (overall) mean term; ¿i is an effect
10
(deviation from mean) due to treatment i; $\epsilon_{ij}$ are i.i.d. $N(0; \sigma^2)$ residual
terms. There are i = 1 to v treatments, $r_i$ replicates of each, and
Xv
i=1
ri = N,
the total number of items in the experiment.
(b)
\begin{center}
\begin{tabular}{|c|c|c|c|c|}
“Treatment” & $r_i$ & $\sum 
y_{ij}$ & $\sum 
y^2_{ij}$ & $\bar{y}_i$\\ \hline

1	&	6	&	128	&	2792	&	21.33	\\ \hline
2	&	4	&	79.4	&	1582.06	&	19.85	\\ \hline
3	&	4	&	90.7	&	2064.51	&	22.68	\\ \hline
4	&	3	&	60.5	&	1243.25	&	20.17	\\ \hline
	&	17	&	358.6	&	7681.82	&		\\ \hline
\end{tabular}
\end{center}

Although results 1 are rounded to whole numbers, analysis of variance will have to assume that all observations on all treatments have the same variance $\sigma^2$. Also the material used in the trial should have been selected at
random from what was available, and the samples examined under identical conditions in random order.
(i) 
\[\mbox{Total corrected S.S} = 7681.82-\frac{G^2}{N} = 7681.82- 7564.35 = 117.47\]
\[\mbox{ “Treatments” S.S.} = \frac{1282}{6} + \frac{79.4^2 + 90.7^2}{4} \frac{60.5^2}{3}-\frac{G^2}{N}
= 7583.4625 - 7564.35 = 19.11.\]

Analysis of Variance 
\begin{center}
\begin{tabular}{|c|c|c|c|c|}\hline
 & D.F.&  S.S.&  M.S. &  F \\\hline
Treatments &  3&  19.11 & 6.371 &  $< 1$\\
(Storages) & &&&\\ \hline
Residual & 13 & 98.36&  7.566& ( = $\hat{\sigma}^2$)\\ \hline
TOTAL & 16 & 117.47 & & \\ \hline
\end{tabular}
\end{center}


We are not given any specific contrasts among storages to be tested, but
even if the whole Treatments S.S. were due to one contrast this would still
not be significant as F(1;13) ( 19.11
7.566 = 2.52, less than the 5\% point 4.67). We
may say confidently, that there are no significant differences among these
“Treatments”.
(ii) Given the result in (i), there could be no change to the inference. [In a
borderline case, some intelligence in looking at individual differences may be
called for, as $\sigma^2$ may be slightly overestimated.]
\end{document}

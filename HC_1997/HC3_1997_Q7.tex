\documentclass[a4paper,12pt]{article}
%%%%%%%%%%%%%%%%%%%%%%%%%%%%%%%%%%%%%%%%%%%%%%%%%%%%%%%%%%%%%%%%%%%%%%%%%%%%%%%%%%%%%%%%%%%%%%%%%%%%%%%%%%%%%%%%%%%%%%%%%%%%%%%%%%%%%%%%%%%%%%%%%%%%%%%%%%%%%%%%%%%%%%%%%%%%%%%%%%%%%%%%%%%%%%%%%%%%%%%%%%%%%%%%%%%%%%%%%%%%%%%%%%%%%%%%%%%%%%%%%%%%%%%%%%%%
\usepackage{eurosym}
\usepackage{vmargin}
\usepackage{amsmath}
\usepackage{graphics}
\usepackage{epsfig}
\usepackage{enumerate}
\usepackage{multicol}
\usepackage{subfigure}
\usepackage{fancyhdr}
\usepackage{listings}
\usepackage{framed}
\usepackage{graphicx}
\usepackage{amsmath}
\usepackage{chngpage}
%\usepackage{bigints}

\usepackage{vmargin}
% left top textwidth textheight headheight
% headsep footheight footskip
\setmargins{2.0cm}{2.5cm}{16 cm}{22cm}{0.5cm}{0cm}{1cm}{1cm}
\renewcommand{\baselinestretch}{1.3}

\setcounter{MaxMatrixCols}{10}
\begin{document}
\begin{table}[ht!]
 \centering
 \begin{tabular}{|p{15cm}|}
 \hline  
The table below gives the failure times in hours of two makes of equipment. The data were gathered in an attempt to answer the question of whether  the two types differ in average failure time.
(i) Draw box and whisker plots of the two sets of observations and describe briefly what they indicate about the shape of the underlying distributions from which the data have been sampled.
\\ \hline
  \end{tabular}
\end{table}


\begin{enumerate}
\item For type A, min = 171; lower quartile, q = 396:5; median, M = 1
2(568+795) =
681:5; upper quartile, Q = 1158; max = 2415.
For B, min = 212; q = 298:5; M = 447:5; Q = 823:5; max = 1678.

\begin{itemize}
\item The two distributions are distinctly skew, since the medians are not in the
middle of the boxes made by the quartiles, and also the upper whiskers are
very long. 
\item The variability in the distributions appears not to be the same
either.
\end{itemize}


\newpage
\begin{table}[ht!]
 \centering
 \begin{tabular}{|p{15cm}|}
 \hline 
(ii) When testing whether two populations differ in location one may use a parametric test such as the  t-test or a non-parametric test such as the Mann-Whitney test. Carry out an appropriate test to answer the question of whether the two types of equipment differ in average failure time and explain your choice of test. Describe precisely the null hypothesis that is being tested by the test you choose.
Type A Type B
171, 257, 288, 295, 396, 212, 236, 262, 272, 286,
397, 431, 435, 554, 568, 311, 336, 340, 412, 446,
795, 902, 958, 1004, 1104, 449, 670, 686, 786, 811,
1212, 1283, 1378, 1621, 2415 836, 936, 978, 1335, 1678 sum = 16464 sum = 12278 sum of squares = 19648218 sum of squares = 10544940
\\ \hline
  \end{tabular}
\end{table}
\item The t-test requires symmetry (strictly normality) of sets of data and, at least
approximately, the same variance. Since neither of these seems very likely
in the populations from which the samples were drawn, a Mann - Whitney
test is preferred. This requires data to be of similar shape, but that is more
24
reasonable. The Null Hypothesis will be that the populations have the same
median values. The ranks of Type A are: \[\{1, 4, 8, 9, 13, 14, 16, 17, 20, 21,
25, 28, 30, 32, 33, 34, 35, 37, 38, 40\}\]; and of type B: \[\{2, 3, 5, 6, 7, 10, 11, 12,
15, 18, 19, 22, 23, 24, 26, 27, 29, 31, 36, 39.\}\]

\begin{itemize}
\item Rank sums are: A, 455; B, 365. [check: sum = 820 = 1
2 ¢ 40 ¢ 41]. 
\item The mean
of all ranks is 410, and the normal approximation to rank sum has variance
1
12 ¢ 20 ¢ 20 ¢ 41, i.e. s.d.= 36:97.
(This form of the test is usually called Wilcoxan’s Rank Sum test.)
\item Hence r = 455¡410
36:97 = 1:22 is approximately N(0; 1); the value is not significant,
so there is no evidence that medians differ.
\end{itemize}
\end{enumerate}
\end{document}

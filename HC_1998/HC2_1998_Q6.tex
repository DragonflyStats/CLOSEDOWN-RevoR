\documentclass[a4paper,12pt]{article}
%%%%%%%%%%%%%%%%%%%%%%%%%%%%%%%%%%%%%%%%%%%%%%%%%%%%%%%%%%%%%%%%%%%%%%%%%%%%%%%%%%%%%%%%%%%%%%%%%%%%%%%%%%%%%%%%%%%%%%%%%%%%%%%%%%%%%%%%%%%%%%%%%%%%%%%%%%%%%%%%%%%%%%%%%%%%%%%%%%%%%%%%%%%%%%%%%%%%%%%%%%%%%%%%%%%%%%%%%%%%%%%%%%%%%%%%%%%%%%%%%%%%%%%%%%%%
\usepackage{eurosym}
\usepackage{vmargin}
\usepackage{amsmath}
\usepackage{graphics}
\usepackage{epsfig}
\usepackage{enumerate}
\usepackage{multicol}
\usepackage{subfigure}
\usepackage{fancyhdr}
\usepackage{listings}
\usepackage{framed}
\usepackage{graphicx}
\usepackage{amsmath}
\usepackage{chngpage}
%\usepackage{bigints}

\usepackage{vmargin}
% left top textwidth textheight headheight
% headsep footheight footskip
\setmargins{2.0cm}{2.5cm}{16 cm}{22cm}{0.5cm}{0cm}{1cm}{1cm}
\renewcommand{\baselinestretch}{1.3}

\setcounter{MaxMatrixCols}{10}
\begin{document}

%%%%%%%%%%%%%%%%%%%%%%%%%%%%%%%%%%%%%%%%%%%%%%%%%%%%%%%%%%%%%%%%%%%%%%%%%%%%%%%%%%%%%%%%%%%%%%%%%%%%%%%%%%%%%%%%%%%%%
\begin{table}[ht!]
 
\centering
 
\begin{tabular}{|p{15cm}|}
 
\hline  
\large
%% 6. (a)
Explain the use and importance of the central limit theorem in statistical inference.

\\ \hline
  
\end{tabular}

\end{table}


%%%%%%%%%%%%%%%%%%%%%%%%%%%%%%%%%%%%%%%%%%%%%%%%%%%%%%%%%%%%%%%%%%%%%%%%%%%%%%%%%%%%%%%%%%%%%%%%%%%%%%%%%%%%%%%%%%%%%

\begin{itemize}
   
\item When a large sample of data is available from any population, not necessarily normal, and
including discrete data as well as continuous, the sample mean or total follows a distribution that

is approximately normal.

Therefore with large samples of data significance tests of, and confidence
intervals for, a population mean
may be found, at least approximately, without knowing what
distribution the population has.


This extends, for example to proportions in a binomial.



   
\item In practice, when distributions are reasonably symmetrical, even when not normal, samples
may be as small as about 30, while if data are very skew then very large
samples-several hundred may
be required to give acceptable results.


\item An examination of data, possibly by graphical methods,
is a useful guide when applying the approximation.


\item We may treat $\bar{x}$ as $N(\mu ; \sigma^2=n)$ when $n$ is sample size and ¹; $\sigma^2$ are the (known or estimated)
mean and variance of the population distribution.


\item The only theoretical restriction is that ¹ and $\sigma^2$ must be finite.

\end{itemize}

%%%%%%%%%%%%%%%%%%%%%%%%%%%%%%%%%%%%%%%%%%%%%%%%%%%%%

\newpage


\begin{table}[ht!]
 
\centering
 
\begin{tabular}{|p{15cm}|}
 
\hline  
\large
A study was conducted into the number and type of police emergency calls during standard 8 hour shifts in two districts of a large city.  Random samples of shifts were selected from the police records for each district and the following descriptive statistics obtained.
District 1 District 2 Total number of shifts 125 108 Mean number of calls 3.75 2.10 Standard deviation of numbers of calls 2.74 1.40 Number of shifts with a major emergency 26 15

\begin{enumerate}[(i)]
\item (i) Construct a 95\% confidence interval for the difference in the mean number of emergency calls per shift between the two districts and interpret your findings.
\item (ii) Test whether the proportion of shifts with major emergencies differs between the two districts.
\end{enumerate}
\\ \hline
  
\end{tabular}

\end{table}
\large
\begin{itemize}
\item Many estimates of practically important items are the sum of several independent components,
e.g. crop yields of many plants forming a plot, and so their total tends to be normally distributed,
$N(n¹; n\sigma^2)$.


\item For difference in means, a 95\% confidence interval is approximately
(x¯1  -  x¯2) § 1:96
p
s21
=n1 + s22
=n2.
For the given data, this is (3:75  -  2:10) § 1:96
q
2:742
125 + 1:402
108 , i.e. 1:65 § 1:96 £ 0:2797
8
giving 1:65 § 0:55 or (1.10 to 2.20).
\item Since this interval does not contain zero, it is likely that there will be more calls each shift in
district 1 than in 2, the mean difference being between 1.1 and 2.2(with probability 0.95).
\item For difference in proportions,q pˆ1 - pˆ2
p1(1 - p1)
n1
+p2(1 - p2)
n2
» N(0; 1).
The NH is ”p1 = p2”.
Hence 26
125  -  15
108 = 0:208  -  0:139 = 0:069 is the estimated difference.
\item Its variance is 0:208£0:792
125 + 0:139£0:861
108 = 2:426 £ 10 - 3; S.E.=0.049.
Hence 0:069
0:049 = 1:40 n.s. as N(0,1)and there is no significant evidence of a difference in proportions.

\end{itemize}
\end{document}

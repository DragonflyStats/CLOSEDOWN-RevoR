\documentclass[a4paper,12pt]{article}
%%%%%%%%%%%%%%%%%%%%%%%%%%%%%%%%%%%%%%%%%%%%%%%%%%%%%%%%%%%%%%%%%%%%%%%%%%%%%%%%%%%%%%%%%%%%%%%%%%%%%%%%%%%%%%%%%%%%%%%%%%%%%%%%%%%%%%%%%%%%%%%%%%%%%%%%%%%%%%%%%%%%%%%%%%%%%%%%%%%%%%%%%%%%%%%%%%%%%%%%%%%%%%%%%%%%%%%%%%%%%%%%%%%%%%%%%%%%%%%%%%%%%%%%%%%%
\usepackage{eurosym}
\usepackage{vmargin}
\usepackage{amsmath}
\usepackage{graphics}
\usepackage{epsfig}
\usepackage{enumerate}
\usepackage{multicol}
\usepackage{subfigure}
\usepackage{fancyhdr}
\usepackage{listings}
\usepackage{framed}
\usepackage{graphicx}
\usepackage{amsmath}
\usepackage{chngpage}
%\usepackage{bigints}

\usepackage{vmargin}
% left top textwidth textheight headheight
% headsep footheight footskip
\setmargins{2.0cm}{2.5cm}{16 cm}{22cm}{0.5cm}{0cm}{1cm}{1cm}
\renewcommand{\baselinestretch}{1.3}

\setcounter{MaxMatrixCols}{10}
\begin{document}

%%%%%%%%%%%%%%%%%%%%%%%%%%%%%%%%%%%%%%%%%%%%%%%%%%%%%%%%%%%%%%%%%%%%%%%%%%%%%%%%%%%%%%%%%%%%%%%%%%%%%%%%%%%%%%%%%%%%%
\begin{table}[ht!]
 
\centering
 
\begin{tabular}{|p{15cm}|}
 
\hline  

7. One of the tasks undertaken in a particular laboratory is the measurement of the nitrogen content of various chemical preparations.  
It has been established that, when experienced workers repeatedly test the same preparation, the standard deviation of their measurements is 0.025 g%.
Two new laboratory workers A and B are given initial training.  They are then each required to make ten repeat 
analyses of a test preparation, which is known to have an exact nitrogen content of 1.81 g%.  The results (in g%) are as follows:

Worker A 1.73 1.75 1.80 1.83 1.79 1.88 1.85 1.79 1.78 1.80
Worker B 1.85 1.86 1.80 1.83 1.87 1.85 1.90 1.84 1.84 1.86
For each worker is there evidence that
(i) his/her experimental technique is more variable than the general laboratory standard?
(ii) his/her results are biased?
Using these results what comments would you make to the manager of the laboratory about the accuracy of measurements made by each worker?
\\ \hline
  
\end{tabular}

\end{table}


%%%%%%%%%%%%%%%%%%%%%%%%%%%%%%%%%%%%%%%%%%%%%%%%%%%%%%%%%%%%%%%%%%%%%%%%%%%%%%%%%%%%%%%%%%%%%%%%%%%%%%%%%%%%%%%%%%%%%
%%%%%%%%%%%%%%%%%%%%%%%%%%%%%%%%%%%%%%%%%%%%%%%%%%%%%%%%%%%%%%%%%%%%%%%%%%%%%%%%%%%%
\begin{enumerate}[(a)]
\item
.Given ¹ = 1:81; ¾2 = (0:025)2. n=10.
(i)For A, ¯x=1.80 and s2=0.001977.
For B, ¯x=1.85 and s2=0.000689.
(n¡1)s2
A
¾2 = 9£0:001977
0:0252 = 28:47¤¤¤ » Â2
(9), giving very strong evidence to reject an NH that A’s
variability is the same as the laboratory standard, and to accept an AH that it is greater.
(n¡1)s2
B
¾2 = 9£0:000689
0:0252 = 9:92; n:s: as Â2
(9), so there is no statistical evidence that B’s variability
is unacceptable.
\item For A,p x¯¡1:81
0:001977=10
= ¡0:01
0:014 = ¡0:71 n:s: as t(9).
No evidence that A’s results are biased.
For B,p x¯¡1:81
0:000689=10
= 0:04
0:0083 = 4:82¤¤¤ as t(9).
\begin{itemize}
    \item B’s results do seem to be biased, because this value of t(9) leads us to reject the N.H. “mean=1.81”
\item Hence worker A produces results which are unbiased but very variable, while B is biased but precise.
\end{itemize}

\end{enumerate}
\end{document}

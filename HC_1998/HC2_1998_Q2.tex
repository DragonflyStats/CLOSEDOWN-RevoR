\documentclass[a4paper,12pt]{article}
%%%%%%%%%%%%%%%%%%%%%%%%%%%%%%%%%%%%%%%%%%%%%%%%%%%%%%%%%%%%%%%%%%%%%%%%%%%%%%%%%%%%%%%%%%%%%%%%%%%%%%%%%%%%%%%%%%%%%%%%%%%%%%%%%%%%%%%%%%%%%%%%%%%%%%%%%%%%%%%%%%%%%%%%%%%%%%%%%%%%%%%%%%%%%%%%%%%%%%%%%%%%%%%%%%%%%%%%%%%%%%%%%%%%%%%%%%%%%%%%%%%%%%%%%%%%
\usepackage{eurosym}
\usepackage{vmargin}
\usepackage{amsmath}
\usepackage{graphics}
\usepackage{epsfig}
\usepackage{enumerate}
\usepackage{multicol}
\usepackage{subfigure}
\usepackage{fancyhdr}
\usepackage{listings}
\usepackage{framed}
\usepackage{graphicx}
\usepackage{amsmath}
\usepackage{chngpage}
%\usepackage{bigints}

\usepackage{vmargin}
% left top textwidth textheight headheight
% headsep footheight footskip
\setmargins{2.0cm}{2.5cm}{16 cm}{22cm}{0.5cm}{0cm}{1cm}{1cm}
\renewcommand{\baselinestretch}{1.3}

\setcounter{MaxMatrixCols}{10}
\begin{document}
%%%%%%%%%%%%%%%%%%%%%%%%%%%%%%%%%%%%%%%%%%%%%%%%%%%%%%%%%%%%%%%%%%%%%%%%%%%%%%%%%%%%%%%%%%%%%%%%%%%%%%%%%%%%%%%%%%%%%
\begin{table}[ht!]
 
\centering
 
\begin{tabular}{|p{15cm}|}
 
\hline  

2. The times in minutes for a random sample of 70 factory workers to complete a standard task were summarised as follows:
\begin{center}
\begin{tabular}{|r|c|}
      Time in minutes & Number of workers \\
$<$ 10  & 0 \\
$\geq$ 10 but $<$11 &  3\\ 
$\geq$ 11 but $<$12 &  7 \\
$\geq$ 12 but $<$15 & 33 \\
$\geq$ 15 but $<$18 & 18 \\
$\geq$ 18 but $<$20  & 9 \\
$\geq$20 &  0\\
\end{tabular}
\end{center}
(i) Construct a histogram of these data and find approximate summary statistics to describe the data.  What do the data and your statistics reveal about the distribution of the number of minutes it takes to complete this task?

\\ \hline
  
\end{tabular}

\end{table}
\begin{table}[ht!]
 
\centering
 
\begin{tabular}{|p{15cm}|}
 
\hline  

(ii) Construct a 95\% confidence interval for the mean number of minutes to complete the task and state any assumptions which you make.

\\ \hline
  
\end{tabular}

\end{table}

%%%%%%%%%%%%%%%%%%%%%%%%%%%%%%%%%%%%%%%%%%%%%%%%%%%%%%%%%%%%%%%%%%%%%%%%%%%%%%%%%%%%%%%%%%%%%%%%%%%%%%%%%%%%%%%%%%%%%
\begin{enumerate}[(a)]


\item Histogram of time taken to complete a standard task.
6
Mid-point of
time interval Frequency Cumulative
(mins:)(t) (f) ft ft2 frequency
10:5 3 31:5 330:75 3
11:5 7 80:5 925:75 10
13:5 33 445:5 6014:25 43
16:5 18 297:0 4900:50 61
19:0 9 171:0 3249:00 70
70 1025:5 15420:25
Mean=1025:5
70 = 14:65.
Median=11:95 + 25
33 £ 3 = 14:22.
(assuming records to nearest 0.05min).
Variance=(15420:25 ¡ 1025:52=70) ¥ 69 = 5:7489. SD=2.40 .
%%%%%%%%%%%%%%%%%%%%%%%%%%%
\begin{itemize}
    \item With few intervals, the histogram alone is not very informative, but the mean and median are
roughly the same, and near to the middle of the range of the data. \item Therefore we may treat the
data as approximately normal, and certainly as sufficiently symmetrical to use large-sample tests.
\end{itemize}
%%%%%%%%%%%%%%%%
\item An approximate 95\% confidence interval is 14:65 § 1:96
q
5:7489
70 i.e. 14:65 § 0:56, which is
(14.09 to 15.21).
\end{enumerate}
\end{document}

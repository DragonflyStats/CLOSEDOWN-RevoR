\documentclass[a4paper,12pt]{article}
%%%%%%%%%%%%%%%%%%%%%%%%%%%%%%%%%%%%%%%%%%%%%%%%%%%%%%%%%%%%%%%%%%%%%%%%%%%%%%%%%%%%%%%%%%%%%%%%%%%%%%%%%%%%%%%%%%%%%%%%%%%%%%%%%%%%%%%%%%%%%%%%%%%%%%%%%%%%%%%%%%%%%%%%%%%%%%%%%%%%%%%%%%%%%%%%%%%%%%%%%%%%%%%%%%%%%%%%%%%%%%%%%%%%%%%%%%%%%%%%%%%%%%%%%%%%
\usepackage{eurosym}
\usepackage{vmargin}
\usepackage{amsmath}
\usepackage{graphics}
\usepackage{epsfig}
\usepackage{enumerate}
\usepackage{multicol}
\usepackage{subfigure}
\usepackage{fancyhdr}
\usepackage{listings}
\usepackage{framed}
\usepackage{graphicx}
\usepackage{amsmath}
\usepackage{chngpage}
%\usepackage{bigints}

\usepackage{vmargin}
% left top textwidth textheight headheight
% headsep footheight footskip
\setmargins{2.0cm}{2.5cm}{16 cm}{22cm}{0.5cm}{0cm}{1cm}{1cm}
\renewcommand{\baselinestretch}{1.3}

\setcounter{MaxMatrixCols}{10}
\begin{document}

%%%%%%%%%%%%%%%%%%%%%%%%%%%%%%%%%%%%%%%%%%%%%%%%%%%%%%%%%%%%%%%%%%%%%%%%%%%%%%%%%%%%%%%%%%%%%%%%%%%%%%%%%%%%%%%%%%%%%
\begin{table}[ht!]
 
\centering
 
\begin{tabular}{|p{15cm}|}
 
\hline  

4. (a) A pharmaceutical company needs to determine whether a new drug alters blood pressure.  
Twelve male volunteers had their diastolic blood pressure measured, in appropriate units before and after receiving the drug, with the following results:

\begin{center}
\begin{tabular}{c|c|c|c|c|c|c|c|c|c|c|c|c|}
Patient & 1&  2&  3 & 4 & 5& 6& 7& 8& 9& 10& 11& 12\\
Before&  120 & 124 & 129 & 118 & 141& 128& 140& 133& 126& 130& 136& 127 \\
After & 125&  126& 137& 117& 144 &128 &146& 131& 127 &135& 136& 131\\

\end{tabular}
\end{center}
 Carry out a suitable analysis of these data stating any assumptions which you make.

\\ \hline
  
\end{tabular}

\end{table}




%%%%%%%%%%%%%%%%%%%%%%%%%%%%%%%%%%%%%%%%%%%%%%%%%%%%%%%%%%%%%%%%%%%%%%%%%%%%%%%%%%%%%%%%%%%%%%%%%%%%%%%%%%%%%%%%%%%%%
\begin{enumerate}
\item Because the measurements are taken on the same volunteers, the paired t-test is appropriate.
Differences (B-A) are: -5, -2, -8, 1, -3, 0, -6, 2, -1, -5, 0, -4.
\begin{itemize}
\item Assuming that these are normally distributed, the N.H. that the mean difference is 0
uses t(11) = ¯ d¡0
s=
p
12
.
\item The observed mean difference is ¯ d = ¡31
12 = ¡2:583. s2 = (3:088)2.
\item Hence t(11) = ¡ 2:583
3:088=3:464 = ¡2:898¤.

    \item Reject the N.H. There is evidence of a change in blood pressure.
\item The estimated mean increase is 2.583 units. 
\item A 95\% confidence interval for this is 2:583§2:201£
3:088=3:464 = 2:583 § 1:962 or (0.62 to 4.55) units.
\end{itemize}
\newpage
\begin{table}[ht!]
 
\centering
 
\begin{tabular}{|p{15cm}|}
 
\hline  

(b) In a trial of anti-inflammatory drugs in the treatment of arthritis, 200 arthritic patients were allocated at random to receive one of two treatments.  After one month each patient was asked to state whether their arthritis had improved.  The results were as follows:
\begin{center}
\begin{tabular}{|c|c|c|}
 & Improved  &          Not improved \\ \hline
Treatment A   &   45 &  55 \\ \hline
Treatment B &     63 &  37\\ \hline
\end{tabular}
\end{center}
Apply a chi-squared test to these data and explain your results.
\\ \hline
  
\end{tabular}

\end{table}
\item On the Null Hypothesis of no difference in improvement under the two treatments, expected
numbers are calculated:
\begin{center}
\begin{tabular}{|c|c|c|c|}
OBS(EXP) & Improved & Not Improved&  TOTAL\\ \hline
A & 45(54) & 55(46) & 100\\ \hline
B & 63(54) & 37(46) & 100\\ \hline
Total & 108 & 92 & 200\\ \hline
\end{tabular}
\end{center}

Â2
(1) =
P (O¡E)2
E = 92( 2
54 + 2
46 ) = 162( 1
54 + 1
46 ) = 6:52¤.
(Â2
(1) = 5:80 if Yates’ correction is used: not essential).
We have evidence to reject the Null Hypothesis at the 5\% significance level. This is an indication
of treatment difference.\end{enumerate}
\end{document}

\documentclass[a4paper,12pt]{article}
%%%%%%%%%%%%%%%%%%%%%%%%%%%%%%%%%%%%%%%%%%%%%%%%%%%%%%%%%%%%%%%%%%%%%%%%%%%%%%%%%%%%%%%%%%%%%%%%%%%%%%%%%%%%%%%%%%%%%%%%%%%%%%%%%%%%%%%%%%%%%%%%%%%%%%%%%%%%%%%%%%%%%%%%%%%%%%%%%%%%%%%%%%%%%%%%%%%%%%%%%%%%%%%%%%%%%%%%%%%%%%%%%%%%%%%%%%%%%%%%%%%%%%%%%%%%
\usepackage{eurosym}
\usepackage{vmargin}
\usepackage{amsmath}
\usepackage{graphics}
\usepackage{epsfig}
\usepackage{enumerate}
\usepackage{multicol}
\usepackage{subfigure}
\usepackage{fancyhdr}
\usepackage{listings}
\usepackage{framed}
\usepackage{graphicx}
\usepackage{amsmath}
\usepackage{chngpage}
%\usepackage{bigints}

\usepackage{vmargin}
% left top textwidth textheight headheight
% headsep footheight footskip
\setmargins{2.0cm}{2.5cm}{16 cm}{22cm}{0.5cm}{0cm}{1cm}{1cm}
\renewcommand{\baselinestretch}{1.3}

\setcounter{MaxMatrixCols}{10}
\begin{document}
\begin{table}[ht!]
 \centering
 \begin{tabular}{|p{15cm}|}
 \hline  
3. (a) (i) Explain what is meant by the non-response problem in a sample survey, giving examples of where it might arise.
(ii) What steps may be taken to reduce the level of non-response?
(b) A survey organisation defines the "true level of business confidence" for a particular sector of economic activity as the proportion of managing directors of all companies in that sector who expect prospects for their company to improve in the next six months.
In the Light Engineering sector the managing directors of 67 out of a random sample of 125 companies stated that they expected prospects for their company to improve in the next six months. For a random sample of 200 companies in the Banking and Financial Services sector the corresponding figure was 126 managing directors reporting that they expected prospects for their company to improve over the same period.
(i) Do these results provide evidence of a difference between true levels of business confidence (proportions expecting an improvement) in the two sectors?

\\ 
\hline
 \end{tabular}
\end{table}
%%%%%%%%%%%%%%%%%%%%%%%%%%%%%%%%%%%%%%%%%%%%%%%%%%%%%%%%%5
\begin{table}[ht!]
 \centering
 \begin{tabular}{|p{15cm}|}
 \hline  
(ii) Calculate a 95 percent confidence interval for the difference between the true levels of business confidence in the two sectors.
(iii) A business analyst wants to calculate a 95\% confidence interval for the level of business confidence in the Light Engineering sector. If the true level is 0.6, what sample size would be required to produce a 95\% confidence interval which has a width of 0.08?\\ \hline
  \end{tabular}
\end{table}
%%%%%%%%%%%%%%%%%%%%%%%%%%%%%%%%%%%%%%%%%%%%%%%%%%%%%%%%%%%%5

3.(a)(i)In a sample survey, people are not compelled to respond and will only do so if the topic of
the enquiry interests them, if they think it is important, and if there is nothing in the approach
or the questionnaire that annoys them or puts them off. Questions of a private or sensitive native
will not be answered. In a postal questionnaire, or if they come early in an interview, this will
usually result in total non-response; at best be some missing data.
There is always some non-response in a postal questionnaire survey because people simply do
not complete and mail it back. Interviews of a selected random sample of people cannot always be
carried out 100% because some individuals refuse or are not available for interview.
(ii)Since non-response is often concentrated in certain parts of the target/study population, it is
important to minimize it to avoid serious bias.
Care over wording of questions, fore-testing of them, reminders is a mail survey, repeated visits
for interviews, keeping a questionnaire or interview as short as possible, even offering a reward or
prize to those who do reply, can sometimes help reduce non-response.
If people cannot be interviewed because they have moved, then can be replaced by ‘reserve’
randomly selected people. But unless there is a good reason replacements should not be made as
they can lead to bias.
(b)(i)pˆE = 67
125 = 0:536. pˆB = 126
200 = 0:630. For comparing these, with the null hypothesis
“ture ¼E=true ¼B”, a 2£2 table is :
OBSERV ED Improve Not EXPECTED
Eng: 67 58 : 125 74:23 50:77
Bankg: 126 74 : 200 118:77 81:23
193 132 325
12
Â2
(1) =
P
all cells of table
(O¡E)2
E = (7:23)2( 1
74:23 + 1
50:77 + 1
118:77 + 1
81:23 )
= 52:2729 £ 0:0539
= 2:82 n:s:
This gives no evidence of difference between the population values of the proportion.
(ii)pˆB ¡ pˆE = 0:094. pE(1¡pE)
nE
= 0:536£0:464
125 = 0:0019896
pB(1¡pB)
nB
= 0:63£0:37
200 = 0:0011655. Variance of difference=0.0031551.
0:094 § 2
p
0:0031551 = 0:094 § 0:112 or -0.018 to 0.206.
With probability 0.95, ¼B ¡ ¼E lies between -0.018 and +0.206.
(iii)When ¼ = 0:6; 2
q
0:6£0:4
n is required to be 0.04.
Hence 4 £ 0:24
n = (0:04)2 or n = 4£0:24
(0:04)2 = 600.
4.(i)y = ct
d+t or yd + yt = ct or d + t = c!.
This can be written as ! = ® + ¯t; where ® = d
c ; ¯ = 1
c .
(ii)
(iii)The fitted line is ! ¡ ! = ˆ ¯(t ¡ ¯t) where
ˆ ¯ =
P
P(!¡!¯)(t¡t¯)
(t¡¯t)2 =
P
!t¡
P
!
P
P t=n
t2¡(
P
t)2=n
= 1190:46022¡156£76:39073=12
2600¡1562=12
= 197:38073


572 = 0:34507
.
13
Hence ˆ® = 76:39073
12 ¡ ˆ ¯ £ 13 = 1:87997.
(iv)ˆc = 1= ˆ ¯ = 0:898 and ˆ d = ˆc ˆ® = 5:448.
%%%%%%%%%%%%%%%%%%%%%%%%%%%%%%%%%%%%%%%%%%%%%%%%%%%%5
\newpage

(v)Using y = 2:898t
5:448+t , when t=16, gives y=2.16.
(vi)The parameters ˆc and ˆ d are non-linear functions of ˆ® and ˆ ¯ so there are no simple formula
for the relations between standard errors.
\end{enumerate}
\end{document}

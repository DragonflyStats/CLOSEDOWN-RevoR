\documentclass{article}
\usepackage[utf8]{inputenc}
\usepackage{enumerate}

\author{kobriendublin }
\date{December 2018}

\begin{document}

%- Higher Certificate, Module 5, 2008. Question 1
\section{Introduction}
\begin{enumerate}[(i)]
\itemQuestion 2
Probability mass function: ()()1,0,1,2,...,01xfxppxp=−=< .
(i)
x
5
4
3
2
1
0
4/27
2/9
1/3
P(X = x)
(ii) Probability generating function G(s) is
()()(){}()001111xxXxxxpGsEssppppsps∞∞==⎡⎤==−=−=⎣⎦−−ΣΣ
(requires (
)1/1 for convergence).
The mean is given by E[X] = G'(1). We have ()(){}2111pGspps−′=−− and inserting s = 1 gives ()21(1)ppGp−′=, i.e. the mean is 1pp−.
The variance is given by Var(X) = G''(1) + mean – mean2. ()()(){}232111ppGsps−′′=−−, so that ()()22211pGp−′′=. Thus the variance is given by ()2222111pppppp−⎛⎞−−+−⎜⎟⎝⎠ ()(){}2221111pppppp−=−+−=.
The solution to part (iii) is on the next page
(iii) Let U = Y + Z. For U to take value r (r = 0, 1, 2, 3, …), we need
Y = 0 and Z = r
or Y = 1 and Z = r – 1
or Y = 2 and Z = r – 2
…
or Y = r – 1 and Z = 1
or Y = r and Z = 0.
Let pi = P(Y = i) and πi = P(Z = i). Then, from the above,
0112211()...rrrrrPUrppppp ππππ−−−==+++++ .
Now,
()01210121......YrYrGEspspspspsps−−==++++++
  and
()01210121......ZrZrGEssssssπππππ−−==++++++ .
Thus the coefficient of sr in GYGZ is 0112211...rrrrrpppppππππ−−− +++++ which equals P(U = r) as required.
Alternatively, argue that because Y and Z are independent we have
()()()YZYZYZYZGEsEsEsGG++===.
Hence, for independent random variables X1 and X2 having the given geometric distribution, we immediately have
()()122221(1)XXXpGGps+==−− .
\end{enumerate}
\end{document}
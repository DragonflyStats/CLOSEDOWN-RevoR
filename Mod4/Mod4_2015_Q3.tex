\documentclass{article}
\usepackage[utf8]{inputenc}
\usepackage{enumerate}

\author{kobriendublin }
\date{December 2018}

\begin{document}

%- Higher Certificate, Module 5, 2008. Question 1
\section{Introduction}
\begin{enumerate}[(i)]
\item Question 3 (i) Sample product moment correlation coecient
rxy =
p Sxy
SxxSyy
; [1]
(or equivalent expression) where Sxy =
P
i(xi �� x)(yi �� y), [1]
Sxx =
P
i(xi �� x)2, Syy =
P
i(yi �� y)2. [1]
\item  zi = axi + b, a > 0
rzy =
p Szy
SzzSyy
[1]
where
Szy =
X
i
(ax + b �� (ax + b))(yi �� y) [1]
= a
X
i
(xi �� x)(yi �� y)
= aSxy; [1]
Szz =
X
i
(axi + b �� (ax + b))2
= a2
X
i
(xi �� x)2 [1]
4
Syy stays the same.
Hence
rzy =
p Szy
SzzSyy
=
q aSxy
a2SxxSyy
= rxy for a > 0 [1]
and for a < 0, rzy = ��rxy. [1]
(iii)
Sxy =
X
i
xiyi ��
(
P
i xi)(
P
i yi)
n
= 19862:6��
1190:0  248:5
15
= 140:333 [1]
Sxx =
X
i
x2i
��
(
P
i xi)2
n
= 95098 ��
1190:02
15
= 691:337 [1]
Syy =
X
i
y2
i ��
(
P
i yi)2
n
= 4161:1 ��
248:52
15
= 40:9693 [1]
rxy =
p Sxy
SxxSyy
=
140:333
p
691:337  40:9693
= 0:834: [1]
H0 : XY = 0 vs H1 : XY > 0
1% point 0.5923 [1]
At 1% signicance level there is strong evidence against H0, which
shows that there is a strong positive correlation between number of
chirps and temperature. [1]
\item  Spearman rank correlation coecient
rS = 1 ��
6
P
d2i
n(n2 �� 1)
[1]
5
y Rank y x Rank x d
14.4 1 76.3 6 -5
14.7 2 69.7 2 0
15.0 3 79.6 7 -4
15.4 4 69.4 1 3
15.5 5 75.2 5 0
15.7 6 71.5 3 3
16.0 7 71.6 4 3
16.1 8 80.5 8 0
16.3 9 83.3 11 -2
17.0 10 83.5 12 -2
17.1 11 80.6 9 2
17.2 12 82.6 10 2
18.4 13 84.3 13 0
19.8 14 93.3 15 -1
20.0 15 88.6 14 1
Correct ranks [1]
(The correct follow through from incorrect ranks get full marks.)
Correct di's and
P
d2i
= 86. [1]
rS = 1 ��
6  86
15(152 �� 1)
= 1 �� 0:1536 = 0:8464 [1]
1% point 0.6036)Strong evidence against H0, which shows that there
is a strong association between the number of chirps and temperature. [1]
\end{enumerate}

\end{document}

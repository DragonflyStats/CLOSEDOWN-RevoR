\documentclass{article}
\usepackage[utf8]{inputenc}
\usepackage{enumerate}

\author{kobriendublin }
\date{December 2018}

\begin{document}

%- Higher Certificate, Module 4, 2015. Question 4
\section{Introduction}
\begin{enumerate}[(i)]
\item
Question 4 (i) A multiple regression model for p explanatory variables
x1; x2; : : : ; xp by
Yi = 0 + 1x1i + 2x2i + : : : + pxp;i + "i:
Marks to be given with or without the subscript i.
Correct coecients 0; 1; : : : ; p of the explanatory variables [1]
Correct use of response and error term in model above. [1]
where Yi is the response variable.
This model has p + 1 unknown parameters 0; 1; : : : ; p. "i are inde-
pendent and normally distributed or uncorrelated [0.5+0.5]
errors have mean zero and constant variance 2. [0.5+0.5]
(i) Test H0 : 1 = 2 = : : : = p = 0 (i.e. all coecients except 0 are
zero) versus H1 : at least one of the coecients is non-zero. [1]
Consider the statistic F? dened by
F? =
regression MS
s2 =
128435
661
= 194:30
6
[1]
F?  F4
30 under H0
[1]
F4
30(0:05) = 2:69 [1]
F? is very large ) very strong evidence against H0. [1]
So 96.3\% of the variation is explained by this model, so it suggests
that the model ts well. [1]

\item H0 : Head = 0 vs H1 : Head 6= 0 [1]
t =
��9:013
4:693
= ��1:92  t30 under H0 [1]
t30(0:025) = 2:042 At[1t]he 5% signicance level no evidence against
H0 )[1e]xclude the variable Head from the model. [1]
90% condence interval for Chest
^ Chest  t30(0:05)se( ^ Chest) [1]
6:255  1:697  1:677
[1]
(3:409; 9:101)
[1]
%%%%%%%%%%%%%%%%%%%%%%%%%%%%%%%%%
\item The regression equation is
^y = ��203 + 0:655Age �� 9:01Head + 11:8Neck + 6:26Chest
^y = ��203 + 0:655  71 �� 9:01  7 + 11:8  27 + 6:26  44 [1]
^y = 374:475 [1]
If you use the coecients given to the number of dp in the question,
this is 375.1545. Please accept anything that rounds to 374 or 375.
7
%%%%%%%%%%%%%%%
\end{enumerate}

\end{document}
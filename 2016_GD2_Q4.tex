4. (a) The continuous random variables X1 and X2 jointly have a bivariate Normal
distribution. X1 has expected value 50 and standard deviation 8. X2 has
expected value 45 and standard deviation 10. The correlation between X1 and
X2 is –0.25.
(i) Obtain the expectation and covariance matrix of the random vector
X = (X1, X2)
T.
(3)
(ii) In different physical units, the same quantities can be recorded as
Y1 = 0.555(X1 – 32) and Y2 = 0.447X2.
Obtain the distribution of the random vector Y = (Y1, Y2)T.
(5)
(iii) Use this example to illustrate one reason why correlation is often
preferred to covariance as a measure of association between two
random variables.
(2)
(b) In a longitudinal study of child growth, measurements X1, X2 and X3 are made
of a baby's length at ages 1 month, 2 months and 3 months respectively. These
random variables are modelled by a multivariate Normal distribution with
2 2 22
2 22
22 2 2
( ) , Cov( ) ,
2
E
   
    
    
                
X X
where  2 > 0 and 0 <  < 1.
(i) Write down the correlations between all possible pairs of these
measurements.
(2)
(ii) The random variable Y is the mean length of an individual child at these
three ages. Specify the distribution of Y.
(5)
(iii) Measurements are to be made, independently, on a sample of n
children. The random vector of measurements on the ith child (i = 1, 2,
…, n) is Xi. State the distribution of the sample mean vector
1
1 n
i
i n 
X X   , giving its expectation vector and covariance matrix.
(3) 

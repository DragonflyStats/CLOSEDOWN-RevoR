\documentclass[a4paper,12pt]{article}
%%%%%%%%%%%%%%%%%%%%%%%%%%%%%%%%%%%%%%%%%%%%%%%%%%%%%%%%%%%%%%%%%%%%%%%%%%%%%%%%%%%%%%%%%%%%%%%%%%%%%%%%%%%%%%%%%%%%%%%%%%%%%%%%%%%%%%%%%%%%%%%%%%%%%%%%%%%%%%%%%%%%%%%%%%%%%%%%%%%%%%%%%%%%%%%%%%%%%%%%%%%%%%%%%%%%%%%%%%%%%%%%%%%%%%%%%%%%%%%%%%%%%%%%%%%%
\usepackage{eurosym}
\usepackage{vmargin}
\usepackage{amsmath}
\usepackage{graphics}
\usepackage{epsfig}
\usepackage{enumerate}
\usepackage{multicol}
\usepackage{subfigure}
\usepackage{fancyhdr}
\usepackage{listings}
\usepackage{framed}
\usepackage{multirow}
% \usepackage{graphicx}
\usepackage{graphicx}
\usepackage{chngpage}
%\usepackage{bigints}

\usepackage{vmargin}
% left top textwidth textheight headheight
% headsep footheight footskip
\setmargins{2.0cm}{2.5cm}{16 cm}{22cm}{0.5cm}{0cm}{1cm}{1cm}
\renewcommand{\baselinestretch}{1.3}

\setcounter{MaxMatrixCols}{10}

%- Higher Certif icate in Statistics
%- May 1999
%- SOLUTIONS Paper I :
%- Statistical Theory
\begin{document}
	\large 
\noindent The game of Snip is played for marbles.  In this game two players, A and B say, each have a coin which, unseen by the other player, they independently choose to display as Heads (H) or Tails (T).  When A and B have made their choices, the coins are disclosed and the outcome of the game is decided in accordance with the table below.
	
	\begin{center}
		\begin{tabular}{|ll|l|l|}\hline
			\multicolumn{2}{|l}{}   & \multicolumn{2}{l|}{B displays} \\ \hline
			\multicolumn{2}{|l|}{}   & H      & T      \\ 
			\multirow{2}{*}{A displays} & H & A wins 3 from B & B wins 2 from A \\ 
			& T & B wins 2 from A & A wins 1 from B \\ \hline
		\end{tabular}
	\end{center}
	
	
	\begin{table}[ht!]
		\centering
		\begin{tabular}{|p{15cm}|}
			\hline  \large 
			\noindent \textbf{Part (a)} \\ \large Suppose that A and B independently both randomly display H and T with probability 1/2.  Show that all four outcome combinations in the table above have the same probability, and deduce that the expected gain to A (or B) is zero. 
			\\ \hline
		\end{tabular}
	\end{table}
	%%%%%%%%%%%%%%%%%%%%%%%%%%%%%%%%%%%%%%%%%%%%%%%%%%
	% Please add the following required packages to your document preamble:
	
	\large 
	\noindent The outcome probabilities are:
	
	$P(H|A) = 1/2$, $P(H|B) = 1/2$ and so $P(T|A)$ = $P(T|B) = 1/2$ .
	\begin{itemize}
		
		\item The expected gain for player $X$ is the 
		\[ E(G_{X}) = \sum W_{ij} P_{ij} \]
		where $W_{ij}$ denoted the winnings from a particular outcome.
		
		\item Probabilities of each pair of outcomes
		
		
		\begin{center}
			\begin{tabular}{|ll|l|l|}\hline
				\multicolumn{2}{|l}{}   & \multicolumn{2}{l|}{B displays} \\ \hline
				\multicolumn{2}{|l|}{}   & H      & T      \\ \hline 
				\multirow{2}{*}{A displays} & H  &  $\frac{1}{2} \times \frac{1}{2}  \;=\; \frac{1}{4}$ & $\frac{1}{2} \times \frac{1}{2}  \;=\; \frac{1}{4}$ \\ 
				&  T &  $\frac{1}{2} \times \frac{1}{2}  \;=\; \frac{1}{4}$ & $\frac{1}{2} \times \frac{1}{2}  \;=\; \frac{1}{4}$ \\ \hline \end{tabular}
		\end{center}
		
		\item Every entry in the body of the table is the product of marginal probabilities (by
		independence),$1/2 \times 1/2 = 1/4$.
		
		\item Writing $G_A$ = Gain for A and $G_B$ = Gain for B ,
		\item $E[G_A] = (3 \times 1/4) - (2 \times 1/4) - (2 \times 1/4) + (1 \times 1/4) \;=\; 0 $
		\item Since this is a two person game $E[G_B] = 0$
	\end{itemize} 
	%%%%%%%%%%%%%%%%%%%%%%%%%%%%%%%%%%%%%%%%%%%%%%%%%%%%%%%%%%%%%%5
	\newpage
	
	\begin{table}[ht!]
		\centering
		\begin{tabular}{|p{15cm}|}
			\hline  
			\noindent \textbf{Part (b)} \\ \large 
			Suppose now that A and B independently both randomly display H and T, but A displays H and T with respective probabilities 2/3 and 1/3 while B displays H and T with respective probabilities 1/3 and 2/3. \\ \smallskip
			\large 
			Calculate the four outcome probabilities under these conditions and find the expected gain to A. 
			\\ \hline 
		\end{tabular}
	\end{table}
	
	\begin{itemize}
		\item Probabilities of each pair of outcomes
		
		
		\begin{center}
			\begin{tabular}{|ll|l|l|}\hline
				\multicolumn{2}{|l}{}   & \multicolumn{2}{l|}{B displays} \\ \hline
				\multicolumn{2}{|l|}{}   & H      & T      \\ \hline 
				\multirow{2}{*}{A displays} & H  &  $\frac{2}{3} \times \frac{1}{3}  \;=\; \frac{2}{9}$ & $\frac{2}{3} \times \frac{2}{3}  \;=\; \frac{4}{9}$ \\ 
				&  T &  $\frac{1}{3} \times \frac{1}{3}  \;=\; \frac{1}{9}$ & $\frac{1}{3} \times \frac{2}{3}  \;=\; \frac{2}{9}$ \\ \hline \end{tabular}
		\end{center}
		
		\item $E[G_A] = (3 \times 2/9) - (2 \times 4/9) - (2 \times 1/9) + (1 \times 2/9) = -2/9$
	\end{itemize}
	
	
	%%%%%%%%%%%%%%%%%%%%
	\newpage
	\begin{table}[ht!]
		\centering
		\begin{tabular}{|p{15cm}|}
			\hline  
			\noindent \textbf{Part (c)} \\ \large 
			Finally, suppose that A and B independently and randomly display H and T with respective probabilities $p_A$ and $1- p_A$;  $p_B$ and $1- p_B$.  Calculate the four outcome probabilities and show that the expected gain to A may be written: 
			
			\[1 \;-\; 3p_A \;-\; 3p_B \;+\; 8p_Ap_B\]
			\large 
			Noting that this may be written as \[1\;-\; p_A(3\;-\;8p_B)\;-\; 3p_B\] or as \[1\;-\; p_B(3 \;-\;8p_A)\;-\; 3p_A,\] discuss whether you would prefer to play as A or B and how you would play. \\ \hline
		\end{tabular}
	\end{table}
	\large
	%-------------------------------------%
	\begin{itemize}
		\item Following from part (b)
		\begin{eqnarray*}
			E[G_A] &=& 3\left[p_Ap_B\right] - 2\left[p_A(1 - p_B)\right] - 2\left[(1 - p_A)p_B\right] + 1\left[(1 - p_A)(1 - p_B)\right]\\ & & \\ &=& 1 - 3p_A -
			3p_B + 8p_Ap_B\\
		\end{eqnarray*}
		\item If $p_B = 3/8$ then $E[G_A] \;=\; 1 - 3p_B$ from the first alternative form. 
		\item In this case \[E[G_A] \;=\; 1 - \frac{9}{8}  = - \frac{1}{8}\]
		\item Hence choose to play as B, with $p_B = 3/8$. 
		\item In the long run will guarantee a win of $1/8$
		(since $E[G_B] = -E[G_A]$)
	\end{itemize}

\end{document}

\documentclass[a4paper,12pt]{article}
%%%%%%%%%%%%%%%%%%%%%%%%%%%%%%%%%%%%%%%%%%%%%%%%%%%%%%%%%%%%%%%%%%%%%%%%%%%%%%%%%%%%%%%%%%%%%%%%%%%%%%%%%%%%%%%%%%%%%%%%%%%%%%%%%%%%%%%%%%%%%%%%%%%%%%%%%%%%%%%%%%%%%%%%%%%%%%%%%%%%%%%%%%%%%%%%%%%%%%%%%%%%%%%%%%%%%%%%%%%%%%%%%%%%%%%%%%%%%%%%%%%%%%%%%%%%
\usepackage{eurosym}
\usepackage{vmargin}
\usepackage{amsmath}
\usepackage{graphics}
\usepackage{epsfig}
\usepackage{enumerate}
\usepackage{multicol}
\usepackage{subfigure}
\usepackage{fancyhdr}
\usepackage{listings}
\usepackage{framed}
\usepackage{graphicx}
\usepackage{amsmath}
\usepackage{chngpage}
%\usepackage{bigints}

\usepackage{vmargin}
% left top textwidth textheight headheight
% headsep footheight footskip
\setmargins{2.0cm}{2.5cm}{16 cm}{22cm}{0.5cm}{0cm}{1cm}{1cm}
\renewcommand{\baselinestretch}{1.3}

\setcounter{MaxMatrixCols}{10}
\begin{document}

\begin{itemize}
\item  Since we have a table of frequencies in various categories, an appropriate Null
Hypothesis is that the ratio Good:Fair:Poor is the same in each area. A Â2
(8)
test is suitable. “Expected” frequencies are calculated from margin totals as
usual, e.g. Ruthven / Good is 680£3689
5842 = 429:39.
OBS(EXP) Good Fair Poor
Area R 459(429:39) 178(210:56) 43(40:04) : 680
M 926(969:29) 506(475:32) 103(90:39) : 1535
W 954(930:77) 442(456:43) 78(86:79) : 1474
D 985(995:82) 507(488:32) 85(92:86) : 1577
A 365(363:72) 176(178:36) 35(33:92) : 576
3689 1809 344 5842
23
(Rounding expected frequencies is to the nearest 0.01).
Â2
(8) =
P (0¡E)2
\[E = 2:04185 + 1:93340 + 0:57977 + 0:11756 + 0:00450 + 5:03492
+1:98027 + 0:45620 + 0:71458 + 0:03123 + 0:21882 + 1:75918
+0:89024 + 0:66530 + 0:03439 = 16:46¤:\]

This indicates that there are departures from a constant ratio in some areas.
Comparing observed and corresponding expected frequencies shows that
\begin{itemize}
\item Ruthven has more ‘Good’ and less ‘Fair’ than expected; 
\item Mossmont has less ‘Good’ and more ‘Fair’ or ‘Poor’; 
\item Windgyle has more ‘Good’ and less ‘Fair’ or ‘Poor’; 
\item Dundonan has more ‘Fair’ and less ‘Good’ or ‘Poor’.
\end{itemize}
\item  Since respondents rate their own health this is very subjective and unlikely
to produce the same ratings for the same condition in different people or
areas. Also we obtain relatively little information per person and so require
a large number of observations.
\begin{itemize}
\item Actual measurements on a smaller number of people could provide data
on blood pressure, cholesterol, weight and many other objective ways of
assessing health, as well as observing the presence or absence of infections,
and the general environmental conditions such as air quality.
\end{itemize}
\end{enumerate}
\end{document}

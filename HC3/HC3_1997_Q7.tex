7. (i) For type A, min = 171; lower quartile, q = 396:5; median, M = 1
2(568+795) =
681:5; upper quartile, Q = 1158; max = 2415.
For B, min = 212; q = 298:5; M = 447:5; Q = 823:5; max = 1678.
The two distributions are distinctly skew, since the medians are not in the
middle of the boxes made by the quartiles, and also the upper whiskers are
very long. The variability in the distributions appears not to be the same
either.
(ii) The t-test requires symmetry (strictly normality) of sets of data and, at least
approximately, the same variance. Since neither of these seems very likely
in the populations from which the samples were drawn, a Mann - Whitney
test is preferred. This requires data to be of similar shape, but that is more
24
reasonable. The Null Hypothesis will be that the populations have the same
median values. The ranks of Type A are: 1, 4, 8, 9, 13, 14, 16, 17, 20, 21,
25, 28, 30, 32, 33, 34, 35, 37, 38, 40; and of type B: 2, 3, 5, 6, 7, 10, 11, 12,
15, 18, 19, 22, 23, 24, 26, 27, 29, 31, 36, 39.
Rank sums are: A, 455; B, 365. [check: sum = 820 = 1
2 ¢ 40 ¢ 41]. The mean
of all ranks is 410, and the normal approximation to rank sum has variance
1
12 ¢ 20 ¢ 20 ¢ 41, i.e. s.d.= 36:97.
(This form of the test is usually called Wilcoxan’s Rank Sum test.)
Hence r = 455¡410
36:97 = 1:22 is approximately N(0; 1); the value is not significant,
so there is no evidence that medians differ.
8. A dot - plot for each set of data, on the same scale, is useful.
(i) 923 for C seems highly unlikely. This level may be physically impossible, to
judge from all the other observations. Or it may be a recording error for 293
(or even 329).
(ii) Residual S.S. = 19 £ 19797 = 376143; hence treatment S.S. = 70262 and
M.S. = 35131. The variance ratio is then 35131
19797 = 1:77.
The analysis, by itself, suggests that there are no significant treatment differences,
and also that the standard deviation of an observation is very large
(
p
19797 = 140:7).
(iii) Revised sums and S.S. are:
A B C TOTAL
Sum 2533 2308 1097 5938
n 8 9 4 21
Sum of squares 826145 602898 305129 1734172
Treatments SS = 25332
8 + 23082
9 + 10972
4 ¡ 59382
21 = 1694737¡1679040 = 15697.
25
and Total SS = 1734172 ¡ 59382=21 = 55132.
Source of variation DF Sum of Squares M:S:
Treatments 2 15697 7849 F(2;18) = 3:58¤
Residual 18 39435 2191
TOTAL 20 55132
F is just significant at 5%. The estimated variance of an observation is 2191,
S.D. = 46.8. Means are: A, 316.6; B, 256.4; C, 274.3. Var[¯xA ¡ ¯xB] =
s2( 1
8 + 1
9 ) = 517:32, S.D. = 22.7, t(18) = 60:2
22:7 = 2:65¤, so A and B appear
to differ. Var[¯xA ¡ ¯xC] = s2( 1
8 + 1
4 ) = 821:63, S.D. = 28.7, t(18) = 42:3
28:7 , not
significant. A and C do not differ; nor will B and C.
(iv) The one doubtful observation greatly increased the variance estimate.
26

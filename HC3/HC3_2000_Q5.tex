\documentclass[a4paper,12pt]{article}
%%%%%%%%%%%%%%%%%%%%%%%%%%%%%%%%%%%%%%%%%%%%%%%%%%%%%%%%%%%%%%%%%%%%%%%%%%%%%%%%%%%%%%%%%%%%%%%%%%%%%%%%%%%%%%%%%%%%%%%%%%%%%%%%%%%%%%%%%%%%%%%%%%%%%%%%%%%%%%%%%%%%%%%%%%%%%%%%%%%%%%%%%%%%%%%%%%%%%%%%%%%%%%%%%%%%%%%%%%%%%%%%%%%%%%%%%%%%%%%%%%%%%%%%%%%%
\usepackage{eurosym}
\usepackage{vmargin}
\usepackage{amsmath}
\usepackage{graphics}
\usepackage{epsfig}
\usepackage{enumerate}
\usepackage{multicol}
\usepackage{subfigure}
\usepackage{fancyhdr}
\usepackage{listings}
\usepackage{framed}
\usepackage{graphicx}
\usepackage{amsmath}
\usepackage{chngpage}
%\usepackage{bigints}

\usepackage{vmargin}
% left top textwidth textheight headheight
% headsep footheight footskip
\setmargins{2.0cm}{2.5cm}{16 cm}{22cm}{0.5cm}{0cm}{1cm}{1cm}
\renewcommand{\baselinestretch}{1.3}

\setcounter{MaxMatrixCols}{10}
\begin{document}
\begin{enumerate}Mean =
1
120
(0 + 30 + 64 + 60 + 52 + 45 = 30 + 14 + 8) =
303
120
= 2:525
P(r) = e¡2:525(2:525)r=r! for r = 0; 1; 2 ¢ ¢ ¢
Expected frequencies are 120P(r).
¿ = 0 1 2 3 4 5 6 ¸ 7
Ei = 9:607 24:258 30:625 25:776 16:271 8:217 3:458 1:788
7 is 1.247; 8 is 0.394;¸ 9 is 0.147.
oi = 8 30 32 20 13 9 8
.
(ii)
x25
=
1:6072
9:607
+
(¡5:742)2
24:258
+
(¡1:375)2
30:625
+
5:7762
25:776
+
3:2712
16:271
+
(¡0:783)2
8:217
+
(¡2:754)2
6:246
= 5:162 n:s:
20
So a Null Hypothesis that the poisson model holds is not rejected.
(iii) Assuming that the Poisson model is valid we use 2.525 as the variance, so
2:525 § 1:96
q
2:525
120 is an approximate 95% confidence interval for the true mean.
(a) This is 2:525 § 1:96 £ 0:145 = 2:525 § 0:284 or (2.24 to 2.81).
Note. it is often specified that this approximation require a mean of at least 5 to be
satisfactory.
(b)P(¸ 1) = 1 ¡ P(0) = 1 ¡ 0:080 = 0:92. This is an estimate of the proportion
of non-zero minutes, and variance is 0:92£0:08
120 which is 0.0006133, whose square root is
0.0248.
An approximate 95% interval for the true proportion is 0:92§1:96£0:0248 or 0:92§0:049,
i.e. 0.87 to 0.97.
\end{enumerate}
\end{document}

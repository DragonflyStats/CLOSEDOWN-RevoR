\documentclass[a4paper,12pt]{article}
%%%%%%%%%%%%%%%%%%%%%%%%%%%%%%%%%%%%%%%%%%%%%%%%%%%%%%%%%%%%%%%%%%%%%%%%%%%%%%%%%%%%%%%%%%%%%%%%%%%%%%%%%%%%%%%%%%%%%%%%%%%%%%%%%%%%%%%%%%%%%%%%%%%%%%%%%%%%%%%%%%%%%%%%%%%%%%%%%%%%%%%%%%%%%%%%%%%%%%%%%%%%%%%%%%%%%%%%%%%%%%%%%%%%%%%%%%%%%%%%%%%%%%%%%%%%
\usepackage{eurosym}
\usepackage{vmargin}
\usepackage{amsmath}
\usepackage{graphics}
\usepackage{epsfig}
\usepackage{enumerate}
\usepackage{multicol}
\usepackage{subfigure}
\usepackage{fancyhdr}
\usepackage{listings}
\usepackage{framed}
\usepackage{graphicx}
\usepackage{amsmath}
\usepackage{chngpage}
%\usepackage{bigints}

\usepackage{vmargin}
% left top textwidth textheight headheight
% headsep footheight footskip
\setmargins{2.0cm}{2.5cm}{16 cm}{22cm}{0.5cm}{0cm}{1cm}{1cm}
\renewcommand{\baselinestretch}{1.3}

\setcounter{MaxMatrixCols}{10}
\begin{document}
\begin{framed}
1. The quarterly profits, in thousands of dollars, of a small but growing company are shown in the table below. The table also includes the centred 4 point moving average of the time series and the differences ( profit − centred moving average ).
Profit ($'000's)
Moving average
Difference
1992       Q1 45.5 Q2 59.3 Q3 82.8 67.900 Q4 69.4 75.163 -5.7625 1993       Q1 74.7 82.125 Q2 88.2 87.762 0.4375 Q3 109.6 92.887 16.7125 Q4 87.7 98.325 -10.6250 1994       Q1 97.4 102.637 -5.2375 Q2 109.0 Q3 123.3 115.250 8.0500 Q4 118.9 121.938 -3.0375 1995       Q1 122.2 129.587 -7.3875 Q2 137.7 136.400 1.3000 Q3 155.8 Q4 140.9 147.700 -6.8000 1996       Q1 144.3 152.475 -8.1750 Q2 161.9 156.913 4.9875 Q3 169.8 Q4 162.4
(i) Complete the calculation of the moving averages and the differences.
(ii) Plot the data together with the moving average.
(iii) Estimate the seasonal effects assuming that an additive model is appropriate for describing the data.
(iv) Explain how the deseasonalized values are calculated, but do not calculate them.

(v) A straight line was fitted to the deseasonalized data and the result was
profit = 50 + 6 t  
where t  is the number of quarters from the start of the series, so that, for example, the value of t for  1993 Q2 is 6. Use this information concerning the fitted line and the seasonal effects to produce profit projections for 1997. What assumptions does this require?

\end{framed}
\begin{enumerate}
    \item Missing entries are (¤ indicates cannot be calculated):
Moving Average ¤, ¤, 108.250, 141.913, ¤, ¤.
Difference ¤, ¤, 14.90000, -7.4250, 0.7500, 13.8875, ¤, ¤.
\item  see next sheet for graph.
\item  Seasonal effects:
\begin{verbatim}
    Quarter 1 2 3 4
¡7:4250 0:4375 14:9000 ¡5:7625
¡5:2375 0:7500 16:7125 ¡10:6250
¡7:3875 1:3000 8:0500 ¡3:0375
¡8:1750 4:9875 13:8875 ¡6:8000
MEAN ¡7:05625 1:86875 13:3875 ¡6:55625 : 1:64375
Correction ¡0:41094 ¡0:41094 ¡0:41094 ¡0:41094 (¼ 0)
SEASONAL ¡7:4672 1:4578 12:9766 ¡6:9672
\end{verbatim}

\item  Since the data are given to 1 decimal place, 7.5 should be added to each Q1
item, 7.0 to each Q4 item, 1.5 subtracted from each Q2 item and 13.0 from
each Q3 item, to “deseasonalise”.
\item 
1997 : Q1 Q2 Q3 Q4
50 + 6t : 176 182 188 194
Seasonalised: 168:5 183:5 201:0 187:0
This assumes same general trend and seasonal effects continue.
\end{enumerate}
\end{document}

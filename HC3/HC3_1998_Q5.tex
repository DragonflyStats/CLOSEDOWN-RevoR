\documentclass[a4paper,12pt]{article}
%%%%%%%%%%%%%%%%%%%%%%%%%%%%%%%%%%%%%%%%%%%%%%%%%%%%%%%%%%%%%%%%%%%%%%%%%%%%%%%%%%%%%%%%%%%%%%%%%%%%%%%%%%%%%%%%%%%%%%%%%%%%%%%%%%%%%%%%%%%%%%%%%%%%%%%%%%%%%%%%%%%%%%%%%%%%%%%%%%%%%%%%%%%%%%%%%%%%%%%%%%%%%%%%%%%%%%%%%%%%%%%%%%%%%%%%%%%%%%%%%%%%%%%%%%%%
\usepackage{eurosym}
\usepackage{vmargin}
\usepackage{amsmath}
\usepackage{graphics}
\usepackage{epsfig}
\usepackage{enumerate}
\usepackage{multicol}
\usepackage{subfigure}
\usepackage{fancyhdr}
\usepackage{listings}
\usepackage{framed}
\usepackage{graphicx}
\usepackage{amsmath}
\usepackage{chngpage}
%\usepackage{bigints}

\usepackage{vmargin}
% left top textwidth textheight headheight
% headsep footheight footskip
\setmargins{2.0cm}{2.5cm}{16 cm}{22cm}{0.5cm}{0cm}{1cm}{1cm}
\renewcommand{\baselinestretch}{1.3}

\setcounter{MaxMatrixCols}{10}
\begin{document}\begin{table}[ht!]
 \centering
 \begin{tabular}{|p{15cm}|}
 \hline  
5. (a) The January prices and volumes, that is the actual number of shares traded in thousands of shares, for four companies are shown for 1995 and 1997 in the table below.
A B C D Price Volume Price Volume Price Volume Price Volume Jan 1995 £2.98 229.7 £4.40 484.3 £3.85 137.8 £11.41 2721.5 Jan 1997 £2.45 167.3 £6.43 777.3 £2.66 165.2 £15.15 3193.7
(i) Calculate a current weighted Price index (Paasche index) for January 1997 using the volumes as weights.
(ii) Express the January 1997 share price of company D as a percentage of its January 1995 price.
(iii) Explain why it is not merely a coincidence that the values obtained in (i) and (ii) are so close.
(b) The value of a particular share (in £ sterling) at close of trading on each of 10 consecutive trading days is shown in the table below.
Day Share value Day Share value 1 3.85 6 3.43 2 3.56 7 3.28 3 3.65 8 3.43 4 3.54 9 3.35 5 3.71 10 3.45
(i) Plot these data.
(ii) Use the method of exponential smoothing with a smoothing parameter of 0.4 to provide one step ahead predictions for each day after day 1 and plot the predictions on the same graph.
(iii) Explain briefly, without doing any further calculations, what would be the effect on the predictions of using a smoothing parameter of 0.9. 
\\ \hline
  \end{tabular}
\end{table}(a)Writing p for price, q for volume, o for January 1995 and l for January 1997, the Paosehe
Price index for 1997 is
(i)
P
Pp1q1
p0q1
= 2:45£167:3+6:43£777:3+2:66£165:2+15:15£3193:7
2:98£167:3+4:40£777:3+3:85£165:2+11:41£3193:7
= 54231:911
40994:811 = 1:3229
.
The index is thus 132.29.
(ii) 15:15
11:41 =1.3278, i.e. 132.78%.
The weight(q) for D is so large that the price change in D dominates the index.
(b)(i) (ii)
Share values and predictions by exponential smoothing
(ii)Exponential smoothing relates forecasts F at successive times t and actual figures as:
Ft = ®xt¡1 + (1 ¡ ®)Ft¡1
.
Calculations for ® = 0:4 are on the next page.
14
(iii)If ® = 0:9, predictions will track actual prices(x) much more closely.
or ® = 0:4, Ft = 0:4 £ xt¡1 + 0:6Ft¡1, for t=2,¢ ¢ ¢,11.
Day 1 2 3 4 5 6 7 8 9 10 11
x 3:85 3:56 3:65 3:54 3:71 3:43 3:28 3:43 3:35 3:45 ¡
F (3:85) 3:85 3:73 3:70 3:64 3:67 3:57 3:45 3:44 3:41 3:42

\end{enumerate}
\end{document}

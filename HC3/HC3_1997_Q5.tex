5. (i) We need to find out whether there are systematic trends along the rows, and
/ or whether one row is likely to do better than the other.
We also want to know whether all the grow-bags came from the same source,
contain the same compost mixture, are the same size, have equally good
drainage, the same thickness of wall so that temperature is likely to be the
same.
Reasons for blocking would be: difference between rows, trend along rows,
different sorts of bag.
(ii) The experimental unit is a bag of 4 plants. We would analyse the total (or
mean) yield of plants per bag. If any plants died, we would need to adjust
for this, so it should be recorded.
22
(iii) If there is no known or suspected systematic variation revealed in the answers
to (i), a completely randomized design may be used, with a fully random
choice of 16 bags for each of the four nutrient solutions. This could be
achieved by using a random number table, reading digits in pairs, discarding
pairs 00, 65 - 99, taking the first 16 positions for treatment A, the next 16
for B, the next 16 for C and the others for D, 01 - 64 represent the two rows
with 32 bags in each.
If the answers to (i) indicate likely differences in the positions, make up 16
blocks each of which is as homogeneous as possible. Number the bags 1,2,3,4
in each block and permute these numbers at random to determine the order
in which the 4 nutrients will be allocated to bags.
(iv) For the completely randomized design, the analysis is:
Source of Variation D.F.
Nutrients 3
Residual 60
TOTAL 63
Using blocks of 4 in a randomized complete block design gives:
Source of Variation D.F.
Blocks 15
Nutrients 3
Residual 45
TOTAL 63
6. Since we have a table of frequencies in various categories, an appropriate Null
Hypothesis is that the ratio Good:Fair:Poor is the same in each area. A Â2
(8)
test is suitable. “Expected” frequencies are calculated from margin totals as
usual, e.g. Ruthven / Good is 680£3689
5842 = 429:39.
OBS(EXP) Good Fair Poor
Area R 459(429:39) 178(210:56) 43(40:04) : 680
M 926(969:29) 506(475:32) 103(90:39) : 1535
W 954(930:77) 442(456:43) 78(86:79) : 1474
D 985(995:82) 507(488:32) 85(92:86) : 1577
A 365(363:72) 176(178:36) 35(33:92) : 576
3689 1809 344 5842
23
(Rounding expected frequencies is to the nearest 0.01).
Â2
(8) =
P (0¡E)2
E = 2:04185 + 1:93340 + 0:57977 + 0:11756 + 0:00450 + 5:03492
+1:98027 + 0:45620 + 0:71458 + 0:03123 + 0:21882 + 1:75918
+0:89024 + 0:66530 + 0:03439 = 16:46¤:
This indicates that there are departures from a constant ratio in some areas.
Comparing observed and corresponding expected frequencies shows that
Ruthven has more ‘Good’ and less ‘Fair’ than expected; Mossmont has less
‘Good’ and more ‘Fair’ or ‘Poor’; Windgyle has more ‘Good’ and less ‘Fair’
or ‘Poor’; Dundonan has more ‘Fair’ and less ‘Good’ or ‘Poor’.
(ii) Since respondents rate their own health this is very subjective and unlikely
to produce the same ratings for the same condition in different people or
areas. Also we obtain relatively little information per person and so require
a large number of observations.
Actual measurements on a smaller number of people could provide data
on blood pressure, cholesterol, weight and many other objective ways of
assessing health, as well as observing the presence or absence of infections,
and the general environmental conditions such as air quality.

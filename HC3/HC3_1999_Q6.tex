\documentclass[a4paper,12pt]{article}
%%%%%%%%%%%%%%%%%%%%%%%%%%%%%%%%%%%%%%%%%%%%%%%%%%%%%%%%%%%%%%%%%%%%%%%%%%%%%%%%%%%%%%%%%%%%%%%%%%%%%%%%%%%%%%%%%%%%%%%%%%%%%%%%%%%%%%%%%%%%%%%%%%%%%%%%%%%%%%%%%%%%%%%%%%%%%%%%%%%%%%%%%%%%%%%%%%%%%%%%%%%%%%%%%%%%%%%%%%%%%%%%%%%%%%%%%%%%%%%%%%%%%%%%%%%%
\usepackage{eurosym}
\usepackage{vmargin}
\usepackage{amsmath}
\usepackage{graphics}
\usepackage{epsfig}
\usepackage{enumerate}
\usepackage{multicol}
\usepackage{subfigure}
\usepackage{fancyhdr}
\usepackage{listings}
\usepackage{framed}
\usepackage{graphicx}
\usepackage{amsmath}
\usepackage{chngpage}
%\usepackage{bigints}

\usepackage{vmargin}
% left top textwidth textheight headheight
% headsep footheight footskip
\setmargins{2.0cm}{2.5cm}{16 cm}{22cm}{0.5cm}{0cm}{1cm}{1cm}
\renewcommand{\baselinestretch}{1.3}

\setcounter{MaxMatrixCols}{10}

\begin{document}\begin{table}[ht!]
 \centering
 \begin{tabular}{|p{15cm}|}
 \hline  
Question Text 3 
\\ \hline
  \end{tabular}
\end{table} A two-sample t-test would not be appropriate because of the correclation.Paired comparisons
can be examined for systolic and for diastolic by examining the two sets of differences,
that are given. Tn each case a one-trail test sets of difference,against a N.H. that the
true difference= 284
15 = 18:93
Estimated variance =
1
14
(6518 ¡
2842
15
) = 81:495
t(14) =
18:93 ¡ 0
p
81:495=15
= 18:93=2:33 = 8:12
\begin{itemize}
    \item very strong evidence against the NH; hence very strong evidence in favor of the drug being
effective.
\item A 95\% confidence interval for the true mean change is 18:93§t14;5% £2:33 or 18:93§2:145£
2:33 which is 18:93 § 5:00; i:e: 13:93 to 23:93:
Diastolic Mean difference=139/15=9.27
s2 =
1
14
(2327 ¡
1392
15
) = 74:210 t(14) =
9:27 ¡ 0
p
74:21=15
= 9:27=2:22 = 4:17
    \item The 95\% confidence interval is 9:27 § 2:145 £ 2:22 = 9:27 § 4:76 which is 4.51 to 14.03
    \item Again there is very strong evidence of a real difference, ie that the drug is effective, but the
reduction in diastolic pressure is less than in systolic.
    \itemGenerally the two pressures decrease together in the same patient, and roughly speaking
the diastolic reduction is about 2/3 as much as the systolic reduction: the slope of a fitted line
would be about +2/3.
\end{itemize}


\end{enumerate}

\end{document}

\documentclass[a4paper,12pt]{article}
%%%%%%%%%%%%%%%%%%%%%%%%%%%%%%%%%%%%%%%%%%%%%%%%%%%%%%%%%%%%%%%%%%%%%%%%%%%%%%%%%%%%%%%%%%%%%%%%%%%%%%%%%%%%%%%%%%%%%%%%%%%%%%%%%%%%%%%%%%%%%%%%%%%%%%%%%%%%%%%%%%%%%%%%%%%%%%%%%%%%%%%%%%%%%%%%%%%%%%%%%%%%%%%%%%%%%%%%%%%%%%%%%%%%%%%%%%%%%%%%%%%%%%%%%%%%
\usepackage{eurosym}
\usepackage{vmargin}
\usepackage{amsmath}
\usepackage{graphics}
\usepackage{epsfig}
\usepackage{enumerate}
\usepackage{multicol}
\usepackage{subfigure}
\usepackage{fancyhdr}
\usepackage{listings}
\usepackage{framed}
\usepackage{graphicx}
\usepackage{amsmath}
\usepackage{chngpage}
%\usepackage{bigints}

\usepackage{vmargin}
% left top textwidth textheight headheight
% headsep footheight footskip
\setmargins{2.0cm}{2.5cm}{16 cm}{22cm}{0.5cm}{0cm}{1cm}{1cm}
\renewcommand{\baselinestretch}{1.3}

\setcounter{MaxMatrixCols}{10}
\begin{document}
\begin{enumerate}
\item N=18 observations. Total (corrected) ss = 286327 ¡ 22252=18 = 11292:28
ss for times =
1
3
(4152 + ¢ ¢ ¢ + 2672) ¡
22252
18
=
840655
3
¡
22252
18
= 5183:61
Analysis of variance:
Source of variation DF Sum of squares Meansquare
Times 5 5183:61 1036:72
Residual 12 6108:67 509:06 F(5; 12) = 2:04n:s:
Total 17 11292:28
This provides no evidence in favor of any time effect.
\item Omitting the given sample,grand total is now 2161,N=17, G2=N = 274701:2353;
sum of all squres = 286327 ¡ 642 = 282231 total for time 36(2 observations)=286,and
ss for times is
2862
2
+
1
3
(4152 + 4092 + 3962 + 3882 + 2672) ¡
G2
N
= 5581:76
Residual now has 11 d.f. and total 16. residual ss=1948.00
DF SS MS
Times 5 5581:76 116:35 F(5; 11) = 6:30
Residual 11 1948:00 177:09 = s2
Total 16
We should now reject the hypothesis that the mean results at each time are all the same.
(0) (24) (36) (48) (72) (120)
Means 138:3 136:3 134:0 132:0 129:3 89:0
18
\begin{itemize}
    \item The reason for the different conclusion is that now 120 stands out from the others.
\item The SE of difference between two means (not including (36)) is
q
2
3 £ 177:09 = 10:87,and
between (36) and any other is
q
( 1
2 + 1
3)(177:09) = 12:15 so t test (11 d.f.) would confirm
this conclusion.
\item The value at 36 hours which has been omitted looked suspiciously low, and could perhaps
have been a measure for recording error. 
\item Now it can not be checked, but if there
are laboratory records available. that may help to decide whether or not to induce it in
the analysis. 
\item It makes a serious difference to the conclusions.
\end{itemize}


\end{enumerate}
\end{document}

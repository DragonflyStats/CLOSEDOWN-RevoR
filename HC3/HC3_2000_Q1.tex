Paper III
Statistical Applications & Practice
1(I)A Â2 test for 2 £ 2table may be used. Expected value on the Null Hypothesis of
no difference between proportion are shown in brackets,(109:14 = 200£191
350 ; etc)
K L Total
+ 127(109:14) 64(81:86) 191
¡ 73(90:86) 86(68:14) 159
200 150 350
Â2
(1) =
127 ¡ 109:142
109:14
+ ¢ ¢ ¢ +
86 ¡ 68:142
68:14
= 15:01
(yate’s correction may be used but is not necessary since the result is in no doubt). This
is clear evidence to reject the N.H.
(ii)The two proportions are:Pk = 127
200 = 0:635; PL = 64
150 = 0:427. Using a normal
approximation to the binomial distribution gives the distributionN(PK ¡PL; PK(1¡PK)
200 +
PL(1¡PL)
150 ) for the true difference ¦K ¡ ¦L between the population proportions. A 95%
confidence interval for ¦K ¡ ¦L is
(PK ¡ PL) § 1:96
s
0:635 £ 0:365
200
+
0:427 £ 0:573
150
i.e. 0:208 § 1:96 £ 0:0528 or 0:208 § 0:104 which is (0.104 to 0.312)
(iii) The proportion of positive response among those showing the K reaction is, with
95% probability, between 10.4% and 31.2% greater than for those showing the L reaction.
(iv) The point estimation is 0.208, or 20.8%. we know that if we took another sample
we would not get exactly the same estimate of the difference, but without the confidence
interval we would not know how much to expect repeated estimation to vary
(v)Interval width depends on the square root of sample size,though the square root
of the variance of (PK ¡ PL) and so to narrow the interval by a factor 1
4 we increase
sample size by 42 = 16. The necessary size is then about 16 £ 350 = 5600.
2(i)
16
(ii) P
y = 424:8 sxy = 17005:66 ¡ (424:8 £ P 393:9)=10 = 272:788
x = 393:9 sxx = 15840:23 ¡ 393:92=10 = 324:509
Hence the slope
ˆb
=
272:788
324:509
= 0:8406
The fitted line is y¡¯y =ˆb(x¡¯x) or y¡42:48 = 0:8406(x¡39:39) i.e. y = 0:8406x+9:3681
or y = 9:37 + 0:84x
(iii)The proportion of total variation (sum of square), that is explained by the relation
of observation to formula is to 0.843.[(229:31=272:18) ¼ 0:8425] this is reasonably
good.
(iv)The calculation is based on the theoretical model y = ®+¯x+",where var(") =
¾2 is estimated by s2 = (2:315)2 = 5:36 It has 8 d.f. var(ˆb) = ¾2=sxx estimated as
5:36
324:509 = 0:016517, so SˆE = 0:1285
17
A 95% interval for ¯ is :
ˆb
§ 2:306 £ 0:1285 = 0:8406 § 0:2963 i:e:(0:544 to 1:137)
[t(8;5%) = 2:306]. A 95% interval for ® is :
ˆa § 2:306
s
s2(
1
10
+
¯x2
sxx
) = 9:368 § 2:306 £
p
5:36 £ 4:8813 = 9:368 § 11:795
This give(-2.427 to +21.16)
Note that this is very imprecisely determined(and is a little use since there are no data
near to zero to confirm whether a linear relation still holds)

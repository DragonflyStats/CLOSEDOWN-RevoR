\documentclass[a4paper,12pt]{article}
%%%%%%%%%%%%%%%%%%%%%%%%%%%%%%%%%%%%%%%%%%%%%%%%%%%%%%%%%%%%%%%%%%%%%%%%%%%%%%%%%%%%%%%%%%%%%%%%%%%%%%%%%%%%%%%%%%%%%%%%%%%%%%%%%%%%%%%%%%%%%%%%%%%%%%%%%%%%%%%%%%%%%%%%%%%%%%%%%%%%%%%%%%%%%%%%%%%%%%%%%%%%%%%%%%%%%%%%%%%%%%%%%%%%%%%%%%%%%%%%%%%%%%%%%%%%
\usepackage{eurosym}
\usepackage{vmargin}
\usepackage{amsmath}
\usepackage{graphics}
\usepackage{epsfig}
\usepackage{enumerate}
\usepackage{multicol}
\usepackage{subfigure}
\usepackage{fancyhdr}
\usepackage{listings}
\usepackage{framed}
\usepackage{graphicx}
\usepackage{amsmath}
\usepackage{chngpage}
%\usepackage{bigints}

\usepackage{vmargin}
% left top textwidth textheight headheight
% headsep footheight footskip
\setmargins{2.0cm}{2.5cm}{16 cm}{22cm}{0.5cm}{0cm}{1cm}{1cm}
\renewcommand{\baselinestretch}{1.3}

\setcounter{MaxMatrixCols}{10}
\begin{document}
\begin{framed}
8. (a) Using examples to illustrate your answer, discuss the uses made of the F distribution in statistical methods.
(b) New working practices have been introduced into a factory and its manager is concerned that, as a consequence, the variation in the time taken to complete a standard task (in minutes) might now differ between men and women.  Times taken in random samples of their work are given below.
Men 21 23 12 18 20 27 23 25 26 10 14 19 Women 18 24 17 21 18 22 18 19 23 24
Is there any statistical evidence that male times are more variable than female?  Explain your conclusion, giving a confidence interval for the ratio of the variances, stating any assumptions which you made.
3
\end{framed}
%%%%%%%%%%%%%%%%%%%%%%%%%%%%%%%%%%%%%%%%%5
\begin{enumerate}
\item The F distribution with (v1; v2) degrees of freedom is the distribution of the
ratio of two Â2 distributions - specifically F(v1; v2) = Â(v1)
v1
=
Â(v2)
v2
. 
\begin{itemize}
\item Therefore,
two independent estimates of variance from the same population, based respectively
on (v1 + 1) and (v2 + 1) observations may be compared in an F
distribution. An example of this is in (b), assuming observations normally
distributed.
\item Also, two sums of squares of normally distributed variables can be compared.
An examples is in the analysis of designed experiments (see Question
2), where the residual sum of squares provides an estimate of natural variation,
¾2, and the treatment sum of squares also provides an estimate of
this on the Null Hypothesis of no treatment differences: the actual estimates
are the sums of squares divided by their degrees of freedom, i.e. the “mean
squares”.
\item Hence these mean squares can be compared in an F-test, as long
as observations are normally distributed.
\item Similarly, in linear regression, the sum of squares of deviations from the fitted
“live” provides a test of fit: sums of squares for regression is Â2
(p¡1) when
p x-variables are used, and the residual sum of squares is Â2
(n¡p). 
\item Hence two
mean squares can be found whose ratio will be F(p ¡ 1; n ¡ p) if a linear fit
is adequate.
\end{itemize}
\item  For men, v1 = 12 and for women, v2 = 10. Calculate s2 = 1
v¡1
P
(xi¡¯x)2 for
each sample. For men, s21
= 30:3333 and for women s22
= 7:3778, 

\[ F(11;9) = \frac{s21}{s22}  = \frac{30:3333}{7:3778} 
= 4:1114\]. 

\begin{itemize}
    \item Since this is significant at 5\% on the Null Hypothesis that the
two variances are equal, we must reject that hypothesis.
\item Each set of data is
assumed to be from a normal population, there is some suggestion that this
may not be true for the men, but rather there are two sub-populations.
\end{itemize}

(v2¡1)s22 ¾2
2
¥ (v1¡1)s21
¾2
1
is the ratio Â(v2¡1)2
Â(v1¡1)2
i.e. F(v2 ¡ 1; v1 ¡ 1). Therefore
v2¡1
v1¡1 ¢ s22
s21
¢ ¾2
1
¾2
2
» F(v2 ¡1; v1 ¡1), or ¾2
1
¾2
2
= (v1¡1)s21
(v2¡1)s22
F(v2 ¡1; v1 ¡1). 
\begin{itemize}
    \item Limits for
¾2
1
¾2
2
are 11
9 ¢4:1114¢F(9; 11), where the upper and lower 21
2\% points of F(9; 11)
are to be used. 
\item The upper point is 3.59. 
\item For the lower point, use the fact
16
that P(F > F¤) = P( 1
F < 1
F¤ ), where F¤ is the critical value.
\item But 1
F is also
an F-variable, with upper and lower degrees of freedom interchanged. \item The
upper 21
2\% point of F(11; 9) is 3.92.
\item Hence the required lower 21
2\% points
1=3:92 = 0:255. 
\item The 95\% confidence interval is 5:025£0:255 to 5:025£3:59

i.e. (1.2814 to 18.04).
\end{itemize}

\end{enumerate}
\end{document}

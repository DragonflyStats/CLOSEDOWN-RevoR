PAPER III : Statistical Applications & Practice
1.(i)
The variance ¾2 appears to be proportional to the mean ¹, whereas for Analysis of Variance
we must assume that all observations have the same variance. The square root transformation will
stabilize the variance when ¾2 / ¹, and has often been found suitable for counts.
(ii)After the square root transformation the variance is quite similar for all batches, and so can
reasonably be pooled in an analysis of variance.
(iii)The total sum of squares of p
yij is the sum of all the yij , most easily calculated as 6 £ the
sum of those means,it is 5955.
The sum of all p
yij is 6£ the sum of those means, 361.73 .
N=24, so G2=N = 361:732=24 = 5452:0247.
Corrected total SS=5955-5452.0247=502.975 .
Transformed batch totals are A, 129.63; B,84.65; C,52.94; D,94.51 .
Batch SS= 1
6 (129:632 + ¢ ¢ ¢ + 94:672) ¡ G2=N = 498:699.
Analysis of variance:
10
SOURCE D:F: SUM OF MEAN
SQUARES SQUARE
Batches 3 498:699 166:233
Residual 20 4:276 0:2138
TOTAL 23 502:975
F(3; 20)¤¤¤¢¢¢ Extremely highly significant.
Clearly there are large differences between batches. This can be explored using least significant
differences, since no further information is available to set up definite comparisons(contrasts) between
batches for testing.
The l.s.d. between two mean=t(20)
q
2£0:2138
6 = 0:267 £
8><
>:
2:086(5%)
2:845(1%)
3:850(0:1%)
=
8><
>:
0:557(5%)
0:759(10%)
1:028(0:1%)
Batch means are: C B D A
8:8233 14:1083 15:7517 21:6050
.
Hence all batches differ very significantly from are another.
2.For sub-adults, nS = 15; for females, nF = 23; for males, nM = 20. Medians and quartiles are:
MS = 3329(13M:
item); MF = 2859(12M:
item); MM = 1693(averrage of 10M:
and 11M:
items):
Transformed: MS = 3:52; MF = 3:46; MM = 3:23.
Lower quartiles: qS = 1
2 (1846+1960) = 1903 or 3:28; qF = 1409 or 3:15; qM = 1
2 (916+1089) =
1002:5 or 3:00.
Upper quartiles: QS = 5602 or 3:75; QF = 4397 or 3:64; QM = 1
2 (2520 + 2888) = 2704 or 3:43.
(i) (ii)
These three distributions are all distinctly skew to the right.
The diagrams for log10(data) show much more symmetry, and much more constant variability.
The basic assumptions for analysis of variance are therefore much more reasonable in these units.
(iii)Analysis of Variance
11
SOURCE DF SUM OF SQUARES MEAN SQUARE
Between groups 2 0:8930 0:4465
Residual 65 5:4844 0:0844
Total 67 6:3774
F(2;65) = 5:29¤¤
There are substantial differences between the three groups of animals.
(iv)Using the log data, we may compare the rations of chlordane levels in the different groups.
The 95% limits in log10 units are:
Sub adults : 3:51 § 2:00
q
0:0844
25 or 3:51 § 0:12; i:e: 3:39 to 3:63
Females : 3:42 § 2:00
q
0:0844
23 or 3:42 § 0:12; i:e: 3:30 to 3:54
Males : 3:23 § 2:00
q
0:0844
20 or 3:23 § 0:13; i:e: 3:10 to 3:36
To answer the questions the investigators had in mind, the sub-adults could be compared
with adults of each sex in significant tests. The confidence intervals in this particular experiment
indicate what the results of these comparisons would be: sub-adults and adult females show no
real difference(intervals overlap considerably) but adult males and sub-adults do show significant
difference(no overlap of intervals).
[NOTE that males and females differ at the 5% significance level; but it not clear that we need
to make this comparison, and the confidence intervals alone do not tell us this. ]
(v)Anti-logs to base 10 give the intervals as follows:
Males 1260 to 2290; Females 2000 to 3470;
Sub-adults 2450 to 4270 (to nearest 10 ng/g).

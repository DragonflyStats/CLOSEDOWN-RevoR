\documentclass[a4paper,12pt]{article}
%%%%%%%%%%%%%%%%%%%%%%%%%%%%%%%%%%%%%%%%%%%%%%%%%%%%%%%%%%%%%%%%%%%%%%%%%%%%%%%%%%%%%%%%%%%%%%%%%%%%%%%%%%%%%%%%%%%%%%%%%%%%%%%%%%%%%%%%%%%%%%%%%%%%%%%%%%%%%%%%%%%%%%%%%%%%%%%%%%%%%%%%%%%%%%%%%%%%%%%%%%%%%%%%%%%%%%%%%%%%%%%%%%%%%%%%%%%%%%%%%%%%%%%%%%%%
\usepackage{eurosym}
\usepackage{vmargin}
\usepackage{amsmath}
\usepackage{graphics}
\usepackage{epsfig}
\usepackage{enumerate}
\usepackage{multicol}
\usepackage{subfigure}
\usepackage{fancyhdr}
\usepackage{listings}
\usepackage{framed}
\usepackage{graphicx}
\usepackage{amsmath}
\usepackage{chngpage}
%\usepackage{bigints}

\usepackage{vmargin}
% left top textwidth textheight headheight
% headsep footheight footskip
\setmargins{2.0cm}{2.5cm}{16 cm}{22cm}{0.5cm}{0cm}{1cm}{1cm}
\renewcommand{\baselinestretch}{1.3}

\setcounter{MaxMatrixCols}{10}
\begin{document}
% PAPER III : Statistical Applications & Practice

\begin{enumerate}[(a)]
\item The variance ¾2 appears to be proportional to the mean ¹, whereas for Analysis of Variance
we must assume that all observations have the same variance. The square root transformation will
stabilize the variance when ¾2 / ¹, and has often been found suitable for counts.
\item After the square root transformation the variance is quite similar for all batches, and so can
reasonably be pooled in an analysis of variance.
\item The total sum of squares of p
$y_{ij}$ is the sum of all the $y_{ij}$ , most easily calculated as 6 £ the
sum of those means,it is 5955.
The sum of all p
yij is 6£ the sum of those means, 361.73 .
N=24, so G2=N = 361:732=24 = 5452:0247.
\[\mbox{Corrected total }SS=5955-5452.0247=502.975 .\]
Transformed batch totals are A, 129.63; B,84.65; C,52.94; D,94.51 .
Batch SS= 1
6 (129:632 + ¢ ¢ ¢ + 94:672) ¡ G2=N = 498:699.
Analysis of variance:

SOURCE D:F: SUM OF MEAN
SQUARES SQUARE
Batches 3 498:699 166:233
Residual 20 4:276 0:2138
TOTAL 23 502:975
F(3; 20)¤¤¤¢¢¢ Extremely highly significant.
\begin{itemize}
    \item Clearly there are large differences between batches. 
    \item This can be explored using least significant
differences, since no further information is available to set up definite comparisons(contrasts) between
batches for testing.
\end{itemize}

The l.s.d. between two mean=t(20)
q
2£0:2138
6 = 0:267 £
8><
>:
2:086(5%)
2:845(1%)
3:850(0:1%)
=
8><
>:
0:557(5%)
0:759(10%)
1:028(0:1%)
Batch means are: C B D A
8:8233 14:1083 15:7517 21:6050
.
Hence all batches differ very significantly from are another.
\end{enumerate}
\end{document}

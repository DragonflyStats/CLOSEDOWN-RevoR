\documentclass[a4paper,12pt]{article}
%%%%%%%%%%%%%%%%%%%%%%%%%%%%%%%%%%%%%%%%%%%%%%%%%%%%%%%%%%%%%%%%%%%%%%%%%%%%%%%%%%%%%%%%%%%%%%%%%%%%%%%%%%%%%%%%%%%%%%%%%%%%%%%%%%%%%%%%%%%%%%%%%%%%%%%%%%%%%%%%%%%%%%%%%%%%%%%%%%%%%%%%%%%%%%%%%%%%%%%%%%%%%%%%%%%%%%%%%%%%%%%%%%%%%%%%%%%%%%%%%%%%%%%%%%%%
\usepackage{eurosym}
\usepackage{vmargin}
\usepackage{amsmath}
\usepackage{graphics}
\usepackage{epsfig}
\usepackage{enumerate}
\usepackage{multicol}
\usepackage{subfigure}
\usepackage{fancyhdr}
\usepackage{listings}
\usepackage{framed}
\usepackage{graphicx}
\usepackage{amsmath}
\usepackage{chngpage}
%\usepackage{bigints}

\usepackage{vmargin}
% left top textwidth textheight headheight
% headsep footheight footskip
\setmargins{2.0cm}{2.5cm}{16 cm}{22cm}{0.5cm}{0cm}{1cm}{1cm}
\renewcommand{\baselinestretch}{1.3}

\setcounter{MaxMatrixCols}{10}
\begin{document}
% PAPER III : Statistical Applications & Practice
\begin{table}[ht!]
 \centering
 \begin{tabular}{|p{15cm}|}
 \hline  
1. Six cartons were sampled randomly from each of four batches of milk and were stored for 14 days. After this time they were tested for the presence of bacteria which cause spoilage. The bacterial counts for samples are given in the table below together with the mean and variance for each batch.
Batch Bacterial count Mean Variance A 439, 480, 497, 464, 460, 462 467.000 387.200 B 207, 204, 186, 186, 208, 204 199.167 106.567 C 74,   71,   72,   87,   89,   75 78.000 62.400 D 252, 227, 234, 258, 260, 259 248.333 203.467
(i) Make a plot of variance against mean for the four batches. What does this plot reveal about the need for a transformation if the intention is to perform an analysis of variance of the results of the investigation? In particular, which assumption required for the validity of the analysis appears to be violated?
The table below gives the square roots of the counts, together with their sample means and sample variances.
Batch Square root of count Mean Variance A 20.95, 21.91, 22.29, 21.54, 21.45, 21.49 21.6050 0.206550 B 14.39, 14.28, 13.64, 13.64, 14.42, 14.28 14.1083 0.134817 C  8.60,   8.43,   8.49,   9.33,   9.43,   8.66 8.8233 0.193427 D 15.87, 15.07, 15.30, 16.06, 16.12, 16.09 15.7517 0.205577
(ii) Explain why the information contained in this table indicates that the transformation has made the data conform more closely to the assumption which was violated by the un-transformed data.
(iii) Perform an analysis of variance of the transformed data to assess whether there is evidence of differences between the bacterial content of the batches.  Note that since the square root transformation has been applied the sum of squares of the transformed data is equal to the sum of the un-transformed counts.
\\ \hline
  \end{tabular}
\end{table}
\begin{enumerate}[(a)]
\item The variance ¾2 appears to be proportional to the mean ¹, whereas for Analysis of Variance
we must assume that all observations have the same variance. The square root transformation will
stabilize the variance when ¾2 / ¹, and has often been found suitable for counts.
\item After the square root transformation the variance is quite similar for all batches, and so can
reasonably be pooled in an analysis of variance.
\item The total sum of squares of p
$y_{ij}$ is the sum of all the $y_{ij}$ , most easily calculated as 6 £ the
sum of those means,it is 5955.
The sum of all p
yij is 6£ the sum of those means, 361.73 .
N=24, so G2=N = 361:732=24 = 5452:0247.
\[\mbox{Corrected total }SS=5955-5452.0247=502.975 .\]
Transformed batch totals are A, 129.63; B,84.65; C,52.94; D,94.51 .
Batch SS= 1
6 (129:632 + ¢ ¢ ¢ + 94:672) ¡ G2=N = 498:699.
Analysis of variance:

SOURCE D:F: SUM OF MEAN
SQUARES SQUARE
Batches 3 498:699 166:233
Residual 20 4:276 0:2138
TOTAL 23 502:975
F(3; 20)¤¤¤¢¢¢ Extremely highly significant.
\begin{itemize}
    \item Clearly there are large differences between batches. 
    \item This can be explored using least significant
differences, since no further information is available to set up definite comparisons(contrasts) between
batches for testing.
\end{itemize}

The l.s.d. between two mean=t(20)
q
2£0:2138
6 = 0:267 £
8><
>:
2:086(5%)
2:845(1%)
3:850(0:1%)
=
8><
>:
0:557(5%)
0:759(10%)
1:028(0:1%)
Batch means are: C B D A
8:8233 14:1083 15:7517 21:6050
.
Hence all batches differ very significantly from are another.
\end{enumerate}
\end{document}

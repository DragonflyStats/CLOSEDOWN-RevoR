\documentclass[a4paper,12pt]{article}
%%%%%%%%%%%%%%%%%%%%%%%%%%%%%%%%%%%%%%%%%%%%%%%%%%%%%%%%%%%%%%%%%%%%%%%%%%%%%%%%%%%%%%%%%%%%%%%%%%%%%%%%%%%%%%%%%%%%%%%%%%%%%%%%%%%%%%%%%%%%%%%%%%%%%%%%%%%%%%%%%%%%%%%%%%%%%%%%%%%%%%%%%%%%%%%%%%%%%%%%%%%%%%%%%%%%%%%%%%%%%%%%%%%%%%%%%%%%%%%%%%%%%%%%%%%%
\usepackage{eurosym}
\usepackage{vmargin}
\usepackage{amsmath}
\usepackage{graphics}
\usepackage{epsfig}
\usepackage{enumerate}
\usepackage{multicol}
\usepackage{subfigure}
\usepackage{fancyhdr}
\usepackage{listings}
\usepackage{framed}
\usepackage{graphicx}
\usepackage{amsmath}
\usepackage{chngpage}
%\usepackage{bigints}

\usepackage{vmargin}
% left top textwidth textheight headheight
% headsep footheight footskip
\setmargins{2.0cm}{2.5cm}{16 cm}{22cm}{0.5cm}{0cm}{1cm}{1cm}
\renewcommand{\baselinestretch}{1.3}

\setcounter{MaxMatrixCols}{10}
\begin{document}
8.(i)The treatment combinations used in a factorial design are made up of all possible combinations
of levels, or amounts, of several factors that may affect the response being measured. For example,
15
an industrial process may depend on the time for which it runs, T, and the temperature at which it
is operated, U. If several values of T and of U are used, say T1; T2; T3; T4 and U1;U2;U3 a factorial
design requires all twelve combinations T1U1 to T4U3 to be used, usually in two or more complete
replicates.
The response to one factor may take a different pattern at different levels of the other: e.g. at
U1 there may be a linear change from T1 to T4 whereas it is quadratic at U2, and irregular at U3.
This is interaction between T and U and is not discovered unless a factorial design is used.
Totals of responses are required for analysis, i.e. means£ 5.
Goats Red Deer Camelids
Sown grasses 4:535 10:165 5:685
Natural grasses 6:740 12:185 9:865
Heathers 3:230 9:440 4:850
Grand total G=66.695.[MISPRINT ON PAPER]
(ii)Correction term G2=N = 66:6952=45 = 98:8494.
Total S.S.=114.85-G2=N=16.0006 .
Total for Animals are 14.505, 31.790, 20.400; for plants 20.385, 28.790, 17.520.
S.S. Animals= 1
15 (14:5052 + 31:7902 + 20:4002) ¡ G2=N = 10:2945.
S.S. Plants= 1
15 (20:3852 + 28:7902 + 17:5202) ¡ G2=N = 4:5748.
S.S. Animals+S.S. plants +S.S. Interaction= 1
5 (4:5352 + ¢ ¢ ¢ + 4:8502) ¡ G2
N = 15:2509.
Hence the Analysis of Variance:
SOURCE OF D:F: SUM OF MEAN V ARIANCE
V ARIATION SQUARES SQUARE RATIO
Animal Types 2 10:2945 5:1473
Plant Types 2 4:5748 2:2874
Interaction 4 0:3816 0:0954 F(4; 36) = 4:58??
Residual 36 0:7497 0:02083
TOTAL 44 16:0006
There is strong evidence of interaction between Animals and Plants. Graphs of the means for
the nine combinations are required.
In the presence of interaction, the main effects of the factors have little meaning. We may
note that levels for Red Deer are consistently above those for Camelids, which are also above those
for Goats.
16
Camelids show a different pattern from Goats and Red Deer, in that their intake from natural
grasses is relatively higher than from sown grass or heathers; the figure 1.973 is the odd one out
in the table of means.
17
\end{document}

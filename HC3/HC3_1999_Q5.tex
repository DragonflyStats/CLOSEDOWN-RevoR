\documentclass[a4paper,12pt]{article}
%%%%%%%%%%%%%%%%%%%%%%%%%%%%%%%%%%%%%%%%%%%%%%%%%%%%%%%%%%%%%%%%%%%%%%%%%%%%%%%%%%%%%%%%%%%%%%%%%%%%%%%%%%%%%%%%%%%%%%%%%%%%%%%%%%%%%%%%%%%%%%%%%%%%%%%%%%%%%%%%%%%%%%%%%%%%%%%%%%%%%%%%%%%%%%%%%%%%%%%%%%%%%%%%%%%%%%%%%%%%%%%%%%%%%%%%%%%%%%%%%%%%%%%%%%%%
\usepackage{eurosym}
\usepackage{vmargin}
\usepackage{amsmath}
\usepackage{graphics}
\usepackage{epsfig}
\usepackage{enumerate}
\usepackage{multicol}
\usepackage{subfigure}
\usepackage{fancyhdr}
\usepackage{listings}
\usepackage{framed}
\usepackage{graphicx}
\usepackage{amsmath}
\usepackage{chngpage}
%\usepackage{bigints}

\usepackage{vmargin}
% left top textwidth textheight headheight
% headsep footheight footskip
\setmargins{2.0cm}{2.5cm}{16 cm}{22cm}{0.5cm}{0cm}{1cm}{1cm}
\renewcommand{\baselinestretch}{1.3}

\setcounter{MaxMatrixCols}{10}

\begin{document}
\begin{table}[ht!]
 \centering
 \begin{tabular}{|p{15cm}|}
 \hline  
In a study of whether two forms of iron (Fe2+ and Fe3+) are retained in different amounts within the body, 18 mice were randomly selected from a group of 36 to receive Fe2+ at a 0.3 millimolar concentration as an addition to their food and the remaining 18 mice received Fe3+ at the same concentration as an addition to their food.  The percentage of each type of iron retained was measured for each mouse. The data are listed in the table below. 
 
Fe3+ Fe2+ 2.25 4.71 3.93 5.43 5.08 6.38 5.82 6.74 5.84 8.32 6.89 9.04 8.50 9.56 8.56 10.01 9.44 10.08 10.52 10.62 13.46 13.80 13.57 14.35 14.76 14.90 18.41 15.25 26.96 17.32 27.56 19.87 32.82 31.60 39.13 37.25 
 
On a single diagram, draw boxplots for the two samples.  Using this diagram, comment on the following
 


\\ \hline
  \end{tabular}
\end{table}

\begin{table}[ht!]
 \centering
 \begin{tabular}{|p{15cm}|}
 \hline  
 (i) whether there appears to be a difference in the general level of retention of the two forms of iron 
 \\ \hline
  \end{tabular}
\end{table}


\begin{table}[ht!]
 \centering
 \begin{tabular}{|p{15cm}|}
 \hline  
 
 (ii) whether the distributions of values appear symmetrical. 

 
The investigators decided to perform a nonparametric test rather than the standard t test for comparison of means.  Explain why this is an appropriate choice. (3) 
 
Carry out the appropriate test and state clearly any conclusions you reach. 
(7) \\ \hline
  \end{tabular}
\end{table}

5 F3+
e :min=2.25, lower quartile=5.84, median=9.98, upper quartile=18.41, max=39.13.
F2+
e :min=4.71, lower quartile=8.32, median=10.35, upper quartile=15.25, max=37.25.
(i) There does not seen to be a large difference in retention as measure by the middle parts
of the data sets (between quartiles).
(ii) The distributions do not appear at all symmetrical, and F2+
e has two suspect outliers, well
above the other 16 figures. Because of the lack of symmetry and the extreme values in the

data, a mann-whitney test will be more satisfactory than a t-test; the two distributions
seem roughly similar shapes. labelling F3+
e A and F2+
e B, the joint ranking of the 36
mice is in the order:
\[AABABAABBABAABABBBABAABBABBBABAABABA\]
U = number of times an A precedes a B
= 18 + 18 + 17 + 16 + 16 + 14 + 13 + 13 + 12 + 9 + 8 + 8 + 6 + 3 + 2 + 2 + 1 + 0 = 176
The rank-sum statistic T = U+1
2£18£19 = 176+171 = 347, and for these sample sizes is
approximately normal with mean 1
2 £18£37 = 333 and variance 1
12 £18£18£37 = 999
Therefore Z = 347¡333
sqrt999 » N(0; 1) approximately, Z = 14
31:4 = 0:44 This provides no
evidence against a Null Hypothesis of the same median values of iron retention.The data
therefore provide no evidence of difference.



different from one another.
19
20
\end{enumerate}

\end{document}

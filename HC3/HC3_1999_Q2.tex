\documentclass[a4paper,12pt]{article}
%%%%%%%%%%%%%%%%%%%%%%%%%%%%%%%%%%%%%%%%%%%%%%%%%%%%%%%%%%%%%%%%%%%%%%%%%%%%%%%%%%%%%%%%%%%%%%%%%%%%%%%%%%%%%%%%%%%%%%%%%%%%%%%%%%%%%%%%%%%%%%%%%%%%%%%%%%%%%%%%%%%%%%%%%%%%%%%%%%%%%%%%%%%%%%%%%%%%%%%%%%%%%%%%%%%%%%%%%%%%%%%%%%%%%%%%%%%%%%%%%%%%%%%%%%%%
\usepackage{eurosym}
\usepackage{vmargin}
\usepackage{amsmath}
\usepackage{graphics}
\usepackage{epsfig}
\usepackage{enumerate}
\usepackage{multicol}
\usepackage{subfigure}
\usepackage{fancyhdr}
\usepackage{listings}
\usepackage{framed}
\usepackage{graphicx}
\usepackage{amsmath}
\usepackage{chngpage}
%\usepackage{bigints}

\usepackage{vmargin}
% left top textwidth textheight headheight
% headsep footheight footskip
\setmargins{2.0cm}{2.5cm}{16 cm}{22cm}{0.5cm}{0cm}{1cm}{1cm}
\renewcommand{\baselinestretch}{1.3}

\setcounter{MaxMatrixCols}{10}

\begin{document}

\section{Introduction}
\begin{enumerate}
    \item A Â2-test has to uses exact frequencies, not percentages. It can test the Null Hypothesis
that the ratios of frequencies between the six categories were the same in both years. On-this
NH, expected frequencies area given in brackets:
Category 1 2 3 4 5 6
1996 60(60) 250(205) 160(155) 240(225) 270(320) 20(35) 1000
1998 60(60) 160(205) 150(155) 210(225) 370(32:) 50(35) 1000
120 410 310 450 640 70 2000
Â2
(5) = 0 + 0 + (250¡205)2
205 + (160¡205)2
205 + (160¡155)2
155 + (150¡155)2
155 + (270¡320)2
320 + (370¡320)2
320 +
(20¡35)2
35 + (50¡35)2
35 = 48:56
There is very strong evidence against the NH.
Looking at the percentages through the categories, the numbers opposing have increased over
the period, ”slight support ”having dropped substantially and ” great opposition” increased.
On the face of it, the magazine was justified.
But much more information could be extracted by attaching a scoring scale, such as ¡2; ¡1; 0;
1; 2 to the first five categories, omitting the ”don’t knows” and looking at scores, or some others
suitable statistic. If the linear scale is (approximately)valid, this will extract more information
than simple categorization. But it may be unwise to assume linearity, and also not easy to
13
decide how to score very strong views.
n1 = n2 = 1000; p1 = 0:51; p2 = 0:58; p2 ¡ p1 = 0:07
V [p2 ¡ p1] =
0:58 £ 0:42
1000
+
0:51 £ 0:49
1000
= 0:0004935; SE = 0:0222
95% confidence interval for true value of p2¡p1 is 0:07§1:96£0:0222 = 0:07§0:0435 or 0:026 to 0:114
i.e. with 95% probability the difference lies between 2:6% and 11:4%. The evidence is that
there has been a shift against, of this extent.

\documentclass[a4paper,12pt]{article}
%%%%%%%%%%%%%%%%%%%%%%%%%%%%%%%%%%%%%%%%%%%%%%%%%%%%%%%%%%%%%%%%%%%%%%%%%%%%%%%%%%%%%%%%%%%%%%%%%%%%%%%%%%%%%%%%%%%%%%%%%%%%%%%%%%%%%%%%%%%%%%%%%%%%%%%%%%%%%%%%%%%%%%%%%%%%%%%%%%%%%%%%%%%%%%%%%%%%%%%%%%%%%%%%%%%%%%%%%%%%%%%%%%%%%%%%%%%%%%%%%%%%%%%%%%%%
\usepackage{eurosym}
\usepackage{vmargin}
\usepackage{amsmath}
\usepackage{graphics}
\usepackage{epsfig}
\usepackage{enumerate}
\usepackage{multicol}
\usepackage{subfigure}
\usepackage{fancyhdr}
\usepackage{listings}
\usepackage{framed}
\usepackage{graphicx}
\usepackage{amsmath}
\usepackage{chngpage}
%\usepackage{bigints}

\usepackage{vmargin}
% left top textwidth textheight headheight
% headsep footheight footskip
\setmargins{2.0cm}{2.5cm}{16 cm}{22cm}{0.5cm}{0cm}{1cm}{1cm}
\renewcommand{\baselinestretch}{1.3}

\setcounter{MaxMatrixCols}{10}
\begin{document}
\begin{enumerate}
    \item Given that ¹ = 0:52 and ¾ = 23:43, the cumulative frequencies in a normal distribution are
given by 520 ©(Z);
for -67.5, ©(¡67:5¡0:52
23:43 ) = ©(¡2:903) = 0:00185; CF = 0:962;
for -52.5, ©(¡52:5¡0:52
23:43 ) = ©(¡2:263) = 0:01182; CF = 6:146.
Because of the very small expected frequency in the first group we shall combine it with the
second, to give Obs.=7, Exp.=6.146. Continuing down the table, the cumulative frequency to -7.5
is 6.146+21.063+57.512+105.632=190.353.
For +7.5, ©(7:5¡0:52
23:43 ) = ©(0:298) = 0:61715; CF = 320:918.
\begin{itemize}
\item Now 320.918-190.353=130.565, which is the frequency in (-7.5,+7.5).
\item For +37.5, ©(37:5¡0:52
23:43 ) = ©(1:578) = 0:94272. Hence the frequency in (22.5,37.5)=490.214-
320.918-108.573=60.723 .
\item Now check that 490.214+(22.873+5.789+1.106)=519.982=: n within acceptable rounding error.
\item For testing normality the last two groups are combined: O=3, E=6.895.
\item There are 9 groups, two parameters were estimated, so Â2 has 6 d f.
\end{enumerate}
Â2
(6) = (7¡6:146)2
6:146 + (26¡21:063)2
21:063 + (54¡57:512)2
57:512 + (90¡105:632)2
105:632 + (147¡130:565)2
130:565 + (102¡108:573)2
108:573
+(66¡60:723)2
60:723 + (25¡22:873)2
22:873 + (3¡6:895)2
6:895 = 9:20 n:s:
This provides no evidence against the fit to a normal distribution.
Unless there are enough observations to combine into several groups, the Â2 is a very poor
approximation, and also the pattern in the data is difficult to detect. Power against an alternative
of non-normality is very low.
\item There will be 30 residuals, which should be i.i.d. N(0; ¾2). If it is only normality that is
tested, and not only of the other assumptions made in analysis of a randomized block, a normal
probability plot is suitable. 

\begin{itemize}
\item The residuals are ranked in order, from largest negative to largest positive.
\item Normal probability paper allows these to be plotted against the order-statistics for a normal
distribution with a sample of 30 items; the ith observed value is plotted against ©¡1(i=31).
\item This
should give roughly a straight line. 
\item Further information comes from identifying which blocks and
treatments give the largest residuals, positive or negative. 
\item If, for example, one treatment seems to
have mostly large residuals it may be indicating that variances differ from one treatment to another.
\end{itemize}

\end{enumerate}
\end{document}

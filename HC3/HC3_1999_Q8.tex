\documentclass[a4paper,12pt]{article}
%%%%%%%%%%%%%%%%%%%%%%%%%%%%%%%%%%%%%%%%%%%%%%%%%%%%%%%%%%%%%%%%%%%%%%%%%%%%%%%%%%%%%%%%%%%%%%%%%%%%%%%%%%%%%%%%%%%%%%%%%%%%%%%%%%%%%%%%%%%%%%%%%%%%%%%%%%%%%%%%%%%%%%%%%%%%%%%%%%%%%%%%%%%%%%%%%%%%%%%%%%%%%%%%%%%%%%%%%%%%%%%%%%%%%%%%%%%%%%%%%%%%%%%%%%%%
\usepackage{eurosym}
\usepackage{vmargin}
\usepackage{amsmath}
\usepackage{graphics}
\usepackage{epsfig}
\usepackage{enumerate}
\usepackage{multicol}
\usepackage{subfigure}
\usepackage{fancyhdr}
\usepackage{listings}
\usepackage{framed}
\usepackage{graphicx}
\usepackage{amsmath}
\usepackage{chngpage}
%\usepackage{bigints}

\usepackage{vmargin}
% left top textwidth textheight headheight
% headsep footheight footskip
\setmargins{2.0cm}{2.5cm}{16 cm}{22cm}{0.5cm}{0cm}{1cm}{1cm}
\renewcommand{\baselinestretch}{1.3}

\setcounter{MaxMatrixCols}{10}

\begin{document}
\begin{table}[ht!]
 \centering
 \begin{tabular}{|p{15cm}|}
 \hline  
The amount of water absorbed by different types of resin used for dental fillings was investigated.  
\begin{itemize}
    \item Ten blocks of resin were prepared with glass fibre added to reinforce the resin.  Ten blocks of resin were prepared without the addition of glass fibre. 
    \item Five blocks of each type were selected at random and hardened by chemical treatment.  The remaining blocks were hardened by heat treatment.
    \item All blocks were placed in water for the same length of time and then the amount of water absorbed by each block was measured, the results being expressed in micrograms of water per cubic millimetre of resin. 
\end{itemize}
 
The experiment was thus a two factor experiment with five replicates at each factor level combination.  The two factors are (1) glass fibre which is either present or absent and (2) method of hardening which is either chemical treatment or heat treatment. 
 
The sample mean and standard error of the mean (s.e.m.) of the water absorption for each of the factor level combinations are given below. 
 
  Hardening method 
 
  Heat mean    (s.e.m.) 
Chemical mean    (s.e.m.) 
  Not added to resin 
 
17.590      (0.417) 
 
16.970      (0.380) 
Glass 
fib re 
   
  Added to resin 
 
20.590      (0.793) 
 
16.390      (0.252) 
 
 Complete the following analysis of variance table and write a brief report on the findings of the analysis.  Your report should include an appropriate diagram to illustrate any interaction there is between the factors. 
 
Analysis of Variance for water absorption by dental resin. 
 
Source D.F. S.S. M.S. V.R. Glass fibre 1 * * * Hardening 1 * * * Interaction 1 * * * Error * 20.21 *  Total * 72.59   
 
 
(20) 
\\ \hline
  \end{tabular}
\end{table}
\begin{enumerate}
    \item The total of all 20 measurements is 5(17:59 + 20:59 + 16:97 + 16:30) = 357:7 The correction
term G2=N for calculated sums of squares is 357:72=20 = 6397:4645 Total for glass fibre are
5(17:59 + 16:97) = 172:8 and 5(20:59 + 16:39) = 184:9 so Fibre s:s: = 1
10(172:82 + 184:92) ¡
6397:4645 = 7:3205 Total for hardening are 5(17:59 + 20:59) = 190:9 and 5(16:97 + 16:34) =
166:8 so Hardening s:s: = 190:92+166:82
10 ¡ 6397:4645 = 29:0405
\begin{verbatim}
    Analysis of Variance
SOURCE DF CUM OF SQUARES MEAN SQUARE F ¡ ratio
Fibre 1 7:32 5:80 ¤
Hardening 1 29:04 22:99 ¤ ¤ ¤
Interaction 1 16:02 12:68 ¤¤
Residual 16 20:21 1:263
Total 19 72:59
\end{verbatim}

\begin{itemize}
    \item The interaction is significant, so the main effects are not studied. 
    \item Using the pooled variance,
the significant difference between any two means at the 5% level is t(16)
p
2 £ 1:263=5 =
2:120 £ 0:711 = 1:51
\item In the table of means it is clear that glass fibre present and heat hardening leads to great
water absorption than any of the other three combinations, whose results are not significantly
\end{itemize}


\end{enumerate}
\end{document}

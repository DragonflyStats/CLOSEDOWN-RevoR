\documentclass[a4paper,12pt]{article}
%%%%%%%%%%%%%%%%%%%%%%%%%%%%%%%%%%%%%%%%%%%%%%%%%%%%%%%%%%%%%%%%%%%%%%%%%%%%%%%%%%%%%%%%%%%%%%%%%%%%%%%%%%%%%%%%%%%%%%%%%%%%%%%%%%%%%%%%%%%%%%%%%%%%%%%%%%%%%%%%%%%%%%%%%%%%%%%%%%%%%%%%%%%%%%%%%%%%%%%%%%%%%%%%%%%%%%%%%%%%%%%%%%%%%%%%%%%%%%%%%%%%%%%%%%%%
\usepackage{eurosym}
\usepackage{vmargin}
\usepackage{amsmath}
\usepackage{graphics}
\usepackage{epsfig}
\usepackage{enumerate}
\usepackage{multicol}
\usepackage{subfigure}
\usepackage{fancyhdr}
\usepackage{listings}
\usepackage{framed}
\usepackage{graphicx}
\usepackage{amsmath}
\usepackage{chngpage}
%\usepackage{bigints}

\usepackage{vmargin}
% left top textwidth textheight headheight
% headsep footheight footskip
\setmargins{2.0cm}{2.5cm}{16 cm}{22cm}{0.5cm}{0cm}{1cm}{1cm}
\renewcommand{\baselinestretch}{1.3}

\setcounter{MaxMatrixCols}{10}

\begin{document}
\section{Introduction}
\begin{enumerate}
    \item The total of all 20 measurements is 5(17:59 + 20:59 + 16:97 + 16:30) = 357:7 The correction
term G2=N for calculated sums of squares is 357:72=20 = 6397:4645 Total for glass fibre are
5(17:59 + 16:97) = 172:8 and 5(20:59 + 16:39) = 184:9 so Fibre s:s: = 1
10(172:82 + 184:92) ¡
6397:4645 = 7:3205 Total for hardening are 5(17:59 + 20:59) = 190:9 and 5(16:97 + 16:34) =
166:8 so Hardening s:s: = 190:92+166:82
10 ¡ 6397:4645 = 29:0405
Analysis of Variance
SOURCE DF CUM OF SQUARES MEAN SQUARE F ¡ ratio
Fibre 1 7:32 5:80 ¤
Hardening 1 29:04 22:99 ¤ ¤ ¤
Interaction 1 16:02 12:68 ¤¤
Residual 16 20:21 1:263
Total 19 72:59
The interaction is significant, so the main effects are not studied. Using the pooled variance,
the significant difference between any two means at the 5% level is t(16)
p
2 £ 1:263=5 =
2:120 £ 0:711 = 1:51
In the table of means it is clear that glass fibre present and heat hardening leads to great
water absorption than any of the other three combinations, whose results are not significantly

\end{enumerate}
\end{document}

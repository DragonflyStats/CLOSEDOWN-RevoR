\documentclass[a4paper,12pt]{article}
%%%%%%%%%%%%%%%%%%%%%%%%%%%%%%%%%%%%%%%%%%%%%%%%%%%%%%%%%%%%%%%%%%%%%%%%%%%%%%%%%%%%%%%%%%%%%%%%%%%%%%%%%%%%%%%%%%%%%%%%%%%%%%%%%%%%%%%%%%%%%%%%%%%%%%%%%%%%%%%%%%%%%%%%%%%%%%%%%%%%%%%%%%%%%%%%%%%%%%%%%%%%%%%%%%%%%%%%%%%%%%%%%%%%%%%%%%%%%%%%%%%%%%%%%%%%
\usepackage{eurosym}
\usepackage{vmargin}
\usepackage{amsmath}
\usepackage{graphics}
\usepackage{epsfig}
\usepackage{enumerate}
\usepackage{multicol}
\usepackage{subfigure}
\usepackage{fancyhdr}
\usepackage{listings}
\usepackage{framed}
\usepackage{graphicx}
\usepackage{amsmath}
\usepackage{chngpage}
%\usepackage{bigints}

\usepackage{vmargin}
% left top textwidth textheight headheight
% headsep footheight footskip
\setmargins{2.0cm}{2.5cm}{16 cm}{22cm}{0.5cm}{0cm}{1cm}{1cm}
\renewcommand{\baselinestretch}{1.3}

\setcounter{MaxMatrixCols}{10}

\begin{document}
\begin{enumerate}
    \item
17
(i) (a) Assuming that the list was consolidated, in any order, e.g. alphabetically, by year
of entry, by faculty, etc. and that the numbers 0001-8002 could be allocated to the
students,then random digits would be used in sets of 4, the corresponding numbered
students being included in the sample. 0000 and any set of digits from 8003 to
9999 would be discarded. Only if a student could not be contacted(e.g. was in
hospital) would another number be chosen. (Digits which came up a second time,
or more,would be discarded-sampling is to be without replacement. )
(b) The numbers required are:
M F Total
Medicine 77 54 131
Science 116 91 207
Engineering 103 40 143
Social Sciences 77 95 172
Arts 70 77 147
443 357 800
From the faculty lists,the appropriate numbers of male and female students would
be selected by a method similar to that above.
(c) The quota sample would be carried out to the same specification as in (b), but no
lists of students would be required. Instead the survey interviewers would only be
asked to interviewer the correct number of students in each category.
(ii) A is relatively cheap, quick and simple to administer, and there is no need for trained
interviewers.
But the questions must be simple and direct, so that they are easy to answer quickly,
and there is minimal scope for misunderstanding. Questions will have to be closed,
with boxes to tick or requests for straight forward information, and some pre-testing for
possible ambiguities will be essential. Unless the topic of the survey is of general interest,
the response rate may be low and follow-up will be needed. Forms may not be returned
to a central point: individual collection is much more efficient. Distribution could be
through mail (by post for the those not living on campus) for a random sample.
18
B will generally have a higher response rate; more questions,and more complex questions,
can be asked; answers can be clarified if necessary; open questions may be included for
a response to be written down or possibly classified by the interviewer; there will be no
distribution problems although refusals are still possible for the random sample. But
there is a cost for using interviewers which may be substantial; if there are only a few
interviewers it may take longer to complete; there is a possibility of interviewer bias, or
of the respondent being biased to give a particular type of answer.
Depending or the purpose of the survey and the size of the questionnaire either method
is possible but B will usually give better quality data, and if the topic is at all complex
the presence of a trained interviewer can be particularly helpful.

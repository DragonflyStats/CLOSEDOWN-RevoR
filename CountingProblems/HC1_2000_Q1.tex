\documentclass[a4paper,12pt]{article}
%%%%%%%%%%%%%%%%%%%%%%%%%%%%%%%%%%%%%%%%%%%%%%%%%%%%%%%%%%%%%%%%%%%%%%%%%%%%%%%%%%%%%%%%%%%%%%%%%%%%%%%%%%%%%%%%%%%%%%%%%%%%%%%%%%%%%%%%%%%%%%%%%%%%%%%%%%%%%%%%%%%%%%%%%%%%%%%%%%%%%%%%%%%%%%%%%%%%%%%%%%%%%%%%%%%%%%%%%%%%%%%%%%%%%%%%%%%%%%%%%%%%%%%%%%%%
\usepackage{eurosym}
\usepackage{vmargin}
\usepackage{amsmath}
\usepackage{graphics}
\usepackage{epsfig}
\usepackage{enumerate}
\usepackage{multicol}
\usepackage{subfigure}
\usepackage{fancyhdr}
\usepackage{listings}
\usepackage{framed}
\usepackage{graphicx}
\usepackage{amsmath}
\usepackage{chngpage}
%\usepackage{bigints}

\usepackage{vmargin}
% left top textwidth textheight headheight
% headsep footheight footskip
\setmargins{2.0cm}{2.5cm}{16 cm}{22cm}{0.5cm}{0cm}{1cm}{1cm}
\renewcommand{\baselinestretch}{1.3}

\setcounter{MaxMatrixCols}{10}
\begin{document}
\large
\begin{table}[ht!]
     \centering
     \begin{tabular}{|p{15cm}|}
     \hline        \large
\noindent An ecological study is to be made of birds of an endangered species. When a bird is captured for the first time, three small, light, coloured bands are put on each leg, in upper, middle and lower positions, and the bird is then released.  Each band may be red, yellow, blue or white. \\
\\ \hline
      \end{tabular}
    \end{table}
    
  \begin{table}[ht!]
     \centering
     \begin{tabular}{|p{15cm}|}
     \hline  \large
\noindent \textbf{Part(a)}\\ \smallskip \large
If, regardless of the colours of other bands, each band position on either leg may have a band of any of the four colours. What is the total possible number of different colour combinations?     \smallskip 
 \\ \hline 
      \end{tabular}
    \end{table}

\bigskip



\begin{enumerate}[(a)]
\item  
\begin{itemize}
\item For one leg, there are 4 possible colours in each of the three positions, making
$4 \times 4 \times 4 = 64$ combinations.
\item For the other leg there will also be 64, and of the 64
on one leg may be combined with any of the 64 on the other to make $64 \times 64 = 4096$
combinations in all.

\item Equivalently, there are 6 positions altogether, and four possible colours may appear
on each, making $4^6 = 4096$
\end{itemize}



\newpage
  \begin{table}[ht!]
     \centering
     \begin{tabular}{|p{15cm}|}
     \hline  \large
\noindent \textbf{Part(b)}\\ \large \smallskip How many different colour combinations are possible if adjacent colour bands on each leg are restricted to be of different colours?    \\
 \\ \hline 
      \end{tabular}
    \end{table} 

\item Consider the middle position on one leg: it can hold any one of 4 colours, there
are only 3 possibilities for each of upper and lower (they could be the same), so each leg
has $3 \times 4 \times 3$ possible patterns, i.e. 36. 
\\ Any of these patterns can also appear on the leg,
making $36 \times 36$ = 1296 possible combinations in all.

%%%%%%%%%%%%

\newpage


  \begin{table}[ht!]
     \centering
     \begin{tabular}{|p{15cm}|}
     \hline  
\noindent \textbf{Part(c)}\\ \smallskip \large How many different colour combinations are possible if on each leg the three colours must all be different?  \\
\smallskip
 \\ \hline 
      \end{tabular}
    \end{table} 
    
\large
\item In this case the possible number of combinations is $4\times 3 \times 2 $ on each leg, i.e. $24\times
24 = 576$ altogether.
%%%%%%%%%%%%
\newpage


  \begin{table}[ht!]
     \centering
     \begin{tabular}{|p{15cm}|}
     \hline  
\noindent \textbf{Part(d)}\\ \smallskip \large The colour sequence (upper, middle, lower) on the right leg is now restricted to be different from that on the left leg. \\ \large Subject to this overriding condition, calculate the numbers of different possible colour combinations in the cases (a), (b) and (c) respectively. \smallskip \\
 \\ \hline 
      \end{tabular}
    \end{table} 
    
    
\item 
\begin{itemize}
\item[(a)] Since there are 64 combinations for one leg. \\ There are only 63 for the other:
$64 \times 63 = 4032$ in all.
\item[(b)] Similarly, there are $36 \times 35 = 1260$.
\item[(c)] Similarly, there are $24 \times 23 = 552$.
\end{itemize}
\end{enumerate}
\end{document}

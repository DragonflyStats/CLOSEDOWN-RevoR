\documentclass{article}
\usepackage[utf8]{inputenc}
\usepackage{enumerate}

\author{kobriendublin }
\date{December 2018}

\begin{document}

%- Higher Certificate, Module 5, 2010. Question 3
\section{Introduction}
\begin{enumerate}[(i)]
\item 

(i) The likelihood is ()()()()111111nnniiiiLxxβββββ−−===−=−ΠΠ.
∴Log likelihood is ()()()loglog1log1iLnxβββ=−−Σ−.
()loglog1idLnxdββ=+Σ− which on setting equal to zero gives solution ()ˆlog1inxβ=−Σ−.
To investigate whether this is a maximum, consider 222log0dLndββ=−<, so this is the maximum likelihood estimator of β.
%%%%%%%%%%%%%%%%%%
\item We have 222logdLnEdββ−=.
Thus, using the usual asymptotic result, ()221ˆVar/nnβββ≈=.
Thus, for large n, 2ˆN,nβββ, approximately.
Thus an approximate 95% confidence interval for β is given byˆˆ1.96nββ±.
[Note. The pivotal quantity method could also be used to find an alternative.]
%%%%%%%%%%%%%%%%%%
\item ()()()0.50.51000.5110.5110.5iPXxdxxβββββ−<=−=−−=−+=−∫.
%%%%%%%%%%%%%%%%%%
\item ()B,10.5Ynβ−.
So ()()()10.50.5ynynPYyyββ−==− and the likelihood is simply
()()10.50.5ynynyββ−−.

\end{enumerate}
\end{document}
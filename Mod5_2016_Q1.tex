\documentclass{article}
\usepackage[utf8]{inputenc}
\usepackage{enumerate}

\author{kobriendublin }
\date{December 2018}

\begin{document}

%- Higher Certificate, Module 5, 2008. Question 1
\section{Introduction}
\begin{enumerate}[(i)]
\item

HC5 2016 - solutions
1(i) P(X  k)  P(X  k 1)  P(X  k  2)  P(X  k  3) ... 1 (method)
= 1 2 (1 ) (1 ) (1 ) ... k k k               1 (terms substituted)
= 2 (1 ) (1 (1 ) (1 ) ...) k        1 (common factor)
Final term is sum to infinity of a GP with first term 1 and common ratio 1 where
0 1 1.
So
(1 )
( ) (1 )
(1 (1 ))
k
k P X k
 



   
 
as required. 1 (correct use of GP)
(Full marks also for correct solution by induction) TOTAL 4
%%%%%%%%%%%%%%%%%%
\item 0 P(X 1)  (1 )  and this is observed for 24 players. 1 (correct term for 24)
1 P(X  2)  (1 ) and this is observed for 48 players. 1 (correct term for 48)
2 P(X  2)  (1 ) and this is observed for the other 128 players.
1 (correct term for 128)

%%%%%%%%%%%%%%%%%%

So the likelihood for the data is given by
 
48 128 24 2 L( )   (1 ) (1 )  1 (combining terms)
= 72 304  (1 ) as required.
l( )  log L( )  72log  304log(1 ). 1 (log likelihood)
( ) 72 304
1
dl
d

  
 

and setting this equal to zero we have 1 (correct derivative), 1 (set to 0)
ˆ ˆ ˆ 72 72(1 ) 304 0.1915.
376
      1 (0.1915)
To check that this gives a maximum we differentiate again:
2
2 2 2
( ) 72 304
(1 )
d l
d

  
  

which is negative. 1 (negative 2nd deriv ) TOTAL 9
%%%%%%%%%%%%%%%%%%
\item The standard error of ˆ
can be estimated by 2
2
1
d l( )
d



where the denominator is
evaluated at ˆ. This gives
2 2
1
72 304
0.1915 0.8085


0.0004118  0.0203
1 (method), 1 (substitute), 1(0.0203 or equiv. variance)
[ If candidates work out the expected information, this gives a variance of
(1 )
n(2 )
 



where
200. n  Substituting ˆinto this gives a variance of 0.0004280. All 3 marks above, and
appropriate follow-on marks below, should be awarded.]
Then the approximate 95\% confidence interval is given by ˆ 1.96s.e(ˆ) i.e. 1 (1.96)
1 (method)
0.19151.960.0203 which is (0.1517, 0.2313). 1(0.1517), 1(0.2313) TOTAL 7
\end{enumerate}
\end{document}

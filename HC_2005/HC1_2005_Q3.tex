\documentclass[a4paper,12pt]{article}

%%%%%%%%%%%%%%%%%%%%%%%%%%%%%%%%%%%%%%%%%%%%%%%%%%%%%%%%%%%%%%%%%%%%%%%%%%%%%%%%%%%%%%%%%%%%%%%%%%%%%%%%%%%%%%%%%%%%%%%%%%%%%%%%%%%%%%%%%%%%%%%%%%%%%%%%%%%%%%%%%%%%%%%%%%%%%%%%%%%%%%%%%%%%%%%%%%%%%%%%%%%%%%%%%%%%%%%%%%%%%%%%%%%%%%%%%%%%%%%%%%%%%%%%%%%%

\usepackage{eurosym}
\usepackage{vmargin}
\usepackage{amsmath}
\usepackage{graphics}
\usepackage{epsfig}
\usepackage{enumerate}
\usepackage{multicol}
\usepackage{subfigure}
\usepackage{fancyhdr}
\usepackage{listings}
\usepackage{framed}
\usepackage{graphicx}
\usepackage{amsmath}
\usepackage{chngpage}

%\usepackage{bigints}
\usepackage{vmargin}

% left top textwidth textheight headheight

% headsep footheight footskip

\setmargins{2.0cm}{2.5cm}{16 cm}{22cm}{0.5cm}{0cm}{1cm}{1cm}
\renewcommand{\baselinestretch}{1.3}
\setcounter{MaxMatrixCols}{10}

\begin{document}

Higher Certificate, Paper I, 2005. Question 3
\begin{enumerate}
\item \[P(T > t) = P(no failure in time t) = P(X = 0) = e^{-\lambda} t .\]
\[\therefore 1 - F(t) = e^{-\lambda} t \] and so the pdf of T is 

\[ \frac{dF (t )}{dt}  =  \frac{dF (e^{-\lambda\,t} )}{dt}  = \lambda\,e^{-\lambda\,t}\]

\item 
\begin{eqnarray*}
P(\mbox{all n systems still functioning at time t}) &=& \prod^{n}_{i=1} P(T_i > t )\\
 &=& \prod^{n}_{i=1}  e^{-\lambda\,t}\\  
 &=& e^{-n\lambda\,t\} 
\end{eqnarray*} 
 


\therefore the pdf of Tmin is
\[ \frac{dF (e^{-\lambda\,t} )}{dt}  = \lambda\,e^{-\lambda\,t}\] (for t > 0).
Thus Tmin is exponential with parameter $n\,\lambda$.
\item  P(all n systems have failed by time t) = ( ) ( )
%%%%%%%%%%%%%%%%%%%%%%%%%%%%%%%
\begin{eqnarray*}
\prod^{n}_{i=1} P(T_i > t )
 &=& \prod^{n}_{i=1}  e^{-\lambda\,t}\\  
 &=& e^{-n\lambda\,t\} 
\\ &=& - e^{-\lambda} t n , 
\end{eqnarray*}

%%%%%%%%%%%%%%%%%%%%%%%%%%%%%%%

and this is $P(Tmax) \leq t$.
\therefore the pdf of Tmax is 

\[ \frac{d}{dt}\left\{ \left(1-e^{-\lambda\,t} \right)^n  \right\}  =  n \lambda\,e^{-\lambda\,t} (1-e^{-\lambda\,t})^{n-1}\]


(for t > 0).
(Note that this is not exponential.)


\begin{itemize}
\item We now have $n = 10$ and $\lambda = 0.002$. 
\item We require $t_1$ and $t_2$ such that $1- e^{-n\lambda t_1} = 0.05$
and $(1- e^{-\lambda t_2})^n = 0.95$
  \item \therefore $1- e^{-n\lambda t_1} = 0.05$ , giving $0.02t_1 = –log(0.95) = 0.051293$ 
\item so that $t_1 = 2.565$.
\item Also, $(1- e^{-\lambda t_2}) = 0.95^{1/10} = 0.994884$ , giving $0.002t_2 = –log(0.005116) = –5.2754$
\item so that $t_2 = 2638$.
\end{itemize}

\end{enumerate}

\end{document}

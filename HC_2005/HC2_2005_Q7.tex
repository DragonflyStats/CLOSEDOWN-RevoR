\documentclass[a4paper,12pt]{article}

%%%%%%%%%%%%%%%%%%%%%%%%%%%%%%%%%%%%%%%%%%%%%%%%%%%%%%%%%%%%%%%%%%%%%%%%%%%%%%%%%%%%%%%%%%%%%%%%%%%%%%%%%%%%%%%%%%%%%%%%%%%%%%%%%%%%%%%%%%%%%%%%%%%%%%%%%%%%%%%%%%%%%%%%%%%%%%%%%%%%%%%%%%%%%%%%%%%%%%%%%%%%%%%%%%%%%%%%%%%%%%%%%%%%%%%%%%%%%%%%%%%%%%%%%%%%

\usepackage{eurosym}
\usepackage{vmargin}
\usepackage{amsmath}
\usepackage{graphics}
\usepackage{epsfig}
\usepackage{enumerate}
\usepackage{multicol}
\usepackage{subfigure}
\usepackage{fancyhdr}
\usepackage{listings}
\usepackage{framed}
\usepackage{graphicx}
\usepackage{amsmath}
\usepackage{chngpage}

%\usepackage{bigints}
\usepackage{vmargin}

% left top textwidth textheight headheight

% headsep footheight footskip

\setmargins{2.0cm}{2.5cm}{16 cm}{22cm}{0.5cm}{0cm}{1cm}{1cm}

\renewcommand{\baselinestretch}{1.3}

\setcounter{MaxMatrixCols}{10}

\begin{document}
Higher Certificate, Paper II, 2005. Question 7
\begin{framed}
 (i) State and explain a linear model that can be used as the basis for a one-way analysis of variance.  Explain clearly what each term in the model represents and state any assumptions required for the analysis to be valid. (5) 

\end{framed}

\begin{framed}
 (ii) A psychology researcher has the hypothesis that effective use of leisure time helps to reduce stress.  In particular, she suggests that play activity is most effective when the subject feels it is free play, not directed by others. 
 
The researcher recruited 36 college students and divided them randomly into three groups.  One group received highly controlled play experience, one received a low level of control and one group performed what they would see as work rather than play. 
 
All subjects first performed a 30-minute stress-producing task, working through mathematics problems while hearing periodic bursts of loud noise through headphones. 
 
Next, each subject had 10 minutes at one of the three play activities described above, "high" or "low" control or "work". 
 
Finally, the subjects attempted to solve two geometric puzzles, one of which was insoluble – but they were not told this.  Persistence on the insoluble puzzle (measured in time in total seconds spent on the puzzle before giving up) was the response variable measured and used to assess the effectiveness of the play period in reducing the stress created by the work task.  The table below gives the results. 
 
High Low Work 347 504 398 567 420 492 424 583   97 239 183 357 256 279 184 682 381 554 435 118 354 666 317 275 825 359 198 102   77 163 601 336 284 384 197 155 
 
 
Perform a one-way analysis of variance on these data and, by computing least significant differences, or otherwise, investigate differences between the three means. 
 
Write a brief report for the researcher to use when interpreting the results. 
(15) 
 

 
9 
\end{framed}
\begin{enumerate}
\item %{}()2,1,2,...,,1,2,...,,~indN0,ijiijiijytikjrμεε=++== .
There are k treatments, indexed by i = 1, 2, …, k. In the experiment or survey, there are ri units (individuals) in the ith group, i.e. receiving the ith treatment. yij is the observation (response) for the jth individual in group i.
μ is the overall population general mean. ti is the population mean effect (departure from μ) due to treatment i, with . 0iitΣ=
The Normally distributed residual (error) terms εij all have variance σ 2 and are uncorrelated (independent).
This is an additive model: the components add together, and together explain all the variation in the responses.
\item The "treatments" here are "high", "low" and "work". $r_1 = r_2 = r_3 = 12$.
Totals are:
\begin{itemize}
\item High : 5528
\item  Low 3754
\item Work: 3511
\end{itemize}




\item The grand total is 12793. $\sum \sum y_{ij}2 = 5719139$.
"Correction factor" is 2127934546134.69436=.

 \begin{itemize}
    \item Therefore total SS = 5719139 – = 1173004.306. 
SS for treatments = \[ \frac{5528^2}{12} + \frac{3754^2}{12} + \frac{3511^2}{12} -   4546134.694 = .
    \item The residual SS is obtained by subtraction.
    \item Hence the analysis of variance table is as follows (SS and MS entries are slightly rounded).
\end{itemize}

\begin{center}
\begin{tabular}{|c|c|c|c|c|l|} \hline 
SOURCE & DF & SS & MS & F value &  \\ \hline \hline
Treatments & 2 & 202067& 101034& 3.43 & Compare F2,33\\ \hline 
Residual & 33& 970937& 29422 &  & = $\hat{\sigma}$ \\ \hline 
TOTAL & 35& 1173004&&&\\ \hline 
\end{tabular}
\end{center}

\begin{itemize}
\item The upper 5\% point of F2,33 is about 3.3; the treatments effect is significant. There is evidence to reject the null hypothesis that all treatments have the same effect.
Solution continued on the next page
\item To investigate treatment differences, first calculate the treatment means, which are (in ascending order, for clarity)
Work : 292.58 Low : 312.83 High : 460.67
\item The least significant difference between any pair of these means is
333322942270.02612tt×= where 332.035at5%2.736at1%3.617at0.1%t⎧⎪=⎨⎪⎩
so the least significant differences are 142.50 for 5\%, 191.59 for 1\% and 253.28 for 0.1\%. 
\item Thus the only apparent difference is that "high" gives a larger mean response than "low" and "work", judged at the 5\% level; "low" and "work" do not differ.
\end{itemize}

\begin{framed}
Report
After carrying out an analysis which compares group means against internal variability of responses in the groups, we find some evidence that "high" shows more persistence than the other two groups, whose results are quite similar. The within-group variability is very high.
\end{framed}
\end{enumerate}
\end{document}

\documentclass[a4paper,12pt]{article}

%%%%%%%%%%%%%%%%%%%%%%%%%%%%%%%%%%%%%%%%%%%%%%%%%%%%%%%%%%%%%%%%%%%%%%%%%%%%%%%%%%%%%%%%%%%%%%%%%%%%%%%%%%%%%%%%%%%%%%%%%%%%%%%%%%%%%%%%%%%%%%%%%%%%%%%%%%%%%%%%%%%%%%%%%%%%%%%%%%%%%%%%%%%%%%%%%%%%%%%%%%%%%%%%%%%%%%%%%%%%%%%%%%%%%%%%%%%%%%%%%%%%%%%%%%%%

\usepackage{eurosym}
\usepackage{vmargin}
\usepackage{amsmath}
\usepackage{graphics}
\usepackage{epsfig}
\usepackage{enumerate}
\usepackage{multicol}
\usepackage{subfigure}
\usepackage{fancyhdr}
\usepackage{listings}
\usepackage{framed}
\usepackage{graphicx}
\usepackage{amsmath}
\usepackage{chngpage}

%\usepackage{bigints}
\usepackage{vmargin}

% left top textwidth textheight headheight

% headsep footheight footskip

\setmargins{2.0cm}{2.5cm}{16 cm}{22cm}{0.5cm}{0cm}{1cm}{1cm}

\renewcommand{\baselinestretch}{1.3}

\setcounter{MaxMatrixCols}{10}

\begin{document}
Higher Certificate, Paper II, 2005. Question 4
\begin{enumerate}[(a)]
    \item McNemar's test is required because the samples are paired.
Denoting the entries in the table by , the test statistic for McNemar's test is abcd()21bcbc−−+, with approximate null distribution , the null hypothesis here being that there is no difference between the proportions (probabilities) for the MAT and ELISA tests. (Notice that McNemar's test uses the information from the "discordant" cells of the table.) 21χ
Thus the test statistic is ()2225411153.409254166−−==+. This is referred to ; the upper 5\% point is 3.84, so there is insufficient evidence to say that there is a real difference. 21χ
    \item  Approximate 95\% confidence intervals for the proportion of positive test results given by each test use the whole data. For MAT, 92462ˆ0.1991Mp==; for ELISA, 108462ˆ0.2338Ep==.
    
\begin{enumerate}[(i)]
    \item  The estimated variance of ˆMp is (0.1991)(0.8009)/462 = 0.000345135, so the estimated standard deviation is 0.0186. Thus a 95\% confidence interval for pM is given by, approximately, $0.1991 \pm (1.96)(0.0186)$, i.e. it is (0.163, 0.236).
\item The estimated variance of ˆEp is (0.2338)(0.7662)/462 = 0.000387744, so the estimated standard deviation is 0.0197 Thus a 95\% confidence interval for pE is given by, approximately, $0.2338 \pm (1.96)(0.0197)$, i.e. it is (0.195, 0.272).
Neither of these intervals contains the proposed value of 0.069 – in fact, the intervals are a considerable distance away from that. So neither is consistent with this value.
\end{enumerate}

\end{enumerate}
\end{document}

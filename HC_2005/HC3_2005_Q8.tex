Higher Certificate, Paper III, 2005. Question 8
An obvious characteristic of the data is the difference between length and width. The "squares" are in every position rectangles, and the range of length sizes in any position (1 to 7) does not even overlap with the range of width sizes. Length is lower in positions 6 and 7 across the tray than it is elsewhere, and width is highest in position 7. Both length and width measurements are about equally variable in all positions.
Height is much more variable, especially in relation to its mean size. It also increases fairly steadily from position 1 to 7 (though 5 goes against this trend), with position 7 being particularly variable, due perhaps to one or two very large values (max 38).
There could be a temperature gradient in the oven, related to width, which affects height, and some other trends which result in length and width not being the same although the original material was presumably squarely placed.
If the appearance and uniformity of the "square" product are important, some attention needs to be given to the operation of the oven.
Summary of average (mean) and range (maximum – minimum in the whole data):
Average Range
Length 86.2 8
Width 77.5 11
Height 29.2 14
The combined effect on volume is to produce larger values in positions 6 and 7, with 1 to 5 showing an increase followed by a decrease. High variability is noted in 7, and fairly high in 2.
Data were collected just after removal from the oven. It is quite possible that after cooling some of the characteristics measured would have settled down more. We might usefully be told how many people were involved in measuring the data, as there could have been a time effect while collecting it.
Solution continued on next page
One useful diagram is to show the mean measurements against pos-w: 765432190807060504030pos-wmean height/width/length
width
length
height
The above figure shows all three sets of data on the same scale. This is very useful in comparing length and width, but putting height on the same diagram hides the detail of the changes in the others because of the vertical scale. Two separate diagrams might be better.
The figure below shows an interesting comparison – volume depends quite closely on height. The numbers in brackets show positions across the width of the tray, 1 to 7. 28293031190000200000210000mean heightmean volume(1)(2)(5)(4)(3)(6)(7)
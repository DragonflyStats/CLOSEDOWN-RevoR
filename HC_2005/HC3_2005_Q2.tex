\documentclass[a4paper,12pt]{article}

%%%%%%%%%%%%%%%%%%%%%%%%%%%%%%%%%%%%%%%%%%%%%%%%%%%%%%%%%%%%%%%%%%%%%%%%%%%%%%%%%%%%%%%%%%%%%%%%%%%%%%%%%%%%%%%%%%%%%%%%%%%%%%%%%%%%%%%%%%%%%%%%%%%%%%%%%%%%%%%%%%%%%%%%%%%%%%%%%%%%%%%%%%%%%%%%%%%%%%%%%%%%%%%%%%%%%%%%%%%%%%%%%%%%%%%%%%%%%%%%%%%%%%%%%%%%

\usepackage{eurosym}
\usepackage{vmargin}
\usepackage{amsmath}
\usepackage{graphics}
\usepackage{epsfig}
\usepackage{enumerate}
\usepackage{multicol}
\usepackage{subfigure}
\usepackage{fancyhdr}
\usepackage{listings}
\usepackage{framed}
\usepackage{graphicx}
\usepackage{amsmath}
\usepackage{chngpage}

%\usepackage{bigints}
\usepackage{vmargin}

% left top textwidth textheight headheight

% headsep footheight footskip

\setmargins{2.0cm}{2.5cm}{16 cm}{22cm}{0.5cm}{0cm}{1cm}{1cm}

\renewcommand{\baselinestretch}{1.3}

\setcounter{MaxMatrixCols}{10}

\begin{document}Higher Certificate, Paper III, 2005. Question 2
%%%%%%%%%%%%%%%%%%%%%%%%%%%%%%%%%%%%%%%%%%%%%%%%%%%%%%%%%%%%%
2.
(i)
An oil company wishes to test a new additive, which they think will decrease
petrol consumption (and believe cannot increase it). An experiment is carried
out in which miles per litre are recorded for different makes of the same type
of car. Twelve cars are selected and divided randomly into two groups of six.
Group A cars use petrol with the additive and Group B cars use petrol without
the additive. The results are:
\begin{center}
\begin{tabular}{|c|c|}
A & \{6.55 ,   8.78 ,10.80 ,9.05 , 7.35, 9.38\} \\

B & \{9.23,10.55 ,  6.73,
7.55 ,
8.23 ,
7.72 \}
\end{tabular}
\end{center}

(a) Test whether there is evidence that the additive increases the mean
mileage per litre. State any assumptions you make.
(6)
(b) Now consider the case where each sample consists of n cars, where n is large, yielding sample means x A and x B . The company wishes to be
95% confident of detecting an increase in mean of 0.5 of a mile per litre when using the additive. Let the mean fuel consumption using petrol
with additive be μ A and the mean fuel consumption using petrol without additive be μ B . Write down a probability statement that represents the
company's objective.

Assuming that the data to hand give a reasonable estimate of the sampling variance of petrol consumption, calculate how large n must
be to meet the company's objective.
\end{framed}
\begin{framed}
(ii)
The Director of Research and Development at the oil company decides that he is not satisfied with the design of the experiment in (i). He decides to conduct
a new experiment in which the same six cars in Group A above are tested again using petrol without the additive. The results obtained are:
With additive
Without additive
6.55
6.15
8.78
7.73
10.80
10.34
9.05
8.16
7.35
7.27
9.38
8.02
(a) Explain why this type of experimental design is better than the one
used in (i). Describe any other improvements you might make if
another experiment was being planned.
(3)
(b) Determine whether or not there is evidence that the additive increases
mean miles per litre.
(5)
3
\end{framed}
%%%%%%%%%%%%%%%%%%%%%%%%%%%%%%%%%%%%%%%%%%%%%%%%%%%%%%%%%%%%%%%%%%%%%%%%%%%%%%%%%%%
\begin{enumerate}
\item If we assume that the two samples of cars are drawn at random from populations having the same variance, and petrol consumption can be assumed Normally distributed, the means of A and B can be compared by a two-sample t test.
The null hypothesis is that the two population means are the same, and the alternative is that mean A > mean B. So this is a one-sided test.
6,6ABnn==;
8.652,8.335;ABxx==
, 22.2865As=21.8578.Bs=
To test for equality of population variances, consider sA2/sB2 = 1.23. This is not significant as an observation from F5,5 so it is reasonable to take the population variances as equal.
The pooled estimate of the common variance is s2 = 2.0722, with 10 d.f.
Thus the test statistic for testing ABμμ= is
1166(0)0.3170.380.831ABxxs−−==+,
which is referred to t10. This is not significant at the 5% level (upper single-tailed 5% point is 1.812), so there is no evidence to reject the null hypothesis – it seems that the population means are the same.
\item Assuming the samples will be reasonably large, we would base the test on the Normal distribution and use test statistic 11ABnnxxzs−=+ where n is the size of each sample and s will be taken as √2.0722 = 1.439. The null hypothesis is rejected (at the 5% level) if the value of z is >1.645. We wish to have probability 0.95 that this will happen if in fact μA – μB = 0.5. Thus we require
20.951.6450.5ABABnXXPsμμ⎛⎞−=>−⎜⎟⎜⎟⎝⎠
= ()21.6450.5ABABnPXXsμμ−>−=.
The underlying distribution of A XX− is taken as 22N,ABnsμμ⎛⎞⎡⎤−⎜⎟⎣⎦⎝⎠, i.e. here it is 22N0.5,ns⎛⎡⎤⎜⎣⎦⎝⎠. So we require
2220.95N0.5,1.645nnPss⎛⎞⎛⎞⎡⎤=>⎜⎟⎜⎟⎣⎦⎝⎠⎝⎠ = ()221.6450.5N0,1)nnsPs⎛⎞−>⎜⎟⎜⎟⎝⎠.
But 0.95 = P(N(0, 1) > –1.645).
So 221.6451.6450.5nnss−= leading to
20.53.29ns= i.e. 3.2920.5sn= i.e. 22(3.29)2.07222(0.5)n××=,
i.e. n = 179.44. Thus n = 180 is required for each group, or 360 in total.
\item  (a) We can now eliminate any systematic differences between cars, and hope to obtain more precise results.
In each pair, the order of using with/without should be chosen at random to avoid time differences. Perhaps the same driver could be used for all, or at least limit the number and design the experiment to balance order and drivers.
\item  We use the paired-sample t test for the differences (6.55 – 6.15 etc); these are 0.40, 1.05, 0.46, 0.89, 0.08, 1.36. We assume that these differences are a sample from a Normal distribution.
We have 0.707dx= and . Thus the test statistic is 20.2252ds=
00.7073.650.194/5ddxs−==,
which is referred to t5. This is significant at the 1% level (upper single-tailed 5% point is 3.365), so there is strong evidence to reject the null hypothesis – it seems that the population means are not the same and that the additive is beneficial.
\end{enumerate}
\end{document}

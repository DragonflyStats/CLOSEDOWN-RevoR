\documentclass[a4paper,12pt]{article}

%%%%%%%%%%%%%%%%%%%%%%%%%%%%%%%%%%%%%%%%%%%%%%%%%%%%%%%%%%%%%%%%%%%%%%%%%%%%%%%%%%%%%%%%%%%%%%%%%%%%%%%%%%%%%%%%%%%%%%%%%%%%%%%%%%%%%%%%%%%%%%%%%%%%%%%%%%%%%%%%%%%%%%%%%%%%%%%%%%%%%%%%%%%%%%%%%%%%%%%%%%%%%%%%%%%%%%%%%%%%%%%%%%%%%%%%%%%%%%%%%%%%%%%%%%%%

\usepackage{eurosym}
\usepackage{vmargin}
\usepackage{amsmath}
\usepackage{graphics}
\usepackage{epsfig}
\usepackage{enumerate}
\usepackage{multicol}
\usepackage{subfigure}
\usepackage{fancyhdr}
\usepackage{listings}
\usepackage{framed}
\usepackage{graphicx}
\usepackage{amsmath}
\usepackage{chngpage}

%\usepackage{bigints}
\usepackage{vmargin}

% left top textwidth textheight headheight

% headsep footheight footskip

\setmargins{2.0cm}{2.5cm}{16 cm}{22cm}{0.5cm}{0cm}{1cm}{1cm}

\renewcommand{\baselinestretch}{1.3}

\setcounter{MaxMatrixCols}{10}

\begin{document}
Higher Certificate, Paper II, 2005. Question 2
\begin{enumerate}
\item 
(i) The ranked data are as follows, with the median and the lower and upper quartiles underlined. [Note. Some slightly different definitions of quartiles are also in use. These would make, at most, only a small difference here.]
1, 1, 2, 3, 3, 4, 4, 5, 5, 6, 6, 6, 7, 7, 8, 9, 14, 17, 19, 19, 20, 44, 82.
Q1 M Q3
We have 1.5(Q3 – Q1) = 1.5(17 – 4) = 19.5. So 44 and 82 may be "outliers". These are indicated by stars in the box and whisker plot.
Length of
stay (days)
0
20
40
60
80
100
Even apart from the two outliers, the distribution is very skew to the right. This may be explained by some of the admissions being in serious enough condition to need extra care.
\item  Here a t test would be used to examine the hypothesis about the mean duration of bed occupancy. It relies on the data being a sample from a Normal distribution, at least approximately, but this assumption is clearly not valid here. More generally, because of the skewness of the underlying distribution, inferences based on the mean (and standard deviation) of a sample will be unreliable unless a very large sample is available. Any statistical test based on the mean of a small sample will be worthless. Even the sample of size 100 in part \item  is not really "large" for a case so skew as this.
The solution to part (iii) is on the next page
\item  We regard the sample of size 100 as being "large" and invoke the Central Limit Theorem so as to use a test based on N(0, 1) [a test based on t99 could also be reasonably justified; t99 is very close to N(0, 1)].
We have 14.88x= and 221148844632227.177499100s⎛⎞=−=⎜⎟⎝⎠.
The null hypothesis is that μ = 14, the alternative hypothesis is μ > 14, where μ is the (population) mean duration of bed occupation.
The test statistic is 14.88140.880.5841.507227.1774100−==, which is clearly not significant as an observation from N(0, 1). There is no evidence that the mean duration is greater than 14 days.
As discussed in parts (i) and (ii), it is clear from the original 23 items of data that the underlying distribution is very skew. Even with a sample of size 100, the result of the test should not be taken as very reliable. (Another illustration of this is provided by calculating a (say) 95% confidence interval in the usual way: 1.961.507x±× gives the interval (11.93, 17.83), which is wide for a sample of this size, indicating imprecise results.) The real problem is that the mean does not give useful information about the "typical" length of stay. There is a wider question as to whether "performance" is validly measured by length of stay in any case.
\end{enumerate}
\end{document}

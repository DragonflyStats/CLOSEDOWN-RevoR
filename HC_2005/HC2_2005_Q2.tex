\documentclass[a4paper,12pt]{article}

%%%%%%%%%%%%%%%%%%%%%%%%%%%%%%%%%%%%%%%%%%%%%%%%%%%%%%%%%%%%%%%%%%%%%%%%%%%%%%%%%%%%%%%%%%%%%%%%%%%%%%%%%%%%%%%%%%%%%%%%%%%%%%%%%%%%%%%%%%%%%%%%%%%%%%%%%%%%%%%%%%%%%%%%%%%%%%%%%%%%%%%%%%%%%%%%%%%%%%%%%%%%%%%%%%%%%%%%%%%%%%%%%%%%%%%%%%%%%%%%%%%%%%%%%%%%

\usepackage{eurosym}
\usepackage{vmargin}
\usepackage{amsmath}
\usepackage{graphics}
\usepackage{epsfig}
\usepackage{enumerate}
\usepackage{multicol}
\usepackage{subfigure}
\usepackage{fancyhdr}
\usepackage{listings}
\usepackage{framed}
\usepackage{graphicx}
\usepackage{amsmath}
\usepackage{chngpage}

%\usepackage{bigints}
\usepackage{vmargin}

% left top textwidth textheight headheight

% headsep footheight footskip

\setmargins{2.0cm}{2.5cm}{16 cm}{22cm}{0.5cm}{0cm}{1cm}{1cm}

\renewcommand{\baselinestretch}{1.3}

\setcounter{MaxMatrixCols}{10}

\begin{document}
Higher Certificate, Paper II, 2005. Question 2
\begin{framed}
 A hospital is interested in whether its stroke admissions are using more than the recommended average bed occupation time of 14 days in acute care for those discharged alive.  The hospital audit department supplies you with bed occupation data from a random sample of 23 such discharges, as follows (in days). 
 
44 20   7 19 14   6   5   3   8   6   6   5   1   4   2   9 17   1   3   7 82   4 19  
 
 
(i) Find the median (M) and the quartiles (Q1 and Q3) of the data.  Using the convention that any observation lying further than 1.5(Q3 – Q1) beyond the nearest quartile is an "outlier", draw a box and whisker plot of the bed occupation data.  Hence comment on the shape of the distribution. (8) 
 \end{framed}
 

\begin{enumerate}
\item 
(i) The ranked data are as follows, with the median and the lower and upper quartiles underlined. [Note. Some slightly different definitions of quartiles are also in use. These would make, at most, only a small difference here.]
1, 1, 2, 3, 3, 4, 4, 5, 5, 6, 6, 6, 7, 7, 8, 9, 14, 17, 19, 19, 20, 44, 82.
Q1 M Q3
We have 1.5(Q3 – Q1) = 1.5(17 – 4) = 19.5. So 44 and 82 may be "outliers". These are indicated by stars in the box and whisker plot.
Length of
stay (days)
0
20
40
60
80
100
Even apart from the two outliers, the distribution is very skew to the right. This may be explained by some of the admissions being in serious enough condition to need extra care.
\item  Here a t test would be used to examine the hypothesis about the mean duration of bed occupancy. It relies on the data being a sample from a Normal distribution, at least approximately, but this assumption is clearly not valid here. More generally, because of the skewness of the underlying distribution, inferences based on the mean (and standard deviation) of a sample will be unreliable unless a very large sample is available. Any statistical test based on the mean of a small sample will be worthless. Even the sample of size 100 in part (i) is not really "large" for a case so skew as this.
%%%%%%%%%%%%%%%%%%%%%%%%%%%%%%%%%%%%%%%%%%%%%%%%%%%%%%%
\newpage 
\begin{framed}
(ii) The audit department wishes to have a test of the hypothesis that the mean duration of bed occupation is not greater than 14 days.  Give a brief justification of the need for a larger sample of data in order to make a valid test of this hypothesis. (3) 
 
(iii) The audit department now supplies you with a sample of 100 such admissions, for whom the duration of bed occupation can be summarised by 
 Σx = 1488,   Σx2 = 44632. 
 
Test the null hypothesis that the mean duration of bed occupation is 14 days against the alternative that it exceeds 14 days. 
 
Discuss critically how reliable this result may be, and whether it is a useful measure of the hospital's performance. (9) 
 \end{framed}
 \item  We regard the sample of size 100 as being "large" and invoke the Central Limit Theorem so as to use a test based on N(0, 1) [a test based on t99 could also be reasonably justified; t99 is very close to N(0, 1)].
We have 14.88x= and 221148844632227.177499100s⎛⎞=−=⎜⎟⎝⎠.
The null hypothesis is that μ = 14, the alternative hypothesis is μ > 14, where μ is the (population) mean duration of bed occupation.

\begin{itemize}
\item The test statistic is 14.88140.880.5841.507227.1774100−==, which is clearly not significant as an observation from N(0, 1). There is no evidence that the mean duration is greater than 14 days.
\item As discussed in parts (i) and (ii), it is clear from the original 23 items of data that the underlying distribution is very skew. Even with a sample of size 100, the result of the test should not be taken as very reliable. 
\item (Another illustration of this is provided by calculating a (say) 95\% confidence interval in the usual way: 1.961.507x±× gives the interval (11.93, 17.83), which is wide for a sample of this size, indicating imprecise results.) 
\item The real problem is that the mean does not give useful information about the "typical" length of stay. There is a wider question as to whether "performance" is validly measured by length of stay in any case.
\end{itemize}

\end{enumerate}
\end{document}

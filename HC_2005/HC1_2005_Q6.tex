\documentclass[a4paper,12pt]{article}

%%%%%%%%%%%%%%%%%%%%%%%%%%%%%%%%%%%%%%%%%%%%%%%%%%%%%%%%%%%%%%%%%%%%%%%%%%%%%%%%%%%%%%%%%%%%%%%%%%%%%%%%%%%%%%%%%%%%%%%%%%%%%%%%%%%%%%%%%%%%%%%%%%%%%%%%%%%%%%%%%%%%%%%%%%%%%%%%%%%%%%%%%%%%%%%%%%%%%%%%%%%%%%%%%%%%%%%%%%%%%%%%%%%%%%%%%%%%%%%%%%%%%%%%%%%%

\usepackage{eurosym}
\usepackage{vmargin}
\usepackage{amsmath}
\usepackage{graphics}
\usepackage{epsfig}
\usepackage{enumerate}
\usepackage{multicol}
\usepackage{subfigure}
\usepackage{fancyhdr}
\usepackage{listings}
\usepackage{framed}
\usepackage{graphicx}
\usepackage{amsmath}
\usepackage{chngpage}

%\usepackage{bigints}
\usepackage{vmargin}

% left top textwidth textheight headheight

% headsep footheight footskip

\setmargins{2.0cm}{2.5cm}{16 cm}{22cm}{0.5cm}{0cm}{1cm}{1cm}

\renewcommand{\baselinestretch}{1.3}

\setcounter{MaxMatrixCols}{10}

\begin{document}Higher Certificate, Paper I, 2005. Question 6
Probability mass function: ( ) (1 ) , 0,1, 2, ... , 0 1 x f x = − p p x = < p < .
\begin{enumerate}
    \item Probability generating function G(s) is
( ) ( ) {( ) } ( ) 0 0
1 1
1 1
X x x x
x x
G s E s s p p p p s p
p s
∞ ∞
= =
=   = − = − = − − Σ Σ
(requires s <1/(1− p) for convergence).
The mean is given by E[X] = G'(1). We have ( )
{ ( ) }2
1
1 1
G s p p
p s
′ = −
− −
and inserting
s = 1 gives ( )
2
1
(1)
p p
G
p
′ − = , i.e. the mean is 1 p
p
− .
The variance is given by Var(X) = G''(1) + mean – mean2. ( ) ( )
{ ( ) }
2
3
2 1
1 1
p p
G s
p s
−
′′ =
− −
, so
that ( ) ( )2
2
2 1
1
p
G
p
−
′′ = . Thus the variance is given by ( ) 2 2
2
2 1 p 1 p 1 p
p p p
− −  −  + − 
 
{( )2 ( )}
2 2
1 1 p p 1 p 1 p
p p
= − + − = − .

1/3
2/9
4/27
P(X = x)
0 1 2 3 4 5
x
%%%%%%%%%%%%%%%%%%%%%%%%%%%%%%%%%%%%%%%%%%%%%%%%%%%%%%
\item  ( ) ( ) ( )
( ) ( ) 1
1 [geometric series] 1
1 1
x
r x
r x
p p
P X x p p p
p
∞
=
−
≥ = − = = −
− − Σ (for x =
0, 1, 2, …). We now use ( ) ( )
( )
P A B
P AB
P B
∩
= and take the event A as "X ≥ l + m"
and the event B as "X ≥ l", so that A∩B = A. Thus
( ) ( )
( )
( ) ( ) 1
1
1
l m
m
l
p
P X l mX l p P X m
p
+ −
≥ + ≥ = = − = ≥
−
.
This is the "lack of memory" property of a geometric distribution.
\item  By independence, ( ) ( ) ( ) (1 ) (1 ) z z P Z ≥ z = P X ≥ z P Y ≥ z = − p −θ .
( ) ( ) ( ) {( )( )} {( )( )} 1 1 1 1 1 1 z z P Z z P Z z P Z z p θ p θ + = = ≥ − ≥ + = − − − − −
{(1 )(1 )} (1 1 ) {(1 )(1 )} ( ) z z = − p −θ − + p +θ − pθ = − p −θ p +θ − pθ , for z = 0, 1, … .
\begin{itemize}
\item This is a geometric distribution as given at the start of the question with p replaced by
p + θ – pθ. 
\item Hence, from part (ii),
E[Z] 1 p p
p p
θ θ
θ θ
= − − +
+ −
, 
\item Var(Z) =
( )2
1 p p
p p
θ θ
θ θ
− − +
+ −
\end{itemize}
.

\end{enumerate}

\end{document}

\documentclass[a4paper,12pt]{article}

%%%%%%%%%%%%%%%%%%%%%%%%%%%%%%%%%%%%%%%%%%%%%%%%%%%%%%%%%%%%%%%%%%%%%%%%%%%%%%%%%%%%%%%%%%%%%%%%%%%%%%%%%%%%%%%%%%%%%%%%%%%%%%%%%%%%%%%%%%%%%%%%%%%%%%%%%%%%%%%%%%%%%%%%%%%%%%%%%%%%%%%%%%%%%%%%%%%%%%%%%%%%%%%%%%%%%%%%%%%%%%%%%%%%%%%%%%%%%%%%%%%%%%%%%%%%

\usepackage{eurosym}
\usepackage{vmargin}
\usepackage{amsmath}
\usepackage{graphics}
\usepackage{epsfig}
\usepackage{enumerate}
\usepackage{multicol}
\usepackage{subfigure}
\usepackage{fancyhdr}
\usepackage{listings}
\usepackage{framed}
\usepackage{graphicx}
\usepackage{amsmath}
\usepackage{chngpage}

%\usepackage{bigints}
\usepackage{vmargin}

% left top textwidth textheight headheight

% headsep footheight footskip

\setmargins{2.0cm}{2.5cm}{16 cm}{22cm}{0.5cm}{0cm}{1cm}{1cm}

\renewcommand{\baselinestretch}{1.3}

\setcounter{MaxMatrixCols}{10}

\begin{document}Higher Certificate, Paper I, 2005. Question 5

%%%%%%%%%%%%%%%%%%%%%%%%%%%%%%%%%%%%%%%%%%%%%%%%%%%%%%%%%%%%%%%%%%
\begin{framed}
5. (i) A social scientist is conducting a survey of drug-taking among students. Her
survey is based on face-to-face interviews with a random sample of n students
from Wackford Squeers University, y of whom tell her that they take drugs.
Assume that the true proportion of drug-takers in the population being sampled
is p, that the responses are truthful, and that the number in the sample who take
drugs follows the binomial distribution B(n, p). Write down the likelihood
function for these data and find the maximum likelihood estimator (MLE) of p,
pˆ say. Also write down Var( pˆ ) and hence obtain an estimate of the standard
error of pˆ , SE( pˆ ) say. Calculate the values of pˆ and SE( pˆ ), given that
n = 100 and y = 20.
\end{framed}
%----------------------------------------------------------------%

%%%%%%%%%%%%%%%%%%%%%%%%%%%%%%%%%%%%%%%%%%%%%%%%%%%%%%%%%%%%%%%%%%
\begin{enumerate}[(a)]
\item We have Y ~ B(n, p), so ( ) ( ) ! (1 )

 probability mass function:

\[{\displaystyle f(y,n,p)=\Pr(k;n,p)=\Pr(Y=y)={\binom {n}{y}}p^{y}(1-p)^{n-y}} \]
for $y = 0, 1, 2, ..., n$, and $0 < p < 1$. 
\begin{itemize}
\item
The likelihood L is simply P(Y = y).
\item
Hence \[\log L = \mbox{constant} + y \log p + (n − y)\log(1− p).\]
\item 
log
1
d L y n y
dp p p
− ∴ = −−
which on setting equal to zero gives that the maximum
likelihood estimate is pˆ y
n
= . 

\item Consideration of
$ \frac{d^2(\log L)}{dp}2

confirms that this is a
maximum.

\item 
( ) ( ) ( ) ( )
2 2
1 1 1 Var ˆ Var 1
p p
p Y np p
n n n
−
= = − = . 
\item We may estimate p by $\hat{p}$ in this and
thus obtain an estimate of the standard error of $\hat{p}$ as 
\[SE (\hat{p}) = \sqrt{\frac{\hat{p} \times (1-\hat{p}) }{n}.\]
\item 
For $n = 100$ and $y = 20$, we have $\hat{p} = 0.2$ and 
\[SE( \hat{p} )  =  \sqrt{ \frac{0.2 \times 0.8}{100} } = 0.04\]
\end{itemize}
%%%%%%%%%%%%%%%%%%%%%%%%%%%%%%%%%%%%%%%%%%%%%%%%%%%%%%%%%%%%%%%%%%%
\begin{framed}
(ii) A statistician now advises her that, because of the sensitivity of the question of
drug-taking, some students may not answer truthfully, so causing bias.
\begin{itemize}
    \item He
suggests using an anonymising device to encourage truthful answers. 
\item He
provides a biased coin with faces labelled "takes drugs" and "does not take
drugs", which show with respective probabilities 0.75 and 0.25 when the coin
is tossed.
\item A second random sample, also of size n, is selected independently of the first.
 
\item Each of the n students is asked to toss the coin, unseen by the interviewer, and
to answer "yes" if he/she is in the group indicated by the coin and "no"
otherwise.
\end{itemize}
Assuming truthful responses, show that the probability of a "yes", \theta say, is
given by
\theta = 0.25 + 0.5p .
Assuming further that the number of students answering "yes", Z say, is
distributed B(n, \theta ) with observed value z, write down the MLE of \theta, \thetaˆ say,
and an estimate of its standard error. Use the relationship between \theta and p to
deduce the MLE of p, p ~ say, and SE(p ~ ). Calculate the values of p ~ and
SE(p ~ ), given that n = 100 and z = 45.
\end{framed}
%----------------------------------------------------------------%

\item P("yes") is given by P(takes drugs and coin shows "takes drugs") + P(does not
take drugs and coin shows "does not take drugs").
\begin{itemize}
    \item Hence \theta = P("yes") = 0.75p + 0.25(1 – p) = 0.25 + 0.5p.
    \item Z ~ B(n, \theta ), so from part (i) we have \hat{theta} = z/n and SE(\hat{theta}) = \hat{theta} (1−\hat{theta})/ n .
    \item We have p = 2\theta – ½, so the MLE of p is 1
2
p�� = 2\hat{theta} − .
    \item Thus SE( p�� ) = 2 SE(\hat{theta}) = 2 \hat{theta} (1−\hat{theta})/ n .
    \item For n = 100 and z = 45, we have \hat{theta} = 0.45 and so p�� = 0.4 and SE( p�� ) = 0.0995 .
\end{itemize}
%%%%%%%%%%%%%%%%%%%%%%%%%%%%%%%%%%%%%%%%%%%%%%%%%%%%%%%%%%%%%%%%%%%
\newpage

\begin{framed}
(iii) A journalist comparing the results of the two surveys says that the first survey
is more reliable. Do you agree? Why, or why not?

\end{framed}
%----------------------------------------------------------------%
\item Both p and the standard error are estimated to be larger by the second survey.
\begin{itemize}
    \item The larger p is plausible, as some people are likely not to admit to taking drugs when
asked directly as in the first survey.
    \item There is a much better expectation of truthful
answers in the second survey.
    \item We should not claim that the first is better just because
the SE is smaller; the first survey is likely to be biased. 
    \item Is it clear what the journalist
means by "reliable"?
\end{itemize}

\end{enumerate}

\end{document}

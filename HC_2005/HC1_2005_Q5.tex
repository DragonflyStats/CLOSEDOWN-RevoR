\documentclass[a4paper,12pt]{article}

%%%%%%%%%%%%%%%%%%%%%%%%%%%%%%%%%%%%%%%%%%%%%%%%%%%%%%%%%%%%%%%%%%%%%%%%%%%%%%%%%%%%%%%%%%%%%%%%%%%%%%%%%%%%%%%%%%%%%%%%%%%%%%%%%%%%%%%%%%%%%%%%%%%%%%%%%%%%%%%%%%%%%%%%%%%%%%%%%%%%%%%%%%%%%%%%%%%%%%%%%%%%%%%%%%%%%%%%%%%%%%%%%%%%%%%%%%%%%%%%%%%%%%%%%%%%

\usepackage{eurosym}
\usepackage{vmargin}
\usepackage{amsmath}
\usepackage{graphics}
\usepackage{epsfig}
\usepackage{enumerate}
\usepackage{multicol}
\usepackage{subfigure}
\usepackage{fancyhdr}
\usepackage{listings}
\usepackage{framed}
\usepackage{graphicx}
\usepackage{amsmath}
\usepackage{chngpage}

%\usepackage{bigints}
\usepackage{vmargin}

% left top textwidth textheight headheight

% headsep footheight footskip

\setmargins{2.0cm}{2.5cm}{16 cm}{22cm}{0.5cm}{0cm}{1cm}{1cm}

\renewcommand{\baselinestretch}{1.3}

\setcounter{MaxMatrixCols}{10}

\begin{document}Higher Certificate, Paper I, 2005. Question 5

\begin{enumerate}
\item We have Y ~ B(n, p), so ( ) ( ) ! (1 )
! !
P Y y n py p n y
y n y
− = = −
−
(for y = 0, 1, 2, …, n
and 0 < p < 1). 
\begin{itemize}
\item
The likelihood L is simply P(Y = y).
\item
Hence \[\log L = constant + y \log p + (n − y)\log(1− p).\]
\item 
log
1
d L y n y
dp p p
− ∴ = −−
which on setting equal to zero gives that the maximum
likelihood estimate is pˆ y
n
= . 

\item Consideration of
2
2
d log L
dp
confirms that this is a
maximum.

\item 
( ) ( ) ( ) ( )
2 2
1 1 1 Var ˆ Var 1
p p
p Y np p
n n n
−
= = − = . 
\item We may estimate p by \hat{p} in this and
thus obtain an estimate of the standard error of pˆ as ( ) ( ) ˆ 1 ˆ
SE ˆ
p p
p
n
−
= .
\item 
For n = 100 and y = 20, we have $\hat{p} = 0.2$ and 
\[SE( \hat{p} )  =  \sqrt{ \frac{0.2 \times 0.8}{100} } = 0.04\]
\end{itemize}
\item P("yes") is given by P(takes drugs and coin shows "takes drugs") + P(does not
take drugs and coin shows "does not take drugs").
\begin{itemize}
    \item Hence θ = P("yes") = 0.75p + 0.25(1 – p) = 0.25 + 0.5p.
    \item Z ~ B(n, θ ), so from part (i) we have \hat{theta} = z/n and SE(\hat{theta}) = \hat{theta} (1−\hat{theta})/ n .
    \item We have p = 2θ – ½, so the MLE of p is 1
2
p�� = 2\hat{theta} − .
    \item Thus SE( p�� ) = 2 SE(\hat{theta}) = 2 \hat{theta} (1−\hat{theta})/ n .
    \item For n = 100 and z = 45, we have \hat{theta} = 0.45 and so p�� = 0.4 and SE( p�� ) = 0.0995 .
\end{itemize}

\item Both p and the standard error are estimated to be larger by the second survey.
\begin{itemize}
    \item The larger p is plausible, as some people are likely not to admit to taking drugs when
asked directly as in the first survey.
    \item There is a much better expectation of truthful
answers in the second survey.
    \item We should not claim that the first is better just because
the SE is smaller; the first survey is likely to be biased. 
    \item Is it clear what the journalist
means by "reliable"?
\end{itemize}

\end{enumerate}

\end{document}

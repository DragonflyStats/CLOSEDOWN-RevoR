\documentclass[a4paper,12pt]{article}

%%%%%%%%%%%%%%%%%%%%%%%%%%%%%%%%%%%%%%%%%%%%%%%%%%%%%%%%%%%%%%%%%%%%%%%%%%%%%%%%%%%%%%%%%%%%%%%%%%%%%%%%%%%%%%%%%%%%%%%%%%%%%%%%%%%%%%%%%%%%%%%%%%%%%%%%%%%%%%%%%%%%%%%%%%%%%%%%%%%%%%%%%%%%%%%%%%%%%%%%%%%%%%%%%%%%%%%%%%%%%%%%%%%%%%%%%%%%%%%%%%%%%%%%%%%%

\usepackage{eurosym}
\usepackage{vmargin}
\usepackage{amsmath}
\usepackage{graphics}
\usepackage{epsfig}
\usepackage{enumerate}
\usepackage{multicol}
\usepackage{subfigure}
\usepackage{fancyhdr}
\usepackage{listings}
\usepackage{framed}
\usepackage{graphicx}
\usepackage{amsmath}
\usepackage{chngpage}

%\usepackage{bigints}
\usepackage{vmargin}

% left top textwidth textheight headheight

% headsep footheight footskip

\setmargins{2.0cm}{2.5cm}{16 cm}{22cm}{0.5cm}{0cm}{1cm}{1cm}

\renewcommand{\baselinestretch}{1.3}

\setcounter{MaxMatrixCols}{10}

\begin{document}
Higher Certificate, Paper I, 2005. Question 2
Cycle ~ N(27, 6.25)
Bus ~ N(13, 20) Walk1 ~ N(7, 4) Walk2 ~ N(5, 1)
Car ~ N(23, 36)
The sum of independent N( μ
i, σ
i 2) distributions is N(Σ μ
i, Σ σ
i 2).

\begin{enumerate}

\item The distribution of total journey time by bus is N(7+13+5, 4+20+1), i.e. N(25,25).


\item

\begin{description}
\item[Cycle: ]
\[P (N(27,6.25) < 30)  = \Phi \left( \frac{30 27}{\sqrt{6.25}} \right)  = \Phi(1.2) = 0.8849\]

\item[Bus:]
\[P (N(25,25) < 30)  = \Phi \left( \frac{30- 25}{\sqrt{25}} \right)  = \Phi(1) = 0.8413\]
\item[Car:]
\[P (N(23,26) < 30)  = \Phi \left( \frac{30- 26}{\sqrt{26}} \right)  = \Phi(1.1667) = 0.8783\]

\end{description}
.
Cycling is best, with a probability of 0.8849.

\item 
\begin{description}
\item[Cycle: ]
\[P (N(27,6.25) >35)  = 1- \Phi \left( \frac{35- 27}{\sqrt{6.25}} \right)  = 1- \Phi(3.2) = 0.0007\]

\item[Bus:]
\[P (N(25,25) >35)  = 1- \Phi \left( \frac{35- 25}{\sqrt{25}} \right)  = 1- \Phi(2) = 0.0228\]
\item[Car:]
\[P (N(23,26) >35)  = 1- \Phi \left( \frac{35- 26}{\sqrt{26}} \right)  = 1- \Phi(2) = 0.0228\]

\end{description}
Again cycling is best, with a probability of 0.0007.
\item  P(cycle) = 0.3 P(bus) = 0.3 P(car) = 0.4
( ) ( ) ( )
( )
30 cycle cycle
cycle 30
30
P P
P
P
<
< =
<
, and similarly for the other modes of travel.
\begin{eqnarray*}
P(< 30) &=& P(< 30 cycle)P(cycle) + P(< 30 bus)P(bus) + P(< 30 car)P(car)\\
&=& (0.8849\times 0.3) + (0.8413\times 0.3) + (0.8783\times 0.4)\\
&=& 0.26547 + 0.25239 + 0.35132 = 0.86918.\\
\end{eqnarray*}

Hence 
\begin{itemize}
\item P(cycle < 30) = 0.26547 / 0.86918 = 0.3054
\item P(bus < 30) = 0.25239 / 0.86918 = 0.2904
\item P(car < 30) = 0.35132 / 0.86918 = 0.4042 .
\end{itemize}
\end{enumerate}
\end{document}

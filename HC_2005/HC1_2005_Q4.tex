\documentclass[a4paper,12pt]{article}

%%%%%%%%%%%%%%%%%%%%%%%%%%%%%%%%%%%%%%%%%%%%%%%%%%%%%%%%%%%%%%%%%%%%%%%%%%%%%%%%%%%%%%%%%%%%%%%%%%%%%%%%%%%%%%%%%%%%%%%%%%%%%%%%%%%%%%%%%%%%%%%%%%%%%%%%%%%%%%%%%%%%%%%%%%%%%%%%%%%%%%%%%%%%%%%%%%%%%%%%%%%%%%%%%%%%%%%%%%%%%%%%%%%%%%%%%%%%%%%%%%%%%%%%%%%%

\usepackage{eurosym}
\usepackage{vmargin}
\usepackage{amsmath}
\usepackage{graphics}
\usepackage{epsfig}
\usepackage{enumerate}
\usepackage{multicol}
\usepackage{subfigure}
\usepackage{fancyhdr}
\usepackage{listings}
\usepackage{framed}
\usepackage{graphicx}
\usepackage{amsmath}
\usepackage{chngpage}

%\usepackage{bigints}
\usepackage{vmargin}

% left top textwidth textheight headheight

% headsep footheight footskip

\setmargins{2.0cm}{2.5cm}{16 cm}{22cm}{0.5cm}{0cm}{1cm}{1cm}

\renewcommand{\baselinestretch}{1.3}

\setcounter{MaxMatrixCols}{10}

\begin{document}Higher Certificate, Paper I, 2005. Question 4
f (x) (1 x) 1 , 0 x 1, 0 \alpha \alpha \alpha − = − < < > .

\begin{enumerate}
\item (i) ( ) ( ) ( ) ( ) 1
0 0
1 1 1 1
x x Fx u du u x \alpha \alpha \alpha \alpha = − − = − −  = − − ∫   (for 0 < x < 1 and \alpha > 0).

\begin{itemize}
\item The median m is given by F(m) = 0.5, so we have $1 – (1 – m)^\alpha = 0.5$ or $(1 – m)^\alpha = 0.5$,
so that 1−m = 2−1/\alpha , i.e. m =1− 2−1/\alpha .
\item When \alpha = 3, ( ) ( )2 f x = 3 1− x and ( ) ( )3 F x =1− 1− x (in [0, 1]).
\[ f(x)
x
1
1
2
3
4\]
\[ F(x)
x
1
1\]
\end{itemize}
\item ( ) ( ) 1 1
1 1
1 1
n n
n
i i
i i
L x x \alpha \alpha \alpha \alpha − −
= =
=  −  = − Π  Π .
Hence ( ) ( )
1
log log 1 log 1
n
i
i
\[L n \alpha \alpha x
=
= + − Σ − .
( )
1
log log 1
n
i
i\]
\[d L n x
d\alpha \alpha =
∴ = +Σ −\] which on setting equal to zero gives that the maximum
likelihood estimate is
\[( )
1
\hat{\alpha}
log 1
n
i
i
n
x
\alpha
=
= −
Σ −\]
, as required. [Consideration of
\[2
2
d log L
d\alpha\]
(see below) confirms that this is a maximum.]
\[2
2 2
d log L n
d\alpha \alpha
= − . \]Hence, using the result quoted in the question, \alpha\hat{\alpha} is approximately
Normally distributed with mean $\alpha$ and variance
$\frac{\alpha^2}{n}$ . We estimate the variance by
$\frac{\hat{\alpha}^2}{n}$
so that we have
$\hat{\alpha}  \sim N \left(\alpha, \frac{\hat{\alpha}^2}{n} \right)$,

 
 
 
, approximately.

\begin{itemize}
\item Hence an approximate 90\% confidence interval is given by
\[
P\left(-1.645 < \left( \frac{ \hat{\alpha} - \alpha }{ \hat{\alpha} / \sqrt{n} }  \right) < 1.645 \right) \approx 0.90
\]

leading to the interval
\[
\left( \hat{\alpha}  - \frac{1.645 \hat{\alpha}}{\sqrt{n}} , \hat{\alpha} + \frac{1.645 \hat{\alpha}}{\sqrt{n}} \right)
\]
 

\item For the given sample, we have n = 5 and the values of 1 – xi are 0.88, 0.57, 0.93, 0.13
and 0.71. 
\item Therefore
\begin{eqnarray*}
\sum (log(1 – x_i)) &=& –0.1278 – 0.5621 – 0.0726 – 2.0402 – 0.3425 \\ 
&=& –3.1452
\end{eqnarray*}
giving $\hat{\alpha} = \frac{5}{3.1452} = 1.5897$

\item Also,
\[ \frac{1.645 \hat{\alpha}}{\sqrt{n}} = \frac{1.645 \times 1.5897}{\sqrt{5}} = 1.1695\] so the confidence interval is (0.420, 2.759).
\end{itemize}
%%%%%%%%%%%%%%%%%%%%%%%%%%%%%%%%%%%%

\end{enumerate}

\end{document}

\documentclass[a4paper,12pt]{article}

%%%%%%%%%%%%%%%%%%%%%%%%%%%%%%%%%%%%%%%%%%%%%%%%%%%%%%%%%%%%%%%%%%%%%%%%%%%%%%%%%%%%%%%%%%%%%%%%%%%%%%%%%%%%%%%%%%%%%%%%%%%%%%%%%%%%%%%%%%%%%%%%%%%%%%%%%%%%%%%%%%%%%%%%%%%%%%%%%%%%%%%%%%%%%%%%%%%%%%%%%%%%%%%%%%%%%%%%%%%%%%%%%%%%%%%%%%%%%%%%%%%%%%%%%%%%

\usepackage{eurosym}
\usepackage{vmargin}
\usepackage{amsmath}
\usepackage{graphics}
\usepackage{epsfig}
\usepackage{enumerate}
\usepackage{multicol}
\usepackage{subfigure}
\usepackage{fancyhdr}
\usepackage{listings}
\usepackage{framed}
\usepackage{graphicx}
\usepackage{amsmath}
\usepackage{chngpage}

%\usepackage{bigints}
\usepackage{vmargin}

% left top textwidth textheight headheight

% headsep footheight footskip

\setmargins{2.0cm}{2.5cm}{16 cm}{22cm}{0.5cm}{0cm}{1cm}{1cm}

\renewcommand{\baselinestretch}{1.3}

\setcounter{MaxMatrixCols}{10}

\begin{document}Higher Certificate, Paper I, 2005. Question 4
%%%%%%%%%%%%%%%%%%%%%%%%%%%%%%%%%%%%%%%%%%%%%%%%%%%%%%%%%%%%%%%%%%
\begin{framed}
%4. (i) 
The continuous random variable X is distributed with probability density
function f (x) given by

\[f(x) =  \alpha (1-x)^{\alpha-1}, \qquad \mbox{ where }0 < x < 1, \qquad 0 \alpha  > 0 .\]

Find the cumulative distribution function of X, F(x) say, and hence obtain the
median of X. Also sketch the graphs of f (x) and F(x) for the case α = 3.
\end{framed}
%----------------------------------------------------------------%

%%%%%%%%%%%%%%%%%%%%%%%%%%%%%%%%%%%%%%%%%%%%%%%%%%%%%%%%%%%%%%%%%%


\begin{enumerate}[(a)]
\item 
{
\large
\begin{eqnarray*}
F(x) &=&  \int^{x}_{0}  \alpha (1-u)^{\alpha-1}  du \\
 &=& \left[ -(1-u)^\alpha  \right]^{x}_{0} \\
 &=& 1- (1-x)^\alpha\\
\end{eqnarray*}
}
 (for $0 < x < 1$ and $\alpha > 0$).

\begin{itemize}
\item The median m is given by $F(m) = 0.5$, so we have $1 – (1 – m)^\alpha = 0.5$ or $(1 – m)^\alpha = 0.5$,
so that $1−m = 2^{−1/\alpha}$ , i.e. $m =1− 2^{−1/\alpha}$ .
\item When $\alpha = 3$, $f(x) = 3(1− x)^2$ and $F x =1−(1− x)^3$ (in [0, 1]).
\[ \mbox{Image HERE} \]
\[ \mbox{Image HERE} \]
\end{itemize}
\item \[
L = \prod^{n}_{i=1} \left[ \alpha(1-x_i)^{n-1} \right]  = \alpha^n\prod^{n}_{i=1}
(1-x_i)^{n-1}\]

%----------------------------------------------------------------%
\newpage
\begin{framed}
(ii) A random sample $x_1, x_2, \ldots, x_n$ is taken from this distribution with a view to
estimating the unknown parameter α. Write down the likelihood function of
these data, $L(x1, x2, ..., xn  α) $ say, and show that the maximum likelihood
estimate (MLE) of $\alpha$ is given by
( )
1
ˆ
log 1
n
i
i
n
x
α
=
= −
Σ −
.

Also obtain 2 ( )
1 2
2
log , , ..., | n d Lx x x
d
α
α
as a function of α and n. 
Assuming that
$\hat{\alpha$ is approximately Normally distributed with mean α and variance
2 ( )
1 2
2
log , , ..., |
1 n d Lx x x
d
α
α
 
−  
 
, deduce an approximate 90\% confidence
interval for $\alpha$. Evaluate this interval for the sample 0.12, 0.43, 0.07, 0.87, 0.29.

\end{framed}
Hence 
\[
\log L = n \log \alpha + (\alpha-1) \sum^{n}_{i=1} \left[ \log (1-x_i) \right]  \]


$ { \displaystyle    \frac{d \log L}{d \alpha} = \frac{n}{\alpha} + \sum^{n}_{i=1} \log(1-x_i)  }$   which on setting equal to zero gives that the maximum likelihood estimate is ${\displaystyle \hat{\alpha} = \frac{-n}{\sum^n_{i=1} \log(1-x_i)  }  }$
, as required. [Consideration of
$ { \displaystyle    \frac{d^2 \log L}{d \alpha^2} }$ 
(see below) confirms that this is a maximum.]
\[ { \displaystyle    \frac{d^2 \log L}{d \alpha^2} }  = -\frac{n}{\alpha^2} \]


Hence, using the result quoted in the question, $\hat{\alpha}$ is approximately
Normally distributed with mean $\alpha$ and variance
$\frac{\alpha^2}{n}$ . We estimate the variance by
$\frac{\hat{\alpha}^2}{n}$
so that we have
$\hat{\alpha}  \sim N \left(\alpha, \frac{\hat{\alpha}^2}{n} \right)$,

 
 
 
, approximately.

\begin{itemize}
\item Hence an approximate 90\% confidence interval is given by
\[
P\left(-1.645 < \left( \frac{ \hat{\alpha} - \alpha }{ \hat{\alpha} / \sqrt{n} }  \right) < 1.645 \right) \approx 0.90
\]

leading to the interval
\[
\left( \hat{\alpha}  - \frac{1.645 \hat{\alpha}}{\sqrt{n}} , \hat{\alpha} + \frac{1.645 \hat{\alpha}}{\sqrt{n}} \right)
\]
 

\item For the given sample, we have n = 5 and the values of $1 – x_i$ are 0.88, 0.57, 0.93, 0.13
and 0.71. 
\item Therefore
\begin{eqnarray*}
\sum (log(1 – x_i)) &=& –0.1278 – 0.5621 – 0.0726 – 2.0402 – 0.3425 \\ 
&=& –3.1452
\end{eqnarray*}
giving $\hat{\alpha} = \frac{5}{3.1452} = 1.5897$

\item Also,
\[ \frac{1.645 \hat{\alpha}}{\sqrt{n}} = \frac{1.645 \times 1.5897}{\sqrt{5}} = 1.1695\] so the confidence interval is (0.420, 2.759).
\end{itemize}
%%%%%%%%%%%%%%%%%%%%%%%%%%%%%%%%%%%%

\end{enumerate}

\end{document}

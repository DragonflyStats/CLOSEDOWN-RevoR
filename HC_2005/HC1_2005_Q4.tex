Higher Certificate, Paper I, 2005. Question 4
f (x) (1 x) 1 , 0 x 1, 0 α α α − = − < < > .
(i) ( ) ( ) ( ) ( ) 1
0 0
1 1 1 1
x x Fx u du u x α α α α = − − = − −  = − − ∫   (for 0 < x < 1 and α > 0).
The median m is given by F(m) = ½, so we have 1 – (1 – m)α = ½ or (1 – m)α = ½,
so that 1−m = 2−1/α , i.e. m =1− 2−1/α .
When α = 3, ( ) ( )2 f x = 3 1− x and ( ) ( )3 F x =1− 1− x (in [0, 1]).
f(x)
x
1
1
2
3
4
F(x)
x
1
1
The solution to part (ii) is on the next page
(ii) ( ) ( ) 1 1
1 1
1 1
n n
n
i i
i i
L x x α α α α − −
= =
=  −  = − Π  Π .
Hence ( ) ( )
1
log log 1 log 1
n
i
i
L n α α x
=
= + − Σ − .
( )
1
log log 1
n
i
i
d L n x
dα α =
∴ = +Σ − which on setting equal to zero gives that the maximum
likelihood estimate is
( )
1
ˆ
log 1
n
i
i
n
x
α
=
= −
Σ −
, as required. [Consideration of
2
2
d log L
dα
(see below) confirms that this is a maximum.]
2
2 2
d log L n
dα α
= − . Hence, using the result quoted in the question, αˆ is approximately
Normally distributed with mean α and variance
2
n
α . We estimate the variance by
ˆ 2
n
α ,
so that we have
ˆ 2 ˆ ~ N ,
n
α α α
 
 
 
, approximately.
Hence an approximate 90% confidence interval is given by
ˆ
0.90 1.645 1.645
ˆ /
P
n
α α
α
 −  ≈ − < < 
 
,
leading to the interval
ˆ 1.645 ˆ ˆ 1.645 ˆ ,
n n
α α α α    − + 
 
.
For the given sample, we have n = 5 and the values of 1 – xi are 0.88, 0.57, 0.93, 0.13
and 0.71. Therefore
Σlog(1 – xi) = –0.1278 – 0.5621 – 0.0726 – 2.0402 – 0.3425 = –3.1452
giving ˆ 5 1.5897
3.1452
α= = .
Also,
1.645 ˆ 1.645 1.5897 1.1695
n 5
α = × = , so the confidence interval is (0.420, 2.759).

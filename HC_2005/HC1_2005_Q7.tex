\documentclass[a4paper,12pt]{article}

%%%%%%%%%%%%%%%%%%%%%%%%%%%%%%%%%%%%%%%%%%%%%%%%%%%%%%%%%%%%%%%%%%%%%%%%%%%%%%%%%%%%%%%%%%%%%%%%%%%%%%%%%%%%%%%%%%%%%%%%%%%%%%%%%%%%%%%%%%%%%%%%%%%%%%%%%%%%%%%%%%%%%%%%%%%%%%%%%%%%%%%%%%%%%%%%%%%%%%%%%%%%%%%%%%%%%%%%%%%%%%%%%%%%%%%%%%%%%%%%%%%%%%%%%%%%

\usepackage{eurosym}
\usepackage{vmargin}
\usepackage{amsmath}
\usepackage{graphics}
\usepackage{epsfig}
\usepackage{enumerate}
\usepackage{multicol}
\usepackage{subfigure}
\usepackage{fancyhdr}
\usepackage{listings}
\usepackage{framed}
\usepackage{graphicx}
\usepackage{amsmath}
\usepackage{chngpage}

%\usepackage{bigints}
\usepackage{vmargin}

% left top textwidth textheight headheight

% headsep footheight footskip

\setmargins{2.0cm}{2.5cm}{16 cm}{22cm}{0.5cm}{0cm}{1cm}{1cm}

\renewcommand{\baselinestretch}{1.3}

\setcounter{MaxMatrixCols}{10}

\begin{document}
Higher Certificate, Paper I, 2005. Question 7

%%%%%%%%%%%%%%%%%%%%%%%%%%%%%%%%%%%%%%%%%%%%%%%%%%%%%%%%%%%%%%%%%%
\begin{framed}
7. The breaking stresses in newtons per square metre of standard samples of pine,
measured at varying angles x° to the grain of the wood, are tabulated below.
Breaking Stress y (N/m2)×10-10 0.987 1.064 1.337 1.912 2.740 5.771 11.494
Angle x° 0 15 30 45 60 75 90
(i) Plot these data, use the summary information given below to calculate the
product-moment coefficient of correlation corr(x, y) between x and y, and
comment briefly.
\end{framed}
%----------------------------------------------------------------%
%----------------------------------------------------------------%
\begin{framed}
(ii) Hankinson's formula for y in terms of x is of the form
1 sin2 x sin2 x 1 y
a b
−  −  =  + 
 
,
where a and b are the standardised breaking stresses parallel to and
perpendicular to the grain respectively. Show how this relationship may be
expressed in the linear form
$Y = A + BX$ ,
where $Y = 1/y$ and $X = \sin^2 x$ and $A$ and $B$ are functions of the unknown
parameters a and b which you should find.
\end{framed}
%----------------------------------------------------------------%

%%%%%%%%%%%%%%%%%%%%%%%%%%%%%%%%%%%%%%%%%%%%%%%%%%%%%%%%%%%%%%%%%%
Note There are many equivalent forms of the formulae that are required to be stated.


The basic expressions are $\sum(x − x)( y − y)$ , $\sum(x − x)^2$ and $\sum( y − y)^2$ . Convenient
computing expressions are $\sum xy − (\sum x\sum y) / n$ and similarly for the others. [Where
appropriate, numerators and denominators of fractions could both be multiplied by n
(7) to avoid possible slight inaccuracies caused by rounding when dividing by 7.]
\begin{enumerate}
\item 
0
2
4
6
8
10
12
14
0 15 30 45 60 75 90
x
y
2 2 ( 2 2 )( 2 2 )
( )( ) /
( ) ( ) ( ) / ( ) /
r x x y y xy x y n
x x y y x x n y y n
= \sum − − = \sum − \sum \sum
\sum − \sum − \sum − \sum \sum − \sum
2 2
1773.795 (315 25.305/7 ) 635.07 0.848
(20475 315 /7 )(180.474 25.305 /7 ) 748.784
= − × = =
− −
.
This indicates a strong linear association between x and y, but nevertheless the scatter
diagram clearly suggests that the relationship is curved.
\item 
We have \[ \frac{1}{y} = \frac{1-\sin^2x}{a} + \frac{sin^2x}{b}
= \frac{1}{a} + \sin^2x \left(\frac{1}{b} - \frac{1}{a} \right) \]
where $X$ and $Y$ are as given.

\[Y = \frac{1}{a} \left(\frac{1}{b} - \frac{1}{a} \right) X]\]

So $A = \frac{1}{a}$   and $B = \frac{1}{b} - \frac{1}{a}$

%%%%%%%%%%%%%%%%%%%%%%%%%%%%%%%%%%%%%%%%%%%%%%%%%%%%%%%%%%%%%%%%%%%%
\newpage

%----------------------------------------------------------------%
\begin{framed}
(iii) Plot Y against X, comment on the suitability of this relationship for regression
analysis, estimate A and B by least squares and deduce the corresponding
estimates of a and b. Also compute $corr(X, Y)$ and compare this with the
correlation computed in part (i).

Note: (A) State clearly any formulae assumed without proof.
(B) $\sum x = 315$, $\sum x2 = 20475$, $\sum y = 25.305$, $\sum y^2 =180.474$, $\sum xy =1773.795$ ,
2
2 4
2
sin x 3.5, sin x 2.75, 1 3.849, 1 2.9136, sin x 1.03385.
y y y
   
= =   =   = =
   
\sum \sum \sum \sum \sum
\end{framed}
\item 
X = sin2x 0 0.067 0.250 0.500 0.750 0.933 1
Y = 1/y 1.013 0.940 0.748 0.523 0.365 0.173 0.087
[Note. Summary statistics for these are given in the question.]
0
0.5
1
0 0.25 0.5 0.75 1
X = sin2x
Y
=1/y
\begin{itemize}
\item 
The scatter diagram indicates that the relationship between X and Y is very close to
linear, with little scatter about a straight line. 
\item Linear regression should be suitable.
\item  We have X = 3.5 / 7 = 0.500 and Y = 3.849 / 7 = 0.550.
\item  So the fitted linear regression is $Y = A + BX$ where

\[ B = \frac{\sum(X-\bar{X})(Y-\bar{Y})}{\sum(X-\bar{X})^2 }\]
\item  Carrying out the calculations as in part (i), we get
\[ B  = \frac{1.03385 - (3.5 \times 3.849/7}{2.75 - (3.5^2/7)} = - \frac{0.89065}{1}
= -0.89065\]
and hence $A = 0.550 + (0.89065)(0.500) = 0.9953$.
\item Thus the corresponding estimates of a and b are given by ˆ 1 1.0047
0.9953
a= = and


\[  \hat{b} = \left( B + \frac{1}{\hat{a} }  \right)^{-1} = \frac{1}{0.10465} = 9.556.\]

\item The correlation coefficient for X and Y is
( )( )
( ) ( ) ( ) 2 2 2 2
0.89065 0.89065 0.9975
1 2.9136 (3.849 /7 )
X X Y Y
X X Y Y Y Y
\sum − − − − = = = −
\sum − \sum − × \sum − −
.
\item This is an even stronger indication of a linear relationship than in part (i), and we can
see from the scatter diagram that the relationship appears almost purely linear (very
little random scatter, certainly no curved component)
\end{itemize}

\end{enumerate}
\end{document}

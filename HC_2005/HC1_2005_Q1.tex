\documentclass[a4paper,12pt]{article}

%%%%%%%%%%%%%%%%%%%%%%%%%%%%%%%%%%%%%%%%%%%%%%%%%%%%%%%%%%%%%%%%%%%%%%%%%%%%%%%%%%%%%%%%%%%%%%%%%%%%%%%%%%%%%%%%%%%%%%%%%%%%%%%%%%%%%%%%%%%%%%%%%%%%%%%%%%%%%%%%%%%%%%%%%%%%%%%%%%%%%%%%%%%%%%%%%%%%%%%%%%%%%%%%%%%%%%%%%%%%%%%%%%%%%%%%%%%%%%%%%%%%%%%%%%%%

\usepackage{eurosym}
\usepackage{vmargin}
\usepackage{amsmath}
\usepackage{graphics}
\usepackage{epsfig}
\usepackage{enumerate}
\usepackage{multicol}
\usepackage{subfigure}
\usepackage{fancyhdr}
\usepackage{listings}
\usepackage{framed}
\usepackage{graphicx}
\usepackage{amsmath}
\usepackage{chngpage}

%\usepackage{bigints}
\usepackage{vmargin}

% left top textwidth textheight headheight

% headsep footheight footskip

\setmargins{2.0cm}{2.5cm}{16 cm}{22cm}{0.5cm}{0cm}{1cm}{1cm}

\renewcommand{\baselinestretch}{1.3}

\setcounter{MaxMatrixCols}{10}

\begin{document}

\section{Introduction}

Higher Certificate, Paper I, 2005. Question 1
\begin{enumerate}
\item (i)


\[  { 49 \choose 6}  = \frac{49!}{6! 43!} = \frac{49.48.47.46.45.44 }{6.5.4.3.2.1} = 13983816 \]

%%%%%%%%%%%%%%%%%%%%%%%%%%%%%%%%%%%%%%%%%%%%%%%%%%

% 2005 HC1 Q1

\begin{itemize}
\item $P(X=0) = e^{\lambda} = e^{-0.7151} = 0.4891 $ 
\item $P(X=1) = \lambda e^{\lambda} = \lambda \times  e^{} = 0.3497 $ 
\end{itemize}

\[ { 31 \choose 6} = \frac{31!}{6! \times 35!}  = \frac{31!}{6!} = 736281\]


\item 

Hence (using also the result of part (i)) we have

\[ P(\mbox{Winning set} \in \{1,2,\ldots 31\} = \frac{ 736281}{13983816} =  0.05265\]

\item  Let U be the number out of the 3,000,000 players choosing from $(1, 2, \ldots, 31)$
who match all 6 winning numbers, and let V be the number out of the 7,000,000
players choosing from all the numbers $(1, 2,\ldots, 49)$ who match all 6 winning
numbers.

%%%%%%%%%%%%%%%%%%%%%%%
Then $Y = U + V$, where (given the winning numbers) U and V are independent.
(a) If all six winning numbers are in the list $(1, 2, \ldots, 31)$, then U is
binomial with n = 3,000,000 and p = 1/736281, which is approximated by
Poisson(np) i.e. $\mbox{Poisson} (\lambda = 4.07453)$.
\begin{itemize}
\item Similarly, V is binomial with n = 7,000,000 and p = 1/13983816, which is
approximated by Poisson(0.50058).
\item Thus, using the result that 
\[Poisson( \lambda_1) + Poisson( \lambda_22) = Poisson( \lambda_1 + \lambda_2)\]
2)
[where the two Poisson distributions are independent], we have that the
approximate distribution of Y is Poisson(4.57511).
\end{itemize}
(b) In this case, U must be zero so we simply have Y = V, i.e.
Poisson(0.50058) approximately.
\end{enumerate}

\end{document}

\documentclass[a4paper,12pt]{article}

%%%%%%%%%%%%%%%%%%%%%%%%%%%%%%%%%%%%%%%%%%%%%%%%%%%%%%%%%%%%%%%%%%%%%%%%%%%%%%%%%%%%%%%%%%%%%%%%%%%%%%%%%%%%%%%%%%%%%%%%%%%%%%%%%%%%%%%%%%%%%%%%%%%%%%%%%%%%%%%%%%%%%%%%%%%%%%%%%%%%%%%%%%%%%%%%%%%%%%%%%%%%%%%%%%%%%%%%%%%%%%%%%%%%%%%%%%%%%%%%%%%%%%%%%%%%

\usepackage{eurosym}
\usepackage{vmargin}
\usepackage{amsmath}
\usepackage{graphics}
\usepackage{epsfig}
\usepackage{enumerate}
\usepackage{multicol}
\usepackage{subfigure}
\usepackage{fancyhdr}
\usepackage{listings}
\usepackage{framed}
\usepackage{graphicx}
\usepackage{amsmath}
\usepackage{chngpage}

%\usepackage{bigints}
\usepackage{vmargin}

% left top textwidth textheight headheight

% headsep footheight footskip

\setmargins{2.0cm}{2.5cm}{16 cm}{22cm}{0.5cm}{0cm}{1cm}{1cm}

\renewcommand{\baselinestretch}{1.3}

\setcounter{MaxMatrixCols}{10}

\begin{document}

Higher Certificate, Paper I, 2005. Question 1
%%%%%%%%%%%%%%%%%%%%%%%%%%%%%%%%%%%%%%%%%%%%%%%%%%%%%%%%%%%%%%%%%%
\begin{framed}
\noindent In the UK National Lottery, players seek to guess six numbers, selected at random
(without replacement) from the list 1, 2, 3, …, 49.
(i) Show that the total number of ways of choosing six numbers from 49 is
13,983,816.
\end{framed}
\begin{enumerate}[(a)]
\item (i)


\[  { 49 \choose 6}  = \frac{49!}{6! 43!} = \frac{49.48.47.46.45.44 }{6.5.4.3.2.1} = 13983816 \]

(Just short of 14 million)
%%%%%%%%%%%%%%%%%%%%%%%%%%%%%%%%%%%%%%%%%%%%%%%%%%%%%%%%%%%%%%%%%%
\begin{framed}
(ii) Suppose that 10,000,000 people play the lottery, and assume that each player
independently chooses, at random and without replacement, six numbers from
the list 1, 2, 3, …, 49. Let X denote the total number of players who match the
six winning numbers. Write down the exact distribution of X, and a Poisson
approximation to this distribution. Hence find, approximately, the probability
(a) that X = 0,
(b) that X = 1.

\end{framed}

%%%%%%%%%%%%%%%%%%%%%%%%%%%%%%%%%%%%%%%%%%%%%%%%%%%%%%%%%%%%%%%%%%



% 2005 HC1 Q1

\begin{itemize}
\item $P(X=0) = e^{\lambda} = e^{-0.7151} = 0.4891 $ 
\item $P(X=1) = \lambda e^{\lambda} = \lambda \times  e^{} = 0.3497 $ 
\end{itemize}



%%%%%%%%%%%%%%%%%%%%%%%%%%%%%%%%%%%%%%%%%%%%%%%%%%%%%%%%%%%%%%%%%%
%----------------------------------------------------------------%
\begin{framed}
(iii) It is believed that many lottery players guess six numbers at random without
replacement from the list 1, 2, 3, …, 31. Calculate the total number of possible
choices without replacement of six numbers from the list 1, 2, 3, …, 31, and
deduce the probability that the winning set of six numbers contains no number
greater than 31.
\end{framed}
\item 
\[ { 31 \choose 6} = \frac{31!}{6! \times 35!}  = \frac{31!}{6!} = 736281\]
Hence (using also the result of part (i)) we have

\[ P(\mbox{Winning set} \in \{1,2,\ldots 31\} = \frac{ 736281}{13983816} =  0.05265\]


%%%%%%%%%%%%%%%%%%%%%%%%%%%%%%%%%%%%%%%%%%%%%%%%%%%%%%%%%%%%%%%%%%%%%
\begin{framed}
(iv) Suppose now that 3,000,000 lottery players choose their six numbers, at
random and without replacement, from the list 1, 2, 3, …, 31, whilst 7,000,000
players choose their six numbers, at random and without replacement, from the
list 1, 2, 3, …, 49. Let Y denote the total number of players who match the six
winning numbers. Write down the (approximate) distribution of Y
(a) when all six winning numbers are in the list 1, 2, 3, …, 31,
(b) when at least one winning number is not in this list.

\end{framed}

%%%%%%%%%%%%%%%%%%%%%%%%%%%%%%%%%%%%%%%%%%%%%%%%%%%%%%%%%%%%%%%%%%%%%
\item  Let U be the number out of the 3,000,000 players choosing from $(1, 2, \ldots, 31)$
who match all 6 winning numbers, and let V be the number out of the 7,000,000
players choosing from all the numbers $(1, 2,\ldots, 49)$ who match all 6 winning
numbers.


Then $Y = U + V$, where (given the winning numbers) U and V are independent.
(a) If all six winning numbers are in the list $(1, 2, \ldots, 31)$, then U is
binomial with n = 3,000,000 and p = 1/736281, which is approximated by
Poisson(np) i.e. $\mbox{Poisson} (\lambda = 4.07453)$.
\begin{itemize}
\item Similarly, V is binomial with n = 7,000,000 and p = 1/13983816, which is
approximated by Poisson(0.50058).



\item Thus, using the result that 
\[Poisson( \lambda_1) + Poisson( \lambda_22) = Poisson( \lambda_1 + \lambda_2)\]
2)
[where the two Poisson distributions are independent], we have that the
approximate distribution of Y is Poisson(4.57511).
\end{itemize}
(b) In this case, U must be zero so we simply have Y = V, i.e.
Poisson(0.50058) approximately.
\end{enumerate}

\end{document}

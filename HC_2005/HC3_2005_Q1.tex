\documentclass[a4paper,12pt]{article}

%%%%%%%%%%%%%%%%%%%%%%%%%%%%%%%%%%%%%%%%%%%%%%%%%%%%%%%%%%%%%%%%%%%%%%%%%%%%%%%%%%%%%%%%%%%%%%%%%%%%%%%%%%%%%%%%%%%%%%%%%%%%%%%%%%%%%%%%%%%%%%%%%%%%%%%%%%%%%%%%%%%%%%%%%%%%%%%%%%%%%%%%%%%%%%%%%%%%%%%%%%%%%%%%%%%%%%%%%%%%%%%%%%%%%%%%%%%%%%%%%%%%%%%%%%%%

\usepackage{eurosym}
\usepackage{vmargin}
\usepackage{amsmath}
\usepackage{graphics}
\usepackage{epsfig}
\usepackage{enumerate}
\usepackage{multicol}
\usepackage{subfigure}
\usepackage{fancyhdr}
\usepackage{listings}
\usepackage{framed}
\usepackage{graphicx}
\usepackage{amsmath}
\usepackage{chngpage}

%\usepackage{bigints}
\usepackage{vmargin}

% left top textwidth textheight headheight

% headsep footheight footskip

\setmargins{2.0cm}{2.5cm}{16 cm}{22cm}{0.5cm}{0cm}{1cm}{1cm}

\renewcommand{\baselinestretch}{1.3}

\setcounter{MaxMatrixCols}{10}

\begin{document}
Higher Certificate, Paper III, 2005. Question 1
\begin{enumerate}
\item Means are: I low 76.25; I high 57.50; II low 73.75; II high 54.25.
Blood
sugar
Insulin 1
Insulin 2
0
50
60
70
80
dose levels
low
high
As the two lines are virtually parallel, there is no evidence of any interaction between insulin type and dose level.
\item Totals for insulins are I: 535, II: 512. Grand total = 1047
Hence SS for insulin = 222535512104733.06258816+−=.
Level totals are 600, 447. So SS for levels = 22260044710471463.06258816+−=.
So analysis of variance table is
SOURCE
DF
SS
MS
F value
Rabbits
3
297.19
99.06
1.33 compare F3,9
Insulin
Dose level
Insulin × Dose
1
1
1
33.06
1463.06
0.57
33.06
1463.06
0.57
0.444 compare F1,9
19.65 …
0.008 …
Treatments
3
1496.69
Residual
9
670.06
74.45
= 2ˆσ
TOTAL
15
2463.94
The standard error of a treatment mean is 74.454.314=
\item The only influential effect on blood sugar is the dose given; there is no evidence of any differences due to types of insulin and certainly not of any interaction of dose level with type of insulin. The higher dose level reduces blood sugar. Results are rather variable, as shown by the size of the standard error. Rabbits do not show any real difference in response.
\item Using the same four rabbits for all treatments eliminates any possible differences between animals (which did not show up in this experiment but may do in others). Treatment effects and differences will be estimated more precisely because of this. But we need to assume that using the same animals for all four treatments does not affect the responses, all of which are still independent of one another. If there were to be reactions or carry-over effects, it would be better to use 16 animals. The results would be obtained more quickly but they would very likely be more variable.
\end{enumerate}
\end{document}

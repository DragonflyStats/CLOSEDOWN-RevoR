\documentclass[a4paper,12pt]{article}

%%%%%%%%%%%%%%%%%%%%%%%%%%%%%%%%%%%%%%%%%%%%%%%%%%%%%%%%%%%%%%%%%%%%%%%%%%%%%%%%%%%%%%%%%%%%%%%%%%%%%%%%%%%%%%%%%%%%%%%%%%%%%%%%%%%%%%%%%%%%%%%%%%%%%%%%%%%%%%%%%%%%%%%%%%%%%%%%%%%%%%%%%%%%%%%%%%%%%%%%%%%%%%%%%%%%%%%%%%%%%%%%%%%%%%%%%%%%%%%%%%%%%%%%%%%%

\usepackage{eurosym}
\usepackage{vmargin}
\usepackage{amsmath}
\usepackage{graphics}
\usepackage{epsfig}
\usepackage{enumerate}
\usepackage{multicol}
\usepackage{subfigure}
\usepackage{fancyhdr}
\usepackage{listings}
\usepackage{framed}
\usepackage{graphicx}
\usepackage{amsmath}
\usepackage{chngpage}

%\usepackage{bigints}
\usepackage{vmargin}

% left top textwidth textheight headheight

% headsep footheight footskip

\setmargins{2.0cm}{2.5cm}{16 cm}{22cm}{0.5cm}{0cm}{1cm}{1cm}

\renewcommand{\baselinestretch}{1.3}

\setcounter{MaxMatrixCols}{10}

\begin{document}
Higher Certificate, Paper II, 2005. Question 5

\begin{framed}
5. (i) State the assumptions on which an independent (unpaired) two-sample t test is based. (4) 
\end{framed} 


\begin{enumerate}[(a)]
    \item 
(i) The two samples should be from independent Normal distributions with the same variance but possibly different means (the null hypothesis is usually that the two means are equal). The samples are random samples and are independent of each other.


 
 
\begin{framed} 
An individual is considering purchasing a two-bedroomed terraced house in one of two adjacent towns in the north of England.  To compare prices, he extracts some price data from one week's issue of the local newspaper's "Property Supplement" for all such properties advertised by estate agents with branches in both towns.  He wishes to determine whether the mean price in one town differs from that in the other.  The data extracted are given in the table below (units are thousands of pounds). 

\begin{tabular}{|c|c|c|}
Town 1      & \{ 77.50 74.95 74.50 60.00 45.00 25.00 25.00\}    & $n_1=7$\\
Town 2      & \{72.95 72.95 65.00 62.50 56.95 54.95 52.95   & $n_2=14$ \\
& 49.95 46.95 35.00 34.95 30.00 29.95 25.00 & \\
\end{tabular} 
       
\end{framed}
    \item 
221112227;54.56,534.6956.14;49.29,261.3001.nxsnxs======
The "pooled estimate" of variance is
\begin{eqnarray*}
s_p^2 &=& \frac{(n_1 -1 )(534.6956) + (n_2-1 )(261.3001) }{(n_1-1) + (n_2-1)}\\ 
&=& \frac{(7 -1 )s^2_1 + (14-1 )s^2_2 }{(7-1) + (14-1)}\\
&=& 347.6355.\]
The test statistic for testing the null hypothesis $\mu_1-\mu_2 = 0$, where $\mu_1$ and $\mu_2$ are the respective population mean prices, is
\[121171405.270.6118.631xxs−−==+,\]
which is referred to $t_{(19)}$. This is not significant, so the null hypothesis cannot be rejected. There is no evidence that the true means in the two towns differ.

%%%%%%%%%%%%%%%%%%%%%%%%%%%%%%%%%%%%%%%%%%%%%%%%%%%%%%%%%%%%%5

\begin{framed}
 
(ii) Assuming that all the necessary assumptions hold, perform an independent two-sample t test and draw your conclusion. 
 
(iii) Town 1 has a considerably larger population, and greater numbers of all types of properties, than town 2.  Taking note of this information, and of the way in which the data were obtained, discuss critically whether there is a valid basis for the test in part (ii). 
 

\end{framed}
    \item  Town 1 has greater population and greater numbers of all types of properties, yet only half the sample size was used compared with town 2.
    \begin{itemize}
        \item    The probabilities of selection in the two towns are thus very different. Also, the samples were restricted to the "Property Supplement" and to agents dealing in both towns. The assumption of randomness is doubtful, even whether we have representative samples. 
        \item The test based on these data must be suspect for practical reasons, even if Normality and constant variance are acceptable. 
        \item The usual F6,13 test for equality of population variances gives a test statistic value of 2.05 which is not significant.
    \end{itemize}
 
\end{enumerate}
\end{document}

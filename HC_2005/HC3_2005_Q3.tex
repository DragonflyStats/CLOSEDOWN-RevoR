\documentclass[a4paper,12pt]{article}

%%%%%%%%%%%%%%%%%%%%%%%%%%%%%%%%%%%%%%%%%%%%%%%%%%%%%%%%%%%%%%%%%%%%%%%%%%%%%%%%%%%%%%%%%%%%%%%%%%%%%%%%%%%%%%%%%%%%%%%%%%%%%%%%%%%%%%%%%%%%%%%%%%%%%%%%%%%%%%%%%%%%%%%%%%%%%%%%%%%%%%%%%%%%%%%%%%%%%%%%%%%%%%%%%%%%%%%%%%%%%%%%%%%%%%%%%%%%%%%%%%%%%%%%%%%%

\usepackage{eurosym}
\usepackage{vmargin}
\usepackage{amsmath}
\usepackage{graphics}
\usepackage{epsfig}
\usepackage{enumerate}
\usepackage{multicol}
\usepackage{subfigure}
\usepackage{fancyhdr}
\usepackage{listings}
\usepackage{framed}
\usepackage{graphicx}
\usepackage{amsmath}
\usepackage{chngpage}

%\usepackage{bigints}
\usepackage{vmargin}

% left top textwidth textheight headheight

% headsep footheight footskip

\setmargins{2.0cm}{2.5cm}{16 cm}{22cm}{0.5cm}{0cm}{1cm}{1cm}

\renewcommand{\baselinestretch}{1.3}

\setcounter{MaxMatrixCols}{10}

\begin{document}
Higher Certificate, Paper III, 2005. Question 3
\begin{framed}

%%%%%%%%%%%%%%%%%%%%%%%%%%%%%%%%%%%%%%%%%%%%%%%%%%%%%%%%%%%%%
A study into student progression on a mathematics degree looked at the performance
of a random sample of 38 students in the year 2 Probability and Statistics (PS)
examination and the year 3 examination in Statistical Modelling (SM). The scatter
plot below shows the results.
100
90
80
70
60
3.
50
40
30
20
10
0
0
10
20
30
40
50
60
70
80
90
100
PS
A regression model is proposed to describe the result on the SM course, y , as a linear function of the result achieved on the PS course, x .
The model is given as y i = α + β x i + e i , where the { e i } are independent and distributed as N(0, σ 2 ).
(i)
38
Find least squares estimates, α̂ and β̂ , of α and β , given the following
summary information.
∑ x i = 2436
i = 1
38
∑ y i = 1670
i = 1
38
∑ x i y i = 116888
i = 1
38
∑ x i = 166991
i = 1
2
38
∑ y
i
2
= 86402
i = 1
(5)
(ii) Interpret the fitted model from a practical point of view and comment on the fit
of the model.
\end{framed}

%%%%%%%%%%%%%%%%%%%%%%%%%%%%%%%%%%%%%%%%%%%%%%%%%%%%%%%%%%%%%

\begin{enumerate}
\item Define xyxySxynΣΣ=Σ−, etc. Then ˆxyxxSSβ= and ˆˆ.yxαβ=−
We have n = 38, so 243664.1138x== and 167043.9538y==.
22436167024361168889832.21.16699110830.58.3838xyxxSS×=−==−=
Hence And ˆ0.9078.β=ˆ43.9564.110.907814.25.α=−×=−
\item The substantial negative value for ˆ,α and the scatter plot, indicate that marks on SM are lower (harder to get) than on PS. There is a clear relation between the two, with ˆβ not far from 1, suggesting that the same types of skill and understanding are being examined in both.
At the upper end, the marks for SM are above the fitted line, which may just be due to having three very good students or it may perhaps suggest trying a curved relation to fit the whole data and tail off towards the origin.

\newpage

\begin{framed}
(iii) Calculate an estimate for σ 2 and calculate the estimated variances of α̂ and β̂ .
(6)
(iv) Give a 95% confidence interval estimate for the mean mark in the SM
examination for students who score 80 on the PS examination. Comment on
your answer.
(5)
4
\end{framed}
\item The total sum of squares 216708640213009.8938yyS=−= (with 37 df).
The regression sum of squares is ()22ˆˆoror8925.68xyxyxxxxSSSSββ= (with 1 df).
Hence the residual SS is 13009.89 – 8925.68 = 4084.21 with 36 df, and the residual mean square is 4084.21/36 = 113.45.
This (113.45) is the estimate (2ˆσ) of 2σ.
()2ˆVarxxSσβ=, so the estimated variance of ˆβ is 2ˆ113.450.01047.10830.58xxSσ==
()221ˆVarxxxnSασ⎛⎞=+⎜⎝⎠, so the estimated variance of ˆα is
2221164.11ˆ113.4546.018.3810830.58xxxnSσ⎛⎞⎛⎞+=+=⎜⎟⎜⎟⎝⎠⎝⎠
Solution continued on next page
[NOTE. The test statistic for the usual F test for "the significance of the regression" is 8925.6878.68113.45=, which is very highly significant as an observation from F1,36.
This indicates that the line fits well.]
\item When x = 80, we have ŷ = –14.25 + (0.9078 × 80) = 58.37.
The estimated variance is ()222801115.89ˆ113.455.63.3810830.58xxxnSσ⎛⎞−⎛⎞⎜⎟+=+=⎜⎟⎜⎟⎝⎠⎝⎠
Thus the 95\% interval for ŷ, the mean mark at x = 80, is 3758.375.63t± where represents the double-tailed 5% point of the t distribution with 37 df; we take this as (approximately) 2.02 here. Thus the interval is 58.37 ± (2.02 37t× 2.37), i.e. (53.6, 63.2).
This is an interval for the mean mark. Individual marks will vary around this [Var(ŷi) has an extra term in it when interpreted as the estimated variance of an individual observation]. The scatter is quite large even though a line fits the data quite well.
\end{enumerate}
\end{document}

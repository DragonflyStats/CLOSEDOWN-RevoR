\documentclass{article}
\usepackage[utf8]{inputenc}
\usepackage{enumerate}

\author{kobriendublin }
\date{December 2018}

\begin{document}

%- Higher Certificate, Module 5, 2008. Question 1
\section{Introduction}
\begin{enumerate}[(i)]
\item
HC5 2016 - solutions
3. (i) Probabilities sum to 1, so 1 (sum to 1)
1
(4 18 12 1 12 24 6 4 3) 84 1 .
84
k          k  k  1 (1/84)
TOTAL 2
(ii) Sample size is only 3, so there cannot be 2 white dice and 2 blue dice. 1 (correct reason)
There are
9
84
3
 
  
 
ways of drawing the sample altogether. If X  2 and Y  0 then there
are 2 white dice and 1 red die drawn. The 2 white dice can be drawn in
3
3
2
 
  
 
ways and the
1 red die in
4
4
1
 
  
 
ways, so the probability is
3 4 12
84 84

 as given.
1 (84 ways), 1 (3 ways), 1 (4 ways), 1 (probability calculation)
TOTAL 5
(iii)
4 18 12 1 35
( 0) .
84 84
P Y
  
   So the conditional distribution of X given that Y = 0 is
given by P(X  x Y  0) 
( , 0)
( 0)
P X x Y
P Y
 

i.e. 1 (35/84), 1 (method)
x 0 1 2 3
P(X=x|Y=0) 4
35
18
35
12
35
1
35 1 (pmf correct)
Then E(X Y  0) 
18 24 3 45 9
.
35 35 7
 
  1 (9/7)
2 E(X Y  0) 
18 48 9 75 15
35 35 7
 
  so Var(X Y  0) 
2
15 9 24
.
7 7 49
 
  
 
1(15/7), 1(24/49)
TOTAL 6
(iv) X marginal distribution: 0 1 2 3
20
84
45
84
18
84
1
84
So
45 36 3
( ) 1.
84
E X
 
 
1 (X marginal correct), 1(E(X)=1)
Y marginal distribution: 0 1 2
35
84
42
84
7
84
So
42 14 56
( ) .
84 84
E Y

 
1 (Y marginal correct), 1(E(Y)=56/84)
24 6 3 42
( ) 1 1 1 2 2 1 .
84 84 84 84
E XY           1(E(XY)=42/84)
So
42 56 14 1
Cov( , ) ( ) ( ) ( ) .
84 84 84 6
X Y  E XY  E X E Y       1 (Cov = -1/6)
Negative covariance makes sense: the more white dice in the sample, the fewer spaces there
are for blue dice. 1 (valid comment)
TOTAL 7
HC5 2016 - solutions

1 (2nd deriv correct), 1 (positive)
TOTAL 7\end{enumerate}
\end{document}
\documentclass[a4paper,12pt]{article}
%%%%%%%%%%%%%%%%%%%%%%%%%%%%%%%%%%%%%%%%%%%%%%%%%%%%%%%%%%%%%%%%%%%%%%%%%%%%%%%%%%%%%%%%%%%%%%%%%%%%%%%%%%%%%%%%%%%%%%%%%%%%%%%%%%%%%%%%%%%%%%%%%%%%%%%%%%%%%%%%%%%%%%%%%%%%%%%%%%%%%%%%%%%%%%%%%%%%%%%%%%%%%%%%%%%%%%%%%%%%%%%%%%%%%%%%%%%%%%%%%%%%%%%%%%%%
\usepackage{eurosym}
\usepackage{vmargin}
\usepackage{amsmath}
\usepackage{graphics}
\usepackage{epsfig}
\usepackage{enumerate}
\usepackage{multicol}
\usepackage{subfigure}
\usepackage{fancyhdr}
\usepackage{listings}
\usepackage{framed}
\usepackage{graphicx}
\usepackage{amsmath}
\usepackage{chngpage}
%\usepackage{bigints}

\usepackage{vmargin}
% left top textwidth textheight headheight
% headsep footheight footskip
\setmargins{2.0cm}{2.5cm}{16 cm}{22cm}{0.5cm}{0cm}{1cm}{1cm}
\renewcommand{\baselinestretch}{1.3}

\setcounter{MaxMatrixCols}{10}

\begin{document}
\begin{table}[ht!]
 \centering
 \begin{tabular}{|p{15cm}|}
 \hline  
4. The quarterly deliveries of phosphate fertilizers to UK agriculture are shown on the worksheet for this question.  A plot of the data is shown below. 
 
Complete the calculation of the centred four point moving averages and the estimated seasonal effects on the worksheet provided.  Use the estimated seasonal effects and the moving average trend values to complete the calculation of the residuals.  Plot the residuals against the quarter number (i.e. 1, 2, 3 or 4). (14) 
 

 
 \\ \hline
  \end{tabular}
\end{table}
\begin{table}[ht!]
 \centering
 \begin{tabular}{|p{15cm}|}
 \hline  
Comment on this residual plot. 
(3) 
 \\ \hline
  \end{tabular}
\end{table}
\begin{table}[ht!]
 \centering
 \begin{tabular}{|p{15cm}|}
 \hline  
 
Discuss whether deliveries for any quarters seem abnormal compared with the general pattern of deliveries. (3)\\ \hline
  \end{tabular}
\end{table}
%%%%%%%%%%%%%%%%%%%%%%%%%%%%%%%%%%%%%%%%%%%%%%%%%
\newpage
\begin{enumerate}
    \item The completed worksheet is attached (see next page). Plot of residuals against quarters:
The 4M quarter residual in 1993 is very large. The quarter’s delivery was much higher than
usual, and was higher than the previous quarter instead of the usual annual pattern for it to
be lower. (There was no compensating reduction the following quarter;hence the prediction is
also very far from observed. )
\end{enumerate}

\begin{verbatim}
    Worksheet for fertilizer deliveries,for use with Question 4:
15
Deliveries M.AV. Y-M.AV. Seasonal Predicted Residual
1990 2 49.5 * * -14.94 * *
3 94.2 * * -0.3475 * *
4 62.6 69.6750 -7.0750 -2.1675 67.5075 -4.9075
1991 1 72.7 66.3875 6.3125 17.455 83.8425 -11.1425
2 48.9 60.7250 -11.8250 -14.94 45.7850 3.1150
3 68.5 58.8875 9.6125 -0.3475 58.5400 9.9600
4 43.0 58.4125 -15.4125 -2.1675 56.2450 -13.2450
1992 1 77.6 55.6750 21.9250 17.455 73.1300 4.4700
2 40.2 53.400 -13.2000 -14.94 38.4600 1.7400
3 55.3 53.8250 1.4750 -0.3475 53.4775 1.8225
4 38.0 54.6125 -16.6125 -2.1675 52.4450 -14.4450
1993 1 86.0 55.1125 30.8875 17.455 72.5675 13.4325
2 38.1 64.3875 -26.2875 -14.94 49.4475 -11.3475
3 61.4 72.2125 -10.8125 -0.3475 71.8650 -10.4660
4 106.1 72.9750 33.1250 -2.1675 70.8075 35.2925
1994 1 80.5 73.0500 7.4500 17.455 90.5050 -10.0050
2 49.7 63.6625 -13.9625 -14.94 48.7225 0.9775
3 50.4 54.3750 -3.9750 -0.3475 54.0275 -3.6275
4 42.0 52.1375 -10.1375 -2.1675 49.9700 -7.9700
1995 1 70.3 51.6375 18.6625 17.455 69.0925 1.2075
2 42.0 53.4625 -11.4625 -14.94 38.5225 3.4775
16
Calculation of seasonal effects
Q1 Q2 Q3 Q4
6:3125 ¡11:8250 9:6125 ¡7:0750
21:9250 ¡13:2000 1:4750 ¡15:4125
30:8875 ¡26:2875 ¡10:8125 ¡16:6125
7:4500 ¡13:9625 ¡3:9750 33:1250
18:6625 ¡11:4625 ¡0:0750 ¡10:1375
¤ ¤ 0:6625
17:0475 ¡15:3475 ¡0:755 ¡2:575 ¡1:63
17:455 ¡14:94 ¡0:3475 ¡2:1675
\end{verbatim}
\end{document}

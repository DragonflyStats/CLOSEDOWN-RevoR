\documentclass[a4paper,12pt]{article}
%%%%%%%%%%%%%%%%%%%%%%%%%%%%%%%%%%%%%%%%%%%%%%%%%%%%%%%%%%%%%%%%%%%%%%%%%%%%%%%%%%%%%%%%%%%%%%%%%%%%%%%%%%%%%%%%%%%%%%%%%%%%%%%%%%%%%%%%%%%%%%%%%%%%%%%%%%%%%%%%%%%%%%%%%%%%%%%%%%%%%%%%%%%%%%%%%%%%%%%%%%%%%%%%%%%%%%%%%%%%%%%%%%%%%%%%%%%%%%%%%%%%%%%%%%%%
\usepackage{eurosym}
\usepackage{vmargin}
\usepackage{amsmath}
\usepackage{graphics}
\usepackage{epsfig}
\usepackage{enumerate}
\usepackage{multicol}
\usepackage{subfigure}
\usepackage{fancyhdr}
\usepackage{listings}
\usepackage{framed}
\usepackage{graphicx}
\usepackage{amsmath}
\usepackage{chngpage}
%\usepackage{bigints}

\usepackage{vmargin}
% left top textwidth textheight headheight
% headsep footheight footskip
\setmargins{2.0cm}{2.5cm}{16 cm}{22cm}{0.5cm}{0cm}{1cm}{1cm}
\renewcommand{\baselinestretch}{1.3}

\setcounter{MaxMatrixCols}{10}

\begin{document}
\begin{table}[ht!]
 \centering
 \begin{tabular}{|p{15cm}|}
 \hline  
1. The activity of the enzyme acid phosphatase was measured at four different pH values.  Five replicate measurements were made at each pH value. 
 
 Enzyme activity 
µ M / min at 
\begin{center}
\begin{tabular}{ccccc}
& pH = 3 & pH = 5 & pH = 7 & pH = 9\\  
& 11.1 & 12.0 & 11.2 & 5.6  \\
& 10.0 & 15.3 & 9.1 & 7.2  \\
& 13.5 & 15.1 & 9.6 & 6.4  \\
& 10.5 & 15.0 & 10.0& 5.9  \\
& 11.3 & 13.2 & 9.8 &6.3 \\
mean & 11.28 & 14.12 & 9.94& 6.28 \\
\end{tabular}
\end{center}
 
 Plot the enzyme activities against pH.            (3) 
 
 A one-way analysis of variance of the enzyme activities using pH as the classification factor gave the following analysis of variance table: 
 
Source of variation Sum of Squares Degrees of freedom Between pH values  3 Error   Total 178.57  
\\ \hline
  \end{tabular}
\end{table}
\begin{table}[ht!]
 \centering
 \begin{tabular}{|p{15cm}|}
 \hline  
Complete the analysis of variance table by filling in the missing values and adding appropriate mean squares and a variance ratio.  Carry out an appropriate statistical test on the variance ratio and state clearly in non-technical language what the analysis reveals.  What assumptions have you made?        (10) 
\\ \hline
  \end{tabular}
\end{table}
\begin{table}[ht!]
 \centering
 \begin{tabular}{|p{15cm}|}
 \hline  
 A simple linear model relating enzyme activity to pH was also fitted to the data and gave the following results. 
 
  The regression equation is  Enzyme activity  =  16.2 – 0.959 (pH value) 
 
 Analysis of Variance for linear regression 

\begin{center}
\begin{tabular}{|c|c|c|c|c|}
Source of variation&  Sum of squares&  Degrees of freedom &  Mean squares&  Variance ratio \\
Linear regression & 91.968 & 1  & 191.968 &  19.12 \\
Error & 86.601  &18 & 4.811 &  \\
Total & 178.570 & 19& & \\
\end{tabular}
\end{center} 
Draw the fitted line on your plot and comment on whether the fitted straight line adequately describes the relation between enzyme activity and pH.          
\\ \hline
  \end{tabular}
\end{table}




\begin{enumerate}[(a)]
\item 
% Paper III
% Statistical Applications and Practice
1 PH totals are 56.4; 70.6; 49.7; 31.4; G=208.1. Between PH’s ss = 1
5(56:42 +70:662 +49:72 +
34:12) ¡ 1
20208:12 = 158:99
12

Analysis of Variance

\begin{center}
\begin{tabular}{|c|c|c|c|c|}
Source of variation&  Sum of squares&  Degrees of freedom &  Mean squares&  Variance ratio \\
Between PH0s & 3&  158:99 & 52:997 &  F(3;16) = 43:30\\
Residual & 16 & 19:58& 1:224&
Total & 19&  178:57&&
\end{tabular}
\end{center}

\item The analysis shows that the bulk of the variability in the data is due to the difference
between the activities at the different PH values.
\item Each observation is assumed to be normally,
distributed about its mean, with the same variance for all. It is assumed that the model:
observation = mean for given PH + random residual $y_{ij} = m_{i} + e_{ij}$ i = 1; 2; 3; 4; explain
the data.
\item The filled live passes through (5,11.4) and (9,7.6), as shown. Although the analysis shows a significant
linear component in the regression, the plot clearly indicates the need for a curve. (The
deviations from-linearity s.s.is 158.99-91.97=67.02, which is also significantly large). 


\begin{table}[ht!]
 \centering
 \begin{tabular}{|p{15cm}|}
 \hline  
 
 
 
The investigator wishes to carry out another series of experiments at four new pH values to determine more precisely the pH at which enzyme activity is at its greatest.  What values would you suggest and why would you suggest them?     (4) 
\\ \hline
  \end{tabular}
\end{table}

%%%%%%%%%%%%%%%%%%%%%%%%%%%%%%%%%%%%%%%%
\item There is
no point in studying PH9, nor PH7 because the maximum does not seen to be near there.
\item The maximum will be in the region of 5, probably on the lower side, so values such as 4.25
by steps of 0.25 to 5.25 (omitting 5), or 4.5 by steps of 0.25 to 5.25 including 5.0 in the same
experiment, would be appropriate.

\end{enumerate}
\end{document}

\documentclass[a4paper,12pt]{article}
%%%%%%%%%%%%%%%%%%%%%%%%%%%%%%%%%%%%%%%%%%%%%%%%%%%%%%%%%%%%%%%%%%%%%%%%%%%%%%%%%%%%%%%%%%%%%%%%%%%%%%%%%%%%%%%%%%%%%%%%%%%%%%%%%%%%%%%%%%%%%%%%%%%%%%%%%%%%%%%%%%%%%%%%%%%%%%%%%%%%%%%%%%%%%%%%%%%%%%%%%%%%%%%%%%%%%%%%%%%%%%%%%%%%%%%%%%%%%%%%%%%%%%%%%%%%
\usepackage{eurosym}
\usepackage{vmargin}
\usepackage{amsmath}
\usepackage{graphics}
\usepackage{epsfig}
\usepackage{enumerate}
\usepackage{multicol}
\usepackage{subfigure}
\usepackage{fancyhdr}
\usepackage{listings}
\usepackage{framed}
\usepackage{graphicx}
\usepackage{amsmath}
\usepackage{chngpage}
%\usepackage{bigints}

\usepackage{vmargin}
% left top textwidth textheight headheight
% headsep footheight footskip
\setmargins{2.0cm}{2.5cm}{16 cm}{22cm}{0.5cm}{0cm}{1cm}{1cm}
\renewcommand{\baselinestretch}{1.3}

\setcounter{MaxMatrixCols}{10}

\begin{document}

 
 
  \begin{table}[ht!]
     \centering
     \begin{tabular}{|p{15cm}|}
     \hline

%%- Question 7. 
\largeThe
\noindent The continuous random variables $X$ and $Y$ are distributed with a joint probability density function (pdf) 
  


\[  f(x,y) = \begin{cases}  \frac{2}{a^2} & x \geq 0, y \geq 0, x + y \leq a \mbox{ for some positive } a,    \\
0,  & \mbox{otherwise} \\ 
  \end{cases}\]

 
 that is to say, (X,Y) follows a continuous uniform distribution over the triangle bounded by the lines $x = 0$,  $y = 0$,  $x+y = a$. 
 
 (i) Sketch a graph to show the region in which $f(x,y) > 0$.         

\\ \hline  
      \end{tabular}
    \end{table}



\begin{enumerate}
\item The distribution is defined in the area shown by the shaded triangle.

\item  \begin{eqnarray*}
f_X(x)
&=& \int^{a-x}_{0} f(x,y) dy \\
&=&  \frac{2}{a^2} \int^{a-x}_{0} dy \\
&=& \frac{2(a-x)}{a^2} \qquad \mbox{ for}  0 \leq x \leq a \mbox{ and } 0 \mbox{ otherwise} \\
\end{eqnarray*}

\begin{eqnarray*}
F_X(x) 
&=& \int^{x}_{0}f_X(u)du \\
&=& \frac{2}{a^2} \int^{x}_{0} (a-u) du \\ 
&=& \frac{2}{a^2}\left[ au - \frac{1}{2} u^2 \right]^{x}_{0} \\
&=& \frac{2ax-x^2}{a^2} \qquad \mbox{ for} 0 \leq x \leq a \\
\end{eqnarray*}
\item  

%%%%%%%%%%%%%%%%%%%%%%%%%%%%%%%%%%%%%%%%%%%%%%%%%
\newpage
  \begin{table}[ht!]
     \centering
     \begin{tabular}{|p{15cm}|}
     \hline    
(iii) Show that $E(X) = \frac{a}{3}$

 and write down $E(Y$).         
\\ \hline
\end{tabular}
\end{table}

\begin{eqnarray*}
E(X) &=& \int^{a}_{0} x \;f_X(x) dx \\
    &=&\int^{a}_{0} x(a-x) dx \\
    &=&  \frac{2}{a^2}\left[ \frac{1}{2}ax^2 - \frac{1}{3}x^3  \right]^{a}_{0}\\
    &=& \frac{a}{3}  
\end{eqnarray*}
Similarly $E(Y) = \frac{a}{3} $

\begin{eqnarray*}
E(X^2) &=& \int^{a}_{0} x^2 \;f_X(x) dx \\
    &=&\int^{a}_{0} x^2(a-x) dx \\
    &=&  \frac{2}{a^2}\left[ \frac{1}{3}ax^3 - \frac{1}{4}x^4  \right]^{a}_{0}\\
    &=& \frac{a^2}{6}  
\end{eqnarray*}
Similarly $E(Y^2) = \frac{a^2}{6} $

\begin{eqnarray*}
Var(X) &=& [E(X^2)] - [E(X)^2] \\
      &=& \frac{a^2}{6} - \left(\frac{a}{3}\right)^2 \\
     &=& \frac{a^2}{18} \\
\end{eqnarray*}
Necessarily $Var(X) = \frac{a^2}{18}$


%%%%%%%%%%%%%%%%%%%%%%%%%%%%%%%%%%%%%%%%%%%%%%%%%%%%
\newpage
  \begin{table}[ht!]
     \centering
     \begin{tabular}{|p{15cm}|}
     \hline
  Show that the marginal pdf of X is given by 



\[  f_X(x,y) = \left[ \begin{cases}  
\frac{2(a-x)}{a^2} &  0 \leq x  \leq a \mbox{ for some positive } a,    \\
0,  & \mbox{otherwise} \\ 
\end{cases}  \right] \]
  
 and hence obtain the marginal cumulative distribution function of $X$, $F_X(x) = P(X \leq x)$\\
\hline

      \end{tabular}
    \end{table}

\item 

\begin{eqnarray*}
E(XY) &=& \int^{\infty}_{0} \int^{\infty}_{0} xy\; f(x,y) \; dydx \\
    &=&\int^{\infty}_{0} \lambda t\;e^{-\lambda} dt \\
&=&
\int^{\infty}_{0} x[ y^2
a^2 ]a-x
0 dx \\ &=& 1
a2
\int^{\infty}_{0} x(a - x)^2dx = 1
a2
\int^{\infty}_{0}(a^2x - 2ax^2 + x^3)dx\\
&=& 1
a2 [ 1
2a^2x^2 - 2
3ax^3 + 1
4x^4]a
0\\ &=& \frac{1}{12}a^2\\
\end{eqnarray*}

%%%%%%%%%%%%%%%%%%%%%%%%%%%%%%%%%%%%%%%%%%%%%%%%%%%%%%%%%%%%%%%%%%%%%%%%%%%%%%%%%%
\newpage

  \begin{table}[ht!]
     \centering
     \begin{tabular}{|p{15cm}|}
     \hline  
 (iv) Show that $E(X^2)) = \frac{a^2}{6}$
and hence obtain $V(X)$ and $V(Y)$.       
 
(v) Show that $E(XY) = \frac{a^2}{12}$
and hence obtain the correlation coefficient 
between X and Y.\\ \hline
\end{tabular}
\end{table}
\[Cov(X; Y ) = E[XY ] - E[X]E[Y ]\] = a2
12 ¡ fraca32 = a2( 1
12 ¡ 1
9 = ¡a2
36 ,
Hence Pxy = pcov(X;Y )
V (X)V (Y )
= ¡a2=36
a2=18 = ¡1=2:
\end{enumerate}
\end{document}

\documentclass[a4paper,12pt]{article}
%%%%%%%%%%%%%%%%%%%%%%%%%%%%%%%%%%%%%%%%%%%%%%%%%%%%%%%%%%%%%%%%%%%%%%%%%%%%%%%%%%%%%%%%%%%%%%%%%%%%%%%%%%%%%%%%%%%%%%%%%%%%%%%%%%%%%%%%%%%%%%%%%%%%%%%%%%%%%%%%%%%%%%%%%%%%%%%%%%%%%%%%%%%%%%%%%%%%%%%%%%%%%%%%%%%%%%%%%%%%%%%%%%%%%%%%%%%%%%%%%%%%%%%%%%%%
\usepackage{eurosym}
\usepackage{vmargin}
\usepackage{amsmath}
\usepackage{graphics}
\usepackage{epsfig}
\usepackage{enumerate}
\usepackage{multicol}
\usepackage{subfigure}
\usepackage{fancyhdr}
\usepackage{listings}
\usepackage{framed}
\usepackage{graphicx}
\usepackage{amsmath}
\usepackage{chngpage}
%\usepackage{bigints}

\usepackage{vmargin}
% left top textwidth textheight headheight
% headsep footheight footskip
\setmargins{2.0cm}{2.5cm}{16 cm}{22cm}{0.5cm}{0cm}{1cm}{1cm}
\renewcommand{\baselinestretch}{1.3}

\setcounter{MaxMatrixCols}{10}

\begin{document}
  \begin{table}[ht!]
     \centering
     \begin{tabular}{|p{15cm}|}
     \hline        
8. (a) Using two examples to illustrate your answers, discuss the uses of the F distribution in statistical methods. (6) 
\\ \hline
      \end{tabular}
    \end{table}
    
%%%%%%%%%%%%%%%%%%%%%%%%%%%%%%%%%%%%%%%%%%%%%%%%%%%%%%%%%%%%%%%%%%%%%%%%%%%%%%%%%%%%%%%%%%%%%%%%%%%%%%%%%%%%%%%%%%%%%%%%%%%%%%%%%
\begin{table}[ht!]
 
\centering
 
\begin{tabular}{|p{15cm}|}
 
\hline  


 
 (b) A manufacturer of pharmaceutical products purchases one particular material from two different suppliers.  A mean level of impurities in the raw material is approximately the same for each supplier, but the manufacturer is concerned about the variability in the level of impurities from shipment to shipment.  If the percentage of impurities varies excessively for one source of supply, it can affect the quality of the pharmaceutical product.  To compare the variation in percentage impurities for the two suppliers, the manufacturer selects 16 shipments at random from each supplier and measures the percentage of impurities in the raw material for each shipment.  The data obtained are given in the following table. 

\newpage 
 
    Percentage of impurities 
\begin{tabular}{c|c}
 
Supplier 1 &  Supplier 2 \\
2.2 & 1.6 \\
1.3 & 1.7 \\
2.3 & 2.1 \\
1.6 & 2.0 \\
1.5 & 2.1 \\
2.3 & 1.6 \\
2.2 & 1.7 \\
2.1 & 1.7 \\
1.6 & 1.9 \\
1.7 & 1.8 \\
1.6 & 2.1 \\
2.4 & 2.1 \\
1.4 & 2.0 \\
2.1 & 1.8 \\
1.5 & 1.9 \\
2.3 & 2.2 \\
\end{tabular}
 
 
Using a suitable statistical test, examine whether there is sufficient evidence to indicate a difference in the variability of the impurity levels between shipments of the raw material for the two suppliers.  Based on the results of your test, what recommendation would you make to the pharmaceutical company? (14) 
\\ \hline
  
\end{tabular}

\end{table}


\begin{enumerate}
    \item If two samples are drawn from Normal distributions N(ui; ¾2
i ) i = 1; 2 and ui; ¾2
i not
known, an estimate of each variance is \[s2
i =
mPi
j=1
(xij ¡ ¯xi)2=(mi ¡1)\] where mi are sample
sizes. A test of the N.H, ”$\sigma^2_1 = \sigma^2_2 $” is given as F(m1¡1;m2¡1) = s21
=s22
, where s21
> s22
to
make use of standard tables. (see the example below)
In the Analysis of Variance (see, e.g., question 5), mean squares computed on the
Null Hypothesis are all estimates of the same residual variation ¾2, and so the ratio
treatments mean square
residual mean square will follows an F-distribution with the appropriate degrees of freedom.
\item 

\begin{center}
\begin{tabular}{c|c}
Supplier 2 &  Supplier 1: \\
m = 16 &  m = 16;\\
$\sum x_i = 30.1$ &  $\sum x_i = 30.3$ \\
$\sum x_i = 58:85$     & $\sum x_i =  57:97$ \\
$s2 = 0:14829$ & $s2 = 0:03929$ \\
\end{tabular}
\end{center}
\[ F(15;15) = \frac{0:14829}{0:03929} = 3.77\]
leads us to reject a N.H. of equal variances in favor of an A.H. $\sigma^2_1 > \sigma^2_2 $. The company is
well advised to use supplier 1.
\end{enumerate}
\end{document}

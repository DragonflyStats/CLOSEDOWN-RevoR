\documentclass[a4paper,12pt]{article}

%%%%%%%%%%%%%%%%%%%%%%%%%%%%%%%%%%%%%%%%%%%%%%%%%%%%%%%%%%%%%%%%%%%%%%%%%%%%%%%%%%%%%%%%%%%%%%%%%%%%%%%%%%%%%%%%%%%%%%%%%%%%%%%%%%%%%%%%%%%%%%%%%%%%%%%%%%%%%%%%%%%%%%%%%%%%%%%%%%%%%%%%%%%%%%%%%%%%%%%%%%%%%%%%%%%%%%%%%%%%%%%%%%%%%%%%%%%%%%%%%%%%%%%%%%%%

\usepackage{eurosym}
\usepackage{vmargin}
\usepackage{amsmath}
\usepackage{graphics}
\usepackage{epsfig}
\usepackage{enumerate}
\usepackage{multicol}
\usepackage{subfigure}
\usepackage{fancyhdr}
\usepackage{listings}
\usepackage{framed}
\usepackage{graphicx}
\usepackage{amsmath}
\usepackage{chngpage}

%\usepackage{bigints}
\usepackage{vmargin}

% left top textwidth textheight headheight

% headsep footheight footskip

\setmargins{2.0cm}{2.5cm}{16 cm}{22cm}{0.5cm}{0cm}{1cm}{1cm}

\renewcommand{\baselinestretch}{1.3}

\setcounter{MaxMatrixCols}{10}

\begin{document}
Higher Certificate, Paper III, 2006. Question 3
\begin{enumerate}[(a)]
\item In a simple random sample (from a finite population) every possible selection of a sample of given size has the same probability of being chosen. This also has the effect that every individual in the population has the same probability of being selected for the sample.
\item Given a large population, of N items listed in some order (e.g. alphabetical) which is not related to trends in the characteristics being observed or measured, a systematic sample will be a valid alternative to simple random sampling. [If n items are required for the sample, with N = nk, take a random starting point among the first k in the list and every kth thereafter.]
\item(a) 100 claim to be regular users, so  $\hat{p}= 0.50$.
\item (b) We have a contingency table as follows. The null hypothesis is that there is no association between faculty and library use. The expected frequencies are shown in brackets in each cell (e.g. 25.0 = 50 × 100 / 200).
Library use
Regular
Non-regular
Total
Engineering
25 (25.0)
25 (25.0)
50
Business
42 (35.0)
28 (35.0)
70
Arts
21 (17.5)
14 (17.5)
35
Faculty
Informatics
12 (22.5)
33 (22.5)
45
Total
100
100
200
The value of the test statistic is
()()()()222222525.02525.04235.03322.5...14.0025.025.035.022.5X−−−−=++++=.
This is referred to . It is highly significant (the 1% critical point is 11.345). There is strong evidence of an association between faculty and library use. 23χ

%%%%%%%%%%%%%%%%%%%%%%%%%%%%%%%%%%%%%%%%%%%%

\item(c) fmpp− is estimated by 356565135ˆˆ0.5380.4810.057fmpp−=−=−=. The estimated variance of ˆˆfmpp− is given by
()()ˆˆ1ˆˆ10.0038230.0018490.005673ffmmfmppppnn−−+=+=.
Thus the approximate 95\% confidence interval for fmpp− is given by 0.057 ± (1.96×√0.005673), i.e. it is (–0.09, 0.21).
This interval contains 0, so there is no real evidence of a difference in library use between the sexes.
\item (d) Several factors make the sample design unlikely to represent the student population closely. 

\begin{itemize}
    \item Faculties probably contain different numbers of students, there will be different male/female ratios in different faculties (though here there does not seem to be a sex difference in the results), and proportional sampling from these different sub-groups would be desirable.
    \item The timing of the survey will be important, as timetables in different faculties, and other student activities, could affect the structure of a sample and bias it towards particular groups.
    \item Students may be accompanied by others of similar interests or habits, and therefore sample units may not be independent.
\end{itemize}


 \end{enumerate}
 \end{document}
 

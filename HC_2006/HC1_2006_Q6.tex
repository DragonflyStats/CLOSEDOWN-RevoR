\documentclass[a4paper,12pt]{article}

%%%%%%%%%%%%%%%%%%%%%%%%%%%%%%%%%%%%%%%%%%%%%%%%%%%%%%%%%%%%%%%%%%%%%%%%%%%%%%%%%%%%%%%%%%%%%%%%%%%%%%%%%%%%%%%%%%%%%%%%%%%%%%%%%%%%%%%%%%%%%%%%%%%%%%%%%%%%%%%%%%%%%%%%%%%%%%%%%%%%%%%%%%%%%%%%%%%%%%%%%%%%%%%%%%%%%%%%%%%%%%%%%%%%%%%%%%%%%%%%%%%%%%%%%%%%

\usepackage{eurosym}
\usepackage{vmargin}
\usepackage{amsmath}
\usepackage{graphics}
\usepackage{epsfig}
\usepackage{enumerate}
\usepackage{multicol}
\usepackage{subfigure}
\usepackage{fancyhdr}
\usepackage{listings}
\usepackage{framed}
\usepackage{graphicx}
\usepackage{amsmath}
\usepackage{chngpage}

%\usepackage{bigints}
\usepackage{vmargin}

% left top textwidth textheight headheight

% headsep footheight footskip

\setmargins{2.0cm}{2.5cm}{16 cm}{22cm}{0.5cm}{0cm}{1cm}{1cm}

\renewcommand{\baselinestretch}{1.3}

\setcounter{MaxMatrixCols}{10}

\begin{document}
Higher Certificate, Paper I, 2006. Question 6
%%%%%%%%%%%%%%%%%%%%%%%%%%%%%%%%%%%%%%%%%%%%%%%%%%%%%%%%%%%%%%%%%%%%%%%%%%%%
%-------------------------------------------------%

\begin{table}[ht!]
     
\centering
     
\begin{tabular}{|p{15cm}|}
     
\hline        

\noindent
The random variable X has the distribution with probability density function
\[ f(x) =  \frac{\lambda}{2}  e^{-\lambda |x|},  \mbox{ where } –\infty < x < \infty .\]
Sketch a graph of this density function.
\\ \hline
      
\end{tabular}
    
\end{table}


%-------------------------------------------------%

%%%%%%%%%%%%%%%%%%%%%%%%%%%%%%%%%%%%%%%%%%%%%%%%%%%%%%%%%%%%%%%%%%%%%%%%%%%%
(),2xfxex$\lambda$$\lambda$−=−∞<<∞
0.00.00xf(
x)lambda/2

\newpage

%-------------------------------------------------%

\begin{table}[ht!]
     
\centering
     
\begin{tabular}{|p{15cm}|}
     
\hline        

\noindent

Write down $E(X)$ and show that $\operatorname{Var}(X) = \frac{2}{\lambda^2}$. Find also the semi-interquartile range of $X$.
\\ \hline
      
\end{tabular}
    
\end{table}


%-------------------------------------------------%
\begin{itemize}
    \item By symmetry, $E(X) = 0$.
Hence 
\begin{eqnarray*}
E(X^2)  &=& \int^{\infty}_{\infty} x^2 \;\frac{\lambda}{2}\;  e^{-\lambda |x|} dx \\
&=& 2 \times  \int^{\infty}_{0} x^2 \;\frac{\lambda}{2}\;  e^{- \lambda x} dx \\ 
&=& \lambda \times  \int^{\infty}_{0} x^2 \;  e^{- \lambda x} dx \\ 
&  & \mbox{(Integration by Parts)}
\end{eqnarray*}

\begin{framed}
\[I = uv - \int vdu\]
\begin{itemize}
    \item $u = x^2 $
    \item $dv = e^{- \lambda x}\;dx$ $\therefore$ $v = \int dv = \int  dxe^{- \lambda x} dx$
\end{itemize}
\end{framed}
\begin{eqnarray}
\operatorname{Var}(X) &=& E(X^2) - [0]^2 \\
&=& \frac{\lambda}{2} \int^{\infty}_{-\infty} x^2 e^{\lambda|x|}dx \\
&=& \frac{\lambda}{2} \int^{}_{-\infty} x^2 e^{\lambda\;x}dx + \frac{\lambda}{2} \int^{\infty}_{0} x^2 e^{-\lambda\;x}dx \\\\
\end{eqnarray}


\item Substituting $u = –x$ in the first integral gives , which is the same as the second.
\begin{eqnarray*} 
E(x^2) &=&  \lambda \int^{\infty}_{0} x^2 e^{-\lambda\;x}dx \\
&=& \lambda \left( \left[ x^2 \frac{e^{-\lambda\;x}}{-\lambda}   \right]^{\infty}_{0}  + \int^{\infty}_{0} 2x e^{-\lambda\;x}dx \rigth)\\
&=& (0-0) + \int^{\infty}_{0} 2x e^{-\lambda\;x}dx \\
&=& MOER
&=& \frac{2}{\lambda^2}
\end{eqnarray*}
\item Hence we get, integrating by parts, 

\begin{eqnarray*}
\int^{\infty}_0 e^{\lambda} dx  &=&  \int^{\infty} e^{\lambda} dx        \\
  &=&  \lambda {e^{\lambda}      \\
  &=&  \int^{\infty}  dx        \\
\end{eqnarray*}

\[
200.2xxeexxdx$\lambda$$\lambda$$\lambda$$\lambda$$\lambda$∞−−∞⎧⎫⎡⎤⎪⎪=+⎨⎬⎢⎥−⎣⎦⎪⎪⎩⎭\]
\[0[00]2xxedx$\lambda$∞−=−+∫
002xxeexdx$\lambda$$\lambda

\begin{eqnarray*}
\int^{\infty}_0 e^{\lambda} dx  &=&  \int^{\infty} e^{\lambda} dx        \\
  &=&  \lambda {e^{\lambda}      \\
  &=&  \int^{\infty}  dx        \\
\end{eqnarray*}
 
%\lambda\lambda$∞−−∞⎧⎫⎡⎤⎪⎪=+⎨⎬⎢⎥−⎣⎦⎪⎪⎩⎭∫
% []0200xe$\lambda$$\lambda$$$∞−⎡⎤=−+⎢⎥−⎣⎦ 22$\lambda$=.\]

\item


If $Q$, $q$ are the upper and lower quartiles, we have


\[  \int^Q_0 \frac{1}{2} \lambda e^{-\lambda x} dx = \frac{1}{4},\]

and $q$ will be the same distance below 0 by symmetry.


\[ \frac{1}{4} = \left[ - \frac{1}{2} e^{-\lambda x} \right]^Q_0 = \frac{1}{2} \left(  -e^{-\lambda Q} + 1 \right)\]  giving
${ \displaystyle \frac{1}{2}  = 1 -e^{-\lambda Q} }$. Therefore $\lambda Q = \log 2$. Hence
the semi-interquartile range is ${ \displaystyle \frac{\log 2}{\lambda} }$.


%-------------------------------------------------%

\begin{table}[ht!]
     
\centering
     
\begin{tabular}{|p{15cm}|}
     
\hline        

\noindent

A random sample $x_1, x_2, \ldots, x_n$ is taken from this distribution. Show that the maximum likelihood estimate of $\lambda$ is given by
\[ \hat{\lambda}  = \frac{n}{\dislaystyle{\sum^{n}_{i=1} |x_i| }} \].
\\ \hline
      
\end{tabular}
    
\end{table}


%-------------------------------------------------%

\item Therefore $\lambda$Q = log 2.
\item Hence the semi-interquartile range is (log 2)/$\lambda$.
\end{itemize}

\[L = \prod^{n}_{i=1} \left( \frac{\lambda}{2} e^{-\lambda|x_i|}  \right)  = \left( \frac{\lambda}{2}  \right) e^{-\lambda \sum|x_i|},\]

and hence 
\[ \log L = \mbox{ constant }+ n \log \lambda − \lambda \sum_{i=1} |x_i| .\]

Therefore

\[  \frac{d \log \lambda L}{d \lambda} = \frac{n}{2} - \sum_{i=1} |x_i|  \]

which on setting equal to zero gives that the maximum
likelihood estimate is $ { \displaystyle \hat{\lambda} = \frac{n}{\sum_{i=1} |x_i|}  }$

Consideration of $ { \displaystyle  \frac{d^2 \log \lambda L}{d \lambda^2} }$ confirms that this is a maximum.

\[ \frac{d^2 \log \lambda L}{d \lambda^2} = \frac{-n}{\lambda^2} <0.\]
\end{document}

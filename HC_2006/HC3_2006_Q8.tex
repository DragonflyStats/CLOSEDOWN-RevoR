\documentclass[a4paper,12pt]{article}

%%%%%%%%%%%%%%%%%%%%%%%%%%%%%%%%%%%%%%%%%%%%%%%%%%%%%%%%%%%%%%%%%%%%%%%%%%%%%%%%%%%%%%%%%%%%%%%%%%%%%%%%%%%%%%%%%%%%%%%%%%%%%%%%%%%%%%%%%%%%%%%%%%%%%%%%%%%%%%%%%%%%%%%%%%%%%%%%%%%%%%%%%%%%%%%%%%%%%%%%%%%%%%%%%%%%%%%%%%%%%%%%%%%%%%%%%%%%%%%%%%%%%%%%%%%%

\usepackage{eurosym}
\usepackage{vmargin}
\usepackage{amsmath}
\usepackage{graphics}
\usepackage{epsfig}
\usepackage{enumerate}
\usepackage{multicol}
\usepackage{subfigure}
\usepackage{fancyhdr}
\usepackage{listings}
\usepackage{framed}
\usepackage{graphicx}
\usepackage{amsmath}
\usepackage{chngpage}

%\usepackage{bigints}
\usepackage{vmargin}

% left top textwidth textheight headheight

% headsep footheight footskip

\setmargins{2.0cm}{2.5cm}{16 cm}{22cm}{0.5cm}{0cm}{1cm}{1cm}

\renewcommand{\baselinestretch}{1.3}

\setcounter{MaxMatrixCols}{10}

\begin{document}
Higher Certificate, Paper III, 2006. Question 8

%%%%%%%%%%%%%%%%%%%%%%%%%%%%%%%%%%%%%%%%%%%%%%%%%%%%%%%%%%%%%%%%%%%%%%%%%%%%
\begin{framed}
8.
In a project designed to see whether testing procedures in different laboratories give
similar results, fatigue tests at three different strain levels (level 1 lowest, level 
highest) were carried out on samples from the same batch of a composite material.
The tests were carried out at 9 different laboratories across Europe. The results
(cycles to fatigue) from each laboratory and the means of the samples at each strain
level are shown in the table below.
laboratory strain level 1
results mean
(1) strain level 
results mean
() A 75, 688, 85 75 16, 169 B 51, 4000, 400 97 977, 115 C 8, 5558, 6910 540 1516, 1607, 1650 1740, 185 1796 447, 718, 95 700
E 48, 69, 890 6517 1488, 1501, 1516 150 674, 681, 781 71
F 400, 418, 40 41 1095, 105, 190 1197 465, 80, 95 41.
G 5000, 545, 545 5 101, 1156, 18 114 80, 408, 454 414
H 510, 604, 811,
900, 986 76 78, 771, 864,
88 814 0, 5, 9 19
J 47, 800, 567 771 957, 1156, 10 1105 5, 487 56
D
strain level 
results mean
()
1515 690, 75 71.5
1065 48, 460, 47 45.7
1591 60, 675 65.5
Two questions are of interest:
(1) Are there substantial differences between the measurements obtained at
different laboratories?
() Can a model be found to predict the cycles to fatigue at strain levels other than
those tested?
Write a preliminary report on your general conclusions about the questions posed in
(1) and (), based on the tabulated data, and supported by suitable graphical evidence.
(You are advised to avoid spending excessive time on detailed statistical analysis or
repetitive calculations.)
Express your report in a way that makes it accessible to non-technical readers.
(0)
\end{framed}
%%%%%%%%%%%%%%%%%%%%%%%%%%%%%%%%%%%%%%%%%%%%%%%%%%%%%%%%%%%%%%%%%%%%%%%%%%%%

\begin{itemize}
\item It is useful to have a graph showing the measurements made by each laboratory on each strain (laboratories A – J are relabelled 1 – 9). Individual responses are plotted.
\item The substantial difference in mean levels between the three strains makes it hard to present the graphical results on a convenient scale. However, several points emerge.
\item (1) Some laboratories have consistently lower readings than others. For example, H has by some way the lowest mean throughout; A, C, E are high; B, J tend to be low. 
\item Variability within laboratories is also very different; this can be seen especially for strain 1, where C, E, J give very wide ranges while F, G, H do not. 
\item The basic material used does not appear to have been so variable, because not all within-laboratory variation is large; technical reasons in respect of resources of equipment or people is a more likely reason.
\item (2) Level 1 is the lowest strain level, and it shows much higher means and much more variation than levels 2 and 3. There is clearly an inverse relationship between cycles to fatigue and strain level. However, a model assuming constant variance would not be suitable; transformations of the y (cycles) variable could be explored, possibly log y. 
\item Prediction will be more accurate at higher strain levels, and should only be attempted within the range of levels already tested; extrapolation below level 1 or above level 3 would be unwise.
\item Overall, the laboratories lack consistency. If the aim is to have a laboratory-independent prediction, laboratory practice needs to be more consistent. If this cannot be achieved, the model used needs to incorporate a laboratory effect.
 \end{itemize}
 \end{document}
 

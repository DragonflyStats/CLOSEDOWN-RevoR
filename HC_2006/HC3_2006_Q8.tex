\documentclass[a4paper,12pt]{article}

%%%%%%%%%%%%%%%%%%%%%%%%%%%%%%%%%%%%%%%%%%%%%%%%%%%%%%%%%%%%%%%%%%%%%%%%%%%%%%%%%%%%%%%%%%%%%%%%%%%%%%%%%%%%%%%%%%%%%%%%%%%%%%%%%%%%%%%%%%%%%%%%%%%%%%%%%%%%%%%%%%%%%%%%%%%%%%%%%%%%%%%%%%%%%%%%%%%%%%%%%%%%%%%%%%%%%%%%%%%%%%%%%%%%%%%%%%%%%%%%%%%%%%%%%%%%

\usepackage{eurosym}
\usepackage{vmargin}
\usepackage{amsmath}
\usepackage{graphics}
\usepackage{epsfig}
\usepackage{enumerate}
\usepackage{multicol}
\usepackage{subfigure}
\usepackage{fancyhdr}
\usepackage{listings}
\usepackage{framed}
\usepackage{graphicx}
\usepackage{amsmath}
\usepackage{chngpage}

%\usepackage{bigints}
\usepackage{vmargin}

% left top textwidth textheight headheight

% headsep footheight footskip

\setmargins{2.0cm}{2.5cm}{16 cm}{22cm}{0.5cm}{0cm}{1cm}{1cm}

\renewcommand{\baselinestretch}{1.3}

\setcounter{MaxMatrixCols}{10}

\begin{document}
Higher Certificate, Paper III, 2006. Question 8
\begin{itemize}
\item It is useful to have a graph showing the measurements made by each laboratory on each strain (laboratories A – J are relabelled 1 – 9). Individual responses are plotted.
\item The substantial difference in mean levels between the three strains makes it hard to present the graphical results on a convenient scale. However, several points emerge.
\item (1) Some laboratories have consistently lower readings than others. For example, H has by some way the lowest mean throughout; A, C, E are high; B, J tend to be low. Variability within laboratories is also very different; this can be seen especially for strain 1, where C, E, J give very wide ranges while F, G, H do not. 
\item The basic material used does not appear to have been so variable, because not all within-laboratory variation is large; technical reasons in respect of resources of equipment or people is a more likely reason.
\item (2) Level 1 is the lowest strain level, and it shows much higher means and much more variation than levels 2 and 3. There is clearly an inverse relationship between cycles to fatigue and strain level. However, a model assuming constant variance would not be suitable; transformations of the y (cycles) variable could be explored, possibly log y. \item Prediction will be more accurate at higher strain levels, and should only be attempted within the range of levels already tested; extrapolation below level 1 or above level 3 would be unwise.
\item Overall, the laboratories lack consistency. If the aim is to have a laboratory-independent prediction, laboratory practice needs to be more consistent. If this cannot be achieved, the model used needs to incorporate a laboratory effect.
 \end{itemize}
 \end{document}
 

\documentclass[a4paper,12pt]{article}

%%%%%%%%%%%%%%%%%%%%%%%%%%%%%%%%%%%%%%%%%%%%%%%%%%%%%%%%%%%%%%%%%%%%%%%%%%%%%%%%%%%%%%%%%%%%%%%%%%%%%%%%%%%%%%%%%%%%%%%%%%%%%%%%%%%%%%%%%%%%%%%%%%%%%%%%%%%%%%%%%%%%%%%%%%%%%%%%%%%%%%%%%%%%%%%%%%%%%%%%%%%%%%%%%%%%%%%%%%%%%%%%%%%%%%%%%%%%%%%%%%%%%%%%%%%%

\usepackage{eurosym}
\usepackage{vmargin}
\usepackage{amsmath}
\usepackage{graphics}
\usepackage{epsfig}
\usepackage{enumerate}
\usepackage{multicol}
\usepackage{subfigure}
\usepackage{fancyhdr}
\usepackage{listings}
\usepackage{framed}
\usepackage{graphicx}
\usepackage{amsmath}
\usepackage{chngpage}

%\usepackage{bigints}
\usepackage{vmargin}

% left top textwidth textheight headheight

% headsep footheight footskip

\setmargins{2.0cm}{2.5cm}{16 cm}{22cm}{0.5cm}{0cm}{1cm}{1cm}

\renewcommand{\baselinestretch}{1.3}

\setcounter{MaxMatrixCols}{10}

\begin{document}

%- Higher Certificate, Paper I, 2006. Question 1
\begin{table}[ht!]

 \centering

 \begin{tabular}{|p{15cm}|}

 \hline
\noindent Credit card holders with a certain bank are each assigned a 4-digit PIN (personal identity number). How many different possible PINs are there under each of the following conditions?
\\ \hline

  \end{tabular}

\end{table}




\begin{table}[ht!]

 \centering

 \begin{tabular}{|p{15cm}|}

 \hline

\noindent \textbf{Part (a)}

Condition: The first digit may not be zero.
\\ \hline

  \end{tabular}

\end{table}
%%%%%%%%%%%%%%%%%%%%%%%%%%%%%%%%%%%%

\begin{enumerate}[(a)]
\item The first place can be occupied by 9 different digits, 1 to 9. Each of the other three places can be occupied by 10 digits, 0 to 9.
Hence there are $9 × 10 × 10 × 10 = 9000$ possible PINs.

\begin{table}[ht!]
 \centering
 \begin{tabular}{|p{15cm}|}
 \hline
\noindent \textbf{Part (b)}\\
Condition: The first digit may not be zero, and the four digits may not all be the same.
\\ \hline
  \end{tabular}
\end{table}
\item All of the combinations in (i) are allowed except $\{1111, 2222, \ldots, 9999\}$, so there are $9000 - 9 = 8991$ possibilities.

%%%%%%%%%%%%%%%%%%%%%%%%%%%%%%%%%%%%%%%%%%%%%%%%%%%%%%%%%%%%%%%%%%%%%%%%%%%%%%%%%%%%%%%
\newpage
\begin{table}[ht!]
 \centering
 \begin{tabular}{|p{15cm}|}
 \hline
\noindent \textbf{Part (c)}\\
Condition: No zeros are allowed in any position and all four digits must be different.
\\ \hline
  \end{tabular}
\end{table}
\item Only the 9 digits 1 to 9 can be used. The first place can be filled in 9 ways, the second in 8, the third in 7 and the last in 6. So there are 9 × 8 × 7 × 6 = 3024 possibilities.

%%%%%%%%%%%%%%%%%%%%%%%%%%%%%%%%%%%%%%%%%%%%%%%%%%%%%%%%%%%%%%%%%%%%%%
\begin{table}[ht!]
 \centering
 \begin{tabular}{|p{15cm}|}
 \hline
\noindent \textbf{Part (d)}\\
Condition: Zeros are allowed in all positions but sequences of four consecutive digits up (e.g. 2 3 4 5) or down (e.g. 3 2 1 0) are not allowed.
\\ \hline
  \end{tabular}
\end{table}



\item With all 10 digits possible in any position, there would be 104 PINs. There are 7 increasing sequences (0123, 1234, …, 6789) and 7 decreasing sequences $(9876, 8765, \ldots, 3210)$, which are not allowed. The number of possible PINs is therefore $10^4 - 14 = 9986$.




%%%%%%%%%%%%%%%%%%%%%%%%%%%%%%%%%%%%%%%%%%%%%%%%%%%%%%%%%%%%%%%%%%%%%%
\begin{table}[ht!]
 \centering
 \begin{tabular}{|p{15cm}|}
 \hline
\noindent \textbf{Part (e)}\\
Condition: Zeros are allowed in all positions but no digit may occur more than twice.
\\ \hline

  \end{tabular}

\end{table}
\item All of the $10^4$ combinations are allowed except:
(a) the 10 where all 4 digits are the same: $\{0000, 1111, \ldots, 9999\}$;
(b) those where one digit occurs three times and another just once.  \begin{itemize}
\item There are $10 \times 9 = 90$ ways of choosing the two digits. 
\item But note that, for example, 2333, 3233, 3323 and 3332 are four different PINs; whichever two digits occur, the odd one out can be in any of the 4 places in the PIN. \item Therefore there are $4 \times 90 = 360$ PINs of this sort.
\item The number of possible PINs is therefore $10^4 - 10 - 360 = 9630$.
\end{itemize}

\end{enumerate}

\end{document}

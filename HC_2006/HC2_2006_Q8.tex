\documentclass[a4paper,12pt]{article}

%%%%%%%%%%%%%%%%%%%%%%%%%%%%%%%%%%%%%%%%%%%%%%%%%%%%%%%%%%%%%%%%%%%%%%%%%%%%%%%%%%%%%%%%%%%%%%%%%%%%%%%%%%%%%%%%%%%%%%%%%%%%%%%%%%%%%%%%%%%%%%%%%%%%%%%%%%%%%%%%%%%%%%%%%%%%%%%%%%%%%%%%%%%%%%%%%%%%%%%%%%%%%%%%%%%%%%%%%%%%%%%%%%%%%%%%%%%%%%%%%%%%%%%%%%%%

\usepackage{eurosym}
\usepackage{vmargin}
\usepackage{amsmath}
\usepackage{graphics}
\usepackage{epsfig}
\usepackage{enumerate}
\usepackage{multicol}
\usepackage{subfigure}
\usepackage{fancyhdr}
\usepackage{listings}
\usepackage{framed}
\usepackage{graphicx}
\usepackage{amsmath}
\usepackage{chngpage}

%\usepackage{bigints}
\usepackage{vmargin}

% left top textwidth textheight headheight

% headsep footheight footskip

\setmargins{2.0cm}{2.5cm}{16 cm}{22cm}{0.5cm}{0cm}{1cm}{1cm}

\renewcommand{\baselinestretch}{1.3}

\setcounter{MaxMatrixCols}{10}

\begin{document}
Higher Certificate, Paper II, 2006.  Question 8 
 
 
(i) We have a 2×2 contingency table.  The null hypothesis is that there is no association between an individual's sex and the chance of he or she having a recently recorded cholesterol measurement.  The contingency table is as follows, with the expected frequencies in brackets in each cell (e.g. 88.48 = 131 × 206 / 305). 
 
  Cholesterol level recorded    No Yes Total Female 109     (88.48) 22  (42.52) 131    Sex Male   97   (117.52) 77  (56.48) 174  Total 206 99 305 
 
All the differences between observed and expected frequencies are ±20.52, becoming ±20.02 if Yates' correction is used.  Thus the usual test statistic can be calculated as (using Yates' correction) 
 
() 2 111120.02 24.46 88.48 42.52 117.52 56.48 ⎧⎫ + + + = ⎨⎬ ⎩⎭
 
 
(or 25.70 if Yates' correction is not used).  This is referred to .  This is very highly significant (for example, the 1% point is 6.635);  we have very strong evidence to reject the null hypothesis and conclude that there is an association. 2 1χ
 
 
(ii) f m pp − is estimated by 77 22 131 174ˆˆ 0.168 0.443 0.275 fm pp − = − = − =− .  The estimated variance of ˆˆ f m pp − is given by 
 ()() ˆˆ 1 ˆˆ 1 0.001067 0.001418 0.002485ff mm fm pppp nn − − + = + = . 
 
Thus the approximate 95% confidence interval for f m pp − is given by –0.275 ± (1.96×√0.002485) i.e. it is (–0.177, –0.373). 
 
 
(iii) There is clear evidence that the proportions are not the same for men and women.  In part (i), this is interpreted via the very strong evidence of an association.  In part (ii), the confidence interval does not contain 0, indeed it is a long way from 0, again giving very strong evidence of a real difference. 

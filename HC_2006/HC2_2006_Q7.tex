\documentclass[a4paper,12pt]{article}

%%%%%%%%%%%%%%%%%%%%%%%%%%%%%%%%%%%%%%%%%%%%%%%%%%%%%%%%%%%%%%%%%%%%%%%%%%%%%%%%%%%%%%%%%%%%%%%%%%%%%%%%%%%%%%%%%%%%%%%%%%%%%%%%%%%%%%%%%%%%%%%%%%%%%%%%%%%%%%%%%%%%%%%%%%%%%%%%%%%%%%%%%%%%%%%%%%%%%%%%%%%%%%%%%%%%%%%%%%%%%%%%%%%%%%%%%%%%%%%%%%%%%%%%%%%%

\usepackage{eurosym}
\usepackage{vmargin}
\usepackage{amsmath}
\usepackage{graphics}
\usepackage{epsfig}
\usepackage{enumerate}
\usepackage{multicol}
\usepackage{subfigure}
\usepackage{fancyhdr}
\usepackage{listings}
\usepackage{framed}
\usepackage{graphicx}
\usepackage{amsmath}
\usepackage{chngpage}

%\usepackage{bigints}
\usepackage{vmargin}

% left top textwidth textheight headheight

% headsep footheight footskip

\setmargins{2.0cm}{2.5cm}{16 cm}{22cm}{0.5cm}{0cm}{1cm}{1cm}

\renewcommand{\baselinestretch}{1.3}

\setcounter{MaxMatrixCols}{10}

\begin{document}Higher Certificate, Paper II, 2006.  Question 7 
\begin{framed}

6 
Turn over 
7. (i) Compare the uses of the independent samples and paired samples t tests.  Explain clearly in which circumstances each method should be used in preference to the other, illustrating your answer with appropriate examples.  State briefly the appropriate assumptions made in each case. (7) 
 
(ii) In a particular population, it was of interest whether married men were, on average, younger or older than their respective wives.  A random sample of 15 couples was taken, their ages being given in the table below. 
 
Couple number Husband's age Wife's age   1 39 32   2 38 31   3 73 68   4 54 58   5 24 26   6 57 53   7 49 48   8 63 69   9 48 47 10 44 46 11 26 25 12 64 62 13 42 40 14 45 48 15 61 57 
 
 
(a) Use the above data to calculate the mean and standard deviation of the differences between the ages of the husbands and their respective wives. (4) 
 
(b) Is there evidence that the mean difference in ages between husbands and wives is non-zero? (6) 
 
(c) Obtain a 95% confidence interval for the mean difference in ages of the husbands and their respective wives. (3) 
 
 
 
\end{framed}

 
(i) Independent samples are those where different (and unrelated) subjects, or units of experimental material, are used for the two samples.  They are randomly selected from their corresponding populations, and a measurement is taken on each unit.  For example, the units in one sample of seedlings taken from a forestry nursery are given treatment A, and those in the other sample treatment B;  A and B might be different fertiliser or cultivation treatments given at the same stage in plant growth.  The measured response is a size or health measurement taken at the same age of plants. 
 The populations underlying the sets of sample data are assumed to be Normally distributed with the same variance. 
 Paired samples use the same units for both treatments (e.g. a "before-andafter" study), or pairs of units as closely alike as possible.  For example, a medical trial of alternative drugs for a chronic (long-lasting) condition might use pairs of patients whose conditions before treatment are very similar, one of each pair receiving drug A and the other drug B (allocated at random). 
 The underlying distribution of differences within pairs needs to be assumed Normal. 
 
 
(ii) Let H, W represent the ages of Husband and Wife.  We have 15 (paired) observations hi, wi and the differences di are 
 7,  7,  5,  –4,  –2,  4,  1,  –6,  1,  –2,  1,  2,  2,  –3,  4. 
 
 
(a) We have Σdi = 17,  Σdi2 = 235.  Thus d = 17/15 = 1.133 and 
 
2 2 1 17 235 15.4095 14 15d s ⎛⎞ = − = ⎜⎟ ⎝⎠
,  giving sd = 3.925. 
 
 
(b) The test statistic for testing the null hypothesis that H W μ μ = is 
 0 1.133 1.12 1.01415/d d s − == , 
 which is referred to t14.  This is not significant, so the null hypothesis cannot be rejected.  There is no evidence of a difference between the mean ages. 
 
 
(c) The confidence interval is given by 
15 ds dt ± where t is the doubletailed 5% point of t14 i.e. 2.145.  Thus the interval is given by , i.e. 1.1.133 (2.145 1.014) ±× 133 2.175 ± , i.e. it is (–1.04, 3.31). 

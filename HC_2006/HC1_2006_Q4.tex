\documentclass[a4paper,12pt]{article}

%%%%%%%%%%%%%%%%%%%%%%%%%%%%%%%%%%%%%%%%%%%%%%%%%%%%%%%%%%%%%%%%%%%%%%%%%%%%%%%%%%%%%%%%%%%%%%%%%%%%%%%%%%%%%%%%%%%%%%%%%%%%%%%%%%%%%%%%%%%%%%%%%%%%%%%%%%%%%%%%%%%%%%%%%%%%%%%%%%%%%%%%%%%%%%%%%%%%%%%%%%%%%%%%%%%%%%%%%%%%%%%%%%%%%%%%%%%%%%%%%%%%%%%%%%%%

\usepackage{eurosym}
\usepackage{vmargin}
\usepackage{amsmath}
\usepackage{graphics}
\usepackage{epsfig}
\usepackage{enumerate}
\usepackage{multicol}
\usepackage{subfigure}
\usepackage{fancyhdr}
\usepackage{listings}
\usepackage{framed}
\usepackage{graphicx}
\usepackage{amsmath}
\usepackage{chngpage}

%\usepackage{bigints}
\usepackage{vmargin}

% left top textwidth textheight headheight

% headsep footheight footskip

\setmargins{2.0cm}{2.5cm}{16 cm}{22cm}{0.5cm}{0cm}{1cm}{1cm}

\renewcommand{\baselinestretch}{1.3}

\setcounter{MaxMatrixCols}{10}

\begin{document}
% Higher Certificate, Paper I, 2006. Question 4
Let X represent cycling time without delays: $X \sim N(15, 1)$.
\begin{enumerate}
\item ()()17151720.97721PX−⎛⎞≤=Φ=Φ=⎜⎟⎝⎠.
[Φ denotes the cdf of the standard Normal distribution as usual.]
\item  Adding in the delay times, also Normally distributed [N(0.7, 0.09)], and letting $T$ denote the total time:
\begin{itemize}
\item $T \sim N(15.7, 1.09)$, so ()()1715.7171.2450.89341.09PT−⎛⎞≤=Φ=Φ=⎜⎟⎝⎠;
\item $T \sim N(16.4, 1.18)$, so ()()1716.4170.5520.70961.18PT−⎛⎞≤=Φ=Φ=⎜⎟⎝⎠;
\item $T \sim N(17.1, 1.27)$, so ()()1717.1170.08870.46461.27PT−⎛⎞≤=Φ=Φ−=⎜⎟⎝⎠.
\end{itemize}
\item The number of delays is distributed as B(3, ½). Hence the situations in (i), (ii)(a), (ii)(b) and (ii)(c) arise with probabilities 1/8, 3/8, 3/8 and 1/8 respectively, so the (unconditional) mean of the total journey time is
\[1331128.4()1515.716.417.116.0588888ET=×+×+×+×==\] minutes.
\item Mean time 1.502516.05,10TN⎛⎞⎜⎟⎝⎠∼.
()()1716.05172.4510.99290.15025PT−⎛⎞≤=Φ=Φ=⎜⎟⎝⎠.
\end{enumerate}
\end{document}

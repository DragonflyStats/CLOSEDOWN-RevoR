\documentclass[a4paper,12pt]{article}

%%%%%%%%%%%%%%%%%%%%%%%%%%%%%%%%%%%%%%%%%%%%%%%%%%%%%%%%%%%%%%%%%%%%%%%%%%%%%%%%%%%%%%%%%%%%%%%%%%%%%%%%%%%%%%%%%%%%%%%%%%%%%%%%%%%%%%%%%%%%%%%%%%%%%%%%%%%%%%%%%%%%%%%%%%%%%%%%%%%%%%%%%%%%%%%%%%%%%%%%%%%%%%%%%%%%%%%%%%%%%%%%%%%%%%%%%%%%%%%%%%%%%%%%%%%%

\usepackage{eurosym}
\usepackage{vmargin}
\usepackage{amsmath}
\usepackage{graphics}
\usepackage{epsfig}
\usepackage{enumerate}
\usepackage{multicol}
\usepackage{subfigure}
\usepackage{fancyhdr}
\usepackage{listings}
\usepackage{framed}
\usepackage{graphicx}
\usepackage{amsmath}
\usepackage{chngpage}

%\usepackage{bigints}
\usepackage{vmargin}

% left top textwidth textheight headheight

% headsep footheight footskip

\setmargins{2.0cm}{2.5cm}{16 cm}{22cm}{0.5cm}{0cm}{1cm}{1cm}

\renewcommand{\baselinestretch}{1.3}

\setcounter{MaxMatrixCols}{10}

\begin{document}
Higher Certificate, Paper III, 2006. Question 1
%%%%%%%%%%%%%%%%%%%%%%%%%%%%%%%%%%%%%%%%%%%%%%%%%%%%%%%%%%%%%%%%%%%%%%%%%%%%
\begin{framed}

An experiment was carried out on random samples of steel to assess whether
increasing the tempering temperature from 00 o C to 50 o C increases the failure stress.
The samples were treated and tested and gave the following failure stresses (measured
in units of 10 MegaPascals).
Sample mean

(i)

\begin{center}
\begin{tabular}{|ccc|}
	&	Tempering at 00 o C	&	Tempering at 50 o C	\\ \hline 
	&	66	&	54	\\
	&	49	&	63	\\
	&	58	&	43	\\
	&	77	&	56	\\
	&	39	&	47	\\
	&	51	&	97	\\
	&	46	&	85	\\
	&	91	&		\\
	&		&		\\
Sample Mean	&	59.6	&	63.6	\\ \hline 
Sample standard deviation	&	17.4	&	20.1	\\ \hline 
\end{tabular}
\end{center}


\begin{enumerate}
\item Carry out a t test, stating carefully your null and alternative hypotheses.
\item 
(ii) State the assumptions you have made in carrying out the t test and provide any
suitable graphical evidence relevant to these assumptions.
\item 
(iii) Re-analyse the data using a suitable non-parametric test, once again stating
carefully your null and alternative hypotheses.
\item 
(iv) Summarise your conclusions and comment on the methods used in your
analysis.
\end{enumerate}
\end{framed}
%%%%%%%%%%%%%%%%%%%%%%%%%%%%%%%%%%%%%%%%%%%%%%%%%%%%%%%%%%%%%%%%%%%%%%%%%%%%




\begin{itemize}
\item Null hypothesis: there is no difference between the population mean failure stresses of the 200°C and 250°C tempered steel. Alternative hypothesis: there is a difference between the mean failure stresses, that for 250°C being higher.
2221212128,7;59.6,63.6;17.4,20.1.nnxxss======

\item 
The pooled estimate of the assumed common variance (see part (ii)) is ()222(717.4)(620.1)/13s=×+× = 349.491 (so s = 18.69), with 13 d.f.
\item Thus the test statistic for testing that the population means are the same is
211187(0)4.00.419.681xxs−−==+,
which is referred to t13. This is not significant at the 5\% level (upper single-tailed 5\% point is 1.771), so there is no evidence to reject the null hypothesis – it seems that the population means are the same.
\item The two underlying populations are assumed Normally distributed, with the same variance. A dot plot helps to check these assumptions:
250°C
200°C
40
50
60
70
80
90
100
\item Sample sizes are small but the ranges are similar and it may be reasonable to assume similar variances. However, Normality is in some doubt, as there is no central clustering and some skewness and/or outliers may be present.


\item The Wilcoxon rank sum test (or, equivalently, the Mann Whitney U form of this test) is suitable. The null hypothesis is that the population median at 250°C is equal to that at 200°C, and the alternative is that it is greater. We first rank all 15 data items, as follows.
\begin{center}
\begin{tabular}{c|c|c}
Data	&	Ranks	&	Group	\\ \hline
39	&	1	&	A	\\ \hline
43	&	2	&	B	\\ \hline
46	&	3	&	A	\\ \hline
47	&	4	&	B	\\ \hline
49	&	5	&	A	\\ \hline
51	&	6	&	A	\\ \hline
54	&	7	&	B	\\ \hline
56	&	8	&	B	\\ \hline
58	&	9	&	A	\\ \hline
63	&	10	&	B	\\ \hline
66	&	11	&	A	\\ \hline
77	&	12	&	A	\\ \hline
85	&	13	&	B	\\ \hline
91	&	14	&	A	\\ \hline
97	&	15	&	B	\\ \hline
					
\end{tabular}
\end{center}

A refers to 200°C, B to 250°C.
The rank sum for the smaller sample (B) is 2 + 4 + 7 + 8 + 10 + 13 + 15 = 59.
\item The required test is one-sided. For a 5\% test, we refer this to the lower 5\% point for the W7,8 distribution as shown in the Society's statistical tables for use in examinations. This is 41 so, at the 5\% level of significance, we cannot reject the null hypothesis; it appears that the two populations are the same in this regard.
\item (iv) Neither test supports the hypothesis of an increase. The nonparametric test is more suitable for these data because there is doubt regarding the assumption of Normality that underlies the t test.
 \end{itemize}
 \end{document}
 

\documentclass[a4paper,12pt]{article}

%%%%%%%%%%%%%%%%%%%%%%%%%%%%%%%%%%%%%%%%%%%%%%%%%%%%%%%%%%%%%%%%%%%%%%%%%%%%%%%%%%%%%%%%%%%%%%%%%%%%%%%%%%%%%%%%%%%%%%%%%%%%%%%%%%%%%%%%%%%%%%%%%%%%%%%%%%%%%%%%%%%%%%%%%%%%%%%%%%%%%%%%%%%%%%%%%%%%%%%%%%%%%%%%%%%%%%%%%%%%%%%%%%%%%%%%%%%%%%%%%%%%%%%%%%%%

\usepackage{eurosym}
\usepackage{vmargin}
\usepackage{amsmath}
\usepackage{graphics}
\usepackage{epsfig}
\usepackage{enumerate}
\usepackage{multicol}
\usepackage{subfigure}
\usepackage{fancyhdr}
\usepackage{listings}
\usepackage{framed}
\usepackage{graphicx}
\usepackage{amsmath}
\usepackage{chngpage}

%\usepackage{bigints}
\usepackage{vmargin}

% left top textwidth textheight headheight

% headsep footheight footskip

\setmargins{2.0cm}{2.5cm}{16 cm}{22cm}{0.5cm}{0cm}{1cm}{1cm}

\renewcommand{\baselinestretch}{1.3}

\setcounter{MaxMatrixCols}{10}

\begin{document}
Higher Certificate, Paper III, 2006. Question 1

\begin{enumerate}
(i) Null hypothesis: there is no difference between the population mean failure stresses of the 200°C and 250°C tempered steel. Alternative hypothesis: there is a difference between the mean failure stresses, that for 250°C being higher.
2221212128,7;59.6,63.6;17.4,20.1.nnxxss======
The pooled estimate of the assumed common variance (see part (ii)) is ()222(717.4)(620.1)/13s=×+× = 349.491 (so s = 18.69), with 13 d.f.
Thus the test statistic for testing that the population means are the same is
211187(0)4.00.419.681xxs−−==+,
which is referred to t13. This is not significant at the 5% level (upper single-tailed 5% point is 1.771), so there is no evidence to reject the null hypothesis – it seems that the population means are the same.
(ii) The two underlying populations are assumed Normally distributed, with the same variance. A dot plot helps to check these assumptions:
250°C
200°C
40
50
60
70
80
90
100
Sample sizes are small but the ranges are similar and it may be reasonable to assume similar variances. However, Normality is in some doubt, as there is no central clustering and some skewness and/or outliers may be present.
Solution continued on next page
(iii) The Wilcoxon rank sum test (or, equivalently, the Mann Whitney U form of this test) is suitable. The null hypothesis is that the population median at 250°C is equal to that at 200°C, and the alternative is that it is greater. We first rank all 15 data items, as follows.
Data
39
43
46
47
49
51
54
56
58
63
66
77
85
91
97
Ranks
1
2
3
4
5
6
7
8
9
10
11
12
13
14
15
A
B
A
B
A
A
B
B
A
B
A
A
B
A
B
A refers to 200°C, B to 250°C.
The rank sum for the smaller sample (B) is 2 + 4 + 7 + 8 + 10 + 13 + 15 = 59.
The required test is one-sided. For a 5% test, we refer this to the lower 5% point for the W7,8 distribution as shown in the Society's statistical tables for use in examinations. This is 41 so, at the 5% level of significance, we cannot reject the null hypothesis; it appears that the two populations are the same in this regard.
(iv) Neither test supports the hypothesis of an increase. The nonparametric test is more suitable for these data because there is doubt regarding the assumption of Normality that underlies the t test.
 \end{enumerate}
 \end{document}
 

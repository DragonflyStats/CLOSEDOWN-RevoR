\documentclass[a4paper,12pt]{article}

%%%%%%%%%%%%%%%%%%%%%%%%%%%%%%%%%%%%%%%%%%%%%%%%%%%%%%%%%%%%%%%%%%%%%%%%%%%%%%%%%%%%%%%%%%%%%%%%%%%%%%%%%%%%%%%%%%%%%%%%%%%%%%%%%%%%%%%%%%%%%%%%%%%%%%%%%%%%%%%%%%%%%%%%%%%%%%%%%%%%%%%%%%%%%%%%%%%%%%%%%%%%%%%%%%%%%%%%%%%%%%%%%%%%%%%%%%%%%%%%%%%%%%%%%%%%

\usepackage{eurosym}
\usepackage{vmargin}
\usepackage{amsmath}
\usepackage{graphics}
\usepackage{epsfig}
\usepackage{enumerate}
\usepackage{multicol}
\usepackage{subfigure}
\usepackage{fancyhdr}
\usepackage{listings}
\usepackage{framed}
\usepackage{graphicx}
\usepackage{amsmath}
\usepackage{chngpage}

%\usepackage{bigints}
\usepackage{vmargin}

% left top textwidth textheight headheight

% headsep footheight footskip

\setmargins{2.0cm}{2.5cm}{16 cm}{22cm}{0.5cm}{0cm}{1cm}{1cm}

\renewcommand{\baselinestretch}{1.3}

\setcounter{MaxMatrixCols}{10}

\begin{document}
%%%%%%%%%%%%%%%%%%%%%%%%%%%%%%%%%%%%%%%%%%%%%%%%%%%%%%%%%%%%%%%%%%%%%%%%%%%%
%-------------------------------------------------%

\begin{table}[ht!]
     
\centering
     
\begin{tabular}{|p{15cm}|}
     
\hline        

\noindent
2. In a hi-tech company, the members of three research groups (A, B and C) are individually invited to enter a prize competition for the best solution to a technical problem. Group A has 2 staff, B has 3 and C has 5. It is assumed that all staff decide independently whether or not to enter. Members of groups A, B and C enter with respective probabilities 1/2, 1/4 and 1/5.
(i) For each group separately, find the probability of (a) no entries, (b) one entry.
\\ \hline
      
\end{tabular}
    
\end{table}


%-------------------------------------------------%

%%%%%%%%%%%%%%%%%%%%%%%%%%%%%%%%%%%%%%%%%%%%%%%%%%%%%%%%%%%%%%%%%%%%%%%%%%%%
%- Higher Certificate, Paper I, 2006. Question 2
%- HC1 2006 Question 2


\begin{enumerate}
\item 
\begin{description}

\item[A:]  
\begin{itemize}
\item[$\bullet$] No Entries
\[ P( 0 \mbox{ entries}) = \left(\frac{1}{2}\right)^2  = \left(\frac{1}{4}\right) = 0.25\]
\item[$\bullet$] One Entry
\[ P( 1 \mbox{ entry}) = \left( 2 \times \frac{1}{2} \times \frac{1}{2}\right)  = \left(\frac{1}{2}\right) = 0.5\]
\end{itemize}

\item[B:]
\begin{itemize}
\item[$\bullet$] No Entries
\[ P( 0 \mbox{ entries}) = \left(\frac{3}{4}\right)^3  = \left(\frac{27}{64}\right) = 0.4219\]
\item[$\bullet$] One Entry
\[ P( 1 \mbox{ entry}) =  3 \times \left( \frac{1}{4}\right) \times \left( \frac{3}{4}\right)^2  = \left(\frac{27}{64}\right) = 0.4219\]
\end{itemize}

\item[C:]
\begin{itemize}
\item[$\bullet$] No Entries
\[ P( 0 \mbox{ entries}) = \left(\frac{4}{5}\right)^5  = \left(\frac{1024}{3125}\right) = 0.327\]
\item[$\bullet$] One Entry
\[ P( 1 \mbox{ entry}) =  5 \times \left( \frac{1}{5} \right) \times \left( \frac{4}{5}\right)^4\right)  = \left(\frac{256}{625}\right) = 0.4096\]
\end{itemize}
\end{description}

%%%%%%%%%%%%%%%%%%%%%%%%%%%%%%%%%%
%-------------------------------------------------%

\begin{table}[ht!]
     
\centering
     
\begin{tabular}{|p{15cm}|}
     
\hline        

\noindent

(ii) Given that there is just one entry in total, show that the probability that it comes from a member of group A is 8/17.
\\ \hline
      
\end{tabular}
    
\end{table}


%-------------------------------------------------%
\item 
\begin{framed}
\begin{itemize}
    \item $P(1\mbox{ from }A, 0\mbox{ from }B, 0\mbox{ from }C)$ denoted as  $P([1,0,0])$ 
    \item $P(0\mbox{ from }A, 1\mbox{ from }B, 0\mbox{ from }C)$ denoted as $P([0,1,0])$ 
\item $P(0\mbox{ from }A, 0\mbox{ from }B, 1\mbox{ from }C)$ denoted as $P([0,0,1])$
\end{itemize}
\end{framed}


\begin{eqnarray*}
P(1\mbox{ entry in total}) &=& P([1,0,0]) + P([0,1,0]) + P([0,0,1])\\
&=& \left[ 0.5 \times 0.4219 \times 0.327  \right] + 
\left[ 0.5 \times 0.4219 \times 0.327   \right] + 
\left[ 0.5 \times 0.4219 \times 0.4096   \right] 
\end{eqnarray*}

[If worked in decimals, this is 0.1469.]

\begin{eqnarray*}
P(1 \mbox{ from } A | \mbox{1 in total}) &=& \frac{P(\mbox{1 \mbox{ from A and 1 in total}})}{ P(\mbox{1 in total})}\\
&=& \frac{P(\mbox{1 \mbox{ from } A, 0 \mbox{ from B and C} }) }{P(1 \mbox{in total})} \\
&=& 271024126431254593125817××=.\\
\end{eqnarray*}
%%%%%%%%%%%%%%%%%%%%%%%%%%%%%%%%%%


%-------------------------------------------------%

\begin{table}[ht!]
     
\centering
     
\begin{tabular}{|p{15cm}|}
     
\hline        

\noindent

(iii) Explain (but without doing the calculations) the steps that are needed to calculate the probability that there are exactly two entries in total.

\\ \hline
      
\end{tabular}
    
\end{table}


%-------------------------------------------------%
\item  Denote the numbers of entries from A, B, C as (0, 0, 0) etc. Then we need P(2, 0, 0) + P(0, 2, 0) + P(0, 0, 2) + P(1, 1, 0) + P(1, 0, 1) + P(0, 1, 1). Since entries from each group are independent, we have, as an example, P(1, 1, 0) = P(1 from A).P(1 from B).P(0 from C).
\end{enumerate}

\end{document}

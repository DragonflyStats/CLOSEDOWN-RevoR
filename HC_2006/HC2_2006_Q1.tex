\documentclass[a4paper,12pt]{article}

%%%%%%%%%%%%%%%%%%%%%%%%%%%%%%%%%%%%%%%%%%%%%%%%%%%%%%%%%%%%%%%%%%%%%%%%%%%%%%%%%%%%%%%%%%%%%%%%%%%%%%%%%%%%%%%%%%%%%%%%%%%%%%%%%%%%%%%%%%%%%%%%%%%%%%%%%%%%%%%%%%%%%%%%%%%%%%%%%%%%%%%%%%%%%%%%%%%%%%%%%%%%%%%%%%%%%%%%%%%%%%%%%%%%%%%%%%%%%%%%%%%%%%%%%%%%

\usepackage{eurosym}
\usepackage{vmargin}
\usepackage{amsmath}
\usepackage{graphics}
\usepackage{epsfig}
\usepackage{enumerate}
\usepackage{multicol}
\usepackage{subfigure}
\usepackage{fancyhdr}
\usepackage{listings}
\usepackage{framed}
\usepackage{graphicx}
\usepackage{amsmath}
\usepackage{chngpage}

%\usepackage{bigints}
\usepackage{vmargin}

% left top textwidth textheight headheight

% headsep footheight footskip

\setmargins{2.0cm}{2.5cm}{16 cm}{22cm}{0.5cm}{0cm}{1cm}{1cm}

\renewcommand{\baselinestretch}{1.3}

\setcounter{MaxMatrixCols}{10}

\begin{document}Higher Certificate, Paper II, 2006.  Question 1 
 
 \begin{enumerate}
\item The variance of this sample is 
 
2 2 1 15568 15106.93 4054484 256.05 59 60 59
s
⎛⎞ = − = = ⎜⎟ ⎝⎠ . 
 
The null hypothesis to be tested is "σ 2 = 256".  It seems obvious that this null hypothesis is not likely to be rejected (even if the sample had been of considerably smaller size), but continuing with a formal test we use test statistic 
 
2
2 ( 1) 59 256.05 59.01 256 ns σ −× ==
 
 
which is referred to .  The upper 5% point is about 78.  Clearly we cannot reject the null hypothesis as the data give no evidence for doing so. 2 59χ
 
 
\item We have  15568 259.46 60 x ==  and we wish the test the null hypothesis μ = 266.  Taking the value of σ as 16, which seems highly plausible from part (i), we use test statistic 
 266 3.1616 60 x − =− 
 
and refer to N(0, 1). 
 [Alternatively, we could continue to use the sample variance s2 (= 256.05) and refer 266 60 x s − to t59;  this makes hardly any difference in practice in this case.] 
 
This is well beyond the double-tailed 1\% point of N(0, 1);  there is strong evidence against this null hypothesis.  It is reasonable to conclude that this population has a mean different from the "normal" one;  it appears to be less. 
 
Using N(0, 1), we have , giving a p-value of 0.0016. 
\end{enumerate}
\end{document}

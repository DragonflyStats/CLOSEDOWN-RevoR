\documentclass[a4paper,12pt]{article}

%%%%%%%%%%%%%%%%%%%%%%%%%%%%%%%%%%%%%%%%%%%%%%%%%%%%%%%%%%%%%%%%%%%%%%%%%%%%%%%%%%%%%%%%%%%%%%%%%%%%%%%%%%%%%%%%%%%%%%%%%%%%%%%%%%%%%%%%%%%%%%%%%%%%%%%%%%%%%%%%%%%%%%%%%%%%%%%%%%%%%%%%%%%%%%%%%%%%%%%%%%%%%%%%%%%%%%%%%%%%%%%%%%%%%%%%%%%%%%%%%%%%%%%%%%%%

\usepackage{eurosym}
\usepackage{vmargin}
\usepackage{amsmath}
\usepackage{graphics}
\usepackage{epsfig}
\usepackage{enumerate}
\usepackage{multicol}
\usepackage{subfigure}
\usepackage{fancyhdr}
\usepackage{listings}
\usepackage{framed}
\usepackage{graphicx}
\usepackage{amsmath}
\usepackage{chngpage}

%\usepackage{bigints}
\usepackage{vmargin}

% left top textwidth textheight headheight

% headsep footheight footskip

\setmargins{2.0cm}{2.5cm}{16 cm}{22cm}{0.5cm}{0cm}{1cm}{1cm}

\renewcommand{\baselinestretch}{1.3}

\setcounter{MaxMatrixCols}{10}

\begin{document}Higher Certificate, Paper II, 2006.  Question 6 

\begin{framed}


6. A coin is supposed to weigh 5 gm.  A random sample of 100 such coins was taken and weighed, yielding the results tabulated below. 
 
 
Weight (gm) Frequency < 4.895   4 ≥ 4.895 but < 4.925   4 ≥ 4.925 but < 4.955 11 ≥ 4.955 but < 4.985 13 ≥ 4.985 but < 5.015 30 ≥ 5.015 but < 5.045 18 ≥ 5.045 but < 5.075 11 ≥ 5.075 but < 5.105   7 ≥ 5.105   2 
 
 
(i) Construct a histogram of these data, stating any assumptions you make. 
(6) 

 

\end{framed}

\begin{enumerate} 
\item In a histogram for open-ended data, it is common to assume that the intervals at the beginning and end have the same width as others (unless there is good reason to do otherwise), and this has been done here. 
 
 
 
 
  
 
 
 
 
 
 
 
 
 
 
 
 
 
 
 
 
 
 
 
 
 
 30 
10 
Frequency 
20 
  0 
5.135 5.1055.075 5.045 5.015 4.985 4.955 4.925 4.895 4.865 
 
Weight (gm)  

%%%%%%%%%%%%%%%%%%%%%%%%%%%%%%%%%%%%%%% 

(ii) Perform a χ2 goodness-of-fit test of the null hypothesis that the data are from a Normal distribution with mean 5 gm.  [Note:  to save you calculation, the sample standard deviation of these data is 0.055 gm.] (14) 
  
 
\item  Using the sample variance s2 = 0.0552, we work with W ~ N(5, 0.0552).  This gives 5 0.055 WZ − = ~ N(0, 1). 

\begin{itemize}
\item The table on the next page shows the value of w at the end-point of each interval, the corresponding value of z, the probability P(Z < z), and hence the probability of being in the corresponding interval.  
\item The observed and corresponding expected frequencies (o and e) are shown in the last two columns of the table. 
\item 
Weight w w – 5 z = 5 0.055 w− P(Z < z) Prob. in interval
\end{itemize}
 o e 
4.895 –0.105 –1.9091 0.0281 0.0281 (from –∞)   4   2.81 4.925 –0.075 –1.3636 0.0863 0.0582    4   5.82 4.955 –0.045 –0.8182 0.2066 0.1203  11 12.03 4.985 –0.015 –0.2727 0.3925 0.1859  13 18.59 5.015   0.015   0.2727 0.6075 0.2150  30 21.50 5.045   0.045   0.8182 0.7934 0.1859  18 18.59 5.075   0.075   1.3636 0.9137 0.1203  11 12.03 5.105   0.105   1.9091 0.9719 0.0582    7   5.82     0.0281 (to +∞)   2   2.81 
 
 
For the test, the expected frequencies need to be not too small (≥5 is often used as a criterion).  On this basis, we combine the first two cells and the last two cells to get the following table. 
 
 
o 8 11 13 30 18 11 9 e 8.63 12.03 18.59 21.50 18.59 12.03 8.63 
 
 
The test statistic is 
 
     
() 2 2 2 2 2 (8 8.63) (11 12.03) (9 8.63) ... 5.30 8.63 12.03 8.63 oe
X
e − − − − = = + + + = ∑ , 
 which is referred to  (note 5 degrees of freedom because the table has 7 cells and there is one [NB only one, i.e. σ 2] estimated parameter).  This is not significant (the 5\% point is 11.07).  The null hypothesis that a Normal distribution with mean 5 (years) underlies the data cannot be rejected on this evidence. 2 5χ
 \end{enumerate}
 \end{document}
 

\documentclass[a4paper,12pt]{article}

%%%%%%%%%%%%%%%%%%%%%%%%%%%%%%%%%%%%%%%%%%%%%%%%%%%%%%%%%%%%%%%%%%%%%%%%%%%%%%%%%%%%%%%%%%%%%%%%%%%%%%%%%%%%%%%%%%%%%%%%%%%%%%%%%%%%%%%%%%%%%%%%%%%%%%%%%%%%%%%%%%%%%%%%%%%%%%%%%%%%%%%%%%%%%%%%%%%%%%%%%%%%%%%%%%%%%%%%%%%%%%%%%%%%%%%%%%%%%%%%%%%%%%%%%%%%

\usepackage{eurosym}
\usepackage{vmargin}
\usepackage{amsmath}
\usepackage{graphics}
\usepackage{epsfig}
\usepackage{enumerate}
\usepackage{multicol}
\usepackage{subfigure}
\usepackage{fancyhdr}
\usepackage{listings}
\usepackage{framed}
\usepackage{graphicx}
\usepackage{amsmath}
\usepackage{chngpage}

%\usepackage{bigints}
\usepackage{vmargin}

% left top textwidth textheight headheight

% headsep footheight footskip

\setmargins{2.0cm}{2.5cm}{16 cm}{22cm}{0.5cm}{0cm}{1cm}{1cm}

\renewcommand{\baselinestretch}{1.3}

\setcounter{MaxMatrixCols}{10}

\begin{document}
%%%%%%%%%%%%%%%%%%%%%%%%%%%%%%%%%%%%%%%%%%%%%%%%%%%%%%%%%%%%%%%%%%%%%%%%%%%%
%-------------------------------------------------%

\begin{table}[ht!]
     
\centering
     
\begin{tabular}{|p{15cm}|}
     
\hline        

\noindent
8. (i) Write down the model for, and standard assumptions of, simple linear regression analysis. 
State a condition under which the method of least squares is equivalent to the method of maximum likelihood for estimating the regression coefficients.
\\ \hline
      
\end{tabular}
    
\end{table}


%-------------------------------------------------%
%-------------------------------------------------%

\begin{table}[ht!]
     
\centering
     
\begin{tabular}{|p{15cm}|}
     
\hline        

\noindent

(ii) (a) Suppose now that the intercept parameter in the regression model is known to be zero, so that the model becomes
iiyxi e β=+,
where the usual assumptions apply to $e_i$. Show that the least squares estimator of $\beta$ is 2iiixyxΣΣ.
\\ \hline
      
\end{tabular}
    
\end{table}


%-------------------------------------------------%



%%%%%%%%%%%%%%%%%%%%%%%%%%%%%%%%%%%%%%%%%%%%%%%%%%%%%%%%%%%%%%%%%%%%%%%%%%%%
Higher Certificate, Paper I, 2006. Question 8

\begin{enumerate}[(a)]
\item $Yi = a + bx_i + e_i$, i = 1, 2, …, n.
The {ei} are uncorrelated random variables with mean 0 and constant variance σ 2.
The method of least squares is equivalent to the method of maximum likelihood for estimating the regression coefficients (a and b) if the {ei} are Normally distributed.
[If the analysis is to proceed to inference for the regression coefficients, Normality of the {ei} is required.]

\newpage

%-------------------------------------------------%

\begin{table}[ht!]
     
\centering
     
\begin{tabular}{|p{15cm}|}
     
\hline        

\noindent

(b) Over a period of one month, a survey was made on each of ten main roads in a large city. 
Each road was observed for a one-hour period randomly chosen during the working day. 
For each road, the mean traffic flow, $x_i$ (in vehicles per minute), and the number of speed limit violations, $y_i$, $i = 1, 2, \ldots, 10$, were recorded. 
\begin{itemize}
    \item Plot the data shown below on a suitable graph and comment on the suitability of the above model. 
\item Fit the model to the data and hence estimate the expected number of violations on a road 
with an average traffic flow of 20 vehicles per minute.
\item Without any further calculation, comment 
on the suggestion that an intercept should be included in the model.
\end{itemize}

\begin{center}
\begin{tabular}{c|c}
Flow, x	&	Violations, y	\\ \hline
5	&	2	\\ \hline
5	&	1	\\ \hline
5	&	1	\\ \hline
10	&	4	\\ \hline
10	&	2	\\ \hline
15	&	5	\\ \hline
25	&	8	\\ \hline
25	&	2	\\ \hline
30	&	5	\\ \hline
50	&	10	\\ \hline
\end{tabular}
\end{center}
\\ \hline
      
\end{tabular}
    
\end{table}


%-------------------------------------------------%
\item (a) For Yii , we minimise S = 2ieΣ = ()2iiyxβ−Σ.
We have (2iiidS xyxdββ=−−Σ which on setting equal to zero gives 2iiixy β, so the least squares estimate is 2ˆiiixyxβ=ΣΣ.
[Consideration of 22dSdβ confirms that this is a minimum: 22220idSxdβ=>Σ.]
(b) See scatter plot at foot of page. It shows an increasing trend, roughly linear; but there seems to be some increase in variability as x increases. There are not enough data points to be sure.
The usual summary statistics (not all required for the zero intercept model) are
\begin{itemize}
    \item n = 10, 
    \item $\sum (x_i) = 180$, 
    \item $\sum (y_i) = 40$, 
    \item $\sum (x_i)^2 = 5150$, 
    \item $\sum (y_i)^2 = 244$,
    \item $\sum (x_iy_i) = 1055$.
\end{itemize}

ˆβ∴ =
1055/5150 =
0.205. 

\begin{itemize}
\item So the fitted line is y
= 0.205x.
\item Hence the estimated expected number of violations for $x = 20$ is $0.205 \times 20 = 4.1$.
\item Logically, zero traffic flow should imply zero speed violations, so that y should be 0 when x is 0, i.e. the zero intercept model seems reasonable.
\end{itemize} The scatter plot does not contradict this.
504030201001050Traffic Flow per minuteViolations

\end{enumerate}

\end{document}

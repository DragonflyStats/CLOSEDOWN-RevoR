\documentclass[a4paper,12pt]{article}

%%%%%%%%%%%%%%%%%%%%%%%%%%%%%%%%%%%%%%%%%%%%%%%%%%%%%%%%%%%%%%%%%%%%%%%%%%%%%%%%%%%%%%%%%%%%%%%%%%%%%%%%%%%%%%%%%%%%%%%%%%%%%%%%%%%%%%%%%%%%%%%%%%%%%%%%%%%%%%%%%%%%%%%%%%%%%%%%%%%%%%%%%%%%%%%%%%%%%%%%%%%%%%%%%%%%%%%%%%%%%%%%%%%%%%%%%%%%%%%%%%%%%%%%%%%%

\usepackage{eurosym}
\usepackage{vmargin}
\usepackage{amsmath}
\usepackage{graphics}
\usepackage{epsfig}
\usepackage{enumerate}
\usepackage{multicol}
\usepackage{subfigure}
\usepackage{fancyhdr}
\usepackage{listings}
\usepackage{framed}
\usepackage{graphicx}
\usepackage{amsmath}
\usepackage{chngpage}

%\usepackage{bigints}
\usepackage{vmargin}

% left top textwidth textheight headheight

% headsep footheight footskip

\setmargins{2.0cm}{2.5cm}{16 cm}{22cm}{0.5cm}{0cm}{1cm}{1cm}

\renewcommand{\baselinestretch}{1.3}

\setcounter{MaxMatrixCols}{10}

\begin{document}

Higher Certificate, Paper I, 2006. Question 8
(i) Yi = a + bxi + ei, i = 1, 2, …, n.
The {ei} are uncorrelated random variables with mean 0 and constant variance σ 2.
The method of least squares is equivalent to the method of maximum likelihood for estimating the regression coefficients (a and b) if the {ei} are Normally distributed.
[If the analysis is to proceed to inference for the regression coefficients, Normality of the {ei} is required.]
(ii)(a) For Yii , we minimise S = 2ieΣ = ()2iiyxβ−Σ.
We have (2iiidS xyxdββ=−−Σ which on setting equal to zero gives 2iiixy β, so the least squares estimate is 2ˆiiixyxβ=ΣΣ.
[Consideration of 22dSdβ confirms that this is a minimum: 22220idSxdβ=>Σ.]
(b) See scatter plot at foot of page. It shows an increasing trend, roughly linear; but there seems to be some increase in variability as x increases. There are not enough data points to be sure.
The usual summary statistics (not all required for the zero intercept model) are
n = 10, Σxi = 180, Σyi = 40, Σxi2 = 5150, Σyi2 = 244, Σxiyi = 1055.
ˆβ∴ =
1055/5150 =
0.205. 

\begin{itemize}
\item So the fitted line is y
= 0.205x.
\item Hence the estimated expected number of violations for x = 20 is 0.205 × 20 = 4.1.
\item Logically, zero traffic flow should imply zero speed violations, so that y should be 0 when x is 0, i.e. the zero intercept model seems reasonable.
\end{itemize} The scatter plot does not contradict this.
504030201001050Traffic Flow per minuteViolations
\end{document}

\documentclass{article}
\usepackage[utf8]{inputenc}

\title{RSS_Jan_2019_HC_2006}
\author{kobriendublin }
\date{December 2018}

\begin{document}

Higher Certificate, Paper I, 2006. Question 3

\section{Introduction}
\begin{enumerate}
\item We have , so 

% 2006 Question 3

\begin{eqnarray*}
\mbox{Area} 
&=& k \left[ \frac{1}{3}x^3 - \frac{1}{2}x^4 + \frac{1}{5}x^5\]^{1}_{0}\\
&=& k \left[  \left(\frac{1}{3} - 0\right)  -  \left(\frac{1}{2} - 0\right) +  \left(\frac{1}{5} - 0\right) \right]\\
&=& k \left[  \frac{1}{3} - \frac{1}{2} + \frac{1}{5} \right]\\
&=& k \left[  \left(\frac{10}{30} - \frac{15}{30} + \frac{6}{30}\right]\\
&=& k \times \frac{1}{30} \\
\end{eqnarray*}


so k = 30.
f(x) = 0 at x = 0 and at x = 1. f(x) is symmetrical about x = ½. The sketch is as follows. 1.00.90.80.70.60.50.40.30.20.10.0210xf(x)
\item 1()2EX= by symmetry [or by direct integration: 10()xfxdx∫].


%----------%
\begin{eqnarray*}
E(X^2) 
&=& 30 \int^{1}{0} x^2  \times \left[(x^2)(1-x)^2\right] dx\\
&=& 30 \int^{1}{0} (x^4)(1-x)^2 dx\\
&=& 30 \int^{1}{0} (x^4)(1-2x + x^2) dx\\
&=& 30 \int^{1}{0} (x^4 -2x^5 + x^6) dx\\

&=& k \left[  \left(\frac{1}{3} - \frac{1}{2} + \frac{1}{5}\right]


&=& 30 \left[  \left(\frac{1}{5} - 0\right)  -  \left(\frac{1}{3} - 0\right) +  \left(\frac{1}{7} - 0\right) \right]
&=& 30 \left[  \left(\frac{21}{105} - \frac{35}{105} + \frac{15}{105}\right]
&=& 30 \times frac{1}{105} \\
&=&  frac{2}{7} \\


()()1122445600()301302EXxxdxxxx=−=−+∫∫
1566011111113030305375371057xxx⎡⎤⎛⎞=−+=−+=×⎜⎟⎢⎥⎣⎦⎝⎠ .
{}222211Var()()()722XEXEX⎛⎞ ∴=−=−=⎜⎟⎝⎠.
()1/31/323434500113023033PXxxxdxxxx⎛⎞⎡⎤≤=−+=−+⎜⎟ ⎢⎥⎝⎠⎣⎦∫
44511111301130130..132353812158130⎛⎞⎛⎞=−+=−+=×⎜⎟⎜⎟⎝⎠⎝⎠ = 1781 (= 0.2099).



\[ E(X) = \frac{1}{2}\]


\item The required probability is 55176418181⎛⎞⎛−=⎜⎟⎜⎝⎠⎝ = 0.3079.
\item The variance of X for a sample of size 5 is Var()1/28155140X== = 0.00714.
\end{enumerate}
\end{document}
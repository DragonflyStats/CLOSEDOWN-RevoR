\documentclass[a4paper,30pt]{article}

%%%%%%%%%%%%%%%%%%%%%%%%%%%%%%%%%%%%%%%%%%%%%%%%%%%%%%%%%%%%%%%%%%%%%%%%%%%%%%%%%%%%%%%%%%%%%%%%%%%%%%%%%%%%%%%%%%%%%%%%%%%%%%%%%%%%%%%%%%%%%%%%%%%%%%%%%%%%%%%%%%%%%%%%%%%%%%%%%%%%%%%%%%%%%%%%%%%%%%%%%%%%%%%%%%%%%%%%%%%%%%%%%%%%%%%%%%%%%%%%%%%%%%%%%%%%

\usepackage{eurosym}
\usepackage{vmargin}
\usepackage{amsmath}
\usepackage{graphics}
\usepackage{epsfig}
\usepackage{enumerate}
\usepackage{multicol}
\usepackage{subfigure}
\usepackage{fancyhdr}
\usepackage{listings}
\usepackage{framed}
\usepackage{graphicx}
\usepackage{amsmath}
\usepackage{chngpage}

%\usepackage{bigints}
\usepackage{vmargin}

% left top textwidth textheight headheight

% headsep footheight footskip

\setmargins{2.0cm}{2.5cm}{16 cm}{22cm}{0.5cm}{0cm}{1cm}{1cm}

\renewcommand{\baselinestretch}{1.3}

\setcounter{MaxMatrixCols}{10}

\begin{document}

Higher Certificate, Paper I, 2006. Question 3
\begin{table}[ht!]
     \centering
     \begin{tabular}{|p{15cm}|}
     \hline        
\noindent  The continuous random variable X has probability density function given by
\[ f(x) = \begin{cases}  kx^2(1\;-\;x)^2  & 0 \leq x \leq 1\\
0 &  elsewhere
\end{cases}
\]
\\ \hline
      \end{tabular}
    \end{table}

\begin{table}[ht!]
     \centering
     \begin{tabular}{|p{15cm}|}
     \hline        
\noindent (i) Find k and sketch the graph of f (x).


\\ \hline
      \end{tabular}
    \end{table}
    


\begin{enumerate}
\item We have , so 

% 2006 Question 3

\begin{eqnarray*}
\mbox{Area} 
&=& k \left[ \frac{1}{3}x^3 - \frac{1}{2}x^4 + \frac{1}{5}x^5 \right]^{1}_{0}\\
&=& k \left[  \left( \frac{1}{3} - 0 \right)  -  \left(\frac{1}{2} - 0 \right) +  \left( \frac{1}{5} - 0 \right) \right]\\
&=& k \left[  \frac{1}{3} - \frac{1}{2} + \frac{1}{5} \right]\\
&=& k \left[  \left(\frac{10}{30} - \frac{15}{30} + \frac{6}{30}\right)\right]\\
&=& k \times \frac{1}{30} \\
\end{eqnarray*}


so k = 30.
f(x) = 0 at x = 0 and at x = 1. f(x) is symmetrical about x = ½. The sketch is as follows. 1.00.90.80.70.60.50.40.30.20.10.0210xf(x)

    \begin{table}[ht!]
     \centering
     \begin{tabular}{|p{15cm}|}
     \hline        
\noindent (ii) Find $E(X)$ and $Var(X)$, and show that \[  P(X \leq 1/3 ) = \frac{17}{81} \]

\\ \hline
      \end{tabular}
    \end{table}

\item ${\displaystyle E(X) = \frac{1}{2} }$ by symmetry [or by direct integration: 10()xfxdx∫].
{
\large
\begin{eqnarray*}
E(X) &=& \int^{1}_{0} x \;f(x) dx \\
&=& \int^{1}_{0} x[30(x^2)(1\;-\;x)^2] dx \\
&=& 30 \int^{1}_{0} x  \times \left[(x^2)(1-x)^2\right] dx\\
&=& 30 \int^{1}_{0} (x^3)(1-x)^2 dx\\
&=& 30 \int^{1}_{0} (x^3)(1-2x + x^2) dx\\
&=& 30 \int^{1}_{0} (x^3 -2x^4 + x^5) dx \qquad { \mbox{ Standard Definite Integral} \\
&=& 30 \left[ \frac{x^4}{4}  \;-\;  \frac{2x^5}{5} \; + \; \frac{x^6}{6}  \right]^{1}_{0}\\
&=& 30 \left[ \left( \frac{1}{4}- 0 \right) \;-\; \left( \frac{2}{5}- 0 \right) + \left( \frac{1}{6}- 0 \right) \right]\\
&=& 30 \left[  \frac{15}{60} \;-\;  \frac{24}{60} +  \frac{10}{60} \right]\\
&=& 30 \left[  \frac{1}{60}  \right]\\
&=&  \frac{1}{2}\\
\end{eqnarray*}
}
%----------%
\begin{eqnarray*}
E(X^2) 
&=& 30 \int^{1}_{0} x^2  \times \left[(x^2)(1-x)^2 \right] dx\\
&=& 30 \int^{1}_{0} (x^4)(1-x)^2 dx\\
&=& 30 \int^{1}_{0} (x^4)(1-2x + x^2) dx\\
&=& 30 \int^{1}_{0} (x^4 -2x^5 + x^6) dx\\
&=& 30 \left[  \frac{x^5}{5} - \frac{2x^6}{6} + \frac{x^7}{7}\right]^{1}_{0}\\
&=& 30 \left[  \left(\frac{1}{5} - 0\right)  -  \left(\frac{1}{3} - 0\right) +  \left(\frac{1}{7} - 0\right) \right]\\
&=& 30 \left[  \frac{21}{105} - \frac{35}{105} + \frac{15}{105}\right]\\
&=& 30 \times \frac{1}{105} \\
&=&  \frac{2}{7} \\
\end{eqnarray*}

\begin{eqnarray*}
Var(X) &=& E(X^2) - [E(X)]^2 \\
 &=& \frac{2}{7} - \left(\frac{1}{2}\right)^2 \\
 &=& \frac{8}{28} - \frac{7}{28} \\
 &=& \frac{1}{28} \\
\end{eqnarray*}

{
\large
\begin{eqnarray*}
P(X \leq 1/3 ) &=& \int^{1/3}_{0} [30(x^2)(1\;-\;x)^2] dx \\
&=& 30 \int^{1/3}_{0}  \left[(x^2)(1-x)^2\right] dx\\
&=& 30 \int^{1/3}_{0} (x^2 -2x^3 + x^4) dx\\
&=& 30 \left[ \frac{x^3}{3}  \;-\;  \frac{2x^4}{4} \; + \; \frac{x^5}{5}  \right]^{1/3}_{0}\\
&=& 30 \left[ \frac{x^3}{3}  \;-\;  \frac{x^4}{2} \; + \; \frac{x^5}{5}  \right]^{1/3}_{0}\\
&=& 30 \left[ \left( \frac{(1/3)^3}{3}- 0 \right) \;-\; \left( \frac{(1/3)^4}{2}- 0 \right) +  \left( \frac{(1/3)^5}{5}- 0 \right) \right]\\
&=& 30 \left[ \left( \frac{1}{3^4} \right) \;-\; \left( \frac{15}{30} \times \frac{1}{3^4} \right) +  \left( \frac{6}{30} \times \frac{1}{3}\times \frac{1}{3^4} \right) \right]\\
&=& 30 \times \left( \frac{1}{3^4} \right) \times \left[ \frac{30}{30} \;-\;  \frac{15}{30}  + \frac{2}{30}  \right]\\
&=&  \frac{30}{81} \times \frac{17}{30} \\
&=&  \frac{17}{81}\\
\end{eqnarray*}
}

The cumulative distribution function is
\begin{eqnarray*}
F(X) &=& \int^{x}_{0} f(u) du \\
&=& \int^{x}_{0} 30(u^2\;-\;u^3)] du \\
&=& 30 \left[ \frac{u^3}{3} - \frac{u^4}{4}\right]^1_0\\
&=& 30 \left[ \left( \frac{x^3}{3}- 0 \right) - \left( \frac{x^4}{4}- 0 \right)  \right]\\
&=& 30 \int^{1}{0} (x^4 -2x^5 + x^6) dx\\
&=& 30 \left[  \frac{1}{3} - \frac{1}{2} + \frac{1}{5}\right]
\end{eqnarray*}
%%%%%%%%%%%%%%%%%%%%%%%%%%%%%%%%%%%%%%%%%%%%%%%%%%%%%%%%%%%%%%%%%%%%%
\newpage

\begin{table}[ht!]
     \centering
     \begin{tabular}{|p{15cm}|}
     \hline        
\noindent 
(iii) A random sample of size 5 is taken from this distribution. Find, correct to 4 decimal places, the probability that all 5 observations exceed 1/3.


\\ \hline
      \end{tabular}
    \end{table}
    




\item The required probability is 
\[ \left( 1 - \frac{17}{81} \right)^5  = \left( \frac{64}{81} \right)^5 = 0.3079.\]

%%%%%%%%%%%%%%%%%%%%%%%%%%%%%%%%%%%%%%%%%%%%%%%%%%%%%%%%%%%%%
\newpage

    \begin{table}[ht!]
     \centering
     \begin{tabular}{|p{15cm}|}
     \hline        
\noindent (iv) Find, correct to 4 decimal places, the variance of the mean of a random sample of size 5.
\\ \hline
      \end{tabular}
    \end{table}

\item The variance of $X$ for a sample of size 5 is 


\[  \frac{\operatorname{Var}(X)}{5}  = \frac{1/28}{5}  = \frac{1}{140} =  0.00714.\]
\end{enumerate}
\end{document}

%%%%%%%%%%%%%%%%%%%%%%%

f(x) <- (30 * (x^2) * (1-x)^2)

eq = function(x){(30 * (x^2) * (1-x)^2)}
plot(eq(0:1000/1000), type='l')

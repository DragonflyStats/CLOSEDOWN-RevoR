7. (a) The following values are a random sample from the uniform distribution on
(0, 1).
0.0885 0.4096 0.7370 0.9384
Use these values to generate four random variates from each of the following
distributions, carefully explaining the method you use in each case.
(i) Poisson:
2.5 (2.5) ( ) , 0, 1, 2, ... . !
x e PX x x
x

  (5)
(ii) Shifted exponential: 10( 3) ( ) 10 , 3. x fx e x     (5)
(b) Beetle is a party game played using a standard die. The player attempts to
complete a rough sketch of a beetle, consisting of a body, a head, a tail, four
legs, two antennae and two eyes [This is a special beetle for the game, with
only four legs]. On each turn, the player rolls the die once and may draw one
part of the beetle depending on the score on the die as indicated in the
following table.
6 = body 3 = leg
5 = head 2 = antenna
4 = tail 1 = eye
If the player has already drawn all of a particular body part (for example, four
legs), no body part is added on subsequent turns when the corresponding score
(for example, 3) is obtained on the die. In addition, the body itself must be
drawn before any other part may be drawn. The head, tail or legs may then be
attached to the body, but antennae and eyes may only be added after the head.
This means that a turn results in at most one body part being added to the
sketch of the beetle, but often no body part is added.
Use the following sequence of random digits, which should be read from left to
right across each row, to carry out one simulation of a player's attempt to draw
a beetle. Explain your method carefully. State clearly the total number of
random digits and the total number of simulated rolls of a die required to
complete the beetle.
5 2 1 0 9 0 0 9 6 8 6 8 0 8 9 8 6 3 6 4
2 8 8 4 5 4 1 9 3 9 0 0 6 2 6 6 2 0 6 8
7 0 5 4 9 9 6 7 5 6 5 3 9 4 4 3 5 1 0 9
(6)
Describe how you would extend this simulation to find a plausible range of
values for the number of rolls of the die that a player would require in order to
complete a beetle.
(4) 

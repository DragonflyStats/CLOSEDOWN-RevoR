\documentclass[a4paper,12pt]{article}
%%%%%%%%%%%%%%%%%%%%%%%%%%%%%%%%%%%%%%%%%%%%%%%%%%%%%%%%%%%%%%%%%%%%%%%%%%%%%%%%%%%%%%%%%%%%%%%%%%%%%%%%%%%%%%%%%%%%%%%%%%%%%%%%%%%%%%%%%%%%%%%%%%%%%%%%%%%%%%%%%%%%%%%%%%%%%%%%%%%%%%%%%%%%%%%%%%%%%%%%%%%%%%%%%%%%%%%%%%%%%%%%%%%%%%%%%%%%%%%%%%%%%%%%%%%%
\usepackage{eurosym}
\usepackage{vmargin}
\usepackage{amsmath}
\usepackage{graphics}
\usepackage{epsfig}
\usepackage{enumerate}
\usepackage{multicol}
\usepackage{subfigure}
\usepackage{fancyhdr}
\usepackage{listings}
\usepackage{framed}
\usepackage{graphicx}
\usepackage{amsmath}
\usepackage{chngpage}
%\usepackage{bigints}

\usepackage{vmargin}
% left top textwidth textheight headheight
% headsep footheight footskip
\setmargins{2.0cm}{2.5cm}{16 cm}{22cm}{0.5cm}{0cm}{1cm}{1cm}
\renewcommand{\baselinestretch}{1.3}

\setcounter{MaxMatrixCols}{10}
\begin{document}

\begin{table}[ht!]
 \centering
 \begin{tabular}{|p{15cm}|}
 \hline  
7. A comparison of the reliability of engine bearings made from different alloys was performed by testing ten bearings of each type.  The times to failure in units of millions of cycles are given below. 
 
Alloy A Alloy B   5.30   3.19   5.53   4.26   5.60   4.47   6.30   4.53   6.92   4.67 12.51   4.69 12.95   6.79 16.04   9.37 18.21 12.75 18.24 12.78 
 
A computer package produced the five-number summaries in the form (minimum, lower quartile, median, upper quartile, maximum) giving 
 
\[Alloy A   ( 5.30, 5.58, 9.71, 16.58, 18.24 ) \]
 
\[Alloy B   ( 3.19, 4.42, 4.68, 10.22, 12.78 ) \]
 
\begin{enumerate}
    \item (i) Draw dot plots, side by side, of the results for the two alloys and mark the mean failure time for each alloy.  %(6) 
\item 
(ii) Use a non-parametric test to assess whether the distribution of times to failure differs for the two types of alloy.  Explain carefully why your test might be more appropriate than a two sample t-test.  %(10) 
\item 
(iii) State clearly the conclusion that can be drawn from the analysis. 
\end{enumerate}

\\ \hline
  \end{tabular}
\end{table}
\begin{enumerate}
\item 
Because of the very irregular distribution and total table lake of concentration
about the mean, the assumption of normality can not be made and so a t test is not
valid. The summary measure of location, if one were needed, should be the median
(9.71 for A and 4.68 for B ). 
\item The Mann Whitney U test, will examine whether the two
distribution of times differ; so will the Wilcoxon rank same test.
Jocut ranking:

%HC3 - Question 7

\begin{tabular}{|c|c|c|c|c|c|c|c|c|c|}
\hline
(1)&(2)&(3)&(4)&(5)&(6)&(7)&(8)&(9)&(10)\\ \hline
3.19&4.26&4.47&4.53&4.67&4.69&5.30&5.53&5.60&6.30\\ \hline
B&B&B&B&B&B&A&A&A&A\\ \hline
(11)&(12)&(13)&(14)&(15)&(16)&(17)&(18)&(19)&(20)\\ \hline
6.79&6.92&9.37&12.51&12.75&12.78&12.95&16.04&18.21&18.24\\ \hline
B&A&B&A&B&B&A&A&A&A\\ \hline
\end{tabular}

Sums of ranks are: A,134 ;B,76 (check : 134 + 76 = 210 = 1
2 £ 20 £ 21)
22
\[UA = 134 ¡
1
2
£ 10 £ 11 = 79 and UB = 76 ¡
1
2
£ 10 £ 11 = 21 Umin = 21\]
From table this is significant at 5%.
%%%%%%%%%%%%%%%%%%%%%%%%%%%
ALTERNATIVELY, count the number of times member of A come before B’s in the
ranking :U=4+4+4+4+3+2=21.(B’s before A’s gives U=79). A normal approximation,
strictly only for NA; NBboth>10, is that \[U is N( 1
2 £ 10 £ 10; 1
12 £ 10 £ 10 £ 21)\],so
\[2p1¡50
175 = ¡ 29
13:23 is N(0,1)\]; this is 2.19 and so gives the same significance
we may conclude that there is evidence of difference between the time distributions.
(The data suggest B tends to fail earlier)
\end{enumerate}
\end{document}

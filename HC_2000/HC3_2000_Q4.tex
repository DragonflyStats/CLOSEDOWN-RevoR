\documentclass[a4paper,12pt]{article}
%%%%%%%%%%%%%%%%%%%%%%%%%%%%%%%%%%%%%%%%%%%%%%%%%%%%%%%%%%%%%%%%%%%%%%%%%%%%%%%%%%%%%%%%%%%%%%%%%%%%%%%%%%%%%%%%%%%%%%%%%%%%%%%%%%%%%%%%%%%%%%%%%%%%%%%%%%%%%%%%%%%%%%%%%%%%%%%%%%%%%%%%%%%%%%%%%%%%%%%%%%%%%%%%%%%%%%%%%%%%%%%%%%%%%%%%%%%%%%%%%%%%%%%%%%%%
\usepackage{eurosym}
\usepackage{vmargin}
\usepackage{amsmath}
\usepackage{graphics}
\usepackage{epsfig}
\usepackage{enumerate}
\usepackage{multicol}
\usepackage{subfigure}
\usepackage{fancyhdr}
\usepackage{listings}
\usepackage{framed}
\usepackage{graphicx}
\usepackage{amsmath}
\usepackage{chngpage}
%\usepackage{bigints}

\usepackage{vmargin}
% left top textwidth textheight headheight
% headsep footheight footskip
\setmargins{2.0cm}{2.5cm}{16 cm}{22cm}{0.5cm}{0cm}{1cm}{1cm}
\renewcommand{\baselinestretch}{1.3}

\setcounter{MaxMatrixCols}{10}
\begin{document}

\begin{table}[ht!]
 \centering
 \begin{tabular}{|p{15cm}|}
 \hline  
A botanist investigating the dynamics of the formation of colonies of various species of clover performed the following experiment.  Thirty-two plots were sown as follows: 8 plots for each of species A, B, C, D.  Since the intention of the experiment was to examine how the various species developed as time progressed, two plots for each species were randomly allocated to each of the times (i) 9 weeks, (ii) 12 weeks, (iii) 15 weeks, (iv) 18 weeks after sowing.  At these times the chosen plots were examined and the leaf area per unit ground area was determined for each plot.  The results are given below. 
 
  Weeks from sowing   9 12 15 18 A 0.8,  1.3 3.6,  3.1 3.7,  3.3 1.9,  1.2 B 0.3,  0.8 2.5,  2.9 4.7,  3.9 6.5,  5.2 C 0.5,  0.7 2.4,  2.5 2.8,  3.1 1.5,  1.3 Species D 0.2,  0.4 1.0,  1.3 1.6,  1.9 1.8,  1.9 
 
These data were analysed using a computer package and part of the output is shown below. 
 
Analysis of Variance for leaf area per unit ground area 
 
Source  DF        SS      MS 
 
Species    * 18.8012 6.2671 Time    3   *****   ****** Species×Time   * 21.0862  ****** Error    *   ******   ****** Total  31 70.7987 
 
 
(i) Complete the analysis of variance table and use it to assess whether the experiment provides evidence that the four species behave differently with time. (10) 
 
(ii) Draw a suitable diagram to illustrate any species-time interaction that there may be. (5) (iii) Summarise your conclusions in non-technical language that the botanist would understand. (5) \\ \hline
  \end{tabular}
\end{table}
\begin{enumerate}
4 (i) N=32.total G=70.6. week totals:(9),5.0;(12)19.3;(15)25.0;(18)21.3. Hence
SS Time =
1
8
(5:02 + 19:32 + 25:02 + 21:32) ¡ 70:62=32 = 28:7613:
Analysis of variance
Source DF SS MS
Species 3 18:8012 6:2671
Time 3 28:7613 9:5871 F(9; 16) = 17:44
Species £ Time 9 21:0862 2:3429
Residual 16 2:1500 0:1344
Total 31 70:7987
There is clear evidence of an interaction, i.e. species behave differently over time.
Main effects of species and time are not therefore relevant.
(ii)Means:
time (9) (12) (15) (18)
A 1:05 3:35 3:50 1:55
species B 0:55 2:70 4:30 5:85
C 0:60 2:45 2:95 1:40
D 0:30 1:15 1:75 1:85
(iii)A increase up to week (15), then decrease; C has a similar pattern at a lower
level of area, B goes on increasing; so does D,slowly.
19

\end{enumerate}
\end{document}

\documentclass[a4paper,12pt]{article}
%%%%%%%%%%%%%%%%%%%%%%%%%%%%%%%%%%%%%%%%%%%%%%%%%%%%%%%%%%%%%%%%%%%%%%%%%%%%%%%%%%%%%%%%%%%%%%%%%%%%%%%%%%%%%%%%%%%%%%%%%%%%%%%%%%%%%%%%%%%%%%%%%%%%%%%%%%%%%%%%%%%%%%%%%%%%%%%%%%%%%%%%%%%%%%%%%%%%%%%%%%%%%%%%%%%%%%%%%%%%%%%%%%%%%%%%%%%%%%%%%%%%%%%%%%%%
\usepackage{eurosym}
\usepackage{vmargin}
\usepackage{amsmath}
\usepackage{graphics}
\usepackage{epsfig}
\usepackage{enumerate}
\usepackage{multicol}
\usepackage{subfigure}
\usepackage{fancyhdr}
\usepackage{listings}
\usepackage{framed}
\usepackage{graphicx}
\usepackage{amsmath}
\usepackage{chngpage}
%\usepackage{bigints}

\usepackage{vmargin}
% left top textwidth textheight headheight
% headsep footheight footskip
\setmargins{2.0cm}{2.5cm}{16 cm}{22cm}{0.5cm}{0cm}{1cm}{1cm}
\renewcommand{\baselinestretch}{1.3}

\setcounter{MaxMatrixCols}{10}
\begin{document}
\begin{table}[ht!]
 \centering
 \begin{tabular}{|p{15cm}|}
 \hline  
6. The central surface brightness was measured for 68 galactic globular clusters of stars and the results were related to the degree of concentration of the cluster which was categorised as low, medium or high.  Descriptive statistics for these data are given below. 
 
 
Variable: central surface brightness 
 
Concentration of cluster 
N Mean Median St. Dev. Min. Max. Lower quartile 
Upper quartile Low 20 21.7 20.75 2.23 18.9 25.7 20.025 23.975 Medium 26 18.1 17.85 2.30 14.4 24.0 16.600 19.250 High 22 17.3 16.30 3.14 14.1 25.2 15.275 17.650 
 
 
(i) Draw box and whisker plots side by side on one sheet of graph paper for the three categories of globular cluster. (9) 
 
(ii) Explain carefully what the plots suggest about the distribution of central surface brightness and the way in which central surface brightness is related to concentration of the cluster. (6) 
 
(iii) Suppose you wanted to assess the evidence that central surface brightness is related to concentration of the cluster.  Give three reasons why a oneway analysis of variance of the original values might not be appropriate. (5) \\ \hline
  \end{tabular}
\end{table}
\begin{enumerate}\item  (i)
O outside W F and P
\item The whiskers run from minimum value to lower quartile and form maximum value to
upper quartile; the median is marked inside the box. A symmetrical distribution would
have the median at approximately the middle of the box, and the two whiskers of about
the same length.
21
All of these patterns show skewness, positive in direction. For low concentration, the
average magnitude is considerably more than for higher concentration, but there is
skewness; the range of this set of data is less than for the other. The median brightness
appears to reduce as concentration rises. For median concentration the distribution is
nearer to symmetry except for the upper whisker.
For high concentration there is again high skewness(which may include outliers at the
upper end if we had the original data.).
\item An analysis of variance assumes (approximate) normality, and the same variance
in each set of data; neither of these seem to hold here. Because of the skewness, the
mean will not be a good central measure either.
(A transformation such as logarithmic may improve matters, but if there were outliers
these would still affect the analysis ).\end{enumerate}
\end{document}

\documentclass[a4paper,12pt]{article}
%%%%%%%%%%%%%%%%%%%%%%%%%%%%%%%%%%%%%%%%%%%%%%%%%%%%%%%%%%%%%%%%%%%%%%%%%%%%%%%%%%%%%%%%%%%%%%%%%%%%%%%%%%%%%%%%%%%%%%%%%%%%%%%%%%%%%%%%%%%%%%%%%%%%%%%%%%%%%%%%%%%%%%%%%%%%%%%%%%%%%%%%%%%%%%%%%%%%%%%%%%%%%%%%%%%%%%%%%%%%%%%%%%%%%%%%%%%%%%%%%%%%%%%%%%%%
\usepackage{eurosym}
\usepackage{vmargin}
\usepackage{amsmath}
\usepackage{graphics}
\usepackage{epsfig}
\usepackage{enumerate}
\usepackage{multicol}
\usepackage{subfigure}
\usepackage{fancyhdr}
\usepackage{listings}
\usepackage{framed}
\usepackage{graphicx}
\usepackage{amsmath}
\usepackage{chngpage}
%\usepackage{bigints}

\usepackage{vmargin}
% left top textwidth textheight headheight
% headsep footheight footskip
\setmargins{2.0cm}{2.5cm}{16 cm}{22cm}{0.5cm}{0cm}{1cm}{1cm}
\renewcommand{\baselinestretch}{1.3}

\setcounter{MaxMatrixCols}{10}
\begin{document}
%%%%%%%%%%%%%%%%%%%%%%%%%%%%%%%%%%%%%%%%%%%%%%%%%%%%%%%%%%%%%%%%%%%%%%%%%%%%%%%%%%%%%%%%%
\begin{table}[ht!]
     

\centering
     

\begin{tabular}{|p{15cm}|}
     

\hline 
\large
 
7. An experiment was performed in which a Geiger counter was used to measure the level of background radiation.  In the experiment, radiation counts were made during 100 separate ten-second intervals and the results are summarised in the following table. 
 
Radiation Count Number of ten-second intervals 0 24 1 25 2 18 3 12 4 7 5 9 6 5 >6 0 
 
\begin{enumerate}[(i)] 
\item Investigate the hypothesis that the distribution of radiation counts is Poisson using a chi-squared test.  
\item  Use a Kolmogorov-Smirnov test to test the hypothesis that the distribution is Poisson with mean 2.  
\end{enumerate}
Comment on your results. 

\\ \hline


\end{tabular}
    

\end{table}

%%%%%%%%%%%%%%%%%%%%%%%%%%%%%%%%%%%%%%%%%%%%%%%%%%%%%%%%%%%%%%%%%%%%%%%%%%%%%%%%%%%%%%%%%
\begin{itemize}
    \item 
The mean of the data is 1
100(0 + 25 + 36 + 36 + 28 + 45 + 30) = 2.00. 

\item In a
poisson distribution with mean 2, P(0) = e¡2 = 0:1353,so the expected frequency of
zeros is 13.53.


\item Similarly P(1) = 2e¡2 = 0:2707; P(2) = 4e¡2
2 = 0:2707 P(3) = 8e¡2
3 =
0:1804 P(4) = 16e¡2
4 = 0.0902 P(5) = 0.0361; P(¸ 6) = 0.0166: 
\item The table of observed
and corresponding expected frequencies is :
count 0 1 2 3 4 ¸ 5 Total
Obs:freq 24 25 18 12 7 14 : 100
Exp:freq 13:53 27.07 27.07 18.04 9.02 5:27 : 100
\item Comparing these in a $\chi^2$ test, there will be 4 degrees o freedom since we are using an
estimate of the mean.
\begin{eqnarray*}
\chi^2{2,4}
&=&
XO ¡ E
E
\\ &=& 
10:472
13:53
+
2.072 + 9.072
27.07
+
6.042
18.04
+
2.022
9.02
+
8:732
5:27
\\ &=& 28:236
\end{eqnarray*}
\item There is strong evidence against the N.H. of a poisson distribution, which is therefore
rejected.
\end{itemize}

\begin{enumerate}
\item kolmogorov-smirnor uses cumulative probabilities:
0 1 2 3 4 5 6 (1)
OBS 0:24 0:49 0:67 0:79 0:86 0:95 1.00 (¡)
EXP 0:1353 0:4060 0:6767 0:8571 0:9473 0:9843 0:9945 (1.0000)
jO ¡ Ej 0:1047 0.0840 0.0067 0.0671 0.0873 0.0334 0.0046 (¡)
\begin{itemize}
    \item The maximum modulus of difference in cumulative probabilities is 0.1047; for n=100
observations, the upper 5\% tail starts at 1p:36
n i.e. 0.1360, so the observed difference is
not significant and there is no evidence for rejecting the poisson N.H. Chi-squared tests
the pattern of frequencies, which had too heavy a tail that was balanced by too many
zeros for a mean=2; Kolmogrorov-smirnov by using cumulative probabilities is not so
affected by the upper-tail.
\item The gross value added by all manufacturing fell in the period 1991/2/3, so any study
of individual component must allow for this. 
\item Also the data are based on 1995 price, so
change in costs of raw materials, labor and manufacture over the nine years will reflect
this: not all price can be automatically increase to compensate for increase in expenditure.
\item The weightings are 1995,so if there were substantial change over the period.
\end{itemize}
 
\end{enumerate}
\end{document}

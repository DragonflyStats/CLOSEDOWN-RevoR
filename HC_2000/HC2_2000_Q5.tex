\documentclass[a4paper,12pt]{article}
%%%%%%%%%%%%%%%%%%%%%%%%%%%%%%%%%%%%%%%%%%%%%%%%%%%%%%%%%%%%%%%%%%%%%%%%%%%%%%%%%%%%%%%%%%%%%%%%%%%%%%%%%%%%%%%%%%%%%%%%%%%%%%%%%%%%%%%%%%%%%%%%%%%%%%%%%%%%%%%%%%%%%%%%%%%%%%%%%%%%%%%%%%%%%%%%%%%%%%%%%%%%%%%%%%%%%%%%%%%%%%%%%%%%%%%%%%%%%%%%%%%%%%%%%%%%
\usepackage{eurosym}
\usepackage{vmargin}
\usepackage{amsmath}
\usepackage{graphics}
\usepackage{epsfig}
\usepackage{enumerate}
\usepackage{multicol}
\usepackage{subfigure}
\usepackage{fancyhdr}
\usepackage{listings}
\usepackage{framed}
\usepackage{graphicx}
\usepackage{amsmath}
\usepackage{chngpage}
%\usepackage{bigints}

\usepackage{vmargin}
% left top textwidth textheight headheight
% headsep footheight footskip
\setmargins{2.0cm}{2.5cm}{16 cm}{22cm}{0.5cm}{0cm}{1cm}{1cm}
\renewcommand{\baselinestretch}{1.3}

\setcounter{MaxMatrixCols}{10}
\begin{document}

%%%%%%%%%%%%%%%%%%%%%%%%%%%%%%%%%%%%%%%%%%%%%%%%%%%%%%%%%%%%%%%%%%%%%%%%%%%%%%%%%%%%%%%%%
\begin{table}[ht!]
     

\centering
     

\begin{tabular}{|p{15cm}|}
     

\hline 


5. Four different marine paints were compared for their ability to protect ships in a sea-going environment.  Sixteen ships were used, each painted with one of the four paints.  Each of the ships was deployed for 6 months and on the ship's return a score was assigned according to the amount of chipping, peeling and average remaining paint thickness.  A higher score indicated a "better" state of repair.  The scores are given in the following table. 
 7

\begin{center}
\begin{tabular}{|c||c|c|c|c|}
Paint 1 & 80 & 73 & 72&  90 \\ \hline 
Paint 2 & 81 & 82 & 88&  84  \\ \hline
Paint 3 & 93 & 80 & 80&  97 \\ \hline
Paint 4 & 89 & 86 & 96&  99 \\ \hline
\end{tabular}
\end{center}
 
(i) Carry out a one-way analysis of variance of these data, stating the assumptions you have made and explaining what you conclude as a result of your analysis. (10) 
 
(ii) After completing the analysis in part (i) you discover that each column of the above table represents a different geographical area in which the ships were deployed.  Using this additional information re-analyse the data and comment upon whether your conclusions are affected by this additional information. (10) 
 
\\ \hline


\end{tabular}
    

\end{table}

%%%%%%%%%%%%%%%%%%%%%%%%%%%%%%%%%%%%%%%%%%%%%%%%%%%%%%%%%%%%%%%%%%%%%%%%%%%%%%%%%%%%%%%%%

\begin{enumerate}
    \item The 1-way analysis compares the N.H.“all ¹i are the same ”, where f¹ig denote
the means of the sores using the 4 points i=1,2,3,4; The A.H. is that there are differences
among the means
Paint total are 315,335,350,370,grand total=1370, N=16 observations. Sum of squares
of observations=118290.Total corrected sum of squares = 118290 ¡ 13702
16 = 983:75. s.s.
for paint = 1
4(3152 + 3352 + 3502 + 3702) ¡ 13702
16 = 406:25
Analysis Of Variance
Source of V ariation Degrees of freedom Sum of squre Mean square
Paints 3 406:25 135:42
Residual 12 577:50 48:125
Total 15 983:75
F(3; 12) = 2:81 n:s:
\item 
Assuming the scores can be modelled by a normal distribution, the linear model underlining
this analysis is y = ¹i +"ij ; f"ijg all N(0; ¾2). 
\item we do not reject the N.H. that all
¹iare the same.
\item If the columns repeat geographical differences, we should remove these in a 2-way analysis.
\item The area total are 343,321,336,370 and give a sum of square 
\[1
4(3432 +3212 +3362 +
3702) ¡ 13702=16 = 315:25.\]

\end{itemize}
ANALYSIS OF VARIANCE
Source DF SS MS
Paint 3 406:25 136:42 F(3; 9) = 4:65
Areas 3 315:25 105:08 F(3; 9) = 3:61n:s:
Residual 9 262:25 29:139 = ¾ˆ2
Total 15 983:75
\begin{itemize}
    \item Making allowance for area differences reduce the residual variation to that which is
purely random natural variation, so make the paints comparison more precise. 
\item Now the
N.H.should be rejected. there is evidence of a different among the paints.
\end{itemize}

\end{enumerate}
\end{document}

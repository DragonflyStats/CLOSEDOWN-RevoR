\documentclass[a4paper,12pt]{article}
%%%%%%%%%%%%%%%%%%%%%%%%%%%%%%%%%%%%%%%%%%%%%%%%%%%%%%%%%%%%%%%%%%%%%%%%%%%%%%%%%%%%%%%%%%%%%%%%%%%%%%%%%%%%%%%%%%%%%%%%%%%%%%%%%%%%%%%%%%%%%%%%%%%%%%%%%%%%%%%%%%%%%%%%%%%%%%%%%%%%%%%%%%%%%%%%%%%%%%%%%%%%%%%%%%%%%%%%%%%%%%%%%%%%%%%%%%%%%%%%%%%%%%%%%%%%
\usepackage{eurosym}
\usepackage{vmargin}
\usepackage{amsmath}
\usepackage{graphics}
\usepackage{epsfig}
\usepackage{enumerate}
\usepackage{multicol}
\usepackage{subfigure}
\usepackage{fancyhdr}
\usepackage{listings}
\usepackage{framed}
\usepackage{graphicx}
\usepackage{amsmath}
\usepackage{chngpage}
%\usepackage{bigints}

\usepackage{vmargin}
% left top textwidth textheight headheight
% headsep footheight footskip
\setmargins{2.0cm}{2.5cm}{16 cm}{22cm}{0.5cm}{0cm}{1cm}{1cm}
\renewcommand{\baselinestretch}{1.3}

\setcounter{MaxMatrixCols}{10}
\begin{document}\begin{table}[ht!]
     \centering
     \begin{tabular}{|p{15cm}|}
     \hline        
%4. 
\large
Two players, A and B, each simultaneously and independently roll a fair die.  
Let $X$ and $Y$ be random variables denoting the respective scores of A and B on any given roll, so that 

\[   P(X=x,Y=y) = \begin{cases}  \frac{1}{36} & x,y= 1,2,3,4,5,6 \\
0 & \mbox{otherwise} \end{cases}\]

 
 
 
(i) Show that 
\[P(X=Y) = \frac{1}{6}\]
 
 
and that 
 
\[P(X>Y) = \frac{5}{12}\]
 
 \\ \hline
      \end{tabular}
    \end{table}

\begin{enumerate}
    \item 
    
\begin{eqnarray*}
P(x = y) &=& P(x = y = 1) + P(x = y = 2) + \ldots + P(x = y = 6)\\ 
&=& \frac{6}{36}\\
&=& \frac{1}{6}\\
\end{eqnarray*}

\begin{itemize}
    \item $P(x > y \mbox{ or }y > x) = 1 - P(x = y) = 5$
\item By symmetry of the joint distribution,$P(x > y) = P(y > x)$, so each of these must be $ \frac{5}{12}$
\item ALTERNATIVELY enumerate all possibilities
\end{itemize}
\newpage
%%%%%%%%%%%%%%%%%%%%%%%%%%%%%%%%%%%%%%%%%%%%%%%%%%%%%%5

    
  \begin{table}[ht!]
     \centering
     \begin{tabular}{|p{15cm}|}
     \hline  
     \large
     
(ii) Let $Z$ be a random variable denoting the number of times A and B each have to roll their dice for one or both to score a six.  Explain why 
{ 
\large
\[   P(Z=z) = \begin{cases}  { \Large \frac{11}{36} \left( \frac{25}{36} \right)^{z-1} } & z= 1,2,3,\ldots \\
0 & \mbox{otherwise} \end{cases}\]
}
 
 \\ \hline 
      \end{tabular}
    \end{table}
    
\item 
\begin{eqnarray*}
P(Z > k) &=& P(\mbox{neither A or B throws a 6 in first $k - 1$ attempts})\\
&=& \left(\frac{5}{6} \times \frac{5}{6} \right)^{k-1} \\
&=& \left(\frac{25}{36} \right)^{k-1} \qquad k = 1, 2, 3, \ldots
\end{eqnarray*}

\begin{eqnarray*}
P(Z = k) &=& P(Z \geq k) - P(Z \geq k + 1)\\
&=& \left( \frac{25}{36} \right)^{k-1} \left(1 -   \frac{25}{36} \right) \\
&=& \left( \frac{11}{36} \right)\left( \frac{25}{36} \right)^{k-1} \qquad k = 1, 2, 3, \ldots
\end{eqnarray*}

    
     

    
    
\item 
\begin{eqnarray*}
P(Z \leq 4) &=& 1 - P(Z \geq 5) \\
&=& 1 - \left( \frac{25}{36} \right)^4 \\
&=&  0.233
\end{eqnarray*}


\begin{eqnarray*}
E(Z) &=& \sum^{\infty}_{k=1}P(z = k) \\ &=& \frac{11}{36} \times \sum^{\infty}_{k=1} k x 
\left(\frac{25}{36}\right)^{k-1}
\end{eqnarray*}
%%%%%%%%%%%%%%%%%%%%%%%%%%%%%%%%%%%%%%%%%%%%%%%%%%%%%%%%%%%%%%%%%%%%%%%%%%%%%%%%%%%%%%%%%%
\newpage
  \begin{table}[ht!]
     \centering
     \begin{tabular}{|p{15cm}|}
     \hline  
     \large
 Making use of this result, find 
\begin{enumerate}[(i)]
    \item $P(Z \leq 4)$ 
    \item $E(Z)$
\end{enumerate}

giving your answers to 3 significant figures. 
 \\ \hline 
      \end{tabular}
    \end{table}
Now
$\sum^{\infty}_{k=1} k x ^{k-1}$ is derivative of
\[ \sum^{\infty}_{k=1} 
xk = \frac{1}{1-x} \]by the used results for a geometric series.
Thus
\[\sum^{\infty}_{k=1} k x ^{k-1} = \frac{d}{dx} \left( \frac{1}{1-x} \right) = \frac{1}{(1-x)^2} \], we have ${ \displaymode x = \frac{25}{36}}$ ;
Therefore
\begin{eqnarray*}
E(Z) &=& \frac{11}{36} \times \frac{1}{(1- \frac{25}{36})^2} \\ &=& \frac{36}{11} 
\\ &=& 3.27
\end{eqnarray*}
Alternatively consider $E(Z)$ and -25
36E(Z), and add to give $\frac{11}{36}E(Z)$ which is 1 as an
infinite geometric series
\end{enumerate}
\end{document}

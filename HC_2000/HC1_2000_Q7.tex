\documentclass[a4paper,12pt]{article}
%%%%%%%%%%%%%%%%%%%%%%%%%%%%%%%%%%%%%%%%%%%%%%%%%%%%%%%%%%%%%%%%%%%%%%%%%%%%%%%%%%%%%%%%%%%%%%%%%%%%%%%%%%%%%%%%%%%%%%%%%%%%%%%%%%%%%%%%%%%%%%%%%%%%%%%%%%%%%%%%%%%%%%%%%%%%%%%%%%%%%%%%%%%%%%%%%%%%%%%%%%%%%%%%%%%%%%%%%%%%%%%%%%%%%%%%%%%%%%%%%%%%%%%%%%%%
\usepackage{eurosym}
\usepackage{vmargin}
\usepackage{amsmath}
\usepackage{graphics}
\usepackage{epsfig}
\usepackage{enumerate}
\usepackage{multicol}
\usepackage{subfigure}
\usepackage{fancyhdr}
\usepackage{listings}
\usepackage{framed}
\usepackage{graphicx}
\usepackage{amsmath}
\usepackage{chngpage}
%\usepackage{bigints}

\usepackage{vmargin}
% left top textwidth textheight headheight
% headsep footheight footskip
\setmargins{2.0cm}{2.5cm}{16 cm}{22cm}{0.5cm}{0cm}{1cm}{1cm}
\renewcommand{\baselinestretch}{1.3}

\setcounter{MaxMatrixCols}{10}
\begin{document}
\begin{table}[ht!]
     \centering
     \begin{tabular}{|p{15cm}|}
     \hline        
The random variable $X$ follows the discrete uniform distribution on the integers $1, 2, \ldots, k$ so that the probability mass function of $X$ is given by 

\[ p(x)   = \begin{cases}  \frac{1}{k}   & x = 1,2,\ldots k \\
0 & \mbox{ otherwise } \\ \end{cases}\]

 
(i) Show that for a general positive integer $k$ 

 \[E(X)  = \frac{k+1}{2}\] 

 
 \\ \hline
      \end{tabular}
    \end{table}
%%%%%%%%%%%%%%%%%%%%%%%%%%%%%%%%%%%%%%%%%%%%%%%%%%%%%%%%%%%%%%%%%%%%%%%%%%%%%%%%%%%

\begin{enumerate}[(a)]
    \item 
    
\begin{eqnarray*}
E(X)    &=& \sum^{k}_{1=1} \frac{i}{k}  \\
    &=& \frac{1}{k} + \frac{2}{k} + \frac{3}{k} + \ldots + \frac{k}{k}\\
    & & \mbox{ (Arithmetic Series) } \\
%&=&  \frac{1}{k} \frac{1}{2} k(k + 1)\\
&=& \frac{k+1}{2}\\
\end{eqnarray*}

\begin{framed}
\noindent \textbf{Arithmetic Series}
\begin{multicols}{3}
\begin{itemize} 
\item $a  = \frac{1}{k}$
\item $d  = \frac{1}{k}$
\item $n = k$
\end{itemize}
\end{multicols}


\[  S_n \;=\; \frac{n}{2}\left( 2a + (n-1)d \right)  \;=\; \frac{k}{2} \left[ 2 \left( \frac{1}{k} \right)  + (k-1) \left( \frac{1}{k} \right) \right] \]
\end{framed}

\newpage
  \begin{table}[ht!]
     \centering
     \begin{tabular}{|p{15cm}|}
     \hline  
(ii) A random sample of size four, $X_1, X_2, X_3, X_4$, is taken from this distribution, yielding values $x_1, x_2, x_3, x_4$, from which it is intended to estimate the parameter $k$. 
 
(a) Show that the method of moments estimator of $k$ is given by $\hat{k}_1 = 2 \bar{X} - 1$ where $bar{X}$ denotes the sample mean. 

 
(b) Explain clearly why the maximum likelihood estimator of k is given by $\hat{k}_2 = X_{(4)$, the sample maximum.  
 
(c) Calculate $\hat{k}_1$  and $\hat{k}_2$ for the sample $(1, 10, 3, 2)$ and comment on your results.  \\ \hline 
      \end{tabular}
    \end{table}
%%%%%%%%%%%%
\item The moment estimator $\hat{k}_1$ is found from setting $E[x] = \bar{x}$ i.e. $\bar{x}$ = 1
2 ( $\hat{k}_1$ +
1) or \[\hat{k}_1 = 2\bar{x} - 1\]
%%%%%%%%%%%%
\begin{framed}

Explain clearly why the maximum likelihood estimator of k is given by 2 (4) ˆ kX = , the sample maximum. 


\end{framed}
\item Likelihood =
Q4
i=1
f(xi) = 1
k4 provided all xi lie between 1 and k inclusive.
The maximum of this occurs when $\hat{k}_2$ is chosen to be as small as possible given the data
values; i.e. $\hat{k}_2 = x(4)$, the sample maximum.

%%%%%%%%%%%%%%%%%%%%%%%%%%%%%%%%%%%%
\end{framed}
Calculate 1 ˆ k and 2 ˆ k for the sample (1, 10, 3, 2) and comment on your results. (
\begin{framed}
\item Forfxig = f1; 10; 3; 2g; $\hat{k}_1 = 2(16
4 )$ ¡ 1 = 7 $\hat{k}_2 = x(4) = 10$
$\hat{k}_1$ is impossible, since there is a data value above 7. $\hat{k}_2$ is consistent with the data.
\end{enumerate}

\end{document}

\documentclass[a4paper,12pt]{article}
%%%%%%%%%%%%%%%%%%%%%%%%%%%%%%%%%%%%%%%%%%%%%%%%%%%%%%%%%%%%%%%%%%%%%%%%%%%%%%%%%%%%%%%%%%%%%%%%%%%%%%%%%%%%%%%%%%%%%%%%%%%%%%%%%%%%%%%%%%%%%%%%%%%%%%%%%%%%%%%%%%%%%%%%%%%%%%%%%%%%%%%%%%%%%%%%%%%%%%%%%%%%%%%%%%%%%%%%%%%%%%%%%%%%%%%%%%%%%%%%%%%%%%%%%%%%
\usepackage{eurosym}
\usepackage{vmargin}
\usepackage{amsmath}
\usepackage{graphics}
\usepackage{epsfig}
\usepackage{enumerate}
\usepackage{multicol}
\usepackage{subfigure}
\usepackage{fancyhdr}
\usepackage{listings}
\usepackage{framed}
\usepackage{graphicx}
\usepackage{amsmath}
\usepackage{chngpage}
%\usepackage{bigints}

\usepackage{vmargin}
% left top textwidth textheight headheight
% headsep footheight footskip
\setmargins{2.0cm}{2.5cm}{16 cm}{22cm}{0.5cm}{0cm}{1cm}{1cm}
\renewcommand{\baselinestretch}{1.3}

\setcounter{MaxMatrixCols}{10}
\begin{document}

\begin{table}[ht!]
 \centering
 \begin{tabular}{|p{15cm}|}
 \hline  
2. A government agency has developed a formula for calculating the theoretical carrying capacity of an urban roundabout.  Owing to a controversy about the usefulness of the formula, the civil engineering department of a university was asked to investigate the situation.  This they did by selecting ten roundabouts to which the formula was applicable, calculating the theoretical carrying capacity by the formula and observing the actual capacity for each roundabout.  The results of this investigation are shown below. 
 
Carrying capacity (hundreds of vehicles per hour) Theoretical capacity by formula (x) Observed capacity (y) Theoretical capacity by formula (x) Observed capacity  (y) 32.0 33.0 40.0 45.0 33.0 38.4 43.0 43.2 34.0 37.4 45.0 49.0 35.1 42.0 46.9 50.0 36.9 39.4 48.0 47.4 
 
∑∑∑ = == . 66.17005,68.18317,23.15840 22 xy yx 
 
The method of least squares was used to fit a straight line to these data and some of the output from the package is shown below. 
 
The regression equation is y =  ***** + ***** x  
 
Predictor Coef    StDev           Constant  *****     5.114        x   *****   0.1285        
 
S = 2.315       R-Sq = 84.3%      
 
Analysis of Variance 
 
Source  DF      SS         MS             F          P Regression    1 229.31      229.31     42.80    0.000 Error     8   42.87          5.36 Total     9 272.18 
 
(i) Plot the data.               (4) 
 
(ii) Calculate the missing values ***** in the output above (estimates of the parameters of the line) and draw the fitted line on your scatter plot.        (8) 
 
(iii) What information is provided by the value labelled as "R-Sq" in this output?                (3) 
 
(iv) Calculate 95% confidence intervals for the slope and the intercept.        (5) \\ \hline
  \end{tabular}
\end{table}
\begin{enumerate}
\item
16
(ii) P
y = 424:8 sxy = 17005:66 ¡ (424:8 £ P 393:9)=10 = 272:788
x = 393:9 sxx = 15840:23 ¡ 393:92=10 = 324:509
Hence the slope
ˆb
=
272:788
324:509
= 0:8406
The fitted line is y¡¯y =ˆb(x¡¯x) or y¡42:48 = 0:8406(x¡39:39) i.e. y = 0:8406x+9:3681
or y = 9:37 + 0:84x
\item The proportion of total variation (sum of square), that is explained by the relation
of observation to formula is to 0.843.[(229:31=272:18) ¼ 0:8425] this is reasonably
good.
\item The calculation is based on the theoretical model y = ®+¯x+",where var(") =
¾2 is estimated by s2 = (2:315)2 = 5:36 It has 8 d.f. var(ˆb) = ¾2=sxx estimated as
5:36
324:509 = 0:016517, so SˆE = 0:1285
17
A 95% interval for ¯ is :
ˆb
§ 2:306 £ 0:1285 = 0:8406 § 0:2963 i:e:(0:544 to 1:137)
[t(8;5%) = 2:306]. A 95% interval for ® is :
ˆa § 2:306
s
s2(
1
10
+
¯x2
sxx
) = 9:368 § 2:306 £
p
5:36 £ 4:8813 = 9:368 § 11:795
This give(-2.427 to +21.16)
Note that this is very imprecisely determined(and is a little use since there are no data
near to zero to confirm whether a linear relation still holds)
\end{enumerate}
\end{document}

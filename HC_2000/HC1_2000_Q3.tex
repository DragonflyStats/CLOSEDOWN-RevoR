\documentclass[a4paper,12pt]{article}
%%%%%%%%%%%%%%%%%%%%%%%%%%%%%%%%%%%%%%%%%%%%%%%%%%%%%%%%%%%%%%%%%%%%%%%%%%%%%%%%%%%%%%%%%%%%%%%%%%%%%%%%%%%%%%%%%%%%%%%%%%%%%%%%%%%%%%%%%%%%%%%%%%%%%%%%%%%%%%%%%%%%%%%%%%%%%%%%%%%%%%%%%%%%%%%%%%%%%%%%%%%%%%%%%%%%%%%%%%%%%%%%%%%%%%%%%%%%%%%%%%%%%%%%%%%%
\usepackage{eurosym}
\usepackage{vmargin}
\usepackage{amsmath}
\usepackage{graphics}
\usepackage{epsfig}
\usepackage{enumerate}
\usepackage{multicol}
\usepackage{subfigure}
\usepackage{fancyhdr}
\usepackage{listings}
\usepackage{framed}
\usepackage{graphicx}
\usepackage{amsmath}
\usepackage{chngpage}
%\usepackage{bigints}

\usepackage{vmargin}
% left top textwidth textheight headheight
% headsep footheight footskip
\setmargins{2.0cm}{2.5cm}{16 cm}{22cm}{0.5cm}{0cm}{1cm}{1cm}
\renewcommand{\baselinestretch}{1.3}

\setcounter{MaxMatrixCols}{10}
\begin{document}
\begin{table}[ht!]
     \centering
     \begin{tabular}{|p{15cm}|}
     \hline        
     \large
For $i = 1, 2, \ldots, n$, let $X_i$ be Normally distributed with mean $\mu_i$ i and variance $\sigma^2_i$, and let $X_1, X_2, \ldots, X_n$ be independent.  If $a_1, a_2, \ldots$, an are any 
non-zero constants, state the distribution of 
\[  \sum^{n}_{i=1} a_iX_i. \] 
Deduce the  distribution of $X_1 + X_2$ and $X_1 - X_2$ . 
 \\ \hline
      \end{tabular}
    \end{table}
    

    

\begin{enumerate}
    \item 
Linear combinations of normal variables remain normal.
\[
y \sim N( \sum^{n}_{i=1} a_iX_i;  \sum^{n}_{i=1} a_iX_i)
i \sigma^2
i )\]
Hence $x1 + x2 \sim N(\mu_1 + \mu_2; \sigma^2_1 + \sigma^2_2)$; $x1 - x2 \sim N(\mu_1 - \mu_2; \sigma^2_1 + \sigma^2_2)$
%%%%%%%%%%%
\newpage
  \begin{table}[ht!]
     \centering
     \begin{tabular}{|p{15cm}|}
     \hline  
 \large
(b) A mathematician takes, on average, 35 minutes to get to work;  his travelling time is Normally distributed with standard deviation 4 minutes.  

His colleague, the statistician takes, on average, 33 minutes to get to work;  her travelling time is Normally distributed with standard deviation 3 minutes.  Stating any further assumptions you make, find 
 
\begin{enumerate}[(i)] 
\item the probability that the mathematician reaches work before the statistician, supposing that they leave their homes at the same time, 
 
\item how much earlier the statistician must leave her home to be 90\% certain of getting to work before the mathematician, 
 
\item the probability that both get to work within 30 minutes of the times at which they set out. 
\end{enumerate}

 \\ \hline
      \end{tabular}
    \end{table}
\item Let M,S be travelling times of mathematician and statistician.
$M \sim N(35; 16)$ and $S \sim N(33; 9)$
(i)Assuming journey times are independent ,
$M - S \sim N(2; 25)$ and $P(M - s < 0) = P(z <
\frac{0 - 2}{5}$
where $z \sim N(0.; 1)$; this is $P(z < -0.4) = 0.3446$
%%%%%%%%%%%%%%%
\item If statistician leaves t minutes earlier, the difference in arrival times will be $N(2+t,25)$,assuming
that the journey times still have the same distributions as before.
\begin{eqnarray*}
P(\mbox{statistician arrives first}) &=& P(M - S + t > 0)\\
&=& 1 - \Phi \left(\frac{1(2+t)}{5} \right) \geq¸ 0.9 
\end{eqnarray*}
if
$ \frac{t + 2}{5} \geq 1:2826$ from $N(0; 1)$ table i.e. $t = 4.408 min$
\item
\begin{eqnarray*}
P(M < 30) &=& \Phi \left( \frac{30-35}{4} \right) 
 \\ &=& \Phi(-1.25)\\ &=& 0.01056
\end{eqnarray*}

\begin{eqnarray*}
P(S < 30)  &=&  \Phi \left( \frac{30-33}{3} \right) \\ &=&  \Phi(-1) \\ &=&  0.1587
\end{eqnarray*}
and require probability is $0.1056 \times 0.1587 = 0.0168$

%%%%%%%%%%%%%%%%%%%%
\newpage
  \begin{table}[ht!]
     \centering
     \begin{tabular}{|p{15cm}|}
     \hline  
     \large
(c) In an office there are 20 computers, each of which has, independently of the rest, a Poisson incidence of breakdowns at the rate of 0.02 per week.  Breakdowns are invariably repaired with negligible delay.\\\

\medskip Find the probabilities that 
 
\begin{enumerate}
\item (i) a period of 4 weeks passes with no breakdowns, 
\item (ii) in a period of 4 weeks there is at least one breakdown each week. 
\end{enumerate}

Use a suitable approximation to calculate the probability of more than 26 breakdowns over a 52 week period.   
 \\ \hline 
      \end{tabular}
    \end{table}
\item  The number of breakdowns per week will follow a poisson distribution with mean
$20 \times 0.02 = 0.4$
\item 
\[P(\mbox{no breakdowns}) = e^{-0.4}\]
\[P(\mbox{none in 4 weeks}) = (e^{-0.4})4 = e-1:6 = 0.2019\]
\item 
\[P(1 or more in a week) = 1 - e^{-0.4} = 0.39297\]
required probability = $(0.3297)^4 = 0.0118$
\begin{itemize}
    \item In 52 weeks, number of breakdowns in poisson with mean $0.4 \times 52 = 20.8$ 
    \item A normal
approximation $N(20.8,20.8)$ may be used, and using a continuity correction we
require $P(number > 26.5)$
\end{itemize}

\[z = \frac{26.5-20.8}{\sqrt{20.8}} = \frac{5.7}{4.5607} = 1.2498\]
\begin{eqnarray*}
P(z > 1.2498) 
&=& P(z < -1.2498) \\
&=& 0.1056\\
\end{eqnarray*} from tables
\end{enumerate}
\end{document}

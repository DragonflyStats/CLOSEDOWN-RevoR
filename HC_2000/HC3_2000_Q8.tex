\documentclass[a4paper,12pt]{article}
%%%%%%%%%%%%%%%%%%%%%%%%%%%%%%%%%%%%%%%%%%%%%%%%%%%%%%%%%%%%%%%%%%%%%%%%%%%%%%%%%%%%%%%%%%%%%%%%%%%%%%%%%%%%%%%%%%%%%%%%%%%%%%%%%%%%%%%%%%%%%%%%%%%%%%%%%%%%%%%%%%%%%%%%%%%%%%%%%%%%%%%%%%%%%%%%%%%%%%%%%%%%%%%%%%%%%%%%%%%%%%%%%%%%%%%%%%%%%%%%%%%%%%%%%%%%
\usepackage{eurosym}
\usepackage{vmargin}
\usepackage{amsmath}
\usepackage{graphics}
\usepackage{epsfig}
\usepackage{enumerate}
\usepackage{multicol}
\usepackage{subfigure}
\usepackage{fancyhdr}
\usepackage{listings}
\usepackage{framed}
\usepackage{graphicx}
\usepackage{amsmath}
\usepackage{chngpage}
%\usepackage{bigints}

\usepackage{vmargin}
% left top textwidth textheight headheight
% headsep footheight footskip
\setmargins{2.0cm}{2.5cm}{16 cm}{22cm}{0.5cm}{0cm}{1cm}{1cm}
\renewcommand{\baselinestretch}{1.3}

\setcounter{MaxMatrixCols}{10}
\begin{document}
\begin{table}[ht!]
 \centering
 \begin{tabular}{|p{15cm}|}
 \hline  
8. The table below gives the total deaths from lung diseases in the UK for each quarter for the years 1974 to 1979.  
 
1974   1 8291 1977 1 8059  2 6223  2 6155  3 4841  3 4766  4 6785  4 6393 1975 1 8760 1978 1 8779  2 6093  2 6046  3 4548  3 4552  4 6700  4 6492 1976 1 9857 1979 1 9386  2 5227  2 5347  3 4145  3 4259  4 6489  4 6206 
 
Plot the data as a time series and describe the main features of the data. 
(7) The data are also given on the worksheet for this question together with an incomplete set of centred four point moving average values and an incomplete set of differences between the actual deaths and the corresponding moving average values.  Complete the calculation of moving average values and differences. (5) Complete the calculation of seasonal components on the assumption of an additive model (trend plus seasonal plus residual) for the data. (4) The trend line for the data is deaths = 6831 − 31.8t where t = number of quarters since the start of 1974 (for example, for quarter 2 of 1976, t = 10). 
 
Use this information to predict deaths for 1980. 
(4) \\ \hline
  \end{tabular}
\end{table}
\begin{enumerate}8
O outside W F and P
There is a clear seasonal pattern of deaths, highest in quarter 1 and lower in quarter
3 each year. Trend is small, and an additive model should be appropriate.
The centred 4-point average to go in 1974(3) is found by taking the first four item’s
average and setting it at 21
2 , then the average of items 2-3-4-5 set at 31
2,finally averaging
these two to go at 3. So we have 1
8(8291 + 2(6223 + 4841 + 6785) + 8760) = 6593:625
etc. Average the figures for each quarter in the final column gives seasonal means which
total 19.82. Effect are mean - 1
4(19:82) for the four quarters.
23
The trend at t=25,26,27,28 of y=68.31-31.8t is 6036.0, 6004.2,5972 and 5940.6. Predicted
values are trend +seasonal effect, giving for
1980 (1)8542:83; (2)5445:41; (3)4050:41; (4)6014:55:
CALCULATIONS:
Y M.AV. Y-M.AV
1974 1 8291
2 6223
3 4841 6593.63 -1752.63
4 6785 6636.00 149.00
1975 1 8760 6583.13 2176.88
2 6093 6535.88 -442.8
3 4548 6662.38 -2114.38
4 6700 6691.25 8.75
1976 1 9875 6532.5 3324.38
2 5227 6455.88 -1228.88
3 4145 6204.75 -2059.75
4 6489 6096.00 393.00
1977 1 8059 6289.63 1769.83
2 6155 6355.25 -200.25
3 4766 6433.25 -1667.25
4 6393 6509.63 -116.63
1978 1 8779 6469.25 2309.75
2 6046 6454.88 -408.88
3 4552 6543.13 -1991.13
4 6492 6531.63 -39.63
1979 1 9386 6407.63 2978.38
2 5347 6335.25 -988.25
3 4259
4 6206
Calculation of seasonal effects:
Q1 Q2 Q3 Q4
2176.9 -442.88 -1752.63 149.000
3324.4 -1228.88 -2114.38 8.750
detrended data 1769.4 -200.25 -2059.75 393.00
2309.8 -408.88 -1667.25 -116.625
2978.4 -998.25 -1991.13 -39.625
seasonal means 2511.78 -653.83 -1917.03 78.9 19.82
seasonal effects 2506.83 -658.79 -1921.99 73.95
\end{enumerate}
\end{document}

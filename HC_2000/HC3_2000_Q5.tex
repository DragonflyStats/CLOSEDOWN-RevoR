\documentclass[a4paper,12pt]{article}
%%%%%%%%%%%%%%%%%%%%%%%%%%%%%%%%%%%%%%%%%%%%%%%%%%%%%%%%%%%%%%%%%%%%%%%%%%%%%%%%%%%%%%%%%%%%%%%%%%%%%%%%%%%%%%%%%%%%%%%%%%%%%%%%%%%%%%%%%%%%%%%%%%%%%%%%%%%%%%%%%%%%%%%%%%%%%%%%%%%%%%%%%%%%%%%%%%%%%%%%%%%%%%%%%%%%%%%%%%%%%%%%%%%%%%%%%%%%%%%%%%%%%%%%%%%%
\usepackage{eurosym}
\usepackage{vmargin}
\usepackage{amsmath}
\usepackage{graphics}
\usepackage{epsfig}
\usepackage{enumerate}
\usepackage{multicol}
\usepackage{subfigure}
\usepackage{fancyhdr}
\usepackage{listings}
\usepackage{framed}
\usepackage{graphicx}
\usepackage{amsmath}
\usepackage{chngpage}
%\usepackage{bigints}

\usepackage{vmargin}
% left top textwidth textheight headheight
% headsep footheight footskip
\setmargins{2.0cm}{2.5cm}{16 cm}{22cm}{0.5cm}{0cm}{1cm}{1cm}
\renewcommand{\baselinestretch}{1.3}

\setcounter{MaxMatrixCols}{10}
\begin{document}

\begin{table}[ht!]
 \centering
 \begin{tabular}{|p{15cm}|}
 \hline  
 \large
5. The Poisson distribution may be used to provide a simple statistical model for the number of vehicles passing a traffic survey point when traffic is light.  The data below were collected from 120 independent one-minute intervals at a survey point. 
 
Number of vehicles 0 1 2 3 4 5 6 7 8 ≥9 
Number of intervals 8 30 32 20 13 9 5 2 1 0 
 
 
 \begin{enumerate}[(i)]
\item (i) Assuming that a Poisson model is appropriate, calculate the expected numbers of intervals with x vehicles per minute for x = 0, 1, 2, …, 8 and for x ≥ 9. %%--- (7) 
 
\item (ii) Use an appropriate statistical test to assess whether a Poisson model is reasonable for this set of data. (7) 
 
 \item 
(iii) Assuming that a Poisson model is valid, obtain approximate 95% confidence intervals for: 
 
(a) the mean number of vehicles passing the survey point per minute, 
 
(b) the probability of at least one vehicle passing the survey point in a minute. (6) 

\end{enumerate}

\\ \hline
  \end{tabular}
\end{table}
\begin{enumerate}Mean =
1
120
(0 + 30 + 64 + 60 + 52 + 45 = 30 + 14 + 8) =
303
120
= 2:525
P(r) = e¡2:525(2:525)r=r! for r = 0; 1; 2 ¢ ¢ ¢
Expected frequencies are 120P(r).
¿ = 0 1 2 3 4 5 6 ¸ 7
Ei = 9:607 24:258 30:625 25:776 16:271 8:217 3:458 1:788
7 is 1.247; 8 is 0.394;¸ 9 is 0.147.
oi = 8 30 32 20 13 9 8
.
(ii)

%HC3 - Question 5

\begin{eqnarray*}
\chi^2_{TS}
&=&
\frac{(1.607)2 }{9.607}
+
\frac{(-5.742)2 }{24.258}
+
\frac{(-1.375)2}{30.625}
+
\frac{5.776^2}{25.776}
+
\frac{3.271^2}{16.271}
+
\frac{(¡0.783)^2}{8.217}
+
\frac{(¡2.754)^2}{6.246}\\

&=& 5.162 \\ 
\end{eqnarray*}

n.s.

So a Null Hypothesis that the poisson model holds is not rejected.
(iii) Assuming that the Poisson model is valid we use 2.525 as the variance, so
2:525 § 1:96
q
2:525
120 is an approximate 95% confidence interval for the true mean.

\begin{itemize}
    \item (a) This is 2:525 § 1:96 £ 0:145 = 2:525 § 0:284 or (2.24 to 2.81).
\item Note. it is often specified that this approximation require a mean of at least 5 to be
satisfactory.
\item (b)P(¸ 1) = 1 ¡ P(0) = 1 ¡ 0:080 = 0:92. 
\item This is an estimate of the proportion
of non-zero minutes, and variance is 0:92£0:08
120 which is 0.0006133, whose square root is
0.0248.
\item An approximate 95% interval for the true proportion is 0:92§1:96£0:0248 or 0:92§0:049,
i.e. 0.87 to 0.97.
\end{itemize}

\end{enumerate}
\end{document}

\documentclass[a4paper,12pt]{article}
%%%%%%%%%%%%%%%%%%%%%%%%%%%%%%%%%%%%%%%%%%%%%%%%%%%%%%%%%%%%%%%%%%%%%%%%%%%%%%%%%%%%%%%%%%%%%%%%%%%%%%%%%%%%%%%%%%%%%%%%%%%%%%%%%%%%%%%%%%%%%%%%%%%%%%%%%%%%%%%%%%%%%%%%%%%%%%%%%%%%%%%%%%%%%%%%%%%%%%%%%%%%%%%%%%%%%%%%%%%%%%%%%%%%%%%%%%%%%%%%%%%%%%%%%%%%
\usepackage{eurosym}
\usepackage{vmargin}
\usepackage{amsmath}
\usepackage{graphics}
\usepackage{epsfig}
\usepackage{enumerate}
\usepackage{multicol}
\usepackage{subfigure}
\usepackage{fancyhdr}
\usepackage{listings}
\usepackage{framed}
\usepackage{graphicx}
\usepackage{amsmath}
\usepackage{chngpage}
%\usepackage{bigints}

\usepackage{vmargin}
% left top textwidth textheight headheight
% headsep footheight footskip
\setmargins{2.0cm}{2.5cm}{16 cm}{22cm}{0.5cm}{0cm}{1cm}{1cm}
\renewcommand{\baselinestretch}{1.3}

\setcounter{MaxMatrixCols}{10}
\begin{document}

\begin{table}[ht!]
 \centering
 \begin{tabular}{|p{15cm}|}
 \hline  
In an investigation concerned with storage techniques associated with bone marrow transplantation, samples of bone marrow, all from the same subject, were stored at −70°C for various lengths of time.  After return to 37°C, the viability of each sample was determined.  Viability is a measure of the ability of the bone marrow cells to function naturally.  The results of the viability tests are given below;  a high value indicates high viability. 
 
Storage time 
0 hours 
24 hours 
36 hours 
48 hours 
72 hours 
120 hours 
 
150 140 125 
151 135 123 
150   64 136 
145 132 119 
132 114 142 
  74 102   91 Total 415 409 350 396 388 267 
 
Sum of all viability results = 2225. Sum of squares of all viability results = 286327. 
 
(i) Perform an appropriate one-way ANOVA analysis to assess whether storage time affects viability. (12) 
 
(ii) If the sample at 36 hours with viability 64 is omitted from the analysis, the residual sum of squares becomes 1948.  Construct a revised analysis of variance table and a table of means appropriate to this situation.  What is the conclusion indicated by the revised analysis?  How would you explain this to the investigator who provided the data? (8) \\ \hline
  \end{tabular}
\end{table}
\begin{enumerate}
\item N=18 observations. Total (corrected) ss = 286327 ¡ 22252=18 = 11292:28
ss for times =
1
3
(4152 + ¢ ¢ ¢ + 2672) ¡
22252
18
=
840655
3
¡
22252
18
= 5183:61
Analysis of variance:
Source of variation DF Sum of squares Meansquare
Times 5 5183:61 1036:72
Residual 12 6108:67 509:06 F(5; 12) = 2:04n:s:
Total 17 11292:28
This provides no evidence in favor of any time effect.
\item Omitting the given sample,grand total is now 2161,N=17, G2=N = 274701:2353;
sum of all squres = 286327 ¡ 642 = 282231 total for time 36(2 observations)=286,and
ss for times is
2862
2
+
1
3
(4152 + 4092 + 3962 + 3882 + 2672) ¡
G2
N
= 5581:76
Residual now has 11 d.f. and total 16. residual ss=1948.00
DF SS MS
Times 5 5581:76 116:35 F(5; 11) = 6:30
Residual 11 1948:00 177:09 = s2
Total 16
We should now reject the hypothesis that the mean results at each time are all the same.
(0) (24) (36) (48) (72) (120)
Means 138:3 136:3 134:0 132:0 129:3 89:0
18
\begin{itemize}
    \item The reason for the different conclusion is that now 120 stands out from the others.
\item The SE of difference between two means (not including (36)) is
q
2
3 £ 177:09 = 10:87,and
between (36) and any other is
q
( 1
2 + 1
3)(177:09) = 12:15 so t test (11 d.f.) would confirm
this conclusion.
\item The value at 36 hours which has been omitted looked suspiciously low, and could perhaps
have been a measure for recording error. 
\item Now it can not be checked, but if there
are laboratory records available. that may help to decide whether or not to induce it in
the analysis. 
\item It makes a serious difference to the conclusions.
\end{itemize}


\end{enumerate}
\end{document}

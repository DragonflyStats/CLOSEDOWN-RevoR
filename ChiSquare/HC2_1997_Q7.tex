\documentclass[a4paper,12pt]{article}
%%%%%%%%%%%%%%%%%%%%%%%%%%%%%%%%%%%%%%%%%%%%%%%%%%%%%%%%%%%%%%%%%%%%%%%%%%%%%%%%%%%%%%%%%%%%%%%%%%%%%%%%%%%%%%%%%%%%%%%%%%%%%%%%%%%%%%%%%%%%%%%%%%%%%%%%%%%%%%%%%%%%%%%%%%%%%%%%%%%%%%%%%%%%%%%%%%%%%%%%%%%%%%%%%%%%%%%%%%%%%%%%%%%%%%%%%%%%%%%%%%%%%%%%%%%%
\usepackage{eurosym}
\usepackage{vmargin}
\usepackage{amsmath}
\usepackage{graphics}
\usepackage{epsfig}
\usepackage{enumerate}
\usepackage{multicol}
\usepackage{subfigure}
\usepackage{fancyhdr}
\usepackage{listings}
\usepackage{framed}
\usepackage{graphicx}
\usepackage{amsmath}
\usepackage{chngpage}
%\usepackage{bigints}

\usepackage{vmargin}
% left top textwidth textheight headheight
% headsep footheight footskip
\setmargins{2.0cm}{2.5cm}{16 cm}{22cm}{0.5cm}{0cm}{1cm}{1cm}
\renewcommand{\baselinestretch}{1.3}

\setcounter{MaxMatrixCols}{10}
\begin{document}
	%%%%%%%%%%%%%%%%%%%%%%%%%%%%%%%%%%%%%%%%%%%%%%%%%%%%%%%%%%%%%%%%%%%%%%%%%%%%%%%%%%%%%
	\begin{table}[ht!]
		
		\centering
		
		\begin{tabular}{|p{15cm}|}
			
			\hline 
			%%--- Q7. 
			
			\large A factory manager wishes to know whether or not the number of rejects produced by an
			industrial process within a set period follows a Poisson distribution. The previous month’s
			data, given below, are thought to be typical:
			\begin{center}
				\begin{tabular}{|c|c|c|c|c|c|c|c|} \hline
					Number of rejects & 0 & 1 & 2 & 3 & 4 & 5 & 6 or more\\ \hline
					Frequency & 38 & 49 & 43 & 17 & 11 & 2 & 0 \\ \hline
				\end{tabular}
			\end{center}
			\large 
			
			(a) For what reasons might the rejects follow a Poisson distribution?
			
			(b) Test the hypothesis that the distribution of rejects is Poisson, and explain your result to
			the factory manager.
			
			\\ \hline
			
		\end{tabular}
		
	\end{table}
	%%%%%%%%%%%%%%%%%%%%%%%%%%%%%%%%%%%%%%%%%%%%%%%%%%%%%%%%%%%%%%%%%%%%%%%%%%%%%%%%%%%%%
	
	\large 
	\noindent \textbf{Part (a)}
	\begin{itemize}
		\item If the process is producing individual rejects ``at random”, i.e. singly and
		at unpredictable instants of time, but at a constant rate over the period
		of study, then the number of rejects during a fixed time of observation will
		follow a Poisson distribution.
	\end{itemize}
	
		\medskip
	\noindent \textbf{Part (b)}
	\begin{itemize}
 \item The Null hypothesis is that occurrence of rejects follows a Poisson distribution.
 \item Calculate the mean number of rejects per shift: 
		\begin{center}
			\begin{tabular}{|c|c|c|c|c|c|c|c||c|} \hline
				Number of rejects & 0 & 1 & 2 & 3 & 4 & 5 & $\geq 6$ & Total\\ \hline
				Frequency & 38 & 49 & 43 & 17 & 11 & 2 & 0 & 160 \\ \hline  
				Total & 0 & 49 & 86 & 51 & 33 & 10 & 0 & 240 \\ \hline
			\end{tabular}
		\end{center}
		\large 
		\item The total number of inspections was 160.
		\item There were 240 rejected items.
		\item The mean must be estimated from the data:
		\[ \bar{x} \; = \;
		\frac{1}{160}
		(0 + 49 + 86 + 51 + 44 + 10) \;=\;
		\frac{240}{160}
		\;=\; 1.5 \]
		%%%%%%%%%%%%%%%%%%%%%%%%%%%%%%%%%%
		\newpage 
		\item Expected frequencies are ${ \displaystyle 160\; \frac{e^{-1.5}(1.5)^r}{r!} }$ for $r = 0, 1, \ldots$.
		
		\large
		\begin{center}
			\begin{tabular}{|c||c|c|c|c|c|c||c|}\hline
				r : &0 & 1 & 2 & 3 & 4 & $\geq¸ 5$ & TOTAL\\ \hline 
				Obs:& 38 & 49 & 43 & 17 & 11 & 2&  160\\ \hline
				Exp: &35.70& 53.55& 40.16 & 20.08 & 7.53&  2.98 & 160 \\ \hline
			\end{tabular}
		\end{center}
		(The last two cells may be combined, but this is not really necessary.)
		\\
		
		\begin{framed}
			Degree of freedom: In Chi-Square goodness of fit test, the degree of freedom depends on the distribution of the sample.  The following table shows the distribution and an associated degree of freedom:
			
			\begin{center}
				\begin{tabular}{|c|c|c|} \hline 
					Type of distribution 	& No of constraints &	Degree of freedom \\  \hline 
					Binominal distribution 	&1& 	n-1 \\  \hline   
					Poisson distribution 	&2& 	n-2\\  \hline  
					Normal distribution 	&3& 	n-3\\  \hline 
				\end{tabular}
			\end{center}
			
		\end{framed}
		
		$\chi^2$ has 4 d.f.. From the tables, using a 5\% significance level, the critical value is 9.488.
		\\
		\item The value of the test-statistic is 
		\[ {\displaystyle \chi^{2} = \sum_{i=1}^{n}  \frac{(O_{i}-E_{i})^{2}}{E_{i}}} \]
		
		
		
		\begin{eqnarray*}
			\chi^2_{(4)} &=& \frac{(38-35.70)^2}{35.70} \;+\; \frac{(49-53.55)^2}{53.55} \;+\; \frac{(43-40.16)^2}{40.16} +\\ 
			& & \frac{(17-20.08)^2}{20.08} \;+\; \frac{(11-7.53)^2}{7.53} \;+\; \frac{(2-2.98)^2}{2.98}\\\\
			&=& \frac{(38-35.70)^2}{35.70} \;+\; \frac{(49-53.55)^2}{53.55} \;+\; \frac{(43-40.16)^2}{40.16} +  \\ 
			& &  \frac{(17-20.08)^2}{20.08} \;+\; \frac{(11-7.53)^2}{7.53} \;+\; \frac{(2-2.98)^2}{2.98}\\\\
			&=& 3.13.
		\end{eqnarray*}


		\item Is the Test Statistic greater that the Critical Value? No. Not significant.
		
		\large 
		\item There is no reason to reject the hypothesis that the data follow a Poisson
		distribution. 
		\item Therefore the number of rejects per unit time is likely to remain
		reasonably constant and they do not arise in any regular or predictable way.
	\end{itemize}
\end{document}

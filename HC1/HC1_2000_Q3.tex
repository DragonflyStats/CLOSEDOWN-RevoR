\documentclass[a4paper,12pt]{article}
%%%%%%%%%%%%%%%%%%%%%%%%%%%%%%%%%%%%%%%%%%%%%%%%%%%%%%%%%%%%%%%%%%%%%%%%%%%%%%%%%%%%%%%%%%%%%%%%%%%%%%%%%%%%%%%%%%%%%%%%%%%%%%%%%%%%%%%%%%%%%%%%%%%%%%%%%%%%%%%%%%%%%%%%%%%%%%%%%%%%%%%%%%%%%%%%%%%%%%%%%%%%%%%%%%%%%%%%%%%%%%%%%%%%%%%%%%%%%%%%%%%%%%%%%%%%
\usepackage{eurosym}
\usepackage{vmargin}
\usepackage{amsmath}
\usepackage{graphics}
\usepackage{epsfig}
\usepackage{enumerate}
\usepackage{multicol}
\usepackage{subfigure}
\usepackage{fancyhdr}
\usepackage{listings}
\usepackage{framed}
\usepackage{graphicx}
\usepackage{amsmath}
\usepackage{chngpage}
%\usepackage{bigints}

\usepackage{vmargin}
% left top textwidth textheight headheight
% headsep footheight footskip
\setmargins{2.0cm}{2.5cm}{16 cm}{22cm}{0.5cm}{0cm}{1cm}{1cm}
\renewcommand{\baselinestretch}{1.3}

\setcounter{MaxMatrixCols}{10}
\begin{document}

\begin{enumerate}
    \item 
Linear combinations of normal variables remain normal.
y » N(
Xn
i=1
ai¹i;
Xn
i=1
a2
i ¾2
i )
Hence x1 + x2 » N(¹1 + ¹2; ¾2
1 + ¾2
2); x1 ¡ x2 » N(¹1 ¡ ¹2; ¾2
1 + ¾2
2)
%%%%%%%%%%%
\item Let M,S be travelling times of manager and secretary.
M » N(35; 16) and S » N(33; 9)
(i)Assuming journey times are independent ,
M ¡ S » N(2; 25) and P(M ¡ s < 0) = P(z <
0 ¡ 2
5
)
where z » N(0:; 1); this is P(z < ¡0:4) = 0:3446
3
%%%%%%%%%%%%%%%
\item If secretary leaves t minutes earlier, the difference in arrival times will be N(2+t,25),assuming
that the journey times still have the same distributions as before.
P(secretary arrives first) = P(M ¡ S + t > 0)
= 1 ¡ ©(¡1(2+t)
5 ) ¸ 0:9 if
t + 2
5
¸ 1:2826 fromN(0; 1) table i:e: t = 4:408min:
\item
P(M < 30) = Á( 30¡35
4 ) = Á(¡1:25) = 0:01056
P(S < 30) = Á( 30¡33
3 ) = Á(¡1) = 0:1587
and require probability is 0:1056 £ 0:1587 = 0:0168:
%%%%%%%%%%%%%%%%%%%%
\item  The number of breakdowns per week will follow a poisson distribution with mean
20 £ 0:02 = 0:4
\item 
P(no breakdowns) = e¡0:4
P(none in 4 weeks) = (e¡0:4)4 = e¡1:6 = 0:2019
\item 
P(1 or more in a week) = 1 ¡ e¡0:4 = 0:39297
required probability = (0:3297)4 = 0:0118
In 52 weeks, number of breakdowns in poisson with mean 0:4 £ 52 = 20:8 A normal
approximation N(20.8,20.8) may be used, and using a continuity correction we
requireP(number > 26:5)
z = 26p:5¡20:8
20:8 = 5:7
4:5607 = 1:2498
P(z > 1:2498) = P(z < ¡1:2498) = 0:1056 from tables
\end{enumerate}
\end{document}

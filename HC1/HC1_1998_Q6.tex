\documentclass[a4paper,12pt]{article}
%%%%%%%%%%%%%%%%%%%%%%%%%%%%%%%%%%%%%%%%%%%%%%%%%%%%%%%%%%%%%%%%%%%%%%%%%%%%%%%%%%%%%%%%%%%%%%%%%%%%%%%%%%%%%%%%%%%%%%%%%%%%%%%%%%%%%%%%%%%%%%%%%%%%%%%%%%%%%%%%%%%%%%%%%%%%%%%%%%%%%%%%%%%%%%%%%%%%%%%%%%%%%%%%%%%%%%%%%%%%%%%%%%%%%%%%%%%%%%%%%%%%%%%%%%%%
\usepackage{eurosym}
\usepackage{vmargin}
\usepackage{amsmath}
\usepackage{graphics}
\usepackage{epsfig}
\usepackage{enumerate}
\usepackage{multicol}
\usepackage{subfigure}
\usepackage{fancyhdr}
\usepackage{listings}
\usepackage{framed}
\usepackage{graphicx}
\usepackage{amsmath}
\usepackage{chngpage}
%\usepackage{bigints}

\usepackage{vmargin}
% left top textwidth textheight headheight
% headsep footheight footskip
\setmargins{2.0cm}{2.5cm}{16 cm}{22cm}{0.5cm}{0cm}{1cm}{1cm}
\renewcommand{\baselinestretch}{1.3}

\setcounter{MaxMatrixCols}{10}

\maketitle

\section{Introduction}
\begin{enumerate}
    \item If the residual(error) terms are i.i.d. $N(0; \sigma^2)$, then the least squares estimates are also maximum
likelihood. Y = y-y0, X = k(x-x0) transforms \[\hat{Y} = \hat{A}+ \hat{B}X\] into \[\hat{y}-y0 = \hat{A}+ \hat{B}k(x-x0)\]
or \[\hat{y} = ( \hat{A} - \hat{B}kx0) + y0 + \hat{B}kx,\] giving in the usual notation 
\[ˆa = y0 + \hat{A} - \hat{B} kx0\] and
\[ \hat{B} = \hat{B} k .\]
Since the scale of Y is not changed, the estimate of $\sigma^2$ will not be changed: s2 = S2.
%%%%%%%%%%%%%%%%5
\item
P
t=15,
P
w=588, n=6,
P
w2=71360,
P
t2=55, using t=(age-84)/7, w=weihgt-500.
P
wt=1960.

\[w -\bar{w} = \hat{B}(t - \bar{t})\] where

\begin{eqnarray*}
\hat{B} &=&
\frac{P(w-\bar{w})(t-\bar{t})}{(t-\bar{t})2}\\
&=& \frac{1960-15\frac{588}{6} }{
55-\frac{15^2}{6} }\\
&=&  \frac{490}{17:5}\\
&=&  28
\end{eqnarray*}
%%%%%%%%%%%%%%%%%%%%%%%%%%%%%%%%%%%
Hence w-98=28(t-2.5)=28t-70, or w=28t+28 .
This transforms back to (weight-500)=28
7 (age-84)+28
or weight=500+4(age)-336+28 or weight=4(age)+192 .
The fitted values and residuals are:
Age 84 91 98 105 112 119
Weight 528 556 584 612 640 668
Residual -1 -1 1 3 0 -2

\begin{itemize}
\item sum of squares of residuals =16, hence residual mean square with 4 degrees of freedom is 16/4
=4.
\item The residuals go rather systematically up and then down again, which suggests the need for
a curvilinear model, such as adding a (time)2 term, or plotting log(weight) against log(age).
\end{enumerate}
\end{enumerate}
\end{document}

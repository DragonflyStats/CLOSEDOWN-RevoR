\documentclass{article}
\usepackage[utf8]{inputenc}
\usepackage{framed}
\usepackage{enumerate}

\begin{document}

\maketitle

\section{Introduction}
\begin{enumerate}
    \item If the residual(error) terms are i.i.d. N(0; ¾2), then the least squares estimates are also maximum
likelihood. Y = y¡y0, X = k(x¡x0) transforms ˆ Y = ˆ A+ ˆBX into ˆy¡y0 = ˆ A+ ˆBk(x¡x0)
or ˆy = ( ˆ A ¡ ˆBkx0) + y0 + ˆBkx, giving in the usual notation ˆa = y0 + ˆ A ¡ ˆB kx0 andˆb = ˆB k .
Since the scale of Y is not changed, the estimate of ¾2 will not be changed: s2 = S2.
(b)
P
t=15,
P
w=588, n=6,
P
w2=71360,
P
t2=55, using t=(age-84)/7, w=weihgt-500.
P
wt=1960.
w ¡ ¯ w =ˆb(t ¡ ¯t) whereˆb =
P
P(w¡w¯)(t¡t¯)
(t¡¯t)2
= 1960¡15£588=6
55¡152=6
= 490
17:5 = 28
.
Hence w-98=28(t-2.5)=28t-70, or w=28t+28 .
This transforms back to (weight-500)=28
7 (age-84)+28
or weight=500+4(age)-336+28 or weight=4(age)+192 .
The fitted values and residuals are:
Age 84 91 98 105 112 119
Weight 528 556 584 612 640 668
Residual ¡1 ¡1 1 3 0 ¡2
sum of squares of residuals =16, hence residual mean square with 4 degrees of freedom is 16/4
=4.
The residuals go rather systematically up and then down again, which suggests the need for
a curvilinear model, such as adding a (time)2 term, or plotting log(weight) against log(age).
\end{enumerate}
\end{document}

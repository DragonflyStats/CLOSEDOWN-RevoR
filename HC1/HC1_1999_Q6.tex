\documentclass[a4paper,12pt]{article}
%%%%%%%%%%%%%%%%%%%%%%%%%%%%%%%%%%%%%%%%%%%%%%%%%%%%%%%%%%%%%%%%%%%%%%%%%%%%%%%%%%%%%%%%%%%%%%%%%%%%%%%%%%%%%%%%%%%%%%%%%%%%%%%%%%%%%%%%%%%%%%%%%%%%%%%%%%%%%%%%%%%%%%%%%%%%%%%%%%%%%%%%%%%%%%%%%%%%%%%%%%%%%%%%%%%%%%%%%%%%%%%%%%%%%%%%%%%%%%%%%%%%%%%%%%%%
\usepackage{eurosym}
\usepackage{vmargin}
\usepackage{amsmath}
\usepackage{graphics}
\usepackage{epsfig}
\usepackage{enumerate}
\usepackage{multicol}
\usepackage{subfigure}
\usepackage{fancyhdr}
\usepackage{listings}
\usepackage{framed}
\usepackage{graphicx}
\usepackage{amsmath}
\usepackage{chngpage}
%\usepackage{bigints}

\usepackage{vmargin}
% left top textwidth textheight headheight
% headsep footheight footskip
\setmargins{2.0cm}{2.5cm}{16 cm}{22cm}{0.5cm}{0cm}{1cm}{1cm}
\renewcommand{\baselinestretch}{1.3}

\setcounter{MaxMatrixCols}{10}




 

\begin{document}
\large

  \begin{table}[ht!]
     \centering
     \begin{tabular}{|p{15cm}|}
     \hline
     \large
     \smallskip
\noindent \textbf{Part a} \\ \large
An electronic device consists of two identical components which each have, independently, probability $p$ of being faulty.  


The device works unless both components are faulty.  A random sample of $n$ devices is checked. 

 
Write down the probability distribution of R, the number of devices found to work. \\
 \hline
      \end{tabular}
    \end{table}
    
    

\begin{itemize}
    \item P(device fails) = P(both components faulty) = $p^2$, by independence.
    

    \item Hence P(device works) =
$1-p^2$. 
\item Again assuming independence of performance of individual devices, R is Binomial
$(n; 1 - p^2)$
\end{itemize}

\newpage
  \begin{table}[ht!]
     \centering
     \begin{tabular}{|p{15cm}|}
     \hline
 \smallskip      \large
\noindent \textbf{Part b} \\
\large
There are two separate production lines.  Line A uses components from factory A for which $p = 1/2$ and line B uses components from factory B for which $p = 1/3$.  Calculate the probability that a randomly chosen device from the output of line A is found to be faulty, and make the corresponding calculation for line B. \\
 \hline
      \end{tabular}
    \end{table}
    
\begin{itemize}
\item For line A, $P(\mbox{device faulty}) = 1/4$.
\item For line B, $P(\mbox{faulty}) = 1/9$
\end{itemize}



%%%%%%%%%%%%%%%%%%%%%%%%%%%%%%%%%%
\newpage
  \begin{table}[ht!]
     \centering
     \begin{tabular}{|p{15cm}|}
     \hline
     \large
     \smallskip
\noindent \textbf{Part c} \\
\large
A device is chosen at random from the combined output of both production lines and is found to be faulty.  Find the probability that the faulty device came from line A, 
 
\begin{enumerate}[(i)]
\item supposing that lines A and B produce equal numbers of devices, 
\item supposing that line B produces three times as many as line A. 
\end{enumerate}
\\ \hline 
      \end{tabular}
    \end{table} 
    


%%%%%%%%%%%%%%%%%%%%%%%%%%%%%%


\begin{itemize}
\item  Let event F denote failure/faulty.
\item   $P(F|A) = 1/4$; $P(F|B) = 1/9$; from before.

\item Scenario (i) : ${  \displaystyle P(A) = P(B) = \frac{1}{2}  }$; 
\[P(A|F) = \frac{P(F|A) \times P(A)}{P(F)}\] 
\begin{eqnarray*}
P(F) &=&  P(F|A)\times P(A) \;+\; P(F|B) \times P(B) \\ 
& & \\
&=& \left[\frac{1}{4} \times \frac{1}{2} \right] + \left[\frac{1}{9} \times \frac{1}{2} \right] \\
& & \\
&=& \frac{1}{8} + \frac{1}{18}\\
& & \\
&=& \frac{13}{72}
\end{eqnarray*}

\begin{eqnarray*}
P(A|F) &=& \frac{P(F|A)P(A)}{P(F)} \\ 
& & \\
&=& \frac{\frac{1}{4} \times \frac{1}{2}}{\frac{13}{72}}\\
& & \\
&=& \frac{\frac{9}{72} }{\frac{13}{72}}\\
& & \\
&=& \frac{9}{13} 
\end{eqnarray*}
\bigskip 
\item Scenario (ii) :  If $P(A) = 1/4$; $P(B) = 3/4$;

\begin{eqnarray*}
P(F) &=&  P(F|A)\times P(A) \;+\; P(F|B) \times P(B) \\ 
& & \\
&=& \left[\frac{1}{4} \times \frac{1}{4} \right] + \left[\frac{1}{9} \times \frac{3}{4} \right] \\
& & \\
&=& \frac{1}{16} + \frac{1}{12}\\
& & \\
&=& \frac{7}{48}
\end{eqnarray*}


\begin{eqnarray*}
P(A|F) &=& \frac{P(F|A)P(A)}{P(F)} \\ 
& & \\
&=& \frac{\frac{1}{4} \times \frac{1}{4}}{\frac{7}{48}}\\
& & \\
&=& \frac{\frac{3}{48} }{\frac{7}{48}}\\
& & \\
&=& \frac{ 3 }{7}\\
\end{eqnarray*}

\end{itemize}

%%%%%%%%%%%%%%%%%%%%%%%%%%%%%%%%%%%%%%%%%%%%%%%%
\newpage
 \begin{table}[ht!]
     \centering
     \begin{tabular}{|p{15cm}|}
     \hline
     \large
     \smallskip
\noindent \textbf{Part d} \\
\large
 
  Comment on the difference between the answers to scenarios (i) and (ii).\\ 
 \hline
      \end{tabular}
    \end{table}

\noindent Although individually devices from A are more likely to fail, in scenario (ii) more devices
are made on line B which results in a higher probability for scenario /(ii).
\end{document}

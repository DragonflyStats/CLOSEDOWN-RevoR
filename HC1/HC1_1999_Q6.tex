\documentclass[a4paper,12pt]{article}
%%%%%%%%%%%%%%%%%%%%%%%%%%%%%%%%%%%%%%%%%%%%%%%%%%%%%%%%%%%%%%%%%%%%%%%%%%%%%%%%%%%%%%%%%%%%%%%%%%%%%%%%%%%%%%%%%%%%%%%%%%%%%%%%%%%%%%%%%%%%%%%%%%%%%%%%%%%%%%%%%%%%%%%%%%%%%%%%%%%%%%%%%%%%%%%%%%%%%%%%%%%%%%%%%%%%%%%%%%%%%%%%%%%%%%%%%%%%%%%%%%%%%%%%%%%%
\usepackage{eurosym}
\usepackage{vmargin}
\usepackage{amsmath}
\usepackage{graphics}
\usepackage{epsfig}
\usepackage{enumerate}
\usepackage{multicol}
\usepackage{subfigure}
\usepackage{fancyhdr}
\usepackage{listings}
\usepackage{framed}
\usepackage{graphicx}
\usepackage{amsmath}
\usepackage{chngpage}
%\usepackage{bigints}

\usepackage{vmargin}
% left top textwidth textheight headheight
% headsep footheight footskip
\setmargins{2.0cm}{2.5cm}{16 cm}{22cm}{0.5cm}{0cm}{1cm}{1cm}
\renewcommand{\baselinestretch}{1.3}

\setcounter{MaxMatrixCols}{10}




 

\begin{document}
  \begin{table}[ht!]
     \centering
     \begin{tabular}{|p{15cm}|}
     \hline
6. An electronic device consists of two identical components which each have, independently, probability p of being faulty.  The device works unless both components are faulty.  A random sample of n devices is checked. 
 
(i) Write down the probability distribution of R, the number of devices found to work. (4) 
 
 
 
 
 
      \end{tabular}
    \end{table}
    
    
  \begin{table}[ht!]
     \centering
     \begin{tabular}{|p{15cm}|}
     \hline
(ii) There are two separate production lines.  Line A uses components from factory A for which p = ½ and line B uses components from factory B for which p = 1/3.  Calculate the probability that a randomly chosen device from the output of line A is found to be faulty, and make the corresponding calculation for line B. 


      \end{tabular}
    \end{table}

\begin{enumerate}
    \item P(device fails) = P(both components faulty) = p2, by independence.
    
\begin{itemize}
    \item Hence P(device works) =
1-p2. 
\item Again assuming independence of performance of individual devices, R is Binomial
(n; 1 - p2)
\end{itemize}
%%%%%%%%%%%%%%%%%%%%%%%%%%%%%%
\item For line A, $P(\mbox{device faulty}) = 1/4$; and for B; $P(\mbox{faulty}) = 1/9$
%%%%%%%%%%%%%%%%%%%%%%%%%%%%%%%%%%
\item  Let event F denote failure/faulty.
  \begin{table}[ht!]
     \centering
     \begin{tabular}{|p{15cm}|}
     \hline
(iii) A device is chosen at random from the combined output of both production lines and is found to be faulty.  Find the probability that the faulty device came from line A, 
 
(a) supposing that lines A and B produce equal numbers of devices, 
\\ \hline 
      \end{tabular}
    \end{table} 
$P(F|A) = 1/4$; $P(F|B) = 1/9$; from (ii)

(a) If $P(A) = P(B) = 1=2$; 
\[P(A|F) = \frac{P(F|A)P(A)}{P(F)}\] 
\begin{eqnarray*}
P(F) &=&  P(F|A)P(A)+P(F|B)P(B) \\ 
&=& \left[\frac{1}{4} \times \frac{1}{2} \right] + \left[\frac{1}{9} \times \frac{1}{2} \right] 
&=& 9/13
\end{eqnarray*}

\begin{eqnarray*}
P(A|F) = \frac{P(F|A)P(A)}{P(F)} \\ 
&=& \frac{\frac{1}{4} \times \frac{1}{2}}{P(F)}
&=& 9/13
\end{eqnarray*}
 \begin{table}[ht!]
     \centering
     \begin{tabular}{|p{15cm}|}
     \hline
  (b) supposing that line B produces three times as many as line A. 
(4) 
 
  Comment on the difference between your answers to (a) and (b). 
(4) 
      \end{tabular}
    \end{table}
\item  If $P(A) = 1/4$; $P(B) = 3/4$; $P(A|F) = 3/7$
Although individually devices from A are more likely to fail, in case(b) more devices
are made on line B which results in a higher probability for(b).
\end{enumerate}
\end{document}

\documentclass[a4paper,12pt]{article}
%%%%%%%%%%%%%%%%%%%%%%%%%%%%%%%%%%%%%%%%%%%%%%%%%%%%%%%%%%%%%%%%%%%%%%%%%%%%%%%%%%%%%%%%%%%%%%%%%%%%%%%%%%%%%%%%%%%%%%%%%%%%%%%%%%%%%%%%%%%%%%%%%%%%%%%%%%%%%%%%%%%%%%%%%%%%%%%%%%%%%%%%%%%%%%%%%%%%%%%%%%%%%%%%%%%%%%%%%%%%%%%%%%%%%%%%%%%%%%%%%%%%%%%%%%%%
\usepackage{eurosym}
\usepackage{vmargin}
\usepackage{amsmath}
\usepackage{graphics}
\usepackage{epsfig}
\usepackage{enumerate}
\usepackage{multicol}
\usepackage{subfigure}
\usepackage{fancyhdr}
\usepackage{listings}
\usepackage{framed}
\usepackage{graphicx}
\usepackage{amsmath}
\usepackage{chngpage}
%\usepackage{bigints}

\usepackage{vmargin}
% left top textwidth textheight headheight
% headsep footheight footskip
\setmargins{2.0cm}{2.5cm}{16 cm}{22cm}{0.5cm}{0cm}{1cm}{1cm}
\renewcommand{\baselinestretch}{1.3}

\setcounter{MaxMatrixCols}{10}

\begin{document}
\begin{enumerate}
    \item P(device fails) = P(both components faulty) = p2, by independence. Hence P(device works) =
1¡p2. Again assuming independence of performance of individual devices, R is Binomial
(n; 1 ¡ p2)
%%%%%%%%%%%%%%%%%%%%%%%%%%%%%%
\item For line A, P(device faulty) = 1=4; and for B; P(faulty) = 1=9
%%%%%%%%%%%%%%%%%%%%%%%%%%%%%%%%%%
\item  Let event F denote failure/faulty.
$P(F=A) = 1/4$; $P(F=B) = 1/9$; from (ii)

(a) IfP(A) = P(B) = 1=2; P(AjF) = P(FjA)P(A)
P(F) = P(FjA)P(A)
P(FjA)P(A)+P(FjB)P(B) = 9=13
\item  IfP(A) = 1=4; P(B) = 3=4; P(AjF) = 3=7
Although individually devices from A are more likely to fail, in case(b) more devices
are made on line B which results in a higher probability for(b).
\end{enumerate}
\end{document}

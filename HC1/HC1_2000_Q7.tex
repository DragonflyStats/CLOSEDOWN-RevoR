\documentclass[a4paper,12pt]{article}
%%%%%%%%%%%%%%%%%%%%%%%%%%%%%%%%%%%%%%%%%%%%%%%%%%%%%%%%%%%%%%%%%%%%%%%%%%%%%%%%%%%%%%%%%%%%%%%%%%%%%%%%%%%%%%%%%%%%%%%%%%%%%%%%%%%%%%%%%%%%%%%%%%%%%%%%%%%%%%%%%%%%%%%%%%%%%%%%%%%%%%%%%%%%%%%%%%%%%%%%%%%%%%%%%%%%%%%%%%%%%%%%%%%%%%%%%%%%%%%%%%%%%%%%%%%%
\usepackage{eurosym}
\usepackage{vmargin}
\usepackage{amsmath}
\usepackage{graphics}
\usepackage{epsfig}
\usepackage{enumerate}
\usepackage{multicol}
\usepackage{subfigure}
\usepackage{fancyhdr}
\usepackage{listings}
\usepackage{framed}
\usepackage{graphicx}
\usepackage{amsmath}
\usepackage{chngpage}
%\usepackage{bigints}

\usepackage{vmargin}
% left top textwidth textheight headheight
% headsep footheight footskip
\setmargins{2.0cm}{2.5cm}{16 cm}{22cm}{0.5cm}{0cm}{1cm}{1cm}
\renewcommand{\baselinestretch}{1.3}

\setcounter{MaxMatrixCols}{10}
\begin{document}

\begin{framed}
The random variable X follows the discrete uniform distribution on the integers 1, 2, ..., k so that the probability mass function of X is given by 
 
 
1
1,2,...,
()
0 otherwise
xk
kpx  = =   
 
\end{framed}

\begin{framed}
(i) Show that for a general positive integer k 
 
 () 1 2 k EX + = . 
(
\end{framed}
\begin{enumerate}
    \item E[x] =
Pk
i=1
i
k = 1
k
1
2k(k + 1) = k+1
2


\begin{framed}
A random sample of size four, X1, X2, X3, X4, is taken from this distribution, yielding values x1, x2, x3, x4, from which it is intended to estimate the parameter k. 
 
(a) Show that the method of moments estimator of k is given by 
1 ˆ 21 kX =−, where X denotes the sample mean. 
\end{framed}
%%%%%%%%%%%%
\item The moment estimator $\hat{k}_1$ is found from setting $E[x] = \bar{x}$ i.e. $\bar{x}$ = 1
2 ( $\hat{k}_1$ +
1) or \[\hat{k}_1 = 2\bar{x} - 1\]
%%%%%%%%%%%%
\begin{framed}

Explain clearly why the maximum likelihood estimator of k is given by 2 (4) ˆ kX = , the sample maximum. 


\end{framed}
\item Likelihood =
Q4
i=1
f(xi) = 1
k4 provided all xi lie between 1 and k inclusive.
The maximum of this occurs when $\hat{k}_2$ is chosen to be as small as possible given the data
values; i.e. $\hat{k}_2 = x(4)$, the sample maximum.

%%%%%%%%%%%%%%%%%%%%%%%%%%%%%%%%%%%%
\end{framed}
Calculate 1 ˆ k and 2 ˆ k for the sample (1, 10, 3, 2) and comment on your results. (
\begin{framed}
\item Forfxig = f1; 10; 3; 2g; $\hat{k}_1 = 2(16
4 )$ ¡ 1 = 7 $\hat{k}_2 = x(4) = 10$
$\hat{k}_1$ is impossible, since there is a data value above 7. $\hat{k}_2$ is consistent with the data.
\end{enumerate}

\end{document}

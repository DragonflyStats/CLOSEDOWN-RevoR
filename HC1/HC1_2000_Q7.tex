\documentclass[a4paper,12pt]{article}
%%%%%%%%%%%%%%%%%%%%%%%%%%%%%%%%%%%%%%%%%%%%%%%%%%%%%%%%%%%%%%%%%%%%%%%%%%%%%%%%%%%%%%%%%%%%%%%%%%%%%%%%%%%%%%%%%%%%%%%%%%%%%%%%%%%%%%%%%%%%%%%%%%%%%%%%%%%%%%%%%%%%%%%%%%%%%%%%%%%%%%%%%%%%%%%%%%%%%%%%%%%%%%%%%%%%%%%%%%%%%%%%%%%%%%%%%%%%%%%%%%%%%%%%%%%%
\usepackage{eurosym}
\usepackage{vmargin}
\usepackage{amsmath}
\usepackage{graphics}
\usepackage{epsfig}
\usepackage{enumerate}
\usepackage{multicol}
\usepackage{subfigure}
\usepackage{fancyhdr}
\usepackage{listings}
\usepackage{framed}
\usepackage{graphicx}
\usepackage{amsmath}
\usepackage{chngpage}
%\usepackage{bigints}

\usepackage{vmargin}
% left top textwidth textheight headheight
% headsep footheight footskip
\setmargins{2.0cm}{2.5cm}{16 cm}{22cm}{0.5cm}{0cm}{1cm}{1cm}
\renewcommand{\baselinestretch}{1.3}

\setcounter{MaxMatrixCols}{10}
\begin{document}7(i)E[x] =
Pk
i=1
i
k = 1
k
1
2k(k + 1) = k+1
2
(ii)(a)The moment estimator ˆ k1 is found from setting E[x] = ¯x i.e. ¯x = 1
2 ( ˆ k1 +
1) or ˆ k1 = 2¯x ¡ 1
(b)Likelihood =
Q4
i=1
f(xi) = 1
k4 provided all xi lie between 1 and k inclusive.
The maximum of this occurs when ˆ k2 is chosen to be as small as possible given the data
values; i.e. ˆ k2 = x(4), the sample maximum.
(c)Forfxig = f1; 10; 3; 2g; ˆ k1 = 2(16
4 ) ¡ 1 = 7 ˆ k2 = x(4) = 10
ˆ k1 is impossible, since there is a data value above 7. ˆ k2 is consistent with the data.

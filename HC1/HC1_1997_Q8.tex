\documentclass[a4paper,12pt]{article}
%%%%%%%%%%%%%%%%%%%%%%%%%%%%%%%%%%%%%%%%%%%%%%%%%%%%%%%%%%%%%%%%%%%%%%%%%%%%%%%%%%%%%%%%%%%%%%%%%%%%%%%%%%%%%%%%%%%%%%%%%%%%%%%%%%%%%%%%%%%%%%%%%%%%%%%%%%%%%%%%%%%%%%%%%%%%%%%%%%%%%%%%%%%%%%%%%%%%%%%%%%%%%%%%%%%%%%%%%%%%%%%%%%%%%%%%%%%%%%%%%%%%%%%%%%%%
\usepackage{eurosym}
\usepackage{vmargin}
\usepackage{amsmath}
\usepackage{graphics}
\usepackage{epsfig}
\usepackage{enumerate}
\usepackage{multicol}
\usepackage{subfigure}
\usepackage{fancyhdr}
\usepackage{listings}
\usepackage{framed}
\usepackage{graphicx}
\usepackage{amsmath}
\usepackage{chngpage}
%\usepackage{bigints}

\usepackage{vmargin}
% left top textwidth textheight headheight
% headsep footheight footskip
\setmargins{2.0cm}{2.5cm}{16 cm}{22cm}{0.5cm}{0cm}{1cm}{1cm}
\renewcommand{\baselinestretch}{1.3}

\setcounter{MaxMatrixCols}{10}
\begin{document}


\section{Introduction}
\begin{enumerate}
\item A binomial distribution with large n and very small p may be approximated
by a Poisson with $\lambda = np$. 
\begin{itemize}
\item It is desirable that $np$ should be at least 5, but in
addition n should be greater than 20 and p less than 0.1.

\item When all these conditions are met
the approximation will be a good one.
\end{itemize}

\item  A Poisson with large mean can be approximated by $N(\mu, \mu)$. The
approximation will be good for $\mu \geq 10$, but adequate down to $\mu = 5$.

%%%%%%%%%%%%%%%%5
(b) (i) Poisson, mean 5:
\begin{eqnarray*}
P(4) + P(5) + P(6) &=& 
\left[e^{-5} \times \frac{5^4}{4!}\right] + \left[e^{-5}\times \frac{5^5}{5!}\right] +  \left[ e^{-5}\times \frac{5^6}{6!} \right] 
\\
&=& 
e^{-5} \left( \frac{5^4}{4!} + \frac{5^5}{5!} +  \frac{5^6}{6!} \right)
\\
&=& 
e^{-5} \times \frac{5^4}{4!} \times \left( 1 + \frac{5}{5} +  \frac{5^2}{6 \times 5} \right)
\\
&=& 
e^{-5} \times \frac{625}{240} \times \left(\frac{30}{30} + \frac{30}{30} +  \frac{25}{30} \right)
\\
&=& 
e^{-5} \times \frac{625}{240} \times \left(   \frac{85}{30} \right)
\\
&=& 625e¡5( 1
24 + 5
120 + 25
720 )\\ &=& 625e¡5( 1
12 + 5
144 )\\ &=& 0:49716:
\end{eqnarray*}
%%%%%%%%%%%%%%%%%%%%%%%%%
\item  N(5; 5) with a continuity correction is required: find P(31
2 < X < 61
2 ) in
N(5; 5). Corresponding r-values are 3 1
2
¡5
p
5
= ¡0:6708 and 6 1
2
¡5
p
5
= +0:6708.
\[P(z < -0:6708) = P(z > +0.6708) = 0:25117,\] and so the required probability
is \[1 - (2 \times 0.25117) = 0.49766.\] The error is 0.0005, and % error
0:0005
0:49716 £ 100 = 0:1%:
Using the continuity correction with ¹ = 5, and calculating values which we
near to the mean, leads to a very good approximation.
\item  $P(0) = e^{-\lambda t}¸ = P(T > t)$ for the first event observed = $1 - F(t)$. Hence
$F(t) = 1 - e^{-\lambda t}$ and g(t) = F
0(t) = ¸e¡¸t. (t ¸ 0; ¸ > 0).
[g(t) = 0 unless t ¸ 0; ¸ > 0 since neither time of events nor rate of events
occurring can be negative.] 


Use integration by parts.


\begin{eqnarray*}
E(T) &=& \int^{\infty}_{0} \lambda t\;e^{-\lambda} dt \\
    &=&\int^{\infty}_{0} \lambda t\;e^{-\lambda} dt \\
    &=& \left[ -\frac{1}{\lambda}e^{-\lambda t} \right]^{\infty}_{0}\\
    &=& \frac{1}{\lambda}  
\end{eqnarray*}

\begin{eqnarray*}
E(T^2) &=& \int^{\infty}_{0} \lambda t^2\;e^{-\lambda} dt \\
    &=&\int^{\infty}_{0} \lambda t^2\;e^{-\lambda} dt \\
    &=& \left[ -\frac{1}{\lambda}e^{-\lambda t} \right]^{\infty}_{0}\\
    &=& \frac{2}{\lambda}E(T)\\
    &=& \frac{2}{\lambda^2}  
\end{eqnarray*}

\[ Var(X) = [E(X^2)] - [E(X)^2] = \frac{2}{\lambda^2} - \frac{1}{\lambda^2} 
     = \frac{1}{\lambda^2} \]
%%%%%%%%%%%%%%%%%%%5
\begin{itemize}
\item $\lambda = 5$, so that $E[T] = 0.2$ and $V[T] = 0.04$. 

\item For n = 100, a sample mean
$\bar{T}$ is approximately $N(0.2; \frac{0.04}{100})$. The range required is from 0.18 to 0.22,
within 10\% of 0.2. 
\item The corresponding values of r are $\frac{0.18- 0.20}{\sqrt{0.0004}} = -\frac{0.002}{0.002} = -1$,
and the other = +1. 
\item $P(r > 1) = 0.1587 = P(r < -1)$ and so the probability
between these values is $1 - (2 \times 0.1587) = 0.6826$.
\end{itemize}
%%%%%%%%%%%%%%%%%%%5
\end{enumerate}
\end{document}

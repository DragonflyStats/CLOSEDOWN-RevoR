\documentclass[a4paper,12pt]{article}
%%%%%%%%%%%%%%%%%%%%%%%%%%%%%%%%%%%%%%%%%%%%%%%%%%%%%%%%%%%%%%%%%%%%%%%%%%%%%%%%%%%%%%%%%%%%%%%%%%%%%%%%%%%%%%%%%%%%%%%%%%%%%%%%%%%%%%%%%%%%%%%%%%%%%%%%%%%%%%%%%%%%%%%%%%%%%%%%%%%%%%%%%%%%%%%%%%%%%%%%%%%%%%%%%%%%%%%%%%%%%%%%%%%%%%%%%%%%%%%%%%%%%%%%%%%%
\usepackage{eurosym}
\usepackage{vmargin}
\usepackage{amsmath}
\usepackage{graphics}
\usepackage{epsfig}
\usepackage{enumerate}
\usepackage{multicol}
\usepackage{subfigure}
\usepackage{fancyhdr}
\usepackage{listings}
\usepackage{framed}
\usepackage{graphicx}
\usepackage{amsmath}
\usepackage{chngpage}
%\usepackage{bigints}

\usepackage{vmargin}
% left top textwidth textheight headheight
% headsep footheight footskip
\setmargins{2.0cm}{2.5cm}{16 cm}{22cm}{0.5cm}{0cm}{1cm}{1cm}
\renewcommand{\baselinestretch}{1.3}

\setcounter{MaxMatrixCols}{10}

\begin{document}

%%%%%%%%%%%%%%%%
 \begin{table}[ht!]
     \centering
     \begin{tabular}{|p{15cm}|}
     \hline        
 \noindent \textbf{Part (a)}\\
\noindent
The random variable X follows the Geometric distribution such that
\[{\displaystyle p(X = x) ={\begin{cases}q^{x-1}p,&x = 1,2,3, ...,\\0&\mbox{otherwise.}\end{cases}}} 
\]

where $0 < p <1$ and $q = 1 - p$.  Sketch the probability function of this distribution and, by using the result 
\[1+2q + 3q^2 + 4q^3 + ... =  \frac{1}{1-q^2} ,\] or otherwise, find E(X).

\\ \hline
 \end{tabular}
\end{table}
%%%%%%%%%%%%%%%%

\begin{enumerate}
    \item Expected Value


\begin{eqnarray*}
E[X] &=&
\sum_{x=1}^{\infty } x \cdot f(x)   
\\&=&
\sum_{x=1}^{\infty } x \cdot(q)^{k-1}p\cdot   
\\&=&
p \times \sum_{x=1}^{\infty } x \cdot(q)^{k-1}\cdot   
\\&=&
\sum_{x=1}^{\infty }(q)^{k-1}p\cdot k \\&=& p(1 + 2q + 3q^2 + \ldots)
\\&=& \frac{p}{(1 -  q)^2} \\&=& \frac{p}{p^2} \\&=& \frac{1}{p} 
\end{eqnarray*}
 \begin{table}[ht!]
     \centering
     \begin{tabular}{|p{15cm}|}
     \hline        
 \noindent \textbf{Part (b)}\\
\noindent Obtain the cumulative distribution function of X and deduce that the median of X is the
smallest integer not less than − 
qn l 2ln
 .

\\ \hline
 \end{tabular}
\end{table}

\begin{eqnarray*}
F(x)&=&P(X \leq x)\\&=&
Px
n=1
pqn- 1 \\&=&  p1- qx
1- q \\&=& 1 -  qx (x = 1; 2; ¢ ¢ ¢) .
\end{eqnarray*}
(Also F(x)=0 for x<1)
\begin{itemize}
    \item Strictly, F is a step function, changing value for each integer value of x and holding the value
1-q[x] until the next change.
\item ([x] is the integral part of x. )
\item For the median M, F(M)=1
2 .

\item Hence 1-qx = 1
2 or qx = 1
2 so that x ln q = - ln 2 or x = - ln 2= ln q. 
\item M is the smallest integer not
less than this.
\end{itemize}

%%%%%%%%%%%%%%%%%%%%%%%5
\begin{table}[ht!]
     \centering
     \begin{tabular}{|p{15cm}|}
     \hline        
 \noindent \textbf{Part (c)}\\
\noindent The random variable Y has the same distribution as X and is independent of X. Show that
\[P(Y = X) =  \frac{p}{1+q}\]


\\ \hline
 \end{tabular}
\end{table}
%%%%%%%%%%%%%%%%%%%%%%%5

\begin{eqnarray*}
P(Y=X)  
&=&
\sum_{x=1}^{\infty }
p2q2^{(x- 1)} \\&=& p2
1 \sum_{x=1}^{\infty }
q2^{(x- 1)} \\
\\ &=& \frac{p^2}{1 -  q^2} 
\\ &=& \frac{p^2}{(1- q)(1+q)} \\ &=& \frac{p}{1+q}
\end{eqnarray*}
\newpage
\[{\displaystyle {\begin{aligned}\mathrm {E} (Y)&{}=\sum _{k=0}^{\infty }(1-p)^{k}p\cdot k\\&{}=p\sum _{k=0}^{\infty }(1-p)^{k}k\\&{}=p(1-p)\sum _{k=0}^{\infty }(1-p)^{k-1}\cdot k\\&{}=p(1-p)\left[{\frac {d}{dp}}\left(-\sum _{k=0}^{\infty }(1-p)^{k}\right)\right]\\&{}=p(1-p){\frac {d}{dp}}\left(-{\frac {1}{p}}\right)={\frac {1-p}{p}}.\end{aligned}}} 
\]
\end{enumerate}
\end{document}

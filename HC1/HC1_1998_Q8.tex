\documentclass{article}
\usepackage[utf8]{inputenc}

\title{RSS_Jan_2019_HC2}
\author{kobriendublin }
\date{December 2018}

\begin{document}


\section{Introduction}

8.
E[X]=
1P
x=1
xP(X = x) =
1P
x=1
xqx¡1p = p(1 + 2q + 3q2 + ¢ ¢ ¢)
= p=(1 ¡ q)2 = p=p2 = 1=p .
F(x)=P(X · x) =
Px
n=1
pqn¡1 = p1¡qx
1¡q = 1 ¡ qx (x = 1; 2; ¢ ¢ ¢) .
(Also F(x)=0 for x<1)
Strictly, F is a step function, changing value for each integer value of x and holding the value
1-q[x] until the next change. ([x] is the integral part of x. )
For the median M, F(M)=1
2 .
Hence 1-qx = 1
2 or qx = 1
2 so that x ln q = ¡ln 2 or x = ¡ln 2= ln q. M is the smallest integer not
less than this.
P(Y=X)=
1P
x=1
p2q2(x¡1) = p2
1P
x=1
q2(x¡1) = p2=(1 ¡ q2) = p2
(1¡q)(1+q) = p
1+q
———————————————————————————————————————-
\end{document}

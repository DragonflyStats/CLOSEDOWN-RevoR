\documentclass[a4paper,12pt]{article}
%%%%%%%%%%%%%%%%%%%%%%%%%%%%%%%%%%%%%%%%%%%%%%%%%%%%%%%%%%%%%%%%%%%%%%%%%%%%%%%%%%%%%%%%%%%%%%%%%%%%%%%%%%%%%%%%%%%%%%%%%%%%%%%%%%%%%%%%%%%%%%%%%%%%%%%%%%%%%%%%%%%%%%%%%%%%%%%%%%%%%%%%%%%%%%%%%%%%%%%%%%%%%%%%%%%%%%%%%%%%%%%%%%%%%%%%%%%%%%%%%%%%%%%%%%%%
\usepackage{eurosym}
\usepackage{vmargin}
\usepackage{amsmath}
\usepackage{graphics}
\usepackage{epsfig}
\usepackage{enumerate}
\usepackage{multicol}
\usepackage{subfigure}
\usepackage{fancyhdr}
\usepackage{listings}
\usepackage{framed}
\usepackage{graphicx}
\usepackage{amsmath}
\usepackage{chngpage}
%\usepackage{bigints}

\usepackage{vmargin}
% left top textwidth textheight headheight
% headsep footheight footskip
\setmargins{2.0cm}{2.5cm}{16 cm}{22cm}{0.5cm}{0cm}{1cm}{1cm}
\renewcommand{\baselinestretch}{1.3}

\setcounter{MaxMatrixCols}{10}

\begin{document}
\begin{enumerate}
    \item Expected Value
\begin{eqnarray*}
E[X] &=&
1P
x=1
xP(X = x) \\&=&
1P
x=1
xqx- 1p \\&=& p(1 + 2q + 3q2 + ¢ ¢ ¢)
\\&=& \frac{p}{(1 -  q)^2} = \frac{p}{p2} = \frac{1}{p} 
\end{eqnarray*}

\begin{eqnarray*}
F(x)&=&P(X \leq x)\\&=&
Px
n=1
pqn- 1 \\&=&  p1- qx
1- q \\&=& 1 -  qx (x = 1; 2; ¢ ¢ ¢) .
\end{eqnarray*}
(Also F(x)=0 for x<1)
\begin{itemize}
    \item Strictly, F is a step function, changing value for each integer value of x and holding the value
1-q[x] until the next change.
\item ([x] is the integral part of x. )
\item For the median M, F(M)=1
2 .
\item Hence 1-qx = 1
2 or qx = 1
2 so that x ln q = - ln 2 or x = - ln 2= ln q. 
\item M is the smallest integer not
less than this.
\end{itemize}

\begin{eqnarray*}
P(Y=X)  
&=&
1P
x=1
p2q2(x- 1) \\&=& p2
1P
x=1
q2(x- 1) \\&=& p2=(1 -  q2) = p2
(1- q)(1+q) \\ &=& p
1+q
\end{eqnarray*}

\end{enumerate}
\end{document}

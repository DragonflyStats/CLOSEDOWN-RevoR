\documentclass[a4paper,12pt]{article}
%%%%%%%%%%%%%%%%%%%%%%%%%%%%%%%%%%%%%%%%%%%%%%%%%%%%%%%%%%%%%%%%%%%%%%%%%%%%%%%%%%%%%%%%%%%%%%%%%%%%%%%%%%%%%%%%%%%%%%%%%%%%%%%%%%%%%%%%%%%%%%%%%%%%%%%%%%%%%%%%%%%%%%%%%%%%%%%%%%%%%%%%%%%%%%%%%%%%%%%%%%%%%%%%%%%%%%%%%%%%%%%%%%%%%%%%%%%%%%%%%%%%%%%%%%%%
\usepackage{eurosym}
\usepackage{vmargin}
\usepackage{amsmath}
\usepackage{graphics}
\usepackage{epsfig}
\usepackage{enumerate}
\usepackage{multicol}
\usepackage{subfigure}
\usepackage{fancyhdr}
\usepackage{listings}
\usepackage{framed}
\usepackage{graphicx}
\usepackage{amsmath}
\usepackage{chngpage}
%\usepackage{bigints}

\usepackage{vmargin}
% left top textwidth textheight headheight
% headsep footheight footskip
\setmargins{2.0cm}{2.5cm}{16 cm}{22cm}{0.5cm}{0cm}{1cm}{1cm}
\renewcommand{\baselinestretch}{1.3}

\setcounter{MaxMatrixCols}{10}

\begin{document}
\begin{enumerate}
\item  It is assumed that fei : i = 1 to ngare independently normally distributed with constant
variance ¾2, and mean 0.
@S
@¯0
= ¡2(
$\sum^{n}_{i=1}$
i=1(yi ¡¯0 ¡¯1$x_{1i}$ ¡¯2$x_{2i}$) = 0 ie.
$\sum^{n}_{i=1}$
i=1 yi = n¯0 +¯1

\begin{itemize}
    \item P
i $x_{1i}$ +¯2
P
i $x_{2i}$, or
$\bar{y}$ = $\hat{\beta}_0$ + $\hat{\beta}_1$\bar{x}_1 + ˆ ¯2¯x2 which gives $\hat{\beta}_0$ = $\bar{y}$ ¡ $\hat{\beta}_1$\bar{x}_1 ¡ ˆ ¯2¯x2
@S
@¯1
= ¡2
$\sum^{n}_{i}$
=1 $x_{1i}$(yi ¡ ¯0 ¡ ¯i$x_{1i}$ ¡ ¯2$x_{2i}$) = 0 when
$\sum^{n}_{i=1}$
i=1 $x_{1i}$yi =
P
$x_{1i}$ $\hat{\beta}_0$ + $\hat{\beta}_1$

\item P
i x2
1i + ˆ ¯2

\item P
i $x_{1i}$$x_{2i}$ =
P
i $x_{1i}$($\bar{y}$ ¡ $\hat{\beta}_1$\bar{x}_1 ¡ ˆ ¯2¯x2) + $\hat{\beta}_1$
P
i x2
1i +
ˆ ¯2
P
i $x_{1i}$$x_{2i}$:
i.e.
$\sum^{n}_{i=1}$
1 $x_{1i}$(yi ¡ $\bar{y}$) = $\hat{\beta}_1$(
P
i $x_{1i}$
2 - \bar{x}_1
P
i $x_{1i}$) + ˆ ¯2(
\item P
i $x_{1i}$$x_{2i}$ ¡ ¯x2

\item P
$x_{1i}$) = $\hat{\beta}_1$(
P
x2
1i ¡
f
\item P
$x_{1i}$g2
n ) + ˆ ¯2(
P
$x_{1i}$$x_{2i}$ ¡ f
\end{itemize}

P
$x_{1i}$gf
P
$x_{2i}$g
n ) 
\begin{itemize}
\item Now
P
i(yi ¡ $\bar{y})\bar{x}_1 = 0$, Since $\bar{x}_1$ is constant
over the summation and
P
$yi - $\bar{y}\bar{x}_1 = 0$ by definition of $\bar{y}$.

\item Also
P
i x2
i ¡(
P
xi)2
n =
P
(xi¡¯x)2; 
\item So this may be written
\begin{itemize}
\item $\sum^{n}_{i}$
($x_{1i}- \bar{x}_1$)($yi\bar{y}$) = beˆtai
\item $\sum^{n}_{i=1}(x_{1i}$- 
$$\bar{x}_1)2 + ˆ ¯2$
\item $\sum^{n}_{i=1}(x_{1i}$ ¡\bar{x}_1)($x_{2i}$ ¡ ¯x2)
\end{itemize}
\item In exactly the same way, n
i=1($x_{2i}$ ¡ ¯x2)(yi ¡ $\bar{y}$) = $\hat{\beta}_1$
$\sum^{n}_{i=1}$
i=1($x_{1i}$ ¡\bar{x}_1)($x_{2i}$ ¡ ¯x2) + ˆ ¯2
$\sum^{n}_{i=1}$
i=1($x_{2i}$ ¡ ¯xx)2:
5
\end{enumerate}
\item  A simple notation for the equations is S1Y = S11 $\hat{\beta}_1$ + S12 ˆ ¯2 ; S2Y = S12 $\hat{\beta}_1$ + S22 ˆ ¯2

\begin{itemize}
\item S11 = 28028 ¡ (8 £ 59)2=8 = 180 ; 
\item S22 = 29700 ¡ (8 £ 60)2=8 = 900 ; 
\item S12 = 28680 ¡ 8 £
59 £ 60 = 360 ; 
\item S1Y = 26860 ¡ 8 £ 55 £ 59 = 900 ; 
\item S2Y = 28560 ¡ 8 £ 55 £ 60 = 2160:
\end{itemize}
Therefore 900 = 180 $\hat{\beta}_1$ + 360 ˆ ¯2 2160 = 360 $\hat{\beta}_1$ + 900 ˆ ¯2:
These reduce to 5 = $\hat{\beta}_1$ + 2 ˆ ¯2; giving $\hat{\beta}_1$ = 1; ˆ ¯2 = 2 and 12 = 2 $\hat{\beta}_1$ + 5 ˆ ¯2 $\hat{\beta}_0$ =
55 ¡ 59 ¡ (2 £ 60) = ¡124:

\begin{itemize}
\item The multiple P regression equation is y = ¡124 + x1 + 2x2, The total sum of squares
(yi ¡ $\bar{y}$)2 = 29600 ¡ 8 £ 552 = 5400 The residual (error) s.s. will have 8 ¡ 3 = 5d.f.;
hence its value is 5 £ 36 = 180
\item The regression s.s.(2df) is 5400 ¡ 180 = 520
F(2;5) = 5220=2
36 = 72:5(0:1% level);very highly significant and so the model is a satisfactory
fit to the data. ˆ¯1
SE( ˆ¯1)
= p1
1
; n:s: as ts; ˆ¯2
SE( ˆ¯2)
= p2
0:2
= 4:472:
\item Hence ¯1 is not significant in the presence of ¯2; but ¯2 is significant in the presence of
¯1. 
\item From all these tests we can infer that a simple linear regression of y on x2, arithmetic
ability, is adequate as a model.
\end{itemize}
\end{enumerate}

\end{document}

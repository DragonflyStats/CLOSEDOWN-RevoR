\documentclass[a4paper,12pt]{article}
%%%%%%%%%%%%%%%%%%%%%%%%%%%%%%%%%%%%%%%%%%%%%%%%%%%%%%%%%%%%%%%%%%%%%%%%%%%%%%%%%%%%%%%%%%%%%%%%%%%%%%%%%%%%%%%%%%%%%%%%%%%%%%%%%%%%%%%%%%%%%%%%%%%%%%%%%%%%%%%%%%%%%%%%%%%%%%%%%%%%%%%%%%%%%%%%%%%%%%%%%%%%%%%%%%%%%%%%%%%%%%%%%%%%%%%%%%%%%%%%%%%%%%%%%%%%
\usepackage{eurosym}
\usepackage{vmargin}
\usepackage{amsmath}
\usepackage{graphics}
\usepackage{epsfig}
\usepackage{enumerate}
\usepackage{multicol}
\usepackage{subfigure}
\usepackage{fancyhdr}
\usepackage{listings}
\usepackage{framed}
\usepackage{graphicx}
\usepackage{amsmath}
\usepackage{chngpage}
%\usepackage{bigints}

\usepackage{vmargin}
% left top textwidth textheight headheight
% headsep footheight footskip
\setmargins{2.0cm}{2.5cm}{16 cm}{22cm}{0.5cm}{0cm}{1cm}{1cm}
\renewcommand{\baselinestretch}{1.3}

\setcounter{MaxMatrixCols}{10}
\begin{document}
\begin{table}[ht!]
     \centering
     \begin{tabular}{|p{15cm}|}
     \hline        
7. The data ( , ), xy iiwhere x i n i >= 01 , ,..., , are thought to conform to a proportional regression model (linear regression through the origin)
Y x e i n i i i = + = β , ,..., , 1
where the ei are independent Normally distributed error terms with mean zero.
(i) If the ei have constant known variance 
σ 2, show that the maximum likelihood (ML)
estimate of 
β
 minimises ei i n 2 1= ∑ and is given by
ˆ β 1 =
xy
x
ii
i
n
i
i
n =
=
∑
∑
1
2
1
.

\\ \hline
      \end{tabular}
    \end{table}
    
  \begin{table}[ht!]
     \centering
     \begin{tabular}{|p{15cm}|}
     \hline  
(ii) If instead the variance of ei is given by 
σ
2
xi ,  i = 1, …, n,  show that the
ML estimate of 
β
 minimises 
e x i ii n 2 1= ∑ and is given by
ˆ β 2 =   
y x
,
where x and y are the sample mean values of xx n1 ,..., and yy n1 ,..., respectively. \\ \hline 
      \end{tabular}
    \end{table}
  \begin{table}[ht!]
     \centering
     \begin{tabular}{|p{15cm}|}
     \hline  
(iii) Each time a motorcycle is filled with petrol, a record is kept of the amount of petrol in litres (x) used, and the distance travelled in miles (y) since the previous fill-up. Values of x and y recorded on the last 9 occasions were as follows:
x 4.3 4.9 5.7 6.5 7.2 8.3 8.4 9.6 10.1
y 123 156 183 183 204 234 270 273 324
Plot the data and calculate ˆ β 1 and ˆ β 2.  Which of the models (i) or (ii) do you think better represents the data? 
\\ \hline
      \end{tabular}
    \end{table}

\begin{framed}
In the case of simple regression, the formulas for the least squares estimates are 
\[{\displaystyle {\widehat {\beta }}_{1}={\frac {\sum (x_{i}-{\bar {x}})(y_{i}-{\bar {y}})}{\sum (x_{i}-{\bar {x}})^{2}}}{\text{ and }}{\widehat {\beta }}_{0}={\bar {y}}-{\widehat {\beta }}_{1}{\bar {x}}} \]

where 
$ {\displaystyle {\bar {x}}} $
 is the mean (average) of the 
$ {\displaystyle x} $
 values and 
${\displaystyle {\bar {y}}}$ 
 is the mean of the 
$ {\displaystyle y} $
 values.

\end{framed}
\begin{enumerate}[(i)]
\item 7. (i) Yi = ¯xi + ei, i = 1; 2; ¢ ¢ ¢ ; n, feig i.i.d. N(0; ¾2).
Likelihood L =
Yn
i=1
f
1
¾
p
2¼
exp[¡
(yi ¡ ¯xi)2
2¾2 ]g,
\[lnL = ¤ = ¡n ln(¾
p
2¼) ¡ 1
2¾2
Xn
i=1
(yi ¡ ¯xi)2.\]

\[
\frac{\partial \Lambda}{\partial \beta}

= 0 +
1
2¾2
¢ 2
Xn
i=1
(yi ¡ ¯xi)xi\] and is zero when
P
(yi ¡ ˆ ¯xi)xi = 0 i.e.

\[X
yixi = ˆ ¯
X
xi
2 or ¯ˆ1 =
P
Pyixi
x2
i
.\]
\[@2¤
@¯2 = ¡
P
x2
i
¾2 , confirming maximum.\]

%%%%%%%%%%%%%%%%%%%%%%%%%%%%%%%%%%%%
\item  If now feig are N(0; ¾2xi),
L =
Yn
i=1
f
1
¾
p
2¼xi
exp[¡
(yi ¡ ¯xi)2
2xi¾2 ]g
and ¤ = ¡n ln(¾
p
2¼) ¡ n
2
Xn
i=1
ln xi ¡
1
2¾2
Xn
i=1
(yi ¡ ¯xi)2
xi
.
@¤
@¯
= 0 + 0 +
1
2¾2
¢
Xn
i=1
1
xi
2(yi ¡ ¯xi)xi =
1
¾2
Xn
i=1
(yi ¡ ¯xi).
This is zero when
X
(yi ¡ ˆ ¯xi) = 0 i.e.
X
yi = ˆ ¯
X
xi or ¯ˆ2 =
P
Pyi
xi
.
6
@2¤
@¯2 = ¡
P
xi
¾2 , confirming maximum.
\begin{itemize}
\item The first case (i) has L = Constant¡ 1
2¾2
P
e2
i , considered as a function of ¯;
\item similarly case (ii) has L = Constant ¡ 1
2¾2
P e2
i
xi
. 
\item Considering as a function
of ¯. 
\item Thus L is maximized when (i)
P
e2
i or (ii)
P
e2
i =xi is minimized (note
the - sign).
\end{itemize}
%%%%%%%%%%%%%%%%%%%%%%%%%
\item 
SUM
X 4.3 4.9 6.5 5.7 7.2 8.3 8.4 9.6 10.1 65.0
Y 123 156 183 183 204 234 270 273 324 1950
XY 528.9 764.4 1043.1 1189.5 1468.8 1942.2 2268.0 2620.8 3272.4 15098.1
X2 18.49 24.01 32.49 42.25 51.84 68.89 70.56 92.16 102.01 502.70
7
n = 9. ˆ ¯1 = 15098:1
502:7 = 30:034. ˆ ¯2 = 1950
65 = 30:000

%%%%%%%%%%%%%%%%%%%%%%%%%%%5
\begin{itemize}
\item The regression lines are indistinguishable between the two models. 
\item However,
the residuals (difference between y and the value on the line at the same x
- value - i.e. the vertical differences) show a definite tendency to increase as
x increases.
\item For this reason, model (ii) is likely to be better.
\end{itemize}
%%%%%%%%%%%%%%%%%%%%%%%%%%%5
\end{enumerate}
\end{document}

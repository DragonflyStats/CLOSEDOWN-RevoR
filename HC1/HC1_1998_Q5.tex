\documentclass{article}
\usepackage[utf8]{inputenc}

\title{RSS_Jan_2019_HC2}
\author{kobriendublin }
\date{December 2018}

\begin{document}

\maketitle

\section{Introduction}

5.(i)E[X] =
1P
x=0
xe¡¸¸x=x!
=
1P
x=0
e¡¸¸x=(x ¡ 1)!
= ¸
1P
x=0
e¡¸¸x¡1=(x ¡ 1)! (in which the term for x = 0 is o)
= ¸
.
(ii)If P(X=k)=P(X=k+1), e¡¸¸k
k! = e¡¸¸k+1
(k+1)! ; i:e: ¸
k+1 = 1 so that ¸ = k + 1:
(iii)Since the mode has maximum probability, it is unique as in (ii) if ¸ is an integer but otherwise
satisfies P(X=m)
P(X=m¡1) > 1 and P(X=m+1)
P(X=m) < 1, where m is the modal value.
If e¡¸¸m
m! ¢ (m¡1)!
e¡¸¸m¡1 , then ¸
m > 1; i.e. m < ¸;
also if e¡¸¸m+1
(m+1)! ¢ m!
e¡¸¸m < 1, then ¸
m+1 < 1; i.e. ¸ < m + 1 or ¸ ¡ 1 < m ;
hence ¸ ¡ 1 < m < ¸ .
(iv)(a) ¸ = 1, so P(0)=e¡¸=1
e =0.3679 .
(b) P(0)+P(1)+P(2)=e¡1(1+1+1
2)=0.9197 .
(v)Number of faults in 20m2 will follow Poisson with mean 4.
3
(a)P(¸ 3) = 1 ¡ P(0) ¡ P(1) ¡ P(2)
= 1 ¡ e¡4(1 + 4 + 42
2! )
= 1 ¡ 13e¡4
= 1 ¡ 0:2381 = 0:7619
.
(b)Number of rooms with ¸ 3 faults is Binomial(50,0.7619) which can be approximated as
N(50 £ 0:7619; 50 £ 0:7619 £ 0:2381) or N(38:095; 9:0704). The probability of being >40 is the
value corresponding to 40.5(with continuity correction) in this distribution:
Z = 40p:5¡38:095
9:0704
= 2:405
3:0117 = 0:7986
P(Z > 0:7986) = 0:2123
[The answer without a continuity correction would be 0.2635.]
\end{document}

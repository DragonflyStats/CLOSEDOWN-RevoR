\documentclass[a4paper,12pt]{article}
%%%%%%%%%%%%%%%%%%%%%%%%%%%%%%%%%%%%%%%%%%%%%%%%%%%%%%%%%%%%%%%%%%%%%%%%%%%%%%%%%%%%%%%%%%%%%%%%%%%%%%%%%%%%%%%%%%%%%%%%%%%%%%%%%%%%%%%%%%%%%%%%%%%%%%%%%%%%%%%%%%%%%%%%%%%%%%%%%%%%%%%%%%%%%%%%%%%%%%%%%%%%%%%%%%%%%%%%%%%%%%%%%%%%%%%%%%%%%%%%%%%%%%%%%%%%
\usepackage{eurosym}
\usepackage{vmargin}
\usepackage{amsmath}
\usepackage{graphics}
\usepackage{epsfig}
\usepackage{enumerate}
\usepackage{multicol}
\usepackage{subfigure}
\usepackage{fancyhdr}
\usepackage{listings}
\usepackage{framed}
\usepackage{graphicx}
\usepackage{amsmath}
\usepackage{chngpage}
%\usepackage{bigints}

\usepackage{vmargin}
% left top textwidth textheight headheight
% headsep footheight footskip
\setmargins{2.0cm}{2.5cm}{16 cm}{22cm}{0.5cm}{0cm}{1cm}{1cm}
\renewcommand{\baselinestretch}{1.3}

\setcounter{MaxMatrixCols}{10}

\begin{document}
\begin{enumerate}
    \item F(x) = P(X · x) =
R x
0 ¸e¡¸udu = [¡e¡¸u]x
0 = 1 ¡ e¡¸x (x ¸ 0)

%%%%%%%%%%%%%%%%%%%%%%%%%%%%5
\item  L = ¦n i=1¸e¡¸xi = ¸n exp(¡¸
Pn
i=1
xi) = ¸ne¡n¸x ln(L) = n ln ¸ ¡ n¸x
d
d¸ (ln(L)) = 1=¸ ¡ nx = 0 for ˆ¸ = 1=ˆx d2
d¸2 (ln(L)) = ¡n=¸2 < 0 for all ¸, so there is a
maximum for ˆ¸
3
%%%%%%%%%%%%%%%%%%%%%%%%%
\item  Using the result just found ,ˆ¸ » N(¸; ¸2=n) » N(¸; 1=n(x)2), and so approximately
ˆ¸
¡¸
(x
p
n)¡1 » N(0; 1), from which an approximate 95\% confidence interval for ¸ is given by
ˆ¸
§ 1:96=(x
p
n) or 1
x (1 § 1p:96
n )
\end{enumerate}
\end{document}

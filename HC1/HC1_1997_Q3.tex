\documentclass[a4paper,12pt]{article}
%%%%%%%%%%%%%%%%%%%%%%%%%%%%%%%%%%%%%%%%%%%%%%%%%%%%%%%%%%%%%%%%%%%%%%%%%%%%%%%%%%%%%%%%%%%%%%%%%%%%%%%%%%%%%%%%%%%%%%%%%%%%%%%%%%%%%%%%%%%%%%%%%%%%%%%%%%%%%%%%%%%%%%%%%%%%%%%%%%%%%%%%%%%%%%%%%%%%%%%%%%%%%%%%%%%%%%%%%%%%%%%%%%%%%%%%%%%%%%%%%%%%%%%%%%%%
\usepackage{eurosym}
\usepackage{vmargin}
\usepackage{amsmath}
\usepackage{graphics}
\usepackage{epsfig}
\usepackage{enumerate}
\usepackage{multicol}
\usepackage{subfigure}
\usepackage{fancyhdr}
\usepackage{listings}
\usepackage{framed}
\usepackage{graphicx}
\usepackage{amsmath}
\usepackage{chngpage}
%\usepackage{bigints}

\usepackage{vmargin}
% left top textwidth textheight headheight
% headsep footheight footskip
\setmargins{2.0cm}{2.5cm}{16 cm}{22cm}{0.5cm}{0cm}{1cm}{1cm}
\renewcommand{\baselinestretch}{1.3}

\setcounter{MaxMatrixCols}{10}
\begin{document}



\section{Introduction}
\begin{enumerate}
\item $p(B) = p = 1/4$. Family size $n = 5$. 
The distribution of X, the number with blue
eyes is binomial (n = 5; p = 1/4).

\item \[P(X=0) = \left(\frac{3}{4}\right)^5 = \frac{243}{1024} \], so 
\begin{eqnarray*}
P(\mbox{at least 1 with blue eyes}) 
&=& P(X\geq 1) \\
&=& 1 - P(X=0)\\ 
&=& \frac{781}{1024}\\
&=& 0:7627.\\
\end{eqnarray*}
%%%%%%%%%%%%%%%%%%%%%%%%%%%%%%%%%%%%%%%%%%%%%%%%%%%%
\item  P(at least 3 B j at least 1 B)=P(r ¸ 3)=P(r ¸ 1) = P(r¸3)
781=1024 .
P(3) + P(4) + P(5) = ( 5
3 )( 1
4 )3( 3
4 )2 + ( 5
4 )( 1
4 )4( 3
4 ) + ( 1
4 )5
= 1
1024f10 £ 9 + 5 £ 3 + 1g = 106
1024 :
So required answer is \[ \frac{106/1024}{781/1024} = \frac{106}{781} = 0.1357. \]
%%%%%%%%%%%%%%%%%%%%%%%5
\item Given that a particular one - the youngest - has blue eyes means that of the
other four, at least two have blue eyes. This is found as P(2)+P(3)+P(4) in
binomial (4; 1=4): ( 4
2 )( 1
4 )2( 3
4 )2+( 4
3 )( 1
4 )3( 3
4 )+( 1
4 )4 = 1
256f6£9+4£3+1g


\begin{eqnarray*}
 &=& 67/256 \\
 &=& 0:2617.
\end{eqnarray*}

%%%%%%%%%%%%%%%%%%%%%%%%%%%%%%%%%%%%%5
\item  Using binomial (5; 1=4) and excluding r = 0, the expected number is
1
1¡P(0)
X5
r=1
rP(r) =
1024
781
f1£5£(
1
4
)(
3
4
)
4
+2£10£(
1
4
)
2
(
3
4
)
3
+3£10£(
1
4
)
3
(
3
4
)
2
+
4£5£(
1
4
)
4
(
3
4
)+5£(
1
4
)
5
g =
1
781
(5£81+20£27+30£9+20£3+5) =
1280
781
= 1:64.

\item In binomial (4; 1=4), E[r] = np = 1.
So expected number is 1(youngest) + 1(others) = 2.
\item Specific information about one child reduces the “subspace” in which we
have to search for the values of r concerning the others.
\end{enumerate}
\end{document}

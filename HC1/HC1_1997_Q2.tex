\documentclass[a4paper,12pt]{article}
%%%%%%%%%%%%%%%%%%%%%%%%%%%%%%%%%%%%%%%%%%%%%%%%%%%%%%%%%%%%%%%%%%%%%%%%%%%%%%%%%%%%%%%%%%%%%%%%%%%%%%%%%%%%%%%%%%%%%%%%%%%%%%%%%%%%%%%%%%%%%%%%%%%%%%%%%%%%%%%%%%%%%%%%%%%%%%%%%%%%%%%%%%%%%%%%%%%%%%%%%%%%%%%%%%%%%%%%%%%%%%%%%%%%%%%%%%%%%%%%%%%%%%%%%%%%
\usepackage{eurosym}
\usepackage{vmargin}
\usepackage{amsmath}
\usepackage{graphics}
\usepackage{epsfig}
\usepackage{enumerate}
\usepackage{multicol}
\usepackage{subfigure}
\usepackage{fancyhdr}
\usepackage{listings}
\usepackage{framed}
\usepackage{graphicx}
\usepackage{amsmath}
\usepackage{chngpage}
%\usepackage{bigints}

\usepackage{vmargin}
% left top textwidth textheight headheight
% headsep footheight footskip
\setmargins{2.0cm}{2.5cm}{16 cm}{22cm}{0.5cm}{0cm}{1cm}{1cm}
\renewcommand{\baselinestretch}{1.3}

\setcounter{MaxMatrixCols}{10}
\begin{document}
\begin{table}[ht!]
     \centering
     \begin{tabular}{|p{15cm}|}
     \hline        
If a Ruritanian peasant farmer grows cereals, his profit X1 in Ruritanian pounds (£R),  is Normally distributed with mean
 1750 and standard deviation 300.  If he grows beans his profit X2 is Normally distributed with mean 2000 and standard deviation 400.  If he grows a proportion p of cereals and a proportion 1−p of beans, the profit, Y, is pX p X 12 1 +− ()where 0 1 ≤≤ p .
\\
\noindent \textbf{Part (a)}\\
State the distribution of Y, assuming that X1 and X2 are independent.\\

 \hline
      \end{tabular}
    \end{table}
    


\begin{enumerate}[(a)]
\item $Y = pX1 + (1 ¡ p)X2$ is $N(1750p + 2000(1 ¡ p); p2 ¢ 300^2 + (1 ¡ p)2 ¢ 400^2)$
i.e. \[N(2000 - 250p; 10^4(9p^2 + 16(1 - p)^2)\]
or \[N(2000 - 250p; 10000(25p^2 - 32p + 16))\].


%%%%%%%%%%%%%%%%%%%%%%%%%%%%%%%%%%%%%%5
\newpage
  \begin{table}[ht!]
     \centering
     \begin{tabular}{|p{15cm}|}
     \hline  
(ii) State the value of p, $p_1$ say, which maximises the farmer’s expected profit,$E(Y)$. 
\\ \hline 
      \end{tabular}
    \end{table}
    
    
\item $E[Y ] = 2000 - 250p$ and has maximum value (2000) for $p1 = 0$.
\item $V[Y]$ is minimized when 
\[ \frac{d}{dp}(25p^2 - 32p + 16) =0\]

\[i.e. 50p - 32 = 0\] or \[p2 = \frac{16}{25}=0.64\].
The second deviation is > 0, indicating a minimum.
\item  E[Y jp1 = 0] = £2000. 
\begin{eqnarray*}
E\left[Y | p^2 = \frac{16}{25}\right] &=& 2000 - 250 \frac{16}{25}\\ 
&=& \$1840.
\end{eqnarray*}


%%%%%%%%%%%%%%%%%%%%%%%%%%%%%%%%%%%%%%%%%%%%%%%%5
\newpage
  \begin{table}[ht!]
     \centering
     \begin{tabular}{|p{15cm}|}
     \hline  

(iii) Find the value of $p$, $p_2$  say, which minimises the variance of the farmer’s profit, that is, minimises $V(Y)$. \\ \hline 
      \end{tabular}
    \end{table}
\item  (a) On p = p1 = 0, $Y \sim N(2000; 400^2)$:
\begin{eqnarray*}
P(Y < 1480) &=& P(Z < \left( \frac{1480-2000}{400}
\right) \\
&=& P(Z < -1.30)\\ 
&=& 0.0968.\\
\end{eqnarray*}

  \begin{table}[ht!]
     \centering
     \begin{tabular}{|p{15cm}|}
     \hline  
(iv) Calculate the expected profit when pp = 1and when pp = 2 .
(v) The farmer reckons that he will be ruined if his profit is less than £R 1480. Calculate the probability that the farmer will be ruined (a) if he adopts pp = 1, (b) if he adopts pp = 2 .Which course of action do you think is better for the farmer, and why?\\\\ \hline
      \end{tabular}
    \end{table}
(b) On p = p2 = 16=25, 
\[Y \sim N(1840; 104f162
25 ¡ 32£16
25 + 16g).\]
i.e. \[N(1840, 16 £ 104f1 ¡ 16
25g) = N(1840, 240^2)\]:


\begin{eqnarray*}
P(Y < 1480) &=& P\left(Z < \left( \frac{1480-1840}{240} \right)
\right) \\
&=& P(Z < -1.50)\\ 
&=& 0.0668.\\
\end{eqnarray*}
%%%%%%%%%%%%%%%%%%%%%%%%

\begin{itemize}
    \item Z stands for the standardized variate N(0; 1).
    \item Use mixed strategy (b) because
its probability of ruin is only two-thirds of that on (a).
\item His expectation
is lower, but variability is also lower, on (b).
\end{itemize}

\end{enumerate}
\end{document}

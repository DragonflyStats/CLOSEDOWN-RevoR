\documentclass[a4paper,12pt]{article}
%%%%%%%%%%%%%%%%%%%%%%%%%%%%%%%%%%%%%%%%%%%%%%%%%%%%%%%%%%%%%%%%%%%%%%%%%%%%%%%%%%%%%%%%%%%%%%%%%%%%%%%%%%%%%%%%%%%%%%%%%%%%%%%%%%%%%%%%%%%%%%%%%%%%%%%%%%%%%%%%%%%%%%%%%%%%%%%%%%%%%%%%%%%%%%%%%%%%%%%%%%%%%%%%%%%%%%%%%%%%%%%%%%%%%%%%%%%%%%%%%%%%%%%%%%%%
\usepackage{eurosym}
\usepackage{vmargin}
\usepackage{amsmath}
\usepackage{graphics}
\usepackage{epsfig}
\usepackage{enumerate}
\usepackage{multicol}
\usepackage{subfigure}
\usepackage{fancyhdr}
\usepackage{listings}
\usepackage{framed}
\usepackage{graphicx}
\usepackage{amsmath}
\usepackage{chngpage}
%\usepackage{bigints}

\usepackage{vmargin}
% left top textwidth textheight headheight
% headsep footheight footskip
\setmargins{2.0cm}{2.5cm}{16 cm}{22cm}{0.5cm}{0cm}{1cm}{1cm}
\renewcommand{\baselinestretch}{1.3}

\setcounter{MaxMatrixCols}{10}
\begin{document}

\begin{table}[ht!]
     \centering
     \begin{tabular}{|p{15cm}|}
     \hline        
(a) Suppose that a lawyer has been guilty of financial irregularities in 5 of the 50 client accounts that she controls.  An auditor randomly samples 10 accounts to check in detail.  Find the probability that the auditor checks  
 
(i) none, (ii) two or more of the irregular accounts. 
 \\ \hline
      \end{tabular}
    \end{table}
    
  
\begin{framed}
A random variable 
$ {\displaystyle X} $
 follows the hypergeometric distribution if its probability mass function (pmf) is given by
\[ {\displaystyle p_{X}(k)=\Pr(X=k)={\frac {{\binom {K}{k}}{\binom {N-K}{n-k}}}{\binom {N}{n}}},} \]

where 
\begin{itemize}
\item ${\displaystyle N}$ 
 is the population size,
\item ${\displaystyle K}$ 
 is the number of success states in the population,
\item ${\displaystyle n}$ 
 is the number of draws (i.e. quantity drawn in each trial),
\item ${\displaystyle k}$ 
 is the number of observed successes,
\item ${\textstyle \textstyle {a \choose b}}$ 
 is a binomial coefficient.
\end{itemize}

\end{framed}
\begin{enumerate}
\item P(x irregular in sample) =
Ã
10
x
! Ã
40
5 ¡ x
!
=
Ã
50
5
!
For x=0,1,2,3,4,5
The sampling leads to the hypergeometric distribution:
1
Account : Ok Irrigular Total
Sampled 10 ¡ x x 10
Notsampled 35 + x 5 ¡ x 40
Total 45 5 50
So
\begin{eqnarray*}
P(X=0) &=& \frac{ {10 \choose 0} \times {40 \choose 5} }{ {50 \choose 5}}\\
&=& \frac{1 \times 40! \times 5!  \times 45!}{5! \times 35! \times 50|}\\
&=& \frac{ 40 \times 39 \times 38 \times 37 \times 36}{50 \times 49 \times 48 \times 47 \times 46}\\
&=& 0.3106
\end{eqnarray*}
(ii)
P(x ¸ 2) = 1 ¡ P(0) ¡ P(1):
\begin{eqnarray*}
P(X=1) &=& \frac{ {10 \choose 1} \times {40 \choose 4} }{ {50 \choose 5}}\\
&=& \frac{10 \times 40! \times 5!  \times 45!}{4! \times 36! \times 50|}\\
&=& \frac{10 \times 5 \times 40 \times 39 \times 38 \times 37}{50 \times 49 \times 48 \times 47 \times 46}\\
&=& 0.4313
\end{eqnarray*}


and P(x ¸ 2) = 0.2581:
\newpage

\begin{table}[ht!]
     \centering
     \begin{tabular}{|p{15cm}|}
     \hline  
(b) In tennis, when the score reaches deuce (“40 all”) the game is won by the first player to lead by two consecutive points.  Suppose that the outcomes of all points are independent, and that the server is twice as likely to win a point as the receiver.  Show that, once a game has reached deuce, the probability that the server wins the game is 4 5 . (4) 
 
Writing N for the number of points played from when the game first reaches deuce until it ends (i.e. is won by server or receiver), show that 
 
() 52 43 n P N n  == 
 
 
 
and state the range of possible values for N. 
(4) 
 
 
 \\ \hline
      \end{tabular}
    \end{table}
\item P(server wins from deuce)=
P(ww) + P(wlww) + P(lwww) + p(wlwlww) + p(wllwww)
+P(lwwlww) + P(lwlwww) + P(wlwlwlww) + ¢ ¢ ¢
where the number of possible sequence doubles each deuce.this is


\begin{eqnarray*}
P(server wins from deuce) &=& 
( 2
3 )2 + 2(2
3 )2( 1
3 £ 2
3 ) + 4(2
3 )2( 1
3 £ 2
3 )2 + 8(2
3 )2( 1
3 £ 2
3 )3 + ¢ ¢ ¢
\\ &=& ( 2
3 )2[1 + ( 2
3 )2 + ( 2
3 )4 + ( 2
3 )6 + ¢ ¢ ¢]
\\ &=&  9
4 [1 + 9
4 + ( 9
4 )2 + ( 9
4 )3 + ¢ ¢ ¢] \\ &=&  4
9 £ 1
1¡4
9
\\ &=&  4
5
\end{eqnarray*}

\begin{eqnarray*}
P(score does not change after 2 points)  &=&  ( 2
3 £ 1
3 ) + ( 1
3 £ 2
3 ) \\ &=&  4
9 :
2
\end{eqnarray*}

\begin{eqnarray*}
P(game ends after 2 points) &=& ( 2
3 £ 2
3 ) + ( 1
3 £ 1
3 ) \\ &=&  5
9 :
P(N = 2k) = 5
9 £ P(k ¡ 1 sequences of 2 points which do not change score)
\\ &=&  5
9 £ ( 4
9 )k¡1 for k = 1; 2; 3; ¢ ¢ ¢
\end{eqnarray*}

so that N=2,4,6,¢ ¢ ¢
writing 2k=n ,
\begin{eqnarray*} P(N = n) &=& 5
9 £ 9
4 £ ( 4
9 )n=2 \\ &=&  5
4 £ ( 2
3 )n
\end{eqnarray*}

%%%%%%%%%%%%%%%%%%%%%%%%%%%%%%%%    
\newpage    
  \begin{table}[ht!]
     \centering
     \begin{tabular}{|p{15cm}|}
     \hline  
In the large city of Olchester, 30\% of electors are Conservatives, 40\% are Labour supporters, 20\% are Liberal Democrats and 10\% have no affiliation.  Political affiliation is independent of sex, so these percentages apply to males and females equally.  Records show that in a particular election 80\% of the Conservatives voted, as did 60\% of Labour supporters and 90\% of Liberal Democrats, whilst those with no affiliation did not vote.  

\begin{itemize}
\item[(i)]If an elector is chosen at random and it is found that he did not vote in the election, find the probability that he is a Labour supporter.  
 
\item[(i)]A second elector is chosen at random and it is found that she also did not vote;  what is the probability that both people are Labour supporters? 
\end{itemize}
\\ \hline 
 \end{tabular}
    \end{table}
\item

\begin{enumerate}
    \item $P(C) = 0.3 P(V |C) = 0.8: P(L) = 0.4 P(V | L) = 0.6$
    \item $P(D) = 0.2 P(V |D) = 0.9 P(O) = 0.1 P(V |O) = 0$
    \item $P(NV |C) = 0.2 P(NV |L) = 0.4 P(NV |D) = 0.1; P(NV |O) = 1$
\end{enumerate}
\begin{eqnarray*}
P(L|NV ) &=& P(NV |L)P(L)\\ 
&=&
P(x=C;L;D;OP(NV |x)P(x)\\
&=& 0.4£0.4
(0.4£0.4)+(0.3£0.2)+(0.1£0.2)+(1£0.1)\\ &=& 0.16
0.34 \\ &=& 0.4706:
\end{eqnarray*}
\[P(both LjNV ) = 0.4706^2 = 0.2215:\]

\end{enumerate}
\end{document}


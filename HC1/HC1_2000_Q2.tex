\documentclass[a4paper,12pt]{article}
%%%%%%%%%%%%%%%%%%%%%%%%%%%%%%%%%%%%%%%%%%%%%%%%%%%%%%%%%%%%%%%%%%%%%%%%%%%%%%%%%%%%%%%%%%%%%%%%%%%%%%%%%%%%%%%%%%%%%%%%%%%%%%%%%%%%%%%%%%%%%%%%%%%%%%%%%%%%%%%%%%%%%%%%%%%%%%%%%%%%%%%%%%%%%%%%%%%%%%%%%%%%%%%%%%%%%%%%%%%%%%%%%%%%%%%%%%%%%%%%%%%%%%%%%%%%
\usepackage{eurosym}
\usepackage{vmargin}
\usepackage{amsmath}
\usepackage{graphics}
\usepackage{epsfig}
\usepackage{enumerate}
\usepackage{multicol}
\usepackage{subfigure}
\usepackage{fancyhdr}
\usepackage{listings}
\usepackage{framed}
\usepackage{graphicx}
\usepackage{amsmath}
\usepackage{chngpage}
%\usepackage{bigints}

\usepackage{vmargin}
% left top textwidth textheight headheight
% headsep footheight footskip
\setmargins{2.0cm}{2.5cm}{16 cm}{22cm}{0.5cm}{0cm}{1cm}{1cm}
\renewcommand{\baselinestretch}{1.3}

\setcounter{MaxMatrixCols}{10}
\begin{document}

\begin{enumerate}
\item P(x irregular in sample) =
Ã
10
x
! Ã
40
5 ¡ x
!
=
Ã
50
5
!
For x=0,1,2,3,4,5
The sampling leads to the hypergeometric distribution:
1
Account : Ok Irrigular Total
Sampled 10 ¡ x x 10
Notsampled 35 + x 5 ¡ x 40
Total 45 5 50
So
P(x = 0) =
Ã
10
0
! Ã
40
5
!
Ã
50
5
! =
1 £ 40! £ 5! £ 45!
5! £ 35! £ 50!
=
40 £ 39 £ 38 £ 37 £ 36
50 £ 49 £ 48 £ 47 £ 46
= 0:3106
(ii)
P(x ¸ 2) = 1 ¡ P(0) ¡ P(1):
P(x = 1) =
Ã
10
1
! Ã
40
4
!
Ã
50
5
! =
10 £ 40! £ 5! £ 45!
4! £ 36! £ 50!
=
10 £ 5 £ 40 £ 39 £ 38 £ 37
50 £ 49 £ 48 £ 47 £ 46
= 0:4313
and P(x ¸ 2) = 0:2581:
\item P(server wins from deuce)=
P(ww) + P(wlww) + P(lwww) + p(wlwlww) + p(wllwww)
+P(lwwlww) + P(lwlwww) + P(wlwlwlww) + ¢ ¢ ¢
where the number of possible sequence doubles each deuce.this is
( 2
3 )2 + 2(2
3 )2( 1
3 £ 2
3 ) + 4(2
3 )2( 1
3 £ 2
3 )2 + 8(2
3 )2( 1
3 £ 2
3 )3 + ¢ ¢ ¢
= ( 2
3 )2[1 + ( 2
3 )2 + ( 2
3 )4 + ( 2
3 )6 + ¢ ¢ ¢]
= 9
4 [1 + 9
4 + ( 9
4 )2 + ( 9
4 )3 + ¢ ¢ ¢] = 4
9 £ 1
1¡4
9
= 4
5
P(score does not change after 2 points) = ( 2
3 £ 1
3 ) + ( 1
3 £ 2
3 ) = 4
9 :
2
P(game ends after 2 points) = ( 2
3 £ 2
3 ) + ( 1
3 £ 1
3 ) = 5
9 :
P(N = 2k) = 5
9 £ P(k ¡ 1 sequences of 2 points which do not change score)
= 5
9 £ ( 4
9 )k¡1 for k = 1; 2; 3; ¢ ¢ ¢
so that N=2,4,6,¢ ¢ ¢
writing 2k=n ,P(N = n) = 5
9 £ 9
4 £ ( 4
9 )n=2 = 5
4 £ ( 2
3 )n
\item

\begin{enumerate}
    \item P(C) = 0:3 P(V jC) = 0:8: P(L) = 0:4 P(V jL) = 0:6
    \item P(D) = 0:2 P(V jD) = 0:9 P(O) = 0:1 P(V jO) = 0
    \item P(NV jC) = 0:2 P(NV jL) = 0:4 P(NV jD) = 0:1; P(NV jO) = 1
\end{enumerate}

P(LjNV ) = P(NV jL)P(L) =
P
x=C;L;D;OP(NV jx)P(x)
= 0:4£0:4
(0:4£0:4)+(0:3£0:2)+(0:1£0:2)+(1£0:1) = 0:16
0:34 = 0:4706:

\[P(both LjNV ) = 0:4706^2 = 0:2215:\]

\end{enumerate}
\end{document}


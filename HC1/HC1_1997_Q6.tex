\documentclass[a4paper,12pt]{article}
%%%%%%%%%%%%%%%%%%%%%%%%%%%%%%%%%%%%%%%%%%%%%%%%%%%%%%%%%%%%%%%%%%%%%%%%%%%%%%%%%%%%%%%%%%%%%%%%%%%%%%%%%%%%%%%%%%%%%%%%%%%%%%%%%%%%%%%%%%%%%%%%%%%%%%%%%%%%%%%%%%%%%%%%%%%%%%%%%%%%%%%%%%%%%%%%%%%%%%%%%%%%%%%%%%%%%%%%%%%%%%%%%%%%%%%%%%%%%%%%%%%%%%%%%%%%
\usepackage{eurosym}
\usepackage{vmargin}
\usepackage{amsmath}
\usepackage{graphics}
\usepackage{epsfig}
\usepackage{enumerate}
\usepackage{multicol}
\usepackage{subfigure}
\usepackage{fancyhdr}
\usepackage{listings}
\usepackage{framed}
\usepackage{graphicx}
\usepackage{amsmath}
\usepackage{chngpage}
%\usepackage{bigints}

\usepackage{vmargin}
% left top textwidth textheight headheight
% headsep footheight footskip
\setmargins{2.0cm}{2.5cm}{16 cm}{22cm}{0.5cm}{0cm}{1cm}{1cm}
\renewcommand{\baselinestretch}{1.3}

\setcounter{MaxMatrixCols}{10}
\begin{document}
\begin{table}[ht!]
 \centering
 \begin{tabular}{|p{15cm}|}
 \hline
6. The lengths X of offcuts of timber in a carpenter’s workshop follow the continuous uniform distribution with probability density function (pdf)
f x x ( ) , ; ; = ≤ ≤ > 1 00 θ θθ  = 0  

\[f(x) = \int^{\theta}_{0} \frac{1\;dx}{\theta} \]


elsewhere.
(i) Obtain the cumulative distribution function (cdf), F(x) say, and sketch the graphs of
f(x) and F(x).Also find $E(X)$ and $V(X)$.

\\ \hline
  \end{tabular}
\end{table}

 



%%%%%%%%%%%%%%%%%%%%%%%%%%%%%%%%%%%%%%%%%%%%%%5
\begin{enumerate}[(i)]
\item (i) \[ \Pr(X \leq k) = \int^{k}_{0} \frac{1}{\theta} dx\] for $0\leq k \leq \theta$
$ \Pr(X \leq k) = 0 for k < 0$; $\Pr(X \leq k) = 1 for k > \theta$.

\[E(X) = \int^{\theta}_{0} \frac{x\;dx}{\theta} = \frac{1}{2\theta}\left[ x^2 \right]^{\theta}_{0} = \frac{\theta}{2}\]



\[E(X^2) = \int^{\theta}_{0} \frac{x^2\;dx}{\theta} = \frac{1}{3\theta}\left[ x^3 \right]^{\theta}_{0} = \frac{\theta^2}{3}\]

\[ Var(X) = [E(X^2)] - [E(X)^2] = \frac{\theta^2}{3} - \frac{\theta^2}{4} = \frac{\theta^2}{12}\]
%%%%%%%%%%%%%%%%%%%%%%%%%%%%%%%%%%%%%%%%%%%%%%%%%%%%%%%
\newpage
 \begin{table}[ht!]
 \centering
 \begin{tabular}{|p{15cm}|}
 \hline  
(ii) The carpenter takes a random sample of offcuts with lengths XX n1 , ..., . Explain why
P (length of longest offcut in sample ≤ x) = 
x n θ
  
   ,  0 ≤≤ x θ ,
and deduce the pdf of the sample maximum, X n() say.


Show that
\[EX
n nn ( ) , () = + θ 1\]
\[VX
n
nn
n ()
( ) ( ) .() = ++ θ 2 2 12
\]

Write down a multiple of  X n() which is an unbiased estimator of 
θ
 and obtain its
variance.


Is  X n() the maximum likelihood estimate? \\ \hline 
  \end{tabular}
\end{table}
\item Sampled items chosen at random, hence fXig are independent. 
\begin{itemize}
\item P(X · x) =
x
$\theta$ for each item, all are required to be · x, so probability for n items is (x
$\theta$ )n,
when 0 · x · $\theta$. 
\item This is F(X(n)), where X(n) is the sample maximum, and
so f(x(n)) = F
0(x(n)) = nxn¡1
$\theta$n , when 0 · x · $\theta$, =0 otherwise.
\item E[X(n)] =
Z $\theta$
0
nxn
$\theta$n dx = [ nxn+1
(n + 1)$\theta$n ]$\theta$
0 = n$\theta$
n + 1
,
\item E[X2
(n)] =
Z $\theta$
0
nxn+1
$\theta$n dx = [ nxn+2
(n + 2)$\theta$n ]$\theta$
0 = n$\theta$2
n + 2
.

\end{itemize}
Hence
\begin{eqnarray*}
V \left[X_{(n)}\right] 
&=& \frac{n\theta^2}{n+2} - \left( \frac{n\theta}{(n+1)^2} \right)^2\\ 
&=& \theta^2 \left[ \frac{n}{n+2} -  \frac{n^2}{(n+1)^2 }\right]\\
&=& \frac{\theta^2}{(n+1)^2(n+2)} \left[n(n + 1)^2 - n^2(n + 2)\right] \\ 
&=& \frac{n \theta^2}{(n+1)^2(n+2)}\\
\end{eqnarray*}

\begin{itemize}
\item $\displaystyle{ \frac{n+1}{n}X(n) }$ is unbiased for estimating $\theta$.
\item

\begin{eqnarray*}
\operatorname{Var} [\frac{n+1}{n} X_{(n)}]  &=&
\left(\frac{n+1}{n}\right)^2  \frac{n\;\theta^2}{(n+1)^2(n+2)} \\
&=& \frac{\theta^2}{n(n+2)}\\
\end{eqnarray*}

\item The likelihood of a sample $\{x1,x_2,\ldots, x_n\}$ is $\frac{1}{\theta^n}$
$(0 \leq x \leq \theta)$.
\item Setting $\hat{\theta} = X(n)$, where is the lowest value of $\theta$ possible on the evidence of
the sample values, gives the largest possible value of the likelihood.
\item  Hence $X_{(n)}$ is the m.l. estimator.
\end{itemize}
%%%%%%%%%%%%%%%%%%
\item Method of moments estimator $\hat{\theta}$ is found from setting $\bar{x} = E[x]$, i.e., $\bar{x} = 1/2$
$\hat{\theta}$
so that $\hat{\theta} = 2\bar{x}$. 

\[V [\hat{\theta}] = 4V [\bar{x}] = 4
nV [x] = 4\]
n ¢ $\theta$2
12 = $\theta$2=3n.
%%%%%%%%%%%%%%%%%%%%%%%%%%%%%%%5
% 1997 Question 6

\begin{eqnarray*}
V(\hat{\theta}) &=& 4 \times V(\bar{x})\\
 &=& \frac{4}{n} \times V(x) \\
 &=& \frac{4}{n} \times \frac{\theta^2}{12} \\
 &=& \frac{\theta^2}{3n}
\end{eqnarray*}
%%%%%%%%%%%%%%%%%%%%%%%%%%%%%%%%%%%%%%%%%%%%%%%%%%%%%%%%%%%%%%%%
\newpage
  \begin{table}[ht!]
 \centering
 \begin{tabular}{|p{15cm}|}
 \hline  
(iii) Show that 
2 1n
Xi
i
n
= ∑ is the method of moments estimator of 
θ
 and obtain its variance.
(iv) How would you advise the carpenter to estimate 
θ
?
4\\ \hline
  \end{tabular}
\end{table}
%%%%%%%%%%%%%%%%%%
\item $\displaystyle{ \frac{n+1}{n} X_{(n)}}$
is unbiased, and $X_{(n)}$ very nearly so if n is at all large; their
variances are much smaller than that for $\hat{\theta}$, the estimator based on the mean.
Hence, use $X_{(n)}$ if there are a reasonable number of offcuts; if only few,
multiply by the factor n+1
n .
\end{enumerate}
\end{document}

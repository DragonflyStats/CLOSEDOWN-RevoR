\documentclass[a4paper,12pt]{article}
%%%%%%%%%%%%%%%%%%%%%%%%%%%%%%%%%%%%%%%%%%%%%%%%%%%%%%%%%%%%%%%%%%%%%%%%%%%%%%%%%%%%%%%%%%%%%%%%%%%%%%%%%%%%%%%%%%%%%%%%%%%%%%%%%%%%%%%%%%%%%%%%%%%%%%%%%%%%%%%%%%%%%%%%%%%%%%%%%%%%%%%%%%%%%%%%%%%%%%%%%%%%%%%%%%%%%%%%%%%%%%%%%%%%%%%%%%%%%%%%%%%%%%%%%%%%
\usepackage{eurosym}
\usepackage{vmargin}
\usepackage{amsmath}
\usepackage{graphics}
\usepackage{epsfig}
\usepackage{enumerate}
\usepackage{multicol}
\usepackage{subfigure}
\usepackage{fancyhdr}
\usepackage{listings}
\usepackage{framed}
\usepackage{graphicx}
\usepackage{amsmath}
\usepackage{chngpage}
%\usepackage{bigints}

\usepackage{vmargin}
% left top textwidth textheight headheight
% headsep footheight footskip
\setmargins{2.0cm}{2.5cm}{16 cm}{22cm}{0.5cm}{0cm}{1cm}{1cm}
\renewcommand{\baselinestretch}{1.3}

\setcounter{MaxMatrixCols}{10}

\begin{document}

\begin{enumerate}
\item 2 (i) X is binomially distributed B(n; p) and so P(X = x) = ( n
x
)px(1 ¡ p)n¡x ; x =
0; 1; 2 ¢ ¢ ¢ n:
Hence
E[X] =
Xn
x=0
xpx(1 ¡ p)n¡x n!
x!(n ¡ x)!
= np
Xn
x=0
(n ¡ 1)!px¡1(1 ¡ p)n¡x
(x ¡ 1)!(fn ¡ 1g ¡ fx ¡ 1g)!
since the term in E[X] for the value x = 0 is zero. Put Y = (x ¡ 1)
Thus
E[X] = np
nX¡1
y=0
( n ¡ 1
y
)py(1 ¡ p)n¡y = (p + (1 ¡ p))n¡1np = np
The expression for variance is $V [X] = np(1 - p).$
%%%%%%%%%%%%%%%%%%%%%%%%%%%%%%%%%%%%%%%%%5
\item  (a) Y is B(4; 1=2) so P(Y = y) = (
4
y
)1=24 for y = 0; 1; 2; 3; 4
\begin{eqnarray*}
P(Y ¸ 3) &=& P(3) + P(4) = 4 £ 1=16 + 1 £ 1=16 = 5=16
P(Y = 2)\\ &=& 4£3£1 
2£1£16 \\ &=& 3=8.
\end{eqnarray*}
%%%%%%%%%%%%%%%%%%%%%%%%%%%
\item E[Y ] = 2 ; V [Y ] = 1.
\item Now Y = Z1 + z2 + Z3 + Z4, where each Zi is a Bernoulli variable with mean pi
and variable pi(1 ¡ pi) By independence, E[Y ] =
P4
i=1
E[Zi] and V [Y ] =
P4
i=1
V [Zi] so
\begin{itemize}
    \item E[Y ] = 3=4 + 1=3 + 2=3 + 1=4 = 2 (the same as before since the new probabilities
average to 1/2).
\item Also V [Y ] = 3=4£1=4+1=3£2=3+2=3£1=3+1=4£3=4 = 3=16+2=9+2=9+3=16 =
59=72 (less than before)
\item NOTE: for probabilities which average to 1/2, the first case gives the maximum sine
if pi 6= 1=2the product pi(1 ¡ pi) is < 1=4 for each component.
\end{itemize}

\end{enumerate}
\end{document}
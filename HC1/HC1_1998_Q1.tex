\documentclass[a4paper,12pt]{article}
%%%%%%%%%%%%%%%%%%%%%%%%%%%%%%%%%%%%%%%%%%%%%%%%%%%%%%%%%%%%%%%%%%%%%%%%%%%%%%%%%%%%%%%%%%%%%%%%%%%%%%%%%%%%%%%%%%%%%%%%%%%%%%%%%%%%%%%%%%%%%%%%%%%%%%%%%%%%%%%%%%%%%%%%%%%%%%%%%%%%%%%%%%%%%%%%%%%%%%%%%%%%%%%%%%%%%%%%%%%%%%%%%%%%%%%%%%%%%%%%%%%%%%%%%%%%
\usepackage{eurosym}
\usepackage{vmargin}
\usepackage{amsmath}
\usepackage{graphics}
\usepackage{epsfig}
\usepackage{enumerate}
\usepackage{multicol}
\usepackage{subfigure}
\usepackage{fancyhdr}
\usepackage{listings}
\usepackage{framed}
\usepackage{graphicx}
\usepackage{amsmath}
\usepackage{chngpage}
%\usepackage{bigints}

\usepackage{vmargin}
% left top textwidth textheight headheight
% headsep footheight footskip
\setmargins{2.0cm}{2.5cm}{16 cm}{22cm}{0.5cm}{0cm}{1cm}{1cm}
\renewcommand{\baselinestretch}{1.3}

\setcounter{MaxMatrixCols}{10}
\begin{document}

%%%%%%%%%%%%%%%%%%%%%%%%%%%%%%%%%%%%%%%%%%%%%%%%%%%%%%%%%%%%%%%%%%%%%






\begin{table}[ht!]

     \centering

     \begin{tabular}{|p{15cm}|}

     \hline        
A company runs a competition advertised in three national daily newspapers A, B and
C, which have large readerships in the proportions 2:3:1 respectively. Each reader
reads one newspaper only. The proportions of the readerships of A, B and C who enter
the competition are 0.02, 0.01 and 0.05 respectively. You may assume that all entries
to the competition are correct, and if drawn would receive a prize. All entries are
equally likely to be drawn.

\\ \hline

      \end{tabular}

    \end{table}

    


\begin{table}[ht!]

     \centering

     \begin{tabular}{|p{15cm}|}

     \hline        

\noindent \textbf{Part (a)}

\noindent What are the probabilities that the top prize is won by a reader of paper A, B or C
respectively?


\\ \hline

      \end{tabular}

    \end{table}

    
\begin{enumerate}[(a)]


%%%%%%%%%%%%%%%%%%%%%%%%%%%%%%%%%%%
\item The probabilities of reading the three newspapers are
\begin{itemize}
\item P(A) = 1/3
\item P(B) = 1/2 
\item P(C)= 1/6 
\end{itemize}

Let $E$ be the event that an early is made.
$P(E|A)=0.02$, $P(E|B)=0.01$, $P(E|C)=0.05$ .
\begin{eqnarray*}
P(E) &=& P(E|A)P(A) + P(E|B)P(B) + P(E|C)P(C)\\
&=& (0.02 \times  1/3 ) + (0.01 \times  1
/2 ) + (0.05 \times  1/6 )\\
&=& \frac{1}{6}(0.04 + 0.03 + 0.05)
\\&=& \frac{0.12}{6}
 \\ &=& 0.02
\end{eqnarray*}
Since all entries are correct, the probability of winning is $P(A|E)$,$P(B|E)$,$P(C|E)$ respectively for
readers of A, B, C.
\begin{framed}
\noindent \textbf{Bayes Theorem}
\[P(A|E)P(E) =   P(E|A)P(A) \] 
\noindent \textbf{Conditional Probability}
\[P(A|E) = \frac{P(A|E)P(E)}{P(A)}\]
\end{framed}
\[
P(A|E)=\frac{0.02\times (1/3)}{0.02} 
= 1/3 
\]
\[P(B|E)=\frac{P(E|B)P(B)}{P(E)}=\frac{0.01\times( 1/2)}{0.02} = \frac{1}{4}\]

\[P(C|E)=\frac{P(E|C)P(C)}{P(E)}=\frac{0.05\times( 1/6)}{0.02} = \frac{5}{12}\]

[Check: these must sum to 1 .
p
]


\begin{table}[ht!]
     \centering
     \begin{tabular}{|p{15cm}|}
     \hline        
 \noindent \textbf{Part (b)}\\
\noindent What is the probability that the top prize and the second prize are both won by
readers of paper A?
\\ \hline
 \end{tabular}
\end{table}
%%%%%%%%%%%%%%%%%%%%%%%%%%%%%%%%%%%
\item Since the readerships are large, we may ignore the need for a finite population correction, and
so the required probability will be \[ \left( \frac{1}{3}\right)^2 = \frac{1}{9}\]

%%%%%%%%%%%%%%%%%%%%%%%%%%%%%%%%%%%

\begin{table}[ht!]
     \centering
     \begin{tabular}{|p{15cm}|}
     \hline        
\noindent \textbf{Part (c)}\\
\noindent
What is the probability that the top prize and the second prize are won by readers
of two different papers?

\\ \hline
 \end{tabular}
\end{table}

\item Similarly, for any two different newspapers, in either order for first, second, the probability
will be: 
\begin{eqnarray*}
Prob &=& 2 \left[\left( \frac{1}{3} \times \frac{1}{4} \right) + \left( \frac{1}{3} \times \frac{5}{12} \right) + \left( \frac{1}{4} \times \frac{5}{12} \right)\right]\\
&=& \left( \frac{2}{12} + \frac{10}{36} + \frac{10}{48} \right) \\
&=& \left( \frac{12}{72} + \frac{20}{72} + \frac{15}{72} \right) \\
&=& \frac{47}{72} \\
&=& 0.653
\end{eqnarray*}
%%%%%%%%%%%%%%%%%%%%%%%%%%%%%%%%%%%



\begin{table}[ht!]
     \centering
     \begin{tabular}{|p{15cm}|}
     \hline        
\noindent \textbf{Part (d)}\\
\noindent what is the probability that the top three prizes are won by readers of all three
different papers?

\\ \hline
 \end{tabular}
\end{table}

\item There are 6 possible orders for one each of A, B, C, so the probability is \[6\times \frac{1}{3} \times \frac{1}{4}\times  \frac{5}{12}
 = \frac{5}{24} = 0.208 \]
\end{enumerate}
\end{document}

\documentclass[a4paper,12pt]{article}
%%%%%%%%%%%%%%%%%%%%%%%%%%%%%%%%%%%%%%%%%%%%%%%%%%%%%%%%%%%%%%%%%%%%%%%%%%%%%%%%%%%%%%%%%%%%%%%%%%%%%%%%%%%%%%%%%%%%%%%%%%%%%%%%%%%%%%%%%%%%%%%%%%%%%%%%%%%%%%%%%%%%%%%%%%%%%%%%%%%%%%%%%%%%%%%%%%%%%%%%%%%%%%%%%%%%%%%%%%%%%%%%%%%%%%%%%%%%%%%%%%%%%%%%%%%%
\usepackage{eurosym}
\usepackage{vmargin}
\usepackage{amsmath}
\usepackage{graphics}
\usepackage{epsfig}
\usepackage{enumerate}
\usepackage{multicol}
\usepackage{subfigure}
\usepackage{fancyhdr}
\usepackage{listings}
\usepackage{framed}
\usepackage{graphicx}
\usepackage{amsmath}
\usepackage{chngpage}
%\usepackage{bigints}

\usepackage{vmargin}
% left top textwidth textheight headheight
% headsep footheight footskip
\setmargins{2.0cm}{2.5cm}{16 cm}{22cm}{0.5cm}{0cm}{1cm}{1cm}
\renewcommand{\baselinestretch}{1.3}
\begin{document}
\setcounter{MaxMatrixCols}{10}
\begin{document}

\begin{table}[ht!]
     \centering
     \begin{tabular}{|p{15cm}|}
     \hline        
 \noindent \textbf{Part (b)}\\
\noindent 
The joint distribution of the random variables X and Y is specified in the table below, the values tabulated being the probabilities p(X=x, Y=y),
x     y: 1234 1  3 k 10k  21k  36k 2 10 k 36k  78k 136k 3 21 k 78k 171k 300k
where k is a constant.


\\ \hline
 \end{tabular}
\end{table}
\begin{table}[ht!]
     \centering
     \begin{tabular}{|p{15cm}|}
     \hline        
 \noindent \textbf{Part (b)}\\
(a) Find k.

\\ \hline
 \end{tabular}
\end{table}



\begin{enumerate}
\item The total sum of probabilities must be 1. Hence k=1/900.

\item Summing in rows : 

\begin{center}
\begin{tabular}{|c|c|c|c|c|}
x & 1 & 2 & 3 & TOTAL\\
P(x) &  7/90 & 26/90 &  57/90 &  1 \\
\end{tabular}
\end{center}
and in columns : 

\begin{center}
\begin{tabular}{|c|c|c|c|c|}
y & 1 & 2 & 3 & 4 & TOTAL\\
P(y) & 17/450 & 31/225 & 3/10  & 118/225 & 1 \\
\end{tabular}
\end{center} 


These are the marginal distributions of X and Y.

\item 

\begin{eqnarray*}
E[X] &=& (7+52+171)/90 \\ &=& 23/9 \\ &=&2.56).
\end{eqnarray*}

\begin{eqnarray*}
E[Y] &=& 1
450 ((1 \times  17) + (2 \times  62) + (3 \times  135) + (4 \times  236)) \\ &=& \frac{149}{45}
\\ &=&3.31 
\end{eqnarray*}

\begin{eqnarray*}
E[X2]&=& \frac{1}{90}(1 \times  7 + 4 \times  26 + 9 \times  57) 
\\ &=& \frac{624}{90}
\\ &=& \frac{104}{15}
\end{eqnarray*}


\begin{eqnarray*} 
V[X]&=&E[X^2] - (E[X])^2
\\ &=& \frac{104}{15} -  \frac{232}{81} 
\\ &=& 8424-7935
15\times 81 
\\ &=& \frac{163}{405} \\ &=& 0.4025
\end{eqnarray*}
\begin{table}[ht!]
     \centering
     \begin{tabular}{|p{15cm}|}
     \hline        
 \noindent \textbf{Part (c)}\\
 (b) Obtain the marginal distributions of X and Y.

\\ \hline
 \end{tabular}
\end{table}


%%%%%%%%%%%%%%%%%%%%%%%%%%%%%%%%%%%%%55
\begin{eqnarray*}
E[Y 2] &=& \frac{1}{450} (1 \times  17 + 62 \times  4 + 135 \times  9 + 236 \times  16) \\ &=& 
\frac{5256}{450}
\end{eqnarray*}
%%%%%%%%%%%%%%%%%%%%%%%%%%%%%%%%%%%%%55

%%%%%%%%%%%%%%%%%%%%%%%%%%%%%%%%%%%%%55
\begin{eqnarray*} 
V [Y ] &=& E[Y^2]-  (E[Y])^2  \\ &=& 5256
450-  ( 149
45 )2 \\ &=& 5256\times 45¡1492\times 4
45\times 450
= 14510
45\times 450 \\ &=& 1451
2025 \\ &=& 0.7165)
\end{eqnarray*}


\begin{center}
\begin{tabular}{|c|c|c|c|c|c|c|c|c|}\hline
xy : & 1 & 2 & 3 & 4 & 6 & 8 & 9 & 12 \\ \hline
probability : 3/900 & 20/900 & 42/900 & 72/900 & 156/900 & 136=/900 & 171/900 & 300/900  \\ \hline
\end{center}

\begin{eqnarray*} 
E[XY] &=& \frac{1}{900}(3 + 40 + 126 + 288 + 936 + 1088 + 1539 + 3600) \\ &=& \frac{7620}{900} \\ &=& \frac{127}{15} .
\end{eqnarray*} 

\begin{eqnarray*}  
Cov[X,Y] &=&E[XY]-E[X]E[Y]\\ &=&127
15-  23
9 ¢ 149
45 \\ &=& 27\times 127-23\times 149
9\times 45 \\ &=& 2
405 .
\end{eqnarray*}

\[PXY = p 2=405
163
405\times 1451
2025
= 2
p
p 5
163\times 1451 = 0:00920 .\]
\begin{table}[ht!]
     \centering
     \begin{tabular}{|p{15cm}|}
     \hline        
 \noindent \textbf{Part (d)}\\
 (c) Find E(X), V(X), E(Y), V(Y) and the correlation between X and Y.

\\ \hline
 \end{tabular}
\end{table}
%%%%%%%%%%%%%%%%%%%%%
\begin{table}[ht!]
     \centering
     \begin{tabular}{|p{15cm}|}
     \hline        
 \noindent \textbf{Part (d)}\\
Find the conditional distribution of X given Y = 1.

\\ \hline
 \end{tabular}
\end{table}


\item When Y=1, the conditional distribution of X is the first column of the table, scaled to sum to
1.
\begin{tabular}{|c|c|c|c|}
\hline
x & 1 & 2 & 3 \\ \hline 
Probability & 3/34 & 10/34 & 21/34\\ \hline 
\end{tabular}


\begin{table}[ht!]
     \centering
     \begin{tabular}{|p{15cm}|}
     \hline        
 \noindent \textbf{Part (e)}\\
Find $E(X|Y = l).$

\\ \hline
 \end{tabular}
\end{table}

\item \[E[X|Y=1]= 1\]
34 (3 \times  1 + 10 \times  2 + 21 \times  3)\] \[=86/34=43/17=2.529 .\]
\end{enumerate}


\end{document}

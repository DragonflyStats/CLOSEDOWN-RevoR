\documentclass[a4paper,12pt]{article}
%%%%%%%%%%%%%%%%%%%%%%%%%%%%%%%%%%%%%%%%%%%%%%%%%%%%%%%%%%%%%%%%%%%%%%%%%%%%%%%%%%%%%%%%%%%%%%%%%%%%%%%%%%%%%%%%%%%%%%%%%%%%%%%%%%%%%%%%%%%%%%%%%%%%%%%%%%%%%%%%%%%%%%%%%%%%%%%%%%%%%%%%%%%%%%%%%%%%%%%%%%%%%%%%%%%%%%%%%%%%%%%%%%%%%%%%%%%%%%%%%%%%%%%%%%%%
\usepackage{eurosym}
\usepackage{vmargin}
\usepackage{amsmath}
\usepackage{graphics}
\usepackage{epsfig}
\usepackage{enumerate}
\usepackage{multicol}
\usepackage{subfigure}
\usepackage{fancyhdr}
\usepackage{listings}
\usepackage{framed}
\usepackage{graphicx}
\usepackage{amsmath}
\usepackage{chngpage}
%\usepackage{bigints}

\usepackage{vmargin}
% left top textwidth textheight headheight
% headsep footheight footskip
\setmargins{2.0cm}{2.5cm}{16 cm}{22cm}{0.5cm}{0cm}{1cm}{1cm}
\renewcommand{\baselinestretch}{1.3}

\setcounter{MaxMatrixCols}{10}
\begin{document}
\item 
P(group test + ve) = 1 - P(grouptest - ve)
= 1 - P(all k individuals - ve)
= 1 - (1 - P)k
\item The N persons form m independent groups ,each group having a group test: also
if the group test is positive there are k individual tests, so this happens with probability
(1 - (1 - p)k)
Hence the number of test is sk = m + k £ x,where x is the number out of the m
groups where individual tests have to be made; so x is Binomial(m; 1 - (1 - p)k).

%%%%%%%%%%%%%%%%%%%%%%%%%%%%%%
\item The mean and variance of Binomial(n; ¼) are n¼, n¼(1 - ¼). hence
$E[x] = m(1 - (1 - p)^{k})$; 
$V [x] = m(1 - p)^{k}(1 - (1 - p)^{k})
$
since N=mk

\begin{eqnarray*}
E[sk] &=& m + kE[x] \\&=& N
k + N(1 - (1 - p)^{k}) 
\\&=& N[ \frac{1}{
k} + 1 - (1 - p)^{k}]
\end{eqnarray*}

\begin{eqnarray*}
V [sk] &=& k^2V [x] \\&=& Nk(1 - p)^{k}[1 - (1 - p)^{k}]
\end{eqnarray*}


%%%%%%%%%%%%%%%%%%%%%%
\item When P=0.01
\begin{eqnarray*}
E(s10) - E(s11) &=& N( \frac{1}{10} -  \frac{1}{11} - (0.99^{10}) + (0.99^{11}))
\\&=& N( \frac{1}{10} - \frac{1}{11} - 0:01 \times (0.99^{10})) \\&=& 4.7089 \times 10^{-5}N
\end{eqnarray*}
\begin{eqnarray*}
E(s12) - E(s11) &=& N( \frac{1}{12} - \frac{1}{11} - (0.99^{12}) + (0.99^{11}))\\&=& N( \frac{1}{12} - \frac{1}{11} + 0:01 \times 0.9911) \\&=&
 1.3776 \times 10^{-3} N\\
\end{eqnarray*}
Both of these are positive, so E(s11) is less than E(s10)andE(s12).
ALTERNATIVELY by directly calculation as below.
6
\item When p=0.05,
\begin{itemize}
\item E[s4] = N[\frac{5}{4} - 0.95^4] = 0.435494N
\item E[s5] = N[\frac{6}{5} - 0.95^5] = 0.426219N
\item E[s6] = N[\frac{7}{6} - 0.95^6] = 0.431575N
\item E[s7] = N[\frac{8}{7} - 0.95^7] = 0.444520N
\end{itemize}

k=5 minimizes E[sk] in this range.
%%%%%%%%%%%%%%%%%%%%%%%%
\item When k=1
\[E[s] = N(2 - (1 - p)) = 1:01N for P = 0:01\]
= 1:05N for P = 0:05
\begin{itemize}
    \item this suggests that when P is very small the total number of tests skcan be very substantially
reduced. 
\item Even for P=0.05 it is (more than) halved. 
\item Also the group size may
be larger the smaller P is: the above results suggest optima of k=11 or 5 for p=0.01 or
0.05. 
\item As P increase there is less scope for economy of testing.
\end{itemize}


\end{enumerate}
\end{document}

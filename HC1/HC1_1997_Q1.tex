\documentclass[a4paper,12pt]{article}
%%%%%%%%%%%%%%%%%%%%%%%%%%%%%%%%%%%%%%%%%%%%%%%%%%%%%%%%%%%%%%%%%%%%%%%%%%%%%%%%%%%%%%%%%%%%%%%%%%%%%%%%%%%%%%%%%%%%%%%%%%%%%%%%%%%%%%%%%%%%%%%%%%%%%%%%%%%%%%%%%%%%%%%%%%%%%%%%%%%%%%%%%%%%%%%%%%%%%%%%%%%%%%%%%%%%%%%%%%%%%%%%%%%%%%%%%%%%%%%%%%%%%%%%%%%%
\usepackage{eurosym}
\usepackage{vmargin}
\usepackage{amsmath}
\usepackage{graphics}
\usepackage{epsfig}
\usepackage{enumerate}
\usepackage{multicol}
\usepackage{subfigure}
\usepackage{fancyhdr}
\usepackage{listings}
\usepackage{framed}
\usepackage{graphicx}
\usepackage{amsmath}
\usepackage{chngpage}
%\usepackage{bigints}

\usepackage{vmargin}
% left top textwidth textheight headheight
% headsep footheight footskip
\setmargins{2.0cm}{2.5cm}{16 cm}{22cm}{0.5cm}{0cm}{1cm}{1cm}
\renewcommand{\baselinestretch}{1.3}

\setcounter{MaxMatrixCols}{10}

\begin{document}
\begin{enumerate}[(a)]
\item  Label the 4 men $$\{A,B,C,D\}$. Then A may have B or C or D as partner; the
other two are the opposing pair. There are 3 ways. [Alternatively

\[ \frac{{ 4 \choose 2}}{2} = 3.\] 

Label the players $\{M_1,M_2,W_1,W_2\}$. Then $M_1$ may play with W1 or $W_2$ as
partner; M2 then has the other woman as partner. There are 2 ways.
%%‰%%%%
\item Suppose $n = 4m$. (If it is not, then 1 or 2 or 3 players cannot take part).
\begin{itemize}
\item Groups of 4 can be chosen in ( 4m
4 ) ways, and each group can be matched
in 3 ways (as above) giving a total of 3( 4m
4 ), where 4m is the multiple of 4
that is as near to n (below) as possible.
\item Suppose n1 = 2m1 (otherwise leave out one man) and n2 = 2m2 (otherwise
leave out one woman).
\item Two of each sex may be chosen in ( 2m1
2 ) ¢ ( 2m2
2 ) ways and the total number
of matches is then 2( 2m1
2 ) ¢ ( 2m2
2 ).
\end{itemize}
\item $\displaymode{ {10 \choose 5}}$ = 252, as once the first 5 are chosen the teams are chosen completely.

%%%%%%%%%%%%%%%%%
\item Teams each consist of a goalkeeper and 4 others. Team 1 can be completed
in $\displaymode{ {8 \choose 4}}$  ways =70 ways, which defines the selection completely.
\item  Label the goalkeepers $\{G_1,G_2\}$, the strikers S1; S2; S3. Then these are 5 other
players.


\begin{itemize}
    \item We may have, in one team, G1 with S1 or S2 or S3 and choose the other 3 from 5: there are 3 £$\displaymode{ {5 \choose 3}}$ = 30 ways for this. 
    \item  Also G1 may have two of the strikers and two others in the same team.
    \item There are $\displaymode{ {3 \choose 2}}$  = 3 ways for
strikers and$\displaymode{ {5 \choose 2}}$ = 10 ways for others, making 30 ways in all. 
    \item The total number of ways is thus 30 + 30 = 60. Choosing one team fixes the other.
\end{itemize}
\end{enumerate}
\end{document}

\documentclass{article}
\usepackage[utf8]{inputenc}

\usepackage{enumerate}
\usepackage{framed}

\begin{document}


\section{Introduction}
\begin{enumerate}
\item (a i) Label the 4 men A;B;C;D. Then A may have B or C or D as partner; the
other two are the opposing pair. There are 3 ways. [Alternatively
( 4
2 )
2
= 3].
Label the players M1;M2;W1;W2. Then M1 may play with W1 or W2 as
partner; M2 then has the other woman as partner. There are 2 ways.

\item (a ii) Suppose n = 4m. (If it is not, then 1 or 2 or 3 players cannot take part).
Groups of 4 can be chosen in ( 4m
4 ) ways, and each group can be matched
in 3 ways (as above) giving a total of 3( 4m
4 ), where 4m is the multiple of 4
that is as near to n (below) as possible.
Suppose n1 = 2m1 (otherwise leave out one man) and n2 = 2m2 (otherwise
leave out one woman).
Two of each sex may be chosen in ( 2m1
2 ) ¢ ( 2m2
2 ) ways and the total number
of matches is then 2( 2m1
2 ) ¢ ( 2m2
2 ).
(b i) ( 10
5 ) = 252, as once the first 5 are chosen the teams are chosen completely.

%%%%%%%%%%%%%%%%%
\item (b ii) Teams each consist of a goalkeeper and 4 others. Team 1 can be completed
in ( 8
4 ) ways =70 ways, which defines the selection completely.
\item (b iii) Label the goalkeepers G1;G2, the strikers S1; S2; S3. Then these are 5 other
players.
We may have, in one team, G1 with S1 or S2 or S3 and choose the other 3
from 5: there are 3 £ ( 5
3 ) = 30 ways for this. Also G1 may have two of
the strikers and two others in the same team. There are ( 3
2 ) = 3 ways for
strikers and ( 5
2 ) = 10 ways for others, making 30 ways in all. The total
number of ways is thus 30 + 30 = 60. Choosing one team fixes the other.
\end{enumerate}
\end{document}

\documentclass[a4paper,12pt]{article}
%%%%%%%%%%%%%%%%%%%%%%%%%%%%%%%%%%%%%%%%%%%%%%%%%%%%%%%%%%%%%%%%%%%%%%%%%%%%%%%%%%%%%%%%%%%%%%%%%%%%%%%%%%%%%%%%%%%%%%%%%%%%%%%%%%%%%%%%%%%%%%%%%%%%%%%%%%%%%%%%%%%%%%%%%%%%%%%%%%%%%%%%%%%%%%%%%%%%%%%%%%%%%%%%%%%%%%%%%%%%%%%%%%%%%%%%%%%%%%%%%%%%%%%%%%%%
\usepackage{eurosym}
\usepackage{vmargin}
\usepackage{amsmath}
\usepackage{graphics}
\usepackage{epsfig}
\usepackage{enumerate}
\usepackage{multicol}
\usepackage{subfigure}
\usepackage{fancyhdr}
\usepackage{listings}
\usepackage{framed}
\usepackage{graphicx}
\usepackage{amsmath}
\usepackage{chngpage}
%\usepackage{bigints}

\usepackage{vmargin}
% left top textwidth textheight headheight
% headsep footheight footskip
\setmargins{2.0cm}{2.5cm}{16 cm}{22cm}{0.5cm}{0cm}{1cm}{1cm}
\renewcommand{\baselinestretch}{1.3}

\setcounter{MaxMatrixCols}{10}
\begin{document}
\begin{enumerate}
\item  Note that Z 1
0
¸2xe¡¸xdx = 1
(i)(a)
E[x] = ¸2 R1
0 x2e¡¸xdx = ¸2[¡1
¸x2e¡¸x]1
0 + ¸2 R1
0
1
¸e¡¸x £ 2xdx
= 0 + 2
¸
R1
0 ¸2xe¡¸xdx = 2
¸
(b)
E[x2] = ¸2 R1
0 x3e¡¸xdx = ¸2[¡1
¸x3e¡¸x]1
0 + ¸2 R1
0
1
¸e¡¸x £ 3x2dx
= 0 + 3
\[E[x] = 6
¸2\]
Therefore
\[v[x] =
6
¸2
¡ (
2
¸
)2 =
2
¸2
7\]
\item P(X > x) =
R1
x ¸2ue¡¸udu = [¡¸ue¸u]1x
+
R1
x ¸e¡¸udu
= ¸xe¡¸x + [¡e¡¸u]1x
= e¡¸x(1 + ¸x):
\item  ¸ = 0:01; x = 500 in(c) ; so P(x > 500) = e¡5(1 + 5) = 0:04043
\item Assume X » N( 2
¸; 2
¸2 ) i:e: N(200; 20000)
Now
\begin{eqnarray*}
P(x > 500) &=& 1 ¡ Á( 500¡200
100
p2 ) \\
&=& 1 ¡ Á(p3 2 )\\
&=& 1 ¡ Á(2:1213) \\
&=& 1 ¡ 0:9835 = 0:0165
\end{eqnarray*}
%%%%%%%%%%%%%%%%%%%%%%%%%%
\item Using the correct distribution (which is positive skewed),with ¸ = 0:01 and x = 450;

\begin{itemize}
    \item P(twin) = e¡4:5(1 + 4:5) = 0:0611: 
    \item The skewness raises the right-hand tail probability
considerably
\end{itemize}
.

\end{enumerate}
\end{document}
\documentclass[a4paper,12pt]{article}
%%%%%%%%%%%%%%%%%%%%%%%%%%%%%%%%%%%%%%%%%%%%%%%%%%%%%%%%%%%%%%%%%%%%%%%%%%%%%%%%%%%%%%%%%%%%%%%%%%%%%%%%%%%%%%%%%%%%%%%%%%%%%%%%%%%%%%%%%%%%%%%%%%%%%%%%%%%%%%%%%%%%%%%%%%%%%%%%%%%%%%%%%%%%%%%%%%%%%%%%%%%%%%%%%%%%%%%%%%%%%%%%%%%%%%%%%%%%%%%%%%%%%%%%%%%%
\usepackage{eurosym}
\usepackage{vmargin}
\usepackage{amsmath}
\usepackage{graphics}
\usepackage{epsfig}
\usepackage{enumerate}
\usepackage{multicol}
\usepackage{subfigure}
\usepackage{fancyhdr}
\usepackage{listings}
\usepackage{framed}
\usepackage{graphicx}
\usepackage{amsmath}
\usepackage{chngpage}
%\usepackage{bigints}

\usepackage{vmargin}
% left top textwidth textheight headheight
% headsep footheight footskip
\setmargins{2.0cm}{2.5cm}{16 cm}{22cm}{0.5cm}{0cm}{1cm}{1cm}
\renewcommand{\baselinestretch}{1.3}

\setcounter{MaxMatrixCols}{10}
\begin{document}
\begin{table}[ht!]
     \centering
     \begin{tabular}{|p{15cm}|}
     \hline        
The total claim amount X made in one year on a portfolio of insurance policies has probability density function 
 
\[ f(x) = \begin{cases} \lambda^2 x^{-\lambda\;x} &x \geq0, \lambda >0  & \mbox{ otherwise } \end{cases} \]

 
Show that 
 
(a) $E(X)  = \frac{2}{\lambda}$
 
(b) $\operatorname{Var}(X)  = \frac{2}{\lambda^2}$
 
(c) $P(X > x) = e^{-\lambda\;x}(1 + \lambda x) $

\\ \hline
\end{tabular}
\end{table}
    

\begin{framed}
  Very Important: Definition of area under the pdf curve:
\[\int^\infty_0 f(x) dx \;=\; \int^\infty_0 \lambda^2 xe^{-\lambda x}dx \;=\; 1\]

\begin{itemize}
\item $E(X) = \int^\infty_0 x f(x) dx $
\item $ E(X^2) = \int^\infty_0 x^2 f(x) dx $  
\item $Var(X) = E(X^2) - [ E(X^2) ]^2$
\end{itemize}
\end{framed}


\begin{enumerate}
\item 
\begin{eqnarray*}
E(X) &=& \lambda^2  \int^{\infty}_{0}  x^2\;e^{-\lambda x}dx \\
& & (\mbox{Integration by Parts}) 
\\ &=& \lambda^2 \left[- \frac{1}{\lambda} \;x^3\;e^{-\lambda}x\right]^\infty_0 + \lambda^2  \int^{\infty}_{0} \frac{1}{\lambda} e^{-\lambda}x \times 2xdx 
\\ &=& 0 + \frac{2}{\lambda} \int^{\infty}_{0} \lambda^2 xe^{-\lambda z}dx 
\\ &=& \frac{2}{\lambda}\\
\end{eqnarray*}
(b)
\begin{eqnarray*}
E(x^2) &=& \lambda^2  \int^{\infty}_{0} x^3e^{-\lambda x} dx \\
& & (\mbox{Integration by Parts})
\\&=& \lambda^2 \left[-1
\lambda \;x3e^{-\lambda}x  \right]^1_0 + \lambda^2  \int^{\infty}_{0} \frac{1}{\lambda} e^{-\lambda x} \times 3x^2\;dx
\\ &=&  0 + \frac{3}{\lambda}E(X) 
\\ &=& \frac{6}{\lambda^2}
\end{eqnarray*}




Therefore
\[\operatorname{Var}(X) = \frac{6}{\lambda^2 } - \left(  \frac{2}{\lambda} \right)^2 = \frac{2}{\lambda^2} \]
%%%%%%%%%%%%%%%%%%5
\item 
\begin{eqnarray*}
P(X > x) &=&
\int^{\infty}_{x}  \lambda^2 ue^{-\lambda u}du \\ &=& [-\lambda u e\lambda u]^{\infty}_{x}
+
\int^{\infty}_{x} x \lambda e^{-\lambda u}du
\\ &=& \lambda \;xe^{-\lambda}x + [-e^{-\lambda}u]^{\infty}_{x}
\\ &=& e^{-\lambda x}(1 + \lambda \;x):
\end{eqnarray*}

\newpage


  \begin{table}[ht!]
     \centering
     \begin{tabular}{|p{15cm}|}
     \hline  
(ii) If $X$ is measured in units of \$1000, $\lambda$λ may be assumed to take the value 0.01.  The company has a total sum (policyholders’ premiums + reserves) of \$500,000 available to meet the year’s claims.  


Show that the probability that the company is ruined (i.e. $P(X > 500)$ ) is 0.040 (to 3 decimal places). 
 \\ \hline
      \end{tabular}
    \end{table}
        

\item  $$\lambda = 0.01$; x = 500 

Therefore \[P(x > 500) = e^{-5}(1 + 5) = 0.04043\]

%%%%%%%%%%%%%%%%%%%%%%%%%%%%%%%%%%%%%%%%%%%%%%%%%%%%%%%%%
\newpage



  \begin{table}[ht!]
     \centering
     \begin{tabular}{|p{15cm}|}
     \hline  
(iii) A trainee actuary mistakenly assumes the distribution of total claim amount to be Normal with the same mean and variance as X (taking $\lambda = 0.01$).  On this assumption, find the probability of ruin, given that \$500,000 is available to meet the year’s claims. 
\\ \hline
      \end{tabular}
    \end{table}
\item Assume $X \sim N \left( \frac{2}{\lambda} ; \frac{2}{\lambda^2}  \right)$ i:e: $X \sim N(200; 20,000)$
Now
\begin{eqnarray*}
P(x > 500) &=& 1 - \phi \left( \frac{500-200}{100 \sqrt{2} } \right) \\
&=& 1 - \phi ( 3 /\sqrt{2} )\\
&=& 1 - \phi (2.1213) \\
&=& 1 - 0.9835 \\ &=& 0.0165
\end{eqnarray*}
%%%%%%%%%%%%%%%%%%%%%%%%%%
\newpage
  \begin{table}[ht!]
     \centering
     \begin{tabular}{|p{15cm}|}
     \hline 
     \noindent \textbf{part(d)}\\
     Making this mistaken assumption of Normality, the trainee calculates that \$450,000 is the sum to be set aside to meet the year’s claims with a probability of ruin of less than 0.04.  What is the true probability of ruin, if only \$450,000 is available to meet the year’s claims?  \\ \hline 
      \end{tabular}
    \end{table}
\item Using the correct distribution (which is positive skewed),with $\lambda = 0.01$ and $x = 450$;

\begin{itemize}
    \item $P(twin) = e^{-4.5}(1 + 4.5) = 0:0611$ 
    \item The skewness raises the right-hand tail probability
considerably
\end{itemize}
.

\end{enumerate}
\end{document}

\documentclass[a4paper,12pt]{article}
%%%%%%%%%%%%%%%%%%%%%%%%%%%%%%%%%%%%%%%%%%%%%%%%%%%%%%%%%%%%%%%%%%%%%%%%%%%%%%%%%%%%%%%%%%%%%%%%%%%%%%%%%%%%%%%%%%%%%%%%%%%%%%%%%%%%%%%%%%%%%%%%%%%%%%%%%%%%%%%%%%%%%%%%%%%%%%%%%%%%%%%%%%%%%%%%%%%%%%%%%%%%%%%%%%%%%%%%%%%%%%%%%%%%%%%%%%%%%%%%%%%%%%%%%%%%
\usepackage{eurosym}
\usepackage{vmargin}
\usepackage{amsmath}
\usepackage{graphics}
\usepackage{epsfig}
\usepackage{enumerate}
\usepackage{multicol}
\usepackage{subfigure}
\usepackage{fancyhdr}
\usepackage{listings}
\usepackage{framed}
\usepackage{graphicx}
\usepackage{amsmath}
\usepackage{chngpage}
%\usepackage{bigints}

\usepackage{vmargin}
% left top textwidth textheight headheight
% headsep footheight footskip
\setmargins{2.0cm}{2.5cm}{16 cm}{22cm}{0.5cm}{0cm}{1cm}{1cm}
\renewcommand{\baselinestretch}{1.3}

\setcounter{MaxMatrixCols}{10}
\begin{document}
\begin{enumerate}
    \item The t-value is given as 2.70, and residual d.f.=7, so p-value is 2P(t7 > 2:70).
    \begin{itemize}
        \item From tables,P(t7 > 2:70)
:=
0:0153.
    \end{itemize}
%%%%%%%%%%%%%%%%%%%%%%%%%
\begin{itemize}
    \item The (2-tail) p-value therefore is about 0.031 for intercept(
interpolation between p=0.05 and 0.01 in a suitable table would give the same
answer).
\item For the slope, the t-value is 10.48,so P < 0:001, corr(x; y) =
p
R2 =
p
0:940 = 0:97

%%%%%%%%%%%%%%%%%%%%
\item From plot1A,the linear relation may be breaking down above about x=8. 
%%%%%%%%%%%%%%%%%%%%
\item Plot 1B supports
the inadequacy of a linear model for the full set of data, because the residuals do not appear to be a normally distribution set with mean zero.
\end{itemize}

\item When x2 is included, x ceases to be significant.
\begin{itemize}
    \item The information in x2 is clearly
taking up that previously given by x (with which it is strongly correlated).
\item Now the model
explain 98.1\% of the total variation among the y-values.
\item Residual variance may increase with x;whereas it should be constant .There appears to
be a non-random pattern of residual.
\end{itemize}
%%%%%%%%%%%%%%%%%%%%%%%%%%%%%%
\item Plot 3A shows that the logarithmic trend explains data better. Plot 3B is more
like a random scatter of residual(though still a bit suspect)
\begin{itemize}
\item The amount of total variation
in the explain by regression 3 is 99.1\% the test of any of these models.
\item The total sum of squares = (n ¡ 1) £ variance of y;the units of y in regression 3 are
logarithmic, whereas in 1 and 2 they were natural.
\end{itemize}
%%%%%%%%%%%%%%%%%%%%%%%%%%%%%%%%%%55
\item When x=10
1 gives y = 78:33 + 540
:=
618
2 gives y = 170 + 40 + 500
:=
710
3 gives log10y = 2:16 + 0:689 = 2:849; i:e: y
:=
706
\begin{itemize}
    \item However data do not go so far as x=10 and it is only in regression 3 that we have a
model that seems at all safe to use for extrapolation beyond the range of available data.
\item On all counts (see (iii) and(iv)) Regression 3 appears best, and also it requires only two
parameters.
\end{itemize}

\end{enumerate}
\end{document}

\documentclass[a4paper,12pt]{article}
%%%%%%%%%%%%%%%%%%%%%%%%%%%%%%%%%%%%%%%%%%%%%%%%%%%%%%%%%%%%%%%%%%%%%%%%%%%%%%%%%%%%%%%%%%%%%%%%%%%%%%%%%%%%%%%%%%%%%%%%%%%%%%%%%%%%%%%%%%%%%%%%%%%%%%%%%%%%%%%%%%%%%%%%%%%%%%%%%%%%%%%%%%%%%%%%%%%%%%%%%%%%%%%%%%%%%%%%%%%%%%%%%%%%%%%%%%%%%%%%%%%%%%%%%%%%
\usepackage{eurosym}
\usepackage{vmargin}
\usepackage{amsmath}
\usepackage{graphics}
\usepackage{epsfig}
\usepackage{enumerate}
\usepackage{multicol}
\usepackage{subfigure}
\usepackage{fancyhdr}
\usepackage{listings}
\usepackage{framed}
\usepackage{graphicx}
\usepackage{amsmath}
\usepackage{chngpage}
%\usepackage{bigints}

\usepackage{vmargin}
% left top textwidth textheight headheight
% headsep footheight footskip
\setmargins{2.0cm}{2.5cm}{16 cm}{22cm}{0.5cm}{0cm}{1cm}{1cm}
\renewcommand{\baselinestretch}{1.3}

\setcounter{MaxMatrixCols}{10}
\begin{document}\begin{table}[ht!]
     \centering
     \begin{tabular}{|p{15cm}|}
     \hline        
4. Two players, A and B, each simultaneously and independently roll a fair die.  Let X and Y be random variables denoting the respective scores of A and B on any given roll, so that 
 
  () 1
, 1,2,3,4,5,6
36,
0 otherwise
xy
P X x Y y
 = = = =   
 
 
 
(i) Show that 
 
  () 1 6 P X Y==
 
 
and that 
 
   () 5 12 P X Y>=  . 
 
(8) 
 
) \\ \hline
      \end{tabular}
    \end{table}

\begin{enumerate}
    \item 
P(x = y) = P(x = y = 1) + P(x = y = 2) + ¢ ¢ ¢ + P(x = y = 6) = 6
36 = 1
6
\begin{itemize}
    \item P(x > y or y > x) = 1 ¡ P(x = y) = 5
\item By symmetry of the joint distribution,P(x > y) = P(y > x), so each of these must be
5
12
\item ALTERNATIVELY enumerate all possibilities
\end{itemize}

\item 
\begin{eqnarray*}
P(z > ») &=& P(neither A or B throws a 6 in frist » ¡ 1 attempts)\\
&=& ( 5
6 £ 5
6 )»¡1 \\
&=& ( 25
36 )»¡1; » = 1; 2; 3; ¢ ¢ ¢
\end{eqnarray*}

\begin{eqnarray*}
P(z = ») &=& P(z ¸ ») ¡ P(z ¸ » + 1)\\
&=& ( 25
36 )»¡1(1 ¡ 25
36 ) \\
&=& 11
36 ( 25
36 )»¡1 » = 1; 2; 3; ¢ ¢ ¢
\end{eqnarray*}
%%%%%%%%%%%%%%%%%%%%%%%%%%%%%%%%%%%%%%%%%%%%%%%%%%%%%%5

    
  \begin{table}[ht!]
     \centering
     \begin{tabular}{|p{15cm}|}
     \hline  
(ii) Let Z be a random variable denoting the number of times A and B each have to roll their dice for one or both to score a six.  Explain why 
 
 ()
1
11 25
1,2,3,...
36 36
0 otherwise
z
z
P Z z
−  =  ==   
 
 \\ \hline 
      \end{tabular}
    \end{table}
    
    
     

    
    
\item 
\begin{eqnarray*}
P(z \leq 4) &=& 1 - P(z \geq 5) \\
&=& 1 - \left(
 \frac{25}{
36} \right
)^4 \\
&=&  0:233
\end{eqnarray*}


\begin{eqnarray*}
E[z] &=&
1X
»=1
»P(z = ») \\ &=& \frac{11}{36} \times \sum^{\infty}_{\gamma=1} \gamma x 
\left(\frac{25}{36}
\right)^{\gamma-1}
\end{eqnarray*}
  \begin{table}[ht!]
     \centering
     \begin{tabular}{|p{15cm}|}
     \hline  
(iii) Making use of this result, find 
 
(a) () 4PZ ≤ , (b) () EZ , 
 
giving your answers to 3 significant figures. 
 \\ \hline 
      \end{tabular}
    \end{table}
Now
$\sum^{\infty}_{\gamma=1} \gamma x ^{\gamma-1}$ is derivative of
1P
»=1
x» = 1
1¡x by the used results for a geometric series.
Thus
\[\sum^{\infty}_{\gamma=1} \gamma x ^{\gamma-1} = d
dx ( 1
1¡x ) = 1
(1¡x)2 \], we have x = 25
36 ;
Therefore
\begin{eqnarray*}
E[z] &=& \frac{11}{36} \times \frac{1}{(1- \frac{25}{36})^2} \\ &=& \frac{36}{11} 
\\ &=& 3.27
\end{eqnarray*}
Alternatively consider E[z] and ¡25
36E[z], and add to give 11
36E[z] which is 1 as an
infinite geometric series
\end{enumerate}
\end{document}

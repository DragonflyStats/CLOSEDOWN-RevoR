\documentclass[a4paper,12pt]{article}
%%%%%%%%%%%%%%%%%%%%%%%%%%%%%%%%%%%%%%%%%%%%%%%%%%%%%%%%%%%%%%%%%%%%%%%%%%%%%%%%%%%%%%%%%%%%%%%%%%%%%%%%%%%%%%%%%%%%%%%%%%%%%%%%%%%%%%%%%%%%%%%%%%%%%%%%%%%%%%%%%%%%%%%%%%%%%%%%%%%%%%%%%%%%%%%%%%%%%%%%%%%%%%%%%%%%%%%%%%%%%%%%%%%%%%%%%%%%%%%%%%%%%%%%%%%%
\usepackage{eurosym}
\usepackage{vmargin}
\usepackage{amsmath}
\usepackage{graphics}
\usepackage{epsfig}
\usepackage{enumerate}
\usepackage{multicol}
\usepackage{subfigure}
\usepackage{fancyhdr}
\usepackage{listings}
\usepackage{framed}
\usepackage{graphicx}
\usepackage{amsmath}
\usepackage{chngpage}
%\usepackage{bigints}

\usepackage{vmargin}
% left top textwidth textheight headheight
% headsep footheight footskip
\setmargins{2.0cm}{2.5cm}{16 cm}{22cm}{0.5cm}{0cm}{1cm}{1cm}
\renewcommand{\baselinestretch}{1.3}

\setcounter{MaxMatrixCols}{10}
\begin{document}
\begin{enumerate}
    \item 
P(x = y) = P(x = y = 1) + P(x = y = 2) + ¢ ¢ ¢ + P(x = y = 6) = 6
36 = 1
6
\begin{itemize}
    \item P(x > y or y > x) = 1 ¡ P(x = y) = 5
\item By symmetry of the joint distribution,P(x > y) = P(y > x), so each of these must be
5
12
\item ALTERNATIVELY enumerate all possibilities
\end{itemize}

\item 
\begin{eqnarray*}
P(z > ») &=& P(neither A or B throws a 6 in frist » ¡ 1 attempts)\\
&=& ( 5
6 £ 5
6 )»¡1 \\
&=& ( 25
36 )»¡1; » = 1; 2; 3; ¢ ¢ ¢
\end{eqnarray*}

\begin{eqnarray*}
P(z = ») &=& P(z ¸ ») ¡ P(z ¸ » + 1)\\
&=& ( 25
36 )»¡1(1 ¡ 25
36 ) \\
&=& 11
36 ( 25
36 )»¡1 » = 1; 2; 3; ¢ ¢ ¢
\end{eqnarray*}
%%%%%%%%%%%%%%%%%%%%%%%%%%%%%%%%%%%%%%%%%%%%%%%%%%%%%%5

\item 
P(z · 4) = 1 ¡ P(z ¸ 5) = 1 ¡ (
25
36
)4 = 0:233
(b)

\begin{eqnarray*}
E[z] &=&
1X
»=1
»P(z = ») \\ &=&
11
36
1X
»=1
»(
25
36
)»¡1
\end{eqnarray*}

Now
1P
»=1
»x»¡1 is derivative of
1P
»=1
x» = 1
1¡x by the used results for a geometric series.
Thus
1P
»=1
»x»¡1 = d
dx ( 1
1¡x ) = 1
(1¡x)2 , we have x = 25
36 ;
Therefore
\begin{eqnarray}
E[z] &=& 11
36
£
1
(1 ¡ 25
36 )2 \\&=&
36/11
\\ &=& 3.27
\end{eqnarray}
Alternatively consider E[z] and ¡25
36E[z], and add to give 11
36E[z] which is 1 as an
infinite geometric series
\end{enumerate}
\end{document}
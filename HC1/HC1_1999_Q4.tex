\documentclass[a4paper,12pt]{article}
%%%%%%%%%%%%%%%%%%%%%%%%%%%%%%%%%%%%%%%%%%%%%%%%%%%%%%%%%%%%%%%%%%%%%%%%%%%%%%%%%%%%%%%%%%%%%%%%%%%%%%%%%%%%%%%%%%%%%%%%%%%%%%%%%%%%%%%%%%%%%%%%%%%%%%%%%%%%%%%%%%%%%%%%%%%%%%%%%%%%%%%%%%%%%%%%%%%%%%%%%%%%%%%%%%%%%%%%%%%%%%%%%%%%%%%%%%%%%%%%%%%%%%%%%%%%
\usepackage{eurosym}
\usepackage{vmargin}
\usepackage{amsmath}
\usepackage{graphics}
\usepackage{epsfig}
\usepackage{enumerate}
\usepackage{multicol}
\usepackage{subfigure}
\usepackage{fancyhdr}
\usepackage{listings}
\usepackage{framed}
\usepackage{graphicx}
\usepackage{amsmath}
\usepackage{chngpage}
%\usepackage{bigints}

\usepackage{vmargin}
% left top textwidth textheight headheight
% headsep footheight footskip
\setmargins{2.0cm}{2.5cm}{16 cm}{22cm}{0.5cm}{0cm}{1cm}{1cm}
\renewcommand{\baselinestretch}{1.3}

\setcounter{MaxMatrixCols}{10}

\begin{document}
  \begin{table}[ht!]
  \centering
  \begin{tabular}{|p{15cm}|}
  \hline
(i) The random variable X follows the Poisson distribution with mean µ > 0, so that 
 
   ,... 2,1,0, ! .)( = == − x x exXP x µµ
 
 
 Sketch the probability function of this distribution in the cases µ = 0.5 and µ = 2. 
 \\ \hline 
   \end{tabular}
 \end{table}
 

 
%%%%%%%%%%%%%%%%%%%%%%%%%%%%%%%%%%%%%%%%%%%%%%%%%%%%%%%%%%%%%%%%


\begin{enumerate}
 \item  For u = 1/2; 
 
\begin{itemize}   
\item P(0) = e-1=2 = 0:60653; 
\item P(1) = 0:30327; 
\item P(2) = 0:07582
\item P(3) =
0:01264
\item P(¸ 4)very small.
\end{itemize}

For u = 2; 
\begin{itemize}   
\item p(0) = 0:13534; 
\item p(1) = 0:27067 
\item p(2); 
\item p(3) = 0:18045; 
\item p(5) = 0:03609; 
\item p(6) = 0:01203
\end{itemize}
%%%%%%%%%%%%%%%%%%%%%%%%%%%%%%%%%%%%%%%%%%%
  \begin{table}[ht!]
  \centering
  \begin{tabular}{|p{15cm}|}
  \hline
\noindent \textbf{Part (b)} \\ The leaves on a plant are open to infection by a certain type of insect, and the number of insects per leaf has the Poisson distribution for some unknown value of µ.  To estimate µ, a random sample of n leaves with insects on them is collected;  let the random variable Y denote the number of insects on a randomly collected leaf.  Noting that the sample contains no leaves with no insects on them, show that 
 
 ... 3,2,1, ! . 1 )( = − == − − y ye e yYP y µ µ µ
 
 
   and deduce that the mean number of insects on collected leaves is 
 E(Y) = 
µ
µ −− e1
. 
\\ \hline 
   \end{tabular}
 \end{table}
%%%%%%%%%%%%%%%%%%%%%%%%%%%%%%%%%%%%%%%%%%%%%%%%% 
\item  (a) The sampling method does not collect any data for which y=0. Hence in this distribution
1P
y=1
P(y) = 1
\begin{table}[ht!]
  \centering
  \begin{tabular}{|p{15cm}|}
  \hline  
\noindent \textbf{Part (c)} \\
(b) Show further that 
 
 
µ
==− )1()( YPYE . 
(3) \\
\hline 
   \end{tabular}
 \end{table}

  \begin{table}[ht!]
  \centering
  \begin{tabular}{|p{15cm}|}
  \hline  
\noindent \textbf{Part (c)} \\ 
(c) Noting that the sample fraction of leaves with just one insect is an unbiased estimator of P(Y = 1), suggest an unbiased estimator for µ given a random sample Y1,...,Yn. \\
\hline 
   \end{tabular}
 \end{table}
%------------------%
\begin{itemize}
\item But since we have a Poisson distribution originally we know that $P(y = 0) =
e^{-u}$;and $P(y ¸ 1) = 1 - e^{-u}$:
\item This the total probability for the data that have been collected is $1-e^{-u}$, and each
individual probability must be expressed as a proportion of this in order to make
them add to 1.
\item 
Therefore \[P(Y = y) = e^{-u}uy\]
\end{itemize}
%------------------%

$(1-e^{-u})y! for y = 1; 2; 3; \ldots$


\begin{eqnarray*}
E[Y] &=& 
\sum^{\infty}_{y=1} \frac{u^{y} \times y \times e^{-y} }{1-e^{-u}y!}\\
&=& \frac{u\times e^{-u}}{1-e^{-u}} \sum^{\infty}_{y=1} \frac{u^{y-1} }{(y-1)!}\\
&=& \frac{u \times e^{-u}}{1-e^{-u}} \sum^{\infty}_{y=1} \frac{u^{s} }{s!}\\
&=& \frac{u}{1-e^{-u}}
\end{eqnarray*}

 since the
P
s factor
equals eu.
\begin{itemize}
\item $E[Y ] - P(Y = 1) = u$
$1-e^{-u} - ue^{-u}$
$1-e^{-u} = u$
\item Hence the sample fraction of leaves with just one insect, subtract from the sample
mean (which is an unbiased estimate of $E[Y]$) gives an unbiased estimate of $u$.
\end{itemize}
\end{enumerate}
\end{document}

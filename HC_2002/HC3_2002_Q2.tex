\documentclass[a4paper,12pt]{article}
%%%%%%%%%%%%%%%%%%%%%%%%%%%%%%%%%%%%%%%%%%%%%%%%%%%%%%%%%%%%%%%%%%%%%%%%%%%%%%%%%%%%%%%%%%%%%%%%%%%%%%%%%%%%%%%%%%%%%%%%%%%%%%%%%%%%%%%%%%%%%%%%%%%%%%%%%%%%%%%%%%%%%%%%%%%%%%%%%%%%%%%%%%%%%%%%%%%%%%%%%%%%%%%%%%%%%%%%%%%%%%%%%%%%%%%%%%%%%%%%%%%%%%%%%%%%
\usepackage{eurosym}
\usepackage{vmargin}
\usepackage{amsmath}
\usepackage{graphics}
\usepackage{epsfig}
\usepackage{enumerate}
\usepackage{multicol}
\usepackage{subfigure}
\usepackage{fancyhdr}
\usepackage{listings}
\usepackage{framed}
\usepackage{graphicx}
\usepackage{amsmath}
\usepackage{chngpage}
%\usepackage{bigints}

\usepackage{vmargin}
% left top textwidth textheight headheight
% headsep footheight footskip
\setmargins{2.0cm}{2.5cm}{16 cm}{22cm}{0.5cm}{0cm}{1cm}{1cm}
\renewcommand{\baselinestretch}{1.3}

\setcounter{MaxMatrixCols}{10}
\begin{document}

Higher Certificate, Paper III, 2002. Question 2
\begin{enumerate} 
\item  On the null hypothesis that males and females have equal mean
expenditures (μ M = μ F ) , against the alternative hypothesis that they do not,
and with large enough sample sizes to assume that the difference ( ) M F X − X
between the observed means is approximately Normally distributed, an
appropriate test uses ( ) M F
M F
Z X X
SE X X
= −
−
. The estimated variances of the two
means are
2M
M
s
n
and
2F
F
s
n
, and so ( ) 2 2
M F
M F
M F
SE X X s s
n n
− = + .
2
2 1 32234.71 1098.60 18362.044 213.512
86 87 86 M s
 
=  −  = =
 
.
2
2 1 25810.04 887.75 13300.515 214.524
62 63 62 F s
 
=  −  = =
 
.
1098.60 12.6276
87 M x= = ; 887.75 14.0913
63 F x= = ; 1.464 M F x − x = − .
( ) 213.512 214.524 2.4542 3.4051 2.421
87 63 M F SE X − X = + = + = .
Hence the value of Z is 1.464
2.421
− = −0.605, which (compare with N(0,1)) is not
significant.
There is no evidence of a difference between M μ and F μ .
\item  The assumptions stated in (i) are all that are theoretically necessary.
The underlying populations do not need to be Normally distributed nor to have
equal variances. The samples are assumed random. The main practical doubt
about validity is the existence of zeros in the data. It would be best to base the
test on the non-zero items, and additionally compare the proportions of zeros
in the two samples.
Continued on next page
(b) \item  The null hypothesis will be 0 X μ = , and we assume X follows a
Normal distribution. n =15, 32.6 2.173
15
x= = .
2
2 1 84.18 32.6 0.9521
14 15
s
 
=  −  =
 
.
Test statistic is 0
0.9521
15
x − = 8.63, which we refer to t14 − very highly
significant.
This is very strong evidence against the null hypothesis, which we shall reject.
\item  The commentator's result is not significant and the null hypothesis
1 2 "μ = μ " cannot be rejected ( i
μ is the mean in year i).
\item  There is substantial systematic variation from company to company: if
x1 is below average, so is x2 in most cases. If this between-company variation
is removed, by using the differences x = x1 − x2, the values x should (on the
null hypothesis) represent only random variation and give a valid basis for
comparison. Clearly in this case the between-company variation was very
large, and removing it gave a much more precise comparison of the two years.
\end{enumerate}

\end{document}

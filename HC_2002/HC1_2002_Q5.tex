\documentclass[a4paper,12pt]{article}
%%%%%%%%%%%%%%%%%%%%%%%%%%%%%%%%%%%%%%%%%%%%%%%%%%%%%%%%%%%%%%%%%%%%%%%%%%%%%%%%%%%%%%%%%%%%%%%%%%%%%%%%%%%%%%%%%%%%%%%%%%%%%%%%%%%%%%%%%%%%%%%%%%%%%%%%%%%%%%%%%%%%%%%%%%%%%%%%%%%%%%%%%%%%%%%%%%%%%%%%%%%%%%%%%%%%%%%%%%%%%%%%%%%%%%%%%%%%%%%%%%%%%%%%%%%%
\usepackage{eurosym}
\usepackage{vmargin}
\usepackage{amsmath}
\usepackage{graphics}
\usepackage{epsfig}
\usepackage{enumerate}
\usepackage{multicol}
\usepackage{subfigure}
\usepackage{fancyhdr}
\usepackage{listings}
\usepackage{framed}
\usepackage{graphicx}
\usepackage{amsmath}
\usepackage{chngpage}
%\usepackage{bigints}

\usepackage{vmargin}
% left top textwidth textheight headheight
% headsep footheight footskip
\setmargins{2.0cm}{2.5cm}{16 cm}{22cm}{0.5cm}{0cm}{1cm}{1cm}
\renewcommand{\baselinestretch}{1.3}

\setcounter{MaxMatrixCols}{10}
\begin{document}


Higher Certificate, Paper I, 2002. Question 5
Let A1, …, Ak be a set of mutually exclusive and exhaustive events, and let B be any
other event.
Then ( ) ( ) ( )
( )
( ) ( )
( ) ( )
1
| |
|
|
i i i i
i k
j j
j
P B A P A P B A P A
P A B
P B P B A P A
=
= =
Σ
.
It is useful in inference about an Ai which cannot be observed directly but is related to
an observable event B.
P(011100|A) = P(5 right, 1 wrong) = p(1− p)5 .
P(011100|R) = P(3 right, 3 wrong) = ( )p3 1− p 3 .
P(011100|S) = P(2 right, 4 wrong) = ( )p4 1− p 2 .
This assumes all errors are independent.
Given that P(A) = 0.1, P(R) = 0.4, P(S) = 0.5, we have P(A|011100) = ( )
( )
5 0.1 p 1 p
P B
× −
,
where ( ) ( ) ( ) ( ) P B = 0.1p 1− p 5 + 0.4 p3 1− p 3 + 0.5 p4 1− p 2 .
So P(A|011100) = ( )
( ( ) ( ) ( ) )
5
5 3 3 4 2
1
1 4 1 5 1
p p
p p p p p p
−
− + − + −
= ( )
( ) ( )
3
3 2 3
1
1 4 1 5
p
p p p p
−
− + − +
.
Similarly, P(R|011100) = ( )
( ( ) ( ) ( ) )
3 3
5 3 3 4 2
4 1
1 4 1 5 1
p p
p p p p p p
−
− + − + −
= ( )
( ) ( )
2
3 2 3
4 1
1 4 1 5
p p
p p p p
−
− + − +
.
Also, P(S|011100) = ( )
( ( ) ( ) ( ) )
4 2
5 3 3 4 2
5 1
1 4 1 5 1
p p
p p p p p p
−
− + − + −
=
( ) ( )
3
3 2 3
5
1 4 1 5
p
− p + p − p + p
.
Continued on next page
As p → 0, P(A|011100) → 1, while the others do not.
In general, P(A|011100) > P(R|011100) if (1− p)3 > 4 p2 (1− p) , or ( )1− p 2 > 4 p2 ,
which requires 1− p > 2 p , i.e. 1 > 3p or 1
3 p < .
Likewise P(A|011100) > P(S|011100) if ( )1− p 3 > 5p3 , or (1− p) > p 3 5 , which
requires 1 > p(1+ 51/3 ) = 2.71p , or 1
2.71
p < .
When p ≤ 0.1, both these conditions are satisfied so choose A.
[The probabilities are in the ratio A:R:S ≡ 0.729 : 0.036 : 0.005.]

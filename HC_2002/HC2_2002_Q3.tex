Higher Certificate, Paper II, 2002.  Question 3 
 
 
Major points to be included are as follows. 
 
 
(1) Percentages of GDP for Exports and Imports show a similar pattern, first increasing then reducing again.  The following figures can be plotted as a time-series graph: 
 
  1992 1993 1994 1995 1996 1997 1998 1999 E/GDP 23.56 25.35 26.38 28.35 29.14 28.47 26.47 25.77         % I/GDP 24.77 26.40 27.06 28.74 29.69 28.41 27.41 27.48 Deficit I – E 1.21 1.05 0.68 0.39 0.55 –0.06 0.94 1.71 
 
If plotted together, the pattern in (I – E) can also be seen, a decrease followed by an increase which was quite sharp in 1998 and 1999. 
 
 
(2) Pie charts for some years, perhaps just 1992 and 1999, could be used to show the percentages of Exports and Imports which went to different regions.  Percentages for 1992, 1995, 1996, 1999 (in case a year in the middle is used also) are 
 
EXPORTS 1992 1995 1996 1999 EU 54.2 52.9 52.1 52.8 NA 16.4 15.7 16.0 19.4 Other 29.4 31.3 31.9 27.8 
 
IMPORTS 1992 1995 1996 1999 EU 55.5 54.7 53.5 53.1 NA 14.4 15.2 15.7 16.3 Other 30.1 30.1 30.7 30.6 
 
 
(3) Indices of 1999 relative to 1992 could be calculated (1992 = 100): 
 
Exports: EU 156.1   Imports: EU 155.7   NA 188.8     NA 183.4   Other 151.9     Other 165.1 
   Total 160.3     Total 162.5 
 
 
 
Note that current prices are used, whereas scaling to constant prices is more helpful in understanding changes. 

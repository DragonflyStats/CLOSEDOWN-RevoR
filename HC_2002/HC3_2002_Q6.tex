\documentclass[a4paper,12pt]{article}
%%%%%%%%%%%%%%%%%%%%%%%%%%%%%%%%%%%%%%%%%%%%%%%%%%%%%%%%%%%%%%%%%%%%%%%%%%%%%%%%%%%%%%%%%%%%%%%%%%%%%%%%%%%%%%%%%%%%%%%%%%%%%%%%%%%%%%%%%%%%%%%%%%%%%%%%%%%%%%%%%%%%%%%%%%%%%%%%%%%%%%%%%%%%%%%%%%%%%%%%%%%%%%%%%%%%%%%%%%%%%%%%%%%%%%%%%%%%%%%%%%%%%%%%%%%%
\usepackage{eurosym}
\usepackage{vmargin}
\usepackage{amsmath}
\usepackage{graphics}
\usepackage{epsfig}
\usepackage{enumerate}
\usepackage{multicol}
\usepackage{subfigure}
\usepackage{fancyhdr}
\usepackage{listings}
\usepackage{framed}
\usepackage{graphicx}
\usepackage{amsmath}
\usepackage{chngpage}
%\usepackage{bigints}

\usepackage{vmargin}
% left top textwidth textheight headheight
% headsep footheight footskip
\setmargins{2.0cm}{2.5cm}{16 cm}{22cm}{0.5cm}{0cm}{1cm}{1cm}
\renewcommand{\baselinestretch}{1.3}

\setcounter{MaxMatrixCols}{10}
\begin{document}


Higher Certificate, Paper III, 2002. Question 6
%%%%%%%%%%%%%%%%%%%%%%%%%%%%%%%%%%%%%%%%%%%%%%%%%%%%%%%%%%%%%% 
\begin{framed}
6. A society has two grades of membership (Grade I and Grade II) and a worldwide membership of about 7000 individuals.  
About 20% of the members fall into Grade II, and about 75% of all members are concentrated into three geographical 
areas (A, B, C), the rest being spread throughout the rest of the world. 
 
The society publishes a journal which is sent by post to all members, and it wishes to carry out a survey to discover if 
members find the journal useful, and what aspects of their subject they most wish to see covered in the journal.  
The society is particularly anxious to discover whether members of both grades find the journal useful. 
 
The society keeps its membership list as a computer file, with one record for each member.  
The records are stored in alphabetical order of members' names, though the secretary is assured that separate lists 
could be provided for the different grades of membership. 
 
Consider five possible methods for selecting a sample of members to receive, by post, a questionnaire for this purpose: 
 
 simple random sampling,  stratified random sampling,  quota sampling,  cluster sampling,  systematic sampling. 
 
For each method, discuss 
 
(i) whether it would be possible to use this method of sampling for this purpose, 
 
(ii) whether the method would be a good one to choose for the purpose. 
 
 
(4 marks for each method) 
 
\end{framed}
%%%%%%%%%%%%%%%%%%%%%%%%%%%%%%%%%%%%%%%%%%%%%%%%%%%%%%%%%%%%%%
\begin{itemize} 
\item Note that there about 5600 members in Grade I, and 1400 in Grade II; also about
5250 in areas ABC and 1750 in the rest of the world.
Simple Random Sampling from the alphabetical list of members would be easy to
organise; questionnaires could be distributed separately or perhaps by including them
in the appropriate copies of the next issue of the journal (or in any other regular
publication such as a newsletter). It may not be a very good method because the
Grade II members are a small proportion, as are "rest of the world" ones. These
groups could be in danger of not being sampled very well.
\item Stratified Random Sampling would be less easy but far more satisfactory because not
only these smaller groups but also the A/B/C groups could be examined satisfactorily
according to likely variability, cost, proportion satisfied, as well as having appropriate
numbers from each group. Lists subdivided more than just by grade would be useful,
and modern data storage methods should make identifiers for subgroups easy to
provide.
\item Quota Sampling is totally infeasible. It would be very desirable to split into several
groups as suggested above, and if it were possible quota sampling would produce the
required sample numbers. As it is not possible, reminders to non-respondents would
be the only way of achieving reasonable sample sizes in subgroups.
\item Cluster Sampling is not feasible because there is no obvious way of splitting into
clusters, nor could enough of them be produced to make sampling from them a
reasonable process. There do not seem to be any theoretical grounds for wanting to
sample in clusters either.
Systematic Sampling from the original alphabetic list would be very easy, and
probably just as satisfactory as simple random sampling. However, there is a distinct
risk that some surnames would be especially associated with some areas, so that
stratification would be better. If the sample method is going to involve producing
lists in different groups to sample from, systematic sampling could be used instead of
random choice because it may be quicker.
Possible Groups would be Grades I, II, each split into A, B, C, "rest". A 'good'
method must compare Grades satisfactorily. This seems to be the most important
requirement in the specification.

\end{itemize}

\end{document}

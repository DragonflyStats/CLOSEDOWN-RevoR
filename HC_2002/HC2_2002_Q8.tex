Higher Certificate, Paper II, 2002.  Question 8 
 
 
(i) Prices (£000) in rank order are 
 
68, 74.95, 75, 78, 79.95, 82.95, 85, 85, 85.95, 95, 95, 95, 97.5, 99.95, 105, 108, 115, 119.95, 120, 122.95. 
 
The median is the 1 2 "10 'th" observation in ascending order.  The 10th and 11th are both 95, so the median is 95. 
 
The lower quartile is between the 5th and 6th, which are 79.95 and 82.95.  It is acceptable to take the average of these, though other detailed definitions are also in use.  The upper quartile is found in a similar way.  On this basis, we take lower quartile:   () 1 79.95 82.95 81.45 2 +=,      upper quartile:   () 1 105 108 106.5 2 += . 
 
 
 
 
 
 
 
 
 
 
[Note.  This plot might not appear exactly correct, due to screen and/or printer resolution.] 
 
 
 
The median is approximately in the middle of the box, and the two whiskers are about the same length, so the distribution is not far from symmetrical.  There is also some clustering in the middle, near to the median, so a Normal distribution could be proposed as a model for these data. 
 
 
 
 
 
 
 
 
 
 
 
 
Continued on next page 
 
x x
 65  75  85  95  105 115 125 Prices (£000) 

 
 
(ii) For these data (using a pocket calculator;  some care is needed to preserve accuracy in dealing with the large numbers both here and for the adjacent suburb), 1 888 150y =∑ and 2 183 450 567 500 y =∑ , giving s = 16535.5222, so 2 273 423 493.4211 ys = .  (n = 20;  19 degrees of freedom.) 
 
 
For the adjacent suburb, n = 30, 2864490 x=∑ and 2 278 338 961 408 x =∑ , so () 2 22 1 1x x sx nn   =− −   ∑∑ is () 1 4 828 862 738 166 512 508.2069 29 = .  (29 degrees of freedom.) 
 
 
Test statistic for equality of variances is 22 /y xs s = 1.642 which is not significant when referred to F19,29.  There is not sufficient evidence to reject the null hypothesis, which is 22 "" XY σσ = . 
 
 
The F test is valid on the basis of apparent Normality for the original data, and the assumption that this is also true for the other sample. 

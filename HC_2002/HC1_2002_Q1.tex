\documentclass[a4paper,12pt]{article}
%%%%%%%%%%%%%%%%%%%%%%%%%%%%%%%%%%%%%%%%%%%%%%%%%%%%%%%%%%%%%%%%%%%%%%%%%%%%%%%%%%%%%%%%%%%%%%%%%%%%%%%%%%%%%%%%%%%%%%%%%%%%%%%%%%%%%%%%%%%%%%%%%%%%%%%%%%%%%%%%%%%%%%%%%%%%%%%%%%%%%%%%%%%%%%%%%%%%%%%%%%%%%%%%%%%%%%%%%%%%%%%%%%%%%%%%%%%%%%%%%%%%%%%%%%%%
\usepackage{eurosym}
\usepackage{vmargin}
\usepackage{amsmath}
\usepackage{graphics}
\usepackage{epsfig}
\usepackage{enumerate}
\usepackage{multicol}
\usepackage{subfigure}
\usepackage{fancyhdr}
\usepackage{listings}
\usepackage{framed}
\usepackage{graphicx}
\usepackage{amsmath}
\usepackage{chngpage}
%\usepackage{bigints}

\usepackage{vmargin}
% left top textwidth textheight headheight
% headsep footheight footskip
\setmargins{2.0cm}{2.5cm}{16 cm}{22cm}{0.5cm}{0cm}{1cm}{1cm}
\renewcommand{\baselinestretch}{1.3}

\setcounter{MaxMatrixCols}{10}
\begin{document}

Higher Certificate, Paper I, 2002. Question 1
\begin{enumerate}
%%%%%%%%%%%%%%%%%%%%%%%5
\item  P(all 13 from 39 non-spades) =
\begin{eqnarray*}
\[\frac{  {39 \choose 13} }{  {52 \choose 13}}  
&=& \frac{39! \times 13! \times 39!}{13! \times 26! \times 52!}\\
&=& \frac{39 \times 38 \times \ldots 28 \times 27}{52 \times 51 \times 41 \times 40}\\ 
&=& 0.1279\\
\end{eqnarray*}

%%%%%%%%%%%%%%%%%%%%%%%5
\item  There are 32 "low" cards (2 up to 9) and 20 "high" (10 to Ace). Hence all 13
are selected from 32, so probability is
\begin{eqnarray*}
\[\frac{  {32 \choose 13} }{  {52 \choose 13}}  
&=& \frac{32! \times 13! \times 39!}{13! \times 19! \times 52!}\\
&=& \frac{32! \times 39! }{ 19! \times 52!}\\
&=& 0.000547 \\
\end{eqnarray*}

%%%%%%%%%%%%%%%%%%%%%%%5
\item  6 of the 13 Hearts, and 7 of the remaining 39, must be selected.
Probability is ( ) ( )
( )( ) ( )( )
2 2
2
13 39
6 7 13!39!13!39! 13! 39!
52 6!7!7!32!52! 6! 7! 32! 52!
13
  
  
  = =
 
 
 
(= 0.04156).
%%%%%%%%%%%%%%%%%%%%%%%5
\item The division of Clubs must be 3, 3, 3, 4 between players, any one of whom
can be the person who receives 4; there are 4 ways for this. If the first player
receives 3, this can happen in
$ {13 \choose 3}$
 
 
 
ways, then
10
3
 
 
 
,
7
3
 
 
 
and
4
4
 
 
 
for the other
players. Their other cards are then selected from the remaining ones in
39
10
 
 
 
,
29
10
 
 
 
,
19
10
 
 
 
and
9
9
 
 
 
ways respectively. Therefore the probability is
( )4
13 39 10 29 7 19 4 9
. . .
3 10 3 10 3 10 4 9
4
52!
13!
           
           
           
 
 
 
, the divisor being the number of ways of
splitting 52 cards into 4 hands of 13 each.
Probability is 4 13!39! . 10!29! . 7!19! .1 13!13!13!13!
3!10!10!29! 3!7!10!19! 3!4!10!9! 52!
\[ = 4 \frac{(39)!(13!)^5}{(52!)(10!)^3(9!)(3!)^3(4!)}\]
\[  = \frac{(39)!(13!)^5}{(52!)(10!)^3(9!)(3!)^4}
\[ \approx 0.1054\]


\end{enumerate}
\end{document}

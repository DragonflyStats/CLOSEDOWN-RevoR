\documentclass[a4paper,12pt]{article}
%%%%%%%%%%%%%%%%%%%%%%%%%%%%%%%%%%%%%%%%%%%%%%%%%%%%%%%%%%%%%%%%%%%%%%%%%%%%%%%%%%%%%%%%%%%%%%%%%%%%%%%%%%%%%%%%%%%%%%%%%%%%%%%%%%%%%%%%%%%%%%%%%%%%%%%%%%%%%%%%%%%%%%%%%%%%%%%%%%%%%%%%%%%%%%%%%%%%%%%%%%%%%%%%%%%%%%%%%%%%%%%%%%%%%%%%%%%%%%%%%%%%%%%%%%%%
\usepackage{eurosym}
\usepackage{vmargin}
\usepackage{amsmath}
\usepackage{graphics}
\usepackage{epsfig}
\usepackage{enumerate}
\usepackage{multicol}
\usepackage{subfigure}
\usepackage{fancyhdr}
\usepackage{listings}
\usepackage{framed}
\usepackage{graphicx}
\usepackage{amsmath}
\usepackage{chngpage}
%\usepackage{bigints}

\usepackage{vmargin}
% left top textwidth textheight headheight
% headsep footheight footskip
\setmargins{2.0cm}{2.5cm}{16 cm}{22cm}{0.5cm}{0cm}{1cm}{1cm}
\renewcommand{\baselinestretch}{1.3}

\setcounter{MaxMatrixCols}{10}
\begin{document}
Higher Certificate, Paper I, 2002. Question 6
f (x) =λ e−λ x x > 0, λ > 0
\begin{enumerate}
\item ( ) ( )
0 0
tX x tx t x
X M t E e λ e λ dx λ e λ dx =   = ∞ − + = ∞ − −   ∫ ∫
( ) 1
0
1 1
e t x t
t t
λ λ
λ λ λ
 − − ∞ − = − = =  −   −  −      
[| t |<λ ] .
( )
2
2 1 ... x
M t t t
λ λ
= + + + .
\begin{itemize}
\item E[Xk] = coefficient of
!
xk
k
in the expansion.
\item Hence E[X ] 1
λ
= ; also, 2
2

\end{itemize}
E X 2
λ
  = , so ( )
2
2 2
Var X 2 1 1
λ λ λ
= −   =  
 
.
\item ( ) ( ) 1
1 1
| , ..., i exp
n n
x n
n i
i i
L λ x x λ e−λ λ λ x
= =
  = = − 
 
Π Σ .
ln ln (ln ) i L = n λ −λΣx = n λ −λ x
(ln )
0
d L n nx
dλ λ
= − = for ˆ 1
x
λ = .
2 ( )
2 2
d ln L n
dλ λ
= − which confirms maximum of L.
\item  The asymptotic variance of $\hat{\lambda}$ [Cramér-Rao lower bound for variance] is
( )
2
2
2
1
d ln L n
E
d
λ
λ
=
 
− 
 
.
We have
2 ˆ approx N ,
n
λ λ λ
 
∼  
 
, i.e. the estimate of $SE(\hat{\lambda})$ is
ˆ
n
λ , so that
ˆ ˆ ˆ ˆ 1.96 1.96
n n
λ − λ <λ <λ + λ is an approximate 95\% confidence interval for λ when
n is large.
This is 1 1.96 1 1.96
x x n x x n
− <λ < + .
\end{enumerate}
\end{document}
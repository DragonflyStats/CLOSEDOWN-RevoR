\documentclass[a4paper,12pt]{article}
%%%%%%%%%%%%%%%%%%%%%%%%%%%%%%%%%%%%%%%%%%%%%%%%%%%%%%%%%%%%%%%%%%%%%%%%%%%%%%%%%%%%%%%%%%%%%%%%%%%%%%%%%%%%%%%%%%%%%%%%%%%%%%%%%%%%%%%%%%%%%%%%%%%%%%%%%%%%%%%%%%%%%%%%%%%%%%%%%%%%%%%%%%%%%%%%%%%%%%%%%%%%%%%%%%%%%%%%%%%%%%%%%%%%%%%%%%%%%%%%%%%%%%%%%%%%
\usepackage{eurosym}
\usepackage{vmargin}
\usepackage{amsmath}
\usepackage{graphics}
\usepackage{epsfig}
\usepackage{enumerate}
\usepackage{multicol}
\usepackage{subfigure}
\usepackage{fancyhdr}
\usepackage{listings}
\usepackage{framed}
\usepackage{graphicx}
\usepackage{amsmath}
\usepackage{chngpage}
%\usepackage{bigints}

\usepackage{vmargin}
% left top textwidth textheight headheight
% headsep footheight footskip
\setmargins{2.0cm}{2.5cm}{16 cm}{22cm}{0.5cm}{0cm}{1cm}{1cm}
\renewcommand{\baselinestretch}{1.3}

\setcounter{MaxMatrixCols}{10}
\begin{document}
Higher Certificate, Paper I, 2002. Question 7
Y
0 1 2 3
0 k 6k 9k 4k
X 1 8k 18k 12k 2k
2 k 6k 9k 4k
\begin{itemize}
\item  The sum of all the entries in the table is 80k. Hence 1
80
k = .
\item  Row and column totals give the marginal distributions of X and Y:

\begin{center}
\begin{tabular}{|c|c|c|c|c|} \hline
X & 0 & 1 & 2\\ \hline
P(X) & 1/4 & 1/2 & 1/4 \\ \hline
\end{tabular}
\end{center}

\begin{center}
\begin{tabular}{|c|c|c|c|c|} \hline
Y & 0 & 1 & 2 & 3\\ \hline
P(Y)&  1/8 & 3/8 & 3/8 & 1/8\\ \hline
\end{tabular}
\end{center}
\item  For P( X = x |Y = 2) , use ( )
( )
and 2
2
P X x Y
P Y
= =
=
:
X 0 1 2
Probability 9 /80
3/8
= 9/30 = 0.3 12 / 80
3/8
= 0.4 9 /80
3/8
= 0.3
\item E[X] = 1, E[Y] = 1.5, by symmetry. The distribution of XY is:

\begin{center}
\begin{tabular}{|c|c|c|c|c|c|c|} \hline 
XY & 0 & 1 & 2 & 3 & 4 & 6\\ \hline
P(XY) & 29/80&  18/80&  18/80 & 2/80&  9/80&  4/80\\ \hline
\end{tabular}
\end{center}

[ ] 120 1.5
80
E XY = = .
So $E[XY ] = E[X ]E[Y ]$. Therefore Cov( X,Y ) = 0 , so correlation = 0.
\item Zero correlation is not sufficient. Every individual P( X = x, Y = y) in the
table must be the product of its two marginal probabilities.
Consider x = y = 0 . We have (0,0) 1
80
P = k = . But (0) (0) 1 1 1
4 8 32 X Y P P = × = . So
there is not independence.
\end{enumerate}
\end{document}

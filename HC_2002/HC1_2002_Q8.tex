\documentclass[a4paper,12pt]{article}
%%%%%%%%%%%%%%%%%%%%%%%%%%%%%%%%%%%%%%%%%%%%%%%%%%%%%%%%%%%%%%%%%%%%%%%%%%%%%%%%%%%%%%%%%%%%%%%%%%%%%%%%%%%%%%%%%%%%%%%%%%%%%%%%%%%%%%%%%%%%%%%%%%%%%%%%%%%%%%%%%%%%%%%%%%%%%%%%%%%%%%%%%%%%%%%%%%%%%%%%%%%%%%%%%%%%%%%%%%%%%%%%%%%%%%%%%%%%%%%%%%%%%%%%%%%%
\usepackage{eurosym}
\usepackage{vmargin}
\usepackage{amsmath}
\usepackage{graphics}
\usepackage{epsfig}
\usepackage{enumerate}
\usepackage{multicol}
\usepackage{subfigure}
\usepackage{fancyhdr}
\usepackage{listings}
\usepackage{framed}
\usepackage{graphicx}
\usepackage{amsmath}
\usepackage{chngpage}
%\usepackage{bigints}

\usepackage{vmargin}
% left top textwidth textheight headheight
% headsep footheight footskip
\setmargins{2.0cm}{2.5cm}{16 cm}{22cm}{0.5cm}{0cm}{1cm}{1cm}
\renewcommand{\baselinestretch}{1.3}

\setcounter{MaxMatrixCols}{10}
\begin{document}
Higher Certificate, Paper I, 2002. Question 8
45
50
55
60
65
70
16 18 20 22 24 26
Age (years)
y
There appears to be an outlier (25, 69), without which correlation (and the regression
gradient) would be near 0. (Rank correlation may be more appropriate.)
Σx =171 Σy = 477 n = 9 x =19 y = 53
1712 3309 60
9 XX S = − = .
4772 25633 352
9 YY S = − = . 9183 171 477 120
9 XY S = − × = .
y − y = bˆ (x − x ) , and ˆ 120 2
60
XY
XX
b S
S
= = = .
So we have y −53 = 2(x −19) or y = 2x +15
(a) For x = 18, yˆ = 51. (b) For x = 26, yˆ = 67 .
Including the data for I, correlation
2
XY
xy
XX YY
r S
S S
= = 0.826. (Significant at 5% level.)
[Hence the null hypothesis "b = 0" is rejected.]
However, without I, Σx =146 , x =18.25; Σy = 408 , y = 51.00 .
Σx2 = 3309 − 252 = 2684 ; Σy2 = 25633− 692 = 20872 ;
Σxy = 9183− (25×69) = 7458; ( )
2
ˆ 7458 146 408 / 8 12 0.615
2684 146 /8 19.5
b
− ×
= = =
−
.
19.5 XX S = , 12 XY S = ,
4082 20872 64
8 YY S = − = .
12
64 19.5 XY ∴r =
×
= 0.340 which does not approach significance. The line still has
positive gradient, but not significantly different from 0.
\end{document}

Higher Certificate, Paper II, 2002.  Question 6 
 
 
(i) We might reasonably suppose that each trial (i.e. a rat trying a door) has the same probability p of success, independently of all other trials.  These trials continue until there is success, on the xth trial;  there is only one order in which this can occur, namely x − 1 failures followed by one success, so we have ( ) ( ) 11 xP X x p p −= = − . 
 
The possible values of x are 1, 2, 3, … . 
 
In this case, if the rat is "guessing" there will be probability 1/3 of choosing the food door on any trial;  i.e. p is 1/3.  The weakness in this argument may be that the rat does not "guess" because it can detect food, e.g. by smell.  If that occurs, P(food door) is >1/3, and unknown. 
 
 
(ii) Expected frequencies are 
11250 33 x−         
 for x = 1, 2, … on the null hypothesis 
of a geometric distribution with p =1/3. 
 
x 1 2 3 4 5 6 7 ≥ 8 TOTAL Obs 15 11 7 6 5 4 2 0 50 Exp 16.67 11.11 7.41 4.94 3.29 2.19 1.46 2.93       5.48 4.39  
 
Because the geometric distribution tails off very slowly, it is not easy to combine expected values, but the above grouping is better than combining the whole tail from (say) 5 upwards because the pattern is better preserved (and degrees of freedom are saved). 
 
No parameters were estimated, so there will be 5 degrees of freedom for the usual chisquared test using x = 1; 2; 3; 4; (5, 6); ≥ 7. 
 
The test statistic is ()()()()()() 2 2 2 2 2 2 15 16.67 11 11.11 7 7.41 6 4.94 9 5.48 2 4.39 16.67 11.11 7.41 4.94 5.48 4.39 − − − − − − + + + + +
 
 
= 3.98. 
 
This value is not significant when compared with 2 5χ .  The geometric hypothesis is not rejected.  Therefore we may assume the model is satisfactory, and the animals do appear to be "guessing". 
 
\documentclass[a4paper,12pt]{article}
%%%%%%%%%%%%%%%%%%%%%%%%%%%%%%%%%%%%%%%%%%%%%%%%%%%%%%%%%%%%%%%%%%%%%%%%%%%%%%%%%%%%%%%%%%%%%%%%%%%%%%%%%%%%%%%%%%%%%%%%%%%%%%%%%%%%%%%%%%%%%%%%%%%%%%%%%%%%%%%%%%%%%%%%%%%%%%%%%%%%%%%%%%%%%%%%%%%%%%%%%%%%%%%%%%%%%%%%%%%%%%%%%%%%%%%%%%%%%%%%%%%%%%%%%%%%
\usepackage{eurosym}
\usepackage{vmargin}
\usepackage{amsmath}
\usepackage{graphics}
\usepackage{epsfig}
\usepackage{enumerate}
\usepackage{multicol}
\usepackage{subfigure}
\usepackage{fancyhdr}
\usepackage{listings}
\usepackage{framed}
\usepackage{graphicx}
\usepackage{amsmath}
\usepackage{chngpage}
%\usepackage{bigints}

\usepackage{vmargin}
% left top textwidth textheight headheight
% headsep footheight footskip
\setmargins{2.0cm}{2.5cm}{16 cm}{22cm}{0.5cm}{0cm}{1cm}{1cm}
\renewcommand{\baselinestretch}{1.3}

\setcounter{MaxMatrixCols}{10}
\begin{document}Higher Certificate, Paper II, 2002.  Question 7 
%%%%%%%%%%%%%%%%%%%%%%%%%%%%%%%%%%%%%%%%%%%%%%%%%%%%%%%%%%%%%%%%%%%%%%%%%%%%%%%%%%%%%%%%%%%%%%%%%%%%%%%%%%%%%%%%%%%%%%%%%%%%%%%%%%%%%%%%%%% 
\begin{table}[ht!]
 
\centering
 
\begin{tabular}{|p{15cm}|}
 
\hline  


7. (i) Briefly discuss the advantages and disadvantages of using non-parametric rather than parametric methods in statistical analyses. (6) 

\\ \hline
  
\end{tabular}

\end{table}

\begin{table}[ht!]
 
\centering
 
\begin{tabular}{|p{15cm}|}
 
\hline  
  
 (ii) A dietician needs to examine the effectiveness of a new educational intervention programme.  To study the intervention, a random sample of 14 overweight female patients was taken.  Each patient was weighed immediately before and six months after receiving the intervention.  The data obtained were as follows. 
 
Patient Weight before (kg) Weight after (kg) 1 135   75 2 129   85 3   75   65 4   84   85 5   77   80 6   95   90 7 140   88 8   72   80 9 118 104 10   76   66 11   97   82 12   88   57 13 115 101 14 106 103 
 
 
  Investigate whether the new educational intervention programme has an effect on the weight of overweight women using 
 
  (a) a sign test, 
(6) 
 
  (b) a Wilcoxon signed-rank test, 
(6) 
 
  in each case with a significance level of 0.05, and comment on the comparison of your results. (2) 
 
\\ \hline
  
\end{tabular}

\end{table} 
%%%%%%%%%%%%%%%%%%%%%%%%%%%%%%%%%%%%%%%%%%%%%%%%%%%%%%%%%%%%%%%%%%%%%%%%%%%%%%%%%%%%%%%%%%%%%%%%%%%%%%%%%%%%%%%%%%%%%%%%%%%%%%%%%%%%%%%%%%%
 
 
(i) When there is doubt about what distribution may be used to explain a set of data, especially when it is not possible to assume Normality (even after a transformation), methods that do not depend on distributional assumptions are useful.  There are several methods for analysing sets of data using distribution-free ("nonparametric") tests, although they are less powerful than those using distribution theory when the underlying distributions are (at least approximately) Normal, so sample sizes need to be larger for non-parametric tests. 
 
(ii) (a) If there has been no effect, the number of patients who lose weight should be binomially distributed, n = 14, p = ½.  A sign test allocates (say) + sign to those who have lost weight and – sign to those who have not, and does not use any whose weights remain exactly the same.  There are 11 + signs, out of 14. 
 
If B(14, ½) explains the situation, we have () 14 14 14 14 14111,12,13 or 14plus (+) signs 11 12 13 142 P          = + + +                   
 
  
14 1 14.13.12 14.13 14 1 2 3.2.1 2.1     = + + +        
 
()
14
14 1 470 364 91 15 0.0287 22  = + + = =   . 
 
A two-tail test using null hypothesis "no effect" and alternative hypothesis "some effect" (unspecified) therefore has p-value 0.0574, and does not provide evidence on which to reject the null hypothesis. 
 
(b) A Wilcoxon signed-rank test uses the sizes as well as the signs of the differences, and so carries more power to reject the null hypothesis when it is false. 
 
Patient 1 2 3 4 5 6 7 8 9 10 11 12 13 14 Difference 60 44 10 -1 -3 5 52 -8 14 10 15 31 14 3 Rank 14 12 6.5 1 2.5 4 13 5 8.5 6.5 10 11 8.5 2.5 
 
The ranks are those of the absolute differences.  The sums of the ranks for the positive and negative differences are 1 296T + = and 1 28T − = .  The test statistic is () min , 8.5 T T T −+ ==.  The tables, with n = 14 and α = 0.05 (two-sided), give T = 21.  Since 8.5 is (much) lower than this, there is evidence to reject the null hypothesis.  Inspection of the data shows that the negatives (i.e. weight gains) are generally small in size compared with the positives (weight losses). 
\end{document}

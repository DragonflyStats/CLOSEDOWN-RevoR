\documentclass[a4paper,12pt]{article}
%%%%%%%%%%%%%%%%%%%%%%%%%%%%%%%%%%%%%%%%%%%%%%%%%%%%%%%%%%%%%%%%%%%%%%%%%%%%%%%%%%%%%%%%%%%%%%%%%%%%%%%%%%%%%%%%%%%%%%%%%%%%%%%%%%%%%%%%%%%%%%%%%%%%%%%%%%%%%%%%%%%%%%%%%%%%%%%%%%%%%%%%%%%%%%%%%%%%%%%%%%%%%%%%%%%%%%%%%%%%%%%%%%%%%%%%%%%%%%%%%%%%%%%%%%%%
\usepackage{eurosym}
\usepackage{vmargin}
\usepackage{amsmath}
\usepackage{graphics}
\usepackage{epsfig}
\usepackage{enumerate}
\usepackage{multicol}
\usepackage{subfigure}
\usepackage{fancyhdr}
\usepackage{listings}
\usepackage{framed}
\usepackage{graphicx}
\usepackage{amsmath}
\usepackage{chngpage}
%\usepackage{bigints}

\usepackage{vmargin}
% left top textwidth textheight headheight
% headsep footheight footskip
\setmargins{2.0cm}{2.5cm}{16 cm}{22cm}{0.5cm}{0cm}{1cm}{1cm}
\renewcommand{\baselinestretch}{1.3}

\setcounter{MaxMatrixCols}{10}
\begin{document}
Higher Certificate, Paper I, 2002. Question 3

\begin{enumerate}
\item The taster does not know how many there are of each sort, so with random
guessing 1
2 p = is the probability of being correct. Binomial ( 1 )
2 10, is a satisfactory
model if the samples are independent (i.e. presented randomly). Hence the mean is np
= 5, and variance np (1− p) = 2.5 .
\item  SAMPLE
Butter Margarine TOTAL
Butter x 5 − x 5
GUESS Margarine 5 − x x 5
TOTAL 5 5 10

\begin{itemize}
\item The correct value of X would be 5, but with random guessing the actual value of X
may be 0, 1, 2, 3, 4 or 5.
\item With random guessing, all of the
10
5
 
 
 
ways of guessing 5
of each type will be equally likely. 
\item The number of ways of guessing x out of 5 butter
samples is
5
x
 
 
 
, and of guessing (5 − x) out of 5 margarine is
5
5 x
 
 −   
. 
\item The total
number of ways of generating the above table is
5 5
x 5 x
  
  −    
, and each has probability
10
1/
5
 
 
 
, so the required probability distribution is ( ) 5 5 10
5 5
p x
x x
    
=        −   
, for
x = 0, 1, …, 5.
10 10! 252
5 5!5!
 
  = =
 
. 
\item For x = 0 or 5, the numerator in p(x) is 1. \item For x = 1 or 4, the
numerator is
5 5
5 5 25
1 4
  
   = × =
  
; and for x = 2 or 3 it is
5 5
100
2 3
  
    =
  
. 
\end{itemize}
%%%%%%%%%%%%
\begin{itemize}
    \item Therefore the
probability mass function for x is
\begin{center}
\begin{tabular}{|c|c|c|c|c|c|c|}
\hline 
x & 0 & 1& 2& 3& 4& 5\\ \hline
p(x)& 1/252& 25/252& 100/252 &100/252& 25/252 & 1/252\\ \hline 
\end{tabular}
\end{center}

\item By symmetry the mean is 2.5.
2 ( ) 1 ((25 17) (100 ) (1 )) 1750 125
252 252 18
\[ \sum x p(x) = x + x13 + × 25 = = .\]
\item Therefore Var(X) = ( [ ])2 2 125 25 250 225 25
18 4 36 36
E X  − E X = − = − = .
\item The number of samples out of all 10 that are guessed correctly is 2X, which has mean
= 5 and variance = 25
9
.
\end{itemize}

\end{enumerate}
\end{document}
Higher Certificate, Paper II, 2002.  Question 5 
 
 
(i) 
 
The frequency densities in the first 8 intervals are 1.5, 10, 12, 14, 16, 18, 10 and 3 respectively.  The width of the last interval must be chosen arbitrarily;  it could be to 75 (or even more), but as the frequency above 55 drops off sharply it seems reasonable to assume that the last interval ends at 70.  This affects the calculations in part (ii) very little, and the histogram not much.  On this basis, the frequency density in the last interval is 1. 
 
 
 
 
 
 
 
 
 
 
 
 
 
 
 
 
 
 
 
 
 
 
 
 
 
 
 
 
 
 
 
 
 
 
 
 
Continued on next page 
 
10 20 30 40 50 60 70
Yield (kg) 
5 
10 
15 
Frequency density 
0 

 
 
(ii) Modal class is [≥ 40 but < 45], as it has greatest frequency density (allowing for different widths of intervals). 
 
The median is the 1 2 "50 'th" observation in ascending order. This is at 11.5 35 5 16 +× = 38.6. 
 
Yield y 
Frequency f 
Mid-point x 
fx 2 fx Cum freq F 10 ≤ y < 20 3 15 45 675 3 20 ≤ y < 25 10 22.5 225 5062.50 13 25 ≤ y < 30 12 27.5 330 9075 25 30 ≤ y < 35 14 32.5 455 14787.50 39 35 ≤ y < 40 16 37.5 600 22500 55 40 ≤ y < 45 18 42.5 765 32512.50 73 45 ≤ y < 55 20 50 1000 50000 93 55 ≤ y < 65 6 60 360 21600 99 y ≥ 65 1 (67.5) 67.5 4556.25 100    3847.5 160768.75  
 
 
38.475
fx
x
f == ∑ ∑ , or 38.5 to a reasonable level of accuracy. 2 2 1 3847.5 12736.1875 160768.75 128.648 99 100 99
s
 = − = =   , so the standard deviation is 11.34. 
 
 
(iii) In order to calculate x , all the frequency in each interval had to be concentrated at the centre.  This has given an over-estimate, so there must have been more left-of-centre observations in some (or all) intervals.  The median is also an over-estimate, due to assuming a uniform spread of the data in the interval 35 – 40.  This also suggests some skewness in intervals. 

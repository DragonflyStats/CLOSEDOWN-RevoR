\documentclass[a4paper,12pt]{article}
%%%%%%%%%%%%%%%%%%%%%%%%%%%%%%%%%%%%%%%%%%%%%%%%%%%%%%%%%%%%%%%%%%%%%%%%%%%%%%%%%%%%%%%%%%%%%%%%%%%%%%%%%%%%%%%%%%%%%%%%%%%%%%%%%%%%%%%%%%%%%%%%%%%%%%%%%%%%%%%%%%%%%%%%%%%%%%%%%%%%%%%%%%%%%%%%%%%%%%%%%%%%%%%%%%%%%%%%%%%%%%%%%%%%%%%%%%%%%%%%%%%%%%%%%%%%
\usepackage{eurosym}
\usepackage{vmargin}
\usepackage{amsmath}
\usepackage{graphics}
\usepackage{epsfig}
\usepackage{enumerate}
\usepackage{multicol}
\usepackage{subfigure}
\usepackage{fancyhdr}
\usepackage{listings}
\usepackage{framed}
\usepackage{graphicx}
\usepackage{amsmath}
\usepackage{chngpage}
%\usepackage{bigints}

\usepackage{vmargin}
% left top textwidth textheight headheight
% headsep footheight footskip
\setmargins{2.0cm}{2.5cm}{16 cm}{22cm}{0.5cm}{0cm}{1cm}{1cm}
\renewcommand{\baselinestretch}{1.3}

\setcounter{MaxMatrixCols}{10}
\begin{document}

% Higher Certificate, Paper II, 2002.  Question 2 
 
 
(a) In a simple significance test, there is a null hypothesis (NH or H0) which is the basis for calculations and an alternative hypothesis (AH or H1) which is accepted when the NH is rejected.  For example, NH may be that data come from () 2 1 N, µ σ
 and AH that they come from () 2 2 N, µ σ , with 21 µµ > . 
 
 
 
 
 
 
 
 
 
 
 
 
 
(i) Type I error = P(reject H0 when H0 is true). 
 (ii) Type II error = P(not reject H0 when H1 is true). 
 (iii) Level of significance = P(Type I error) = α . 
 (iv) Power = 1 − β , where β = P(Type II error). 
 
 
(b) (i) 9 n= .   222.0 x = .   2 23.50 s = .   0H : 220 µ = ;   1H : 220 µ > .  Test statistic is 222.0 220.0 2.0 1.238 1.61623.50/9 − ==, refer to t8. 
 A one-tail test is required, so the 5% point of t8 is 1.860.  The result is not significant.  There is no evidence that the recommended intake is exceeded, on average. 
 
(ii) If n = 25, with the same values of x and 2 s as in (i), the test statistic is 2.0 2.063 23.50/25 = which is referred to t24.  This is significant as a one-tail test (5% point 1.711).  Therefore we may reject the null hypothesis and accept that the recommended intake is exceeded.  A larger sample size has given a more powerful test. 
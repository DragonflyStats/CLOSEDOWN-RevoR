\documentclass[a4paper,12pt]{article}
%%%%%%%%%%%%%%%%%%%%%%%%%%%%%%%%%%%%%%%%%%%%%%%%%%%%%%%%%%%%%%%%%%%%%%%%%%%%%%%%%%%%%%%%%%%%%%%%%%%%%%%%%%%%%%%%%%%%%%%%%%%%%%%%%%%%%%%%%%%%%%%%%%%%%%%%%%%%%%%%%%%%%%%%%%%%%%%%%%%%%%%%%%%%%%%%%%%%%%%%%%%%%%%%%%%%%%%%%%%%%%%%%%%%%%%%%%%%%%%%%%%%%%%%%%%%
\usepackage{eurosym}
\usepackage{vmargin}
\usepackage{amsmath}
\usepackage{graphics}
\usepackage{epsfig}
\usepackage{enumerate}
\usepackage{multicol}
\usepackage{subfigure}
\usepackage{fancyhdr}
\usepackage{listings}
\usepackage{framed}
\usepackage{graphicx}
\usepackage{amsmath}
\usepackage{chngpage}
%\usepackage{bigints}

\usepackage{vmargin}
% left top textwidth textheight headheight
% headsep footheight footskip
\setmargins{2.0cm}{2.5cm}{16 cm}{22cm}{0.5cm}{0cm}{1cm}{1cm}
\renewcommand{\baselinestretch}{1.3}

\setcounter{MaxMatrixCols}{10}
\begin{document}
Higher Certificate, Paper I, 2002. Question 4
\begin{framed}

\noindent Fatal accidents occur at random at a known 'black spot', following a
Poisson process with mean 4 per year. Draw a diagram of the probability
mass function of X, the actual annual number of fatal accidents, and
calculate the probabilities of the following events.
(a) In a given year there is at most one fatal accident.
(b) In a given 6-month period there are no fatal accidents.
(c) In a given 18-month period there are no fatal accidents.
\end{framed}

%%%%%%%%%%%%%%%%%%%%%%%%%%%%%%%%%%%
\begin{enumerate}
\item ( )
4 4
!
e r P X r
r
−
= = , r = 0, 1, 2, … .
( )
( )
( ) ( )
( ) ( )
( ) ( )
( ) ( )
( ) ( )
( ) ( )
( ) ( )
( ) ( )
4
4
0 0.0183
1 4 0.0733
2 2 1 0.1465
3 4 2 0.1954
3
4 3 0.1954
5 4 4 0.1563
5
6 2 5 0.1042
3
7 4 6 0.0595
7
8 1 7 0.0298
2
9 4 8 0.0132
9

\begin{itemize}
    \item $P(X=0)$
    \item $P(X=1)$
    \item $P(X=2)$
    \item $P(X=3)$
    \item $P(X=4)$
    \item $P(X=5)$
    \item $P(X=6)$
    \item $P(X=7)$
    \item $P(X=8)$
\end{itemize}

P e
P e
P P
P P
P P
P P
P P
P P
P P
P P
−
−
= =
= =
= =
= =
= =
= =
= =
= =
= =
= =
0
0.05
0.1
0.15
0.2
0.25
0 1 2 3 4 5 6 7 8 9
No. of accidents
Probability
and so on (probabilities beyond r = 9 have not been shown on the diagram).
\item P(at most 1 fatal accident) = P(0) + P(1) = 0.0916.
\item In half-year, mean = 2 giving P(0) = e−2 = 0.1353 .
\item In 1
2 1 years, mean = 6 giving P(0) = e−6 = 0.00248 .

\newpage
\begin{framed}
(ii) The annual number, Y say, of non-fatal accidents at the same place may be
assumed to be a Poisson random variable with mean 12, and X and Y are
independent. State the distribution of V = X + Y and write down the mean
and variance of V. Given that there were in total 20 accidents one year,
find the probability that 5 of these were fatal.
\end{framed}

\item  Given that X and Y are independent Poissons, V = X +Y ~ Poisson(16).
E[V ] = Var (V ) =16 .
( ) ( )
( )
4 5 12 15
16 20
5, 15 4 12 20! 5 | 20 . .
20 5! 15! 16
P X Y e e P X V
P V e
− −
−
= =
= = = =
=
=
5 15 20 15 15
20 20 20
4 12 . 20! 4 3 . 20 19 18 17 16 3 .19
16 5!15! 16 5 4 3 2 4
= × × × × = ×17× 48
× × ×
= 0.2023.

\begin{eqnarray*}
P(X=5|V=20) &=& \frac{P(X=5,Y=15)}{P(V=20)}\\
&=& \frac{4^5}{5!} \times \frac{12^{15}}{15!} \times \frac{20!}{16^{20}}\\
&=& 0.2023
\end{eqnarray*}

%%%%%%%%%%%%%%%%%%%%%%%%%%%%%%%%%%%%%%%%%%%%
% Normal Approximation of the Poisson Distribution
\newpage
\begin{framed}
(iii) State the distribution of W, the total number of accidents in a given 4-year
period. Use a suitable approximation to calculate the probability that there
will be more than 70 accidents in the next 4 years.


\end{framed}
\item  W is Poisson with mean $16 \times 4 = 64$. Use Normal approximation $N(64, 64)$.

\[P( W > 50) = P\left( Z \geq \frac{70.5-64}{8}\right)\] 
here $Z \sim N(0,1)$ and a continuity correction is
used. This is $P(Z > 0.8125)$, which is \[1-\Phi(0.8125) =1- 0.7917 = 0.2083.\]
\end{enumerate}
\end{document}

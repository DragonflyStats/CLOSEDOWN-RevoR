\documentclass[a4paper,12pt]{article}
%%%%%%%%%%%%%%%%%%%%%%%%%%%%%%%%%%%%%%%%%%%%%%%%%%%%%%%%%%%%%%%%%%%%%%%%%%%%%%%%%%%%%%%%%%%%%%%%%%%%%%%%%%%%%%%%%%%%%%%%%%%%%%%%%%%%%%%%%%%%%%%%%%%%%%%%%%%%%%%%%%%%%%%%%%%%%%%%%%%%%%%%%%%%%%%%%%%%%%%%%%%%%%%%%%%%%%%%%%%%%%%%%%%%%%%%%%%%%%%%%%%%%%%%%%%%
\usepackage{eurosym}
\usepackage{vmargin}
\usepackage{amsmath}
\usepackage{graphics}
\usepackage{epsfig}
\usepackage{enumerate}
\usepackage{multicol}
\usepackage{subfigure}
\usepackage{fancyhdr}
\usepackage{listings}
\usepackage{framed}
\usepackage{graphicx}
\usepackage{amsmath}
\usepackage{chngpage}
%\usepackage{bigints}

\usepackage{vmargin}
% left top textwidth textheight headheight
% headsep footheight footskip
\setmargins{2.0cm}{2.5cm}{16 cm}{22cm}{0.5cm}{0cm}{1cm}{1cm}
\renewcommand{\baselinestretch}{1.3}

\setcounter{MaxMatrixCols}{10}
\begin{document}

Higher Certificate, Paper III, 2002. Question 3
%%%%%%%%%%%%%%%%%%%%%%%%%%%%%%%%%%%%%%%%%%%%%%%%%%%%%%%%%%%%%% 
\begin{framed}
3. The table below shows the UK Gross Domestic Product (GDP, y) for the years 1989 − 1999, with years also coded as t = year − 1994.  
The figures are given in units of £bn, and are expressed in 1995 prices. 
 
 
Year 1989 1990 1991 1992 1993 1994 1995 1996 1997 1998 1999 t -5 -4 -3 -2 -1 0 1 2 3 4 5 y 655.2 659.5 649.8 650.3 665.4 694.6 714.0 732.2 757.9 777.9 794.4 
 Source:  United Kingdom National Accounts, 2000 edn, table 1.1. 
 Note:    Σy = 7751.2,    Σy2 = 5491108.76,    Σt = 0,    Σt2 = 110,    Σty = 1706.3. 
 
 
 
(i) A model $y = \alpha + \beta t + \varepsilon$ , where ε is a random error term with the usual properties, is proposed for the data. 
 Obtain least squares estimates of α
 and β , and calculate r2 (the coefficient of determination).  Also estimate $\sigma^2$, the variance of ε , and obtain estimates of the standard errors of the coefficients α and β . (6) 
 
(ii) What are "the usual properties" of the errors?  How realistic is the assumption that the errors have these properties?  (You are not expected to describe or conduct any tests.) (2) 
 (iii) Interpret the value of r2, and the values of your estimates of $\alpha$ and $\beta$ . (3) 
 
(iv) Draw a time chart of the data and superimpose your estimated function on it. (5) 
 
(v) Use your estimated model to predict GDP in 2000 and 2010, and comment on your predictions in the light of your graph. (4) 
 
 
\end{framed}
%%%%%%%%%%%%%%%%%%%%%%%%%%%%%%%%%%%%%%%%%%%%%%%%%%%%%%%%%%%%%% 

n =11, 0 i Σt = , 2 110 i Σt = , 7751.2 i Σy = , 2 5491108.76 i Σy = , 1706.3 i i Σt y = .
\begin{enumerate} 
\item  110 tt S = , 1706.3 ty S = (since Σt = 0 ); hence ˆ 1706.3
110
β = = 15.5118.
αˆ = y −βˆt = y = 704.6545 .
7751.22 5491108.76 29190.4473
11 yy S = − = .
2
2 ty
tt yy
S
r
S S
= = 0.9067.
The sum of squares due to fitting the regression line is
2
ty 26467.8154
tt
S
S
= and the
residual SS is therefore
2
ty 2722.6319
yy
tt
S
S
S
− = .
\begin{itemize}
\item This has 11 – 2 = 9 df and so σˆ 2 = 302.5147.
\item Thus estimated variances of $\hat{\alpha}}$ and $\hat{\beta}}$ are given by
( ) ˆ ˆ 2 Var
tt S
β =σ = 2.7501, ( )
2 2
2 ˆ ˆ 1 ˆ Var
11 tt
t
n S
α σ σ
 
=  +  =
 
(since t = 0 ) = 27.5013.
\item Hence SE($\hat{\alpha}}$) = 5.24 and SE($\hat{\beta}}$ ) = 1.658.
\end{itemize}
%%%%%%%%%%%%%%%%%%%%%
\item  The { ε
i} in the linear model are usually assumed to be independent, from the
same Normal distribution, mean 0, variance σ 2. As y increases with t, it is possible
that the variance also increases (heteroscedasticity); and due to likely trade-cycles
and also non-linearity of the model, independence may also be doubtful.
\item  The correlation coefficient r = 0.9067 = 0.952 . This shows a strong linear
relation between y and time. Alternatively, r2 shows that 90.7\% of the variation in the
annual GDP figures can be explained as a linear trend in time, with positive slope
(βˆ =15.52 ). The slope is the average annual increase in GDP over the period, which
is £15.52 bn (expressed at 1995 prices); this is over 2\% of the annual figure.
αˆ = 704.65 was the figure estimated for t = 0 , i.e. 1994, on the basis of all the data
(the actual figure allowing for the random variation was 694.6).
Continued on next page
\item  t = −3 ; yˆ = 704.6545 − (3×15.5118) = 658.12 .
t = +3 ; yˆ = 704.6545 + (3×15.5118) = 751.19 .
630
655
680
705
730
755
780
805
-5 -4 -3 -2 -1 0 1 2 3 4 5
\item  Note that because of the "flat" period in the first 4 years (or the drop back in
1991, however this is interpreted), the linear relation is forced to have a somewhat
smaller slope than seems necessary to explain 1991 – 1999 well.
For 2000, yˆ = 704.6545 + (6×15.5118) = 797.73. Because of the slope not reflecting
the years' GDP in late 1990s, this estimate is probably low (but not unreasonable).
For 2010, yˆ = 704.6545 + (16×15.5118) = 952.84, but this can only be a guess as the
relationship may become curved, or the slope change, or a discontinuity happen (as in
1990/1), and extrapolation so far ahead is not really of any use.
(1989) (1994) (1999)
GDP in £bn at
1995 prices
t

\end{enumerate}

\end{document}

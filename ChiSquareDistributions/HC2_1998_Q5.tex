\documentclass[a4paper,12pt]{article}
%%%%%%%%%%%%%%%%%%%%%%%%%%%%%%%%%%%%%%%%%%%%%%%%%%%%%%%%%%%%%%%%%%%%%%%%%%%%%%%%%%%%%%%%%%%%%%%%%%%%%%%%%%%%%%%%%%%%%%%%%%%%%%%%%%%%%%%%%%%%%%%%%%%%%%%%%%%%%%%%%%%%%%%%%%%%%%%%%%%%%%%%%%%%%%%%%%%%%%%%%%%%%%%%%%%%%%%%%%%%%%%%%%%%%%%%%%%%%%%%%%%%%%%%%%%%
\usepackage{eurosym}
\usepackage{vmargin}
\usepackage{amsmath}
\usepackage{graphics}
\usepackage{epsfig}
\usepackage{enumerate}
\usepackage{multicol}
\usepackage{subfigure}
\usepackage{fancyhdr}
\usepackage{listings}
\usepackage{framed}
\usepackage{graphicx}
\usepackage{amsmath}
\usepackage{chngpage}
%\usepackage{bigints}

\usepackage{vmargin}
% left top textwidth textheight headheight
% headsep footheight footskip
\setmargins{2.0cm}{2.5cm}{16 cm}{22cm}{0.5cm}{0cm}{1cm}{1cm}
\renewcommand{\baselinestretch}{1.3}

\setcounter{MaxMatrixCols}{10}
\begin{document}
	%%%%%%%%%%%%%%%%%%%%%%%%%%%%%%%%%%%%%%%%%%%%%%%%%%%%%%%%%%%%%%%%%%%%%%%%%%%%%%%%%%%%%%%%%%%%%%%%%%%%%%%%%%%%%%%%%%%%%
	\begin{table}[ht!]
		
		\centering
		
		\begin{tabular}{|p{15cm}|}
			
			\hline  
			
			\large
			\noindent A random sample of 100 rats were each put into a maze until they were able to find the correct path.  The number of attempts required by each rat was recorded as follows:
			\begin{center}
				\begin{tabular}{|c|c|c|c|c|c|c|c|}\hline
					Number of attempts &  1& 2& 3 &4 &5 &6 & $>$7\\\hline
					Number of rats& 56& 27& 13& 3 &0 &1& 0\\\hline
				\end{tabular}
			\end{center}
			(a) Explain why the number of attempts taken by each rat might follow a geometric distribution whose probability density function is \[P (X=x) = (1- p)^{x-1} \times p.\]
			(b) Test the hypothesis that the distribution of the number of attempts needed for each rat is geometric, carefully explaining your conclusions.
			
			
			\\ \hline
			
		\end{tabular}
		
	\end{table}
	
	
	%%%%%%%%%%%%%%%%%%%%%%%%%%%%%%%%%%%%%%%%%%%%%%%%%%%%%%%%%%%%%%%%%%%%%%%%%%%%%%%%%%%%%%%%%%%%%%%%%%%%%%%%%%%%%%%%%%%%%
	\large
	\begin{itemize}
		\item If $p$ is the probability of success at any attempt, and the rat does not `learn` which routes are
		failures, so that each result is independent of others, then the geometric distribution explains the
		number of trials needed to gain one success.
		\item The value of $p$ must be estimated from the data.
		\[ \hat{p} = \frac{1}{\bar{x}} ,\] 
		\[\bar{x} = \frac{[(1 \times 56) + (2 \times 27) + (3 \times 13) + (4 \times 3) + (6 \times 1)]}{100} = 1.67.\]
		Hence $\hat{p} = 0.5988$.\\ 
		\item Calculate $P(1)$ etc. on geometric distribution.
		\begin{itemize}
			\item[$\bullet$] P(1) \;=\; $0.5988$ 
			\item[$\bullet$] P(2) \;=\; $0.5988\times0.4012=0.2402$ .
			\item[$\bullet$] P(3) \;=\; $0.5988 \times (0.4012)^2=0.0964$ . 
			\item[$\bullet$] P(4) \;=\; 0.0387 etc.
		\end{itemize}
		
		\begin{center}
			\begin{tabular}{|c|c|c|c|c|c|} \hline
				x & 1 & 2 & 3 & $\geq 4$ & TOTAL \\ \hline 
				Observed & 56 & 27 & 13 & 4 & 100 \\ \hline
				Expected & 59.88 & 24.02 & 9.64 & 6.46 & 100\\ \hline
			\end{tabular}
		\end{center}
		\large
		\item Combine ”$ \geq 4$” into one class to avoid very small expected frequencies. One parameter was
		estimated, so $\chi^{2}$ has 2 d.f. for testing fit to the geometric.
		\[\chi^{2}_{(2)} =
		\frac{(56\;-\; 59.88)^2}{
			59.88}
		+
		\frac{(27  \;-\; 24.02)^2}{24.02}
		+
		\frac{(13 \;-\;  9.64)^2}{9.64}
		+
		\frac{(4 \;-\;  6.46)^2}{6.46}
		\;=\; 2.73\]
		\item From the Chi Square table, the critical value is 5.99.
		\item There is no evidence against the hypothesis of fit to a geometric distribution, nor therefore against
		the conditions stated in (a).
	\end{itemize}
\end{document}

\documentclass[a4paper,12pt]{article}

%%%%%%%%%%%%%%%%%%%%%%%%%%%%%%%%%%%%%%%%%%%%%%%%%%%%%%%%%%%%%%%%%%%%%%%%%%%%%%%%%%%%%%%%%%%%%%%%%%%%%%%%%%%%%%%%%%%%%%%%%%%%%%%%%%%%%%%%%%%%%%%%%%%%%%%%%%%%%%%%%%%%%%%%%%%%%%%%%%%%%%%%%%%%%%%%%%%%%%%%%%%%%%%%%%%%%%%%%%%%%%%%%%%%%%%%%%%%%%%%%%%%%%%%%%%%

\usepackage{eurosym}
\usepackage{vmargin}
\usepackage{amsmath}
\usepackage{graphics}
\usepackage{epsfig}
\usepackage{enumerate}
\usepackage{multicol}
\usepackage{subfigure}
\usepackage{fancyhdr}
\usepackage{listings}
\usepackage{framed}
\usepackage{graphicx}
\usepackage{amsmath}
\usepackage{chngpage}

%\usepackage{bigints}
\usepackage{vmargin}

% left top textwidth textheight headheight

% headsep footheight footskip

\setmargins{2.0cm}{2.5cm}{16 cm}{22cm}{0.5cm}{0cm}{1cm}{1cm}

\renewcommand{\baselinestretch}{1.3}

\setcounter{MaxMatrixCols}{10}

\begin{document}
\large
% Higher Certificate, Paper I, 2006. Question 4
% 4. 
\noindent My cycle journey to work is 3 km, and my cycling time (in minutes) if there are no delays is distributed $N(15, 1)$, i.e. Normally with mean $\mu = 15$ and variance $\sigma^2 = 1$.
\smallskip

\begin{table}[ht!]
 \centering
 \begin{tabular}{|p{15cm}|}
 \hline\large
\noindent \textbf{Part (a)}\\ \large  Find the probability that, if there are no delays, I get to work in at most 17 minutes.

\\ \hline
  \end{tabular}
\end{table}



\large 



\begin{enumerate}[(a)]
\item 

Let $X$ represent cycling time without delays: $X \sim N(15, 1)$.

Compute $P(X \leq 17)$
\begin{framed}
\noindent \textbf{Z-score}
\[Z_{17} \;=\; \frac{17 - \mu}{\sigma} \;= \;\frac{17-15}{1}= 2.00\]
\end{framed}
\begin{eqnarray*} 
P(X \leq 17)  &=& P(Z \leq 2.00) \\
 &=& \Phi(2.00) \qquad (\mbox{Equivalently})\\
 & & (\mbox{From Statistical Tables}) \\
 &=& 0.9772
\end{eqnarray*} 
\large
Here $\Phi$ denotes the cumulative distribution function of the standard Normal distribution (for Statistical Tables.)

\newpage
\large

\begin{table}[ht!]
 \centering
 \begin{tabular}{|p{15cm}|}
 \hline
\noindent \textbf{Part (b)}\\ \large On my route there are three sets of traffic lights.

Each time I meet a red traffic light, I am delayed by a random time that is distributed $N(0.7, 0.09)$. 

These lights operate independently. Find the probability of my getting to work in at most 17 minutes

(1) if just one light is set at red when I reach it,

(2) if just two lights are set at red when I reach them,

(3) if all three lights are set at red when I reach them.

\\ \hline
  \end{tabular}
\end{table}



\large
\item  Adding in the delay times, also Normally distributed [N(0.7, 0.09)], and letting $T$ denote the total time:
\begin{itemize}
\item One red light: $T \sim N(15.7, 1.09)$, so

\[P(T \leq 17) = \Phi \left( \frac{17 - 15.7}{\sqrt{1.09}} \right) = \Phi \left( \frac{1.3}{ 1.0440} \right)  = \Phi(1.245) = 0.8934\]
\item Two red lights: $T \sim N(16.4, 1.18)$, so



\[P(T \leq 17) = \Phi \left(\frac{17 - 16.4}{\sqrt{1.18}} \right) = \Phi \left(\frac{0.6}{1.0863}\right) = \Phi(0.552) = 0.7096\]

\item Three red lights: $T \sim N(17.1, 1.27)$, so

\[P(T \leq 17) = \Phi \left(\frac{17 - 17.1}{\sqrt{1.27}} \right) = \Phi \left( \frac{-0.1}{1.127}\right) = \Phi(-0.0887) = 0.4646\]
.
\end{itemize}

\newpage


\begin{table}[ht!]
 \centering
 \begin{tabular}{|p{15cm}|}
 \hline  \large  
\noindent \textbf{Part (c)}\\ \large  Suppose that, for each set of lights, the chance of delay is 0.5. Deduce that the mean value of 
$T$, my total journey time, is 16.05 minutes.


\\ \hline
  \end{tabular}
\end{table}
 \item The number of delays is distributed as $B(3, 1/2)$. Hence the situations described in the previous parts will arise with probabilities 1/8, 3/8, 3/8 and 1/8 respectively.
 
\begin{center} 
\begin{tabular}{|c|c|c|c|c|} \hline
Scenario  & No Delays & 1 Red light & 2 Red lights  & 3 Red lights  \\ \hline
Expected Time & 15 & 15.7 & 16.4 & 17.1 \\ \hline 
Probability &  1/8 & 3/8 & 3/8 & 1/8  \\ \hline
\end{tabular}
\end{center}

The (unconditional) mean of the total journey time is
\begin{eqnarray*}
E(T) &=& \frac{1}{8} \times 15 + \frac{3}{8} \times 15.7 + \frac{3}{8} \times 16.4  + \frac{1}{8} \times 17.1\\
&=& \frac{128.4}{8} \\
&=& 16.05 \mbox{ minutes.}\\
\end{eqnarray*}
%%%%%%%%%%%%%%%%%%%%%%%%%%%%%%%%%%%%%%%%%
\newpage
\begin{table}[ht!]
 \centering
 \begin{tabular}{|p{15cm}|}
 \hline  \large  
\noindent \textbf{Part (d)}\\ \large Given that $\operatorname{Var}(\bar{T}) = 1.5025$, use a suitable approximation to calculate the probability that, over 10 journeys, my average journey time to work is at most 17 minutes.
\\ \hline
  \end{tabular}
\end{table}


\item Mean time $\bar{T} \sim N(16.05,\frac{1.5025}{10})$ 

\[P(\bar{T} \leq 17) = \Phi \left(\frac{17 - 16.05}{\sqrt{0.15025}} \right) = \Phi \left(\frac{0.95}{0.3876}\right) = \Phi(2.451) = 0.9929\]
\end{enumerate}
\newpage
End
\end{document}
